%%%%%%%%%%%%%%%%%%%%%%%%%%%%%%%%%%%%%%%%%%%%%%%%%%%%%%%%%%%%%%%%%%%%%%%%
%%% documentclass and packages
%%%%%%%%%%%%%%%%%%%%%%%%%%%%%%%%%%%%%%%%%%%%%%%%%%%%%%%%%%%%%%%%%%%%%%%%
\RequirePackage{atbegshi}           % workaround for newer PGF versions
\documentclass{beamer}
% https://sourceforge.net/tracker/index.php?func=detail&aid=1848912&group_id=92412&atid=600660
\usepackage{lmodern}
\usepackage[T1]{fontenc}
\usepackage[utf8]{inputenc}
\usepackage{textcomp}
\usepackage[ngerman]{babel}
\usepackage[babel,english=american,german=guillemets]{csquotes}	% french
\usepackage{microtype}

% colors for listings
\definecolor{lightergray}{gray}{.95}
\definecolor{darkblue}{rgb}{0,0,0.5}
\definecolor{darkgreen}{rgb}{0,0.5,0}
\definecolor{darkred}{rgb}{0.5,0,0}
\definecolor{darkerblue}{rgb}{0,0,0.4}
\definecolor{darkergreen}{rgb}{0,0.4,0}
\definecolor{darkerred}{rgb}{0.4,0,0}

%\usepackage{listings}
%\lstloadlanguages{HTML,XML}
%\lstset{
%    basicstyle=\ttfamily\small\mdseries,
%    keywordstyle=\bfseries\color{darkblue},
%    identifierstyle=,
%    commentstyle=\color{darkgray},
%    stringstyle=\itshape\color{darkred},
%    frame=none,
%    showstringspaces=false,
%    tabsize=4,
%    backgroundcolor=\color{lightergray},
%}

%%%%%%%%%%%%%%%%%%%%%%%%%%%%%%%%%%%%%%%%%%%%%%%%%%%%%%%%%%%%%%%%%%%%%%%%
%%% preparations for beamer
%%%%%%%%%%%%%%%%%%%%%%%%%%%%%%%%%%%%%%%%%%%%%%%%%%%%%%%%%%%%%%%%%%%%%%%%
\useinnertheme{default}
\useoutertheme{infolines}
%\usecolortheme[rgb={0.28,0.37,0.52}]{structure}
\usecolortheme[rgb={0.18,0.23,0.33}]{structure}
%\usecolortheme{beaver}
\usefonttheme{structurebold}

%%% Ränder vergrößern für's Café Central
\setbeamersize{text margin left=1.2cm}
\setbeamersize{text margin right=1.2cm}

%%%%%%%%%%%%%%%%%%%%%%%%%%%%%%%%%%%%%%%%%%%%%%%%%%%%%%%%%%%%%%%%%%%%%%%%
%%% images
%%%%%%%%%%%%%%%%%%%%%%%%%%%%%%%%%%%%%%%%%%%%%%%%%%%%%%%%%%%%%%%%%%%%%%%%
%\pgfdeclareimage[height=0.75\paperheight]{strasse}{strasse}
%\pgfdeclareimage[height=0.75\paperheight]{arbeitsraum}{arbeitsraum}
%\pgfdeclareimage[height=0.75\paperheight]{lounge}{lounge}
%\pgfdeclareimage[height=0.75\paperheight]{mate}{mate}
%\pgfdeclareimage[height=0.75\paperheight]{regal}{regal}
%\pgfdeclareimage[height=0.75\paperheight]{werkstatt}{werkstatt}

%%%%%%%%%%%%%%%%%%%%%%%%%%%%%%%%%%%%%%%%%%%%%%%%%%%%%%%%%%%%%%%%%%%%%%%%
%%% title, author, date
%%%%%%%%%%%%%%%%%%%%%%%%%%%%%%%%%%%%%%%%%%%%%%%%%%%%%%%%%%%%%%%%%%%%%%%%
\title{Linux Installation Party}
\subtitle{powered by FaRaFIN and Netz39 e.\,V.}
\author[Alexander Dahl]{Alexander Dahl \emph{aka} LeSpocky}
\institute[alex@netz39.de]{\url{http://www.netz39.de/}}
\date{2013-04-17}
\subject{subj}
\keywords{Linux, Installation}

%%%%%%%%%%%%%%%%%%%%%%%%%%%%%%%%%%%%%%%%%%%%%%%%%%%%%%%%%%%%%%%%%%%%%%%%
%%% document
%%%%%%%%%%%%%%%%%%%%%%%%%%%%%%%%%%%%%%%%%%%%%%%%%%%%%%%%%%%%%%%%%%%%%%%%
\begin{document}

\begin{frame}
	\titlepage
\end{frame}

%\begin{frame}{Überblick}
    %\tableofcontents
%\end{frame}

\section{Geschichte}

\subsection{GNU is Not Unix}

\begin{frame}{UNIX}
    \begin{itemize}
        \item Bell Laboratories, später AT\&T (1969)
        \item Unixartige Systeme
            \begin{itemize}
                \item HP-UX (Hewlett-Packard)
                \item AIX (IBM)
                \item IRIX (Silicon Graphics)
                \item Solaris (Sun/Oracle)
                \item BSD, Berkeley Software Distribution
                \item Mac OS X (Apple)
            \end{itemize}
        \item Kernel, Dateisystem, Netzwerkstack
        \item \enquote{alles} ist eine Datei
        \item Shell
    \end{itemize}
\end{frame}

\begin{frame}{GNU und die GPL}
    \begin{itemize}
        \item Unix bis Ende der 70er frei verteilt
        \item Richard Stallman (1983)
        \item Free Software Foundation (1985)
        \item POSIX
        \item gcc, gdb, coreutils, Emacs, …
        \item General Public License (v1 1989, v2 1991, v3 2007)
        \item GNU/Hurd (1990)
        \item Free Software Foundation Europe (2001)
    \end{itemize}
\end{frame}

\subsection{Linux}

\begin{frame}{Linus Torvalds}
    \begin{itemize}
        \item Student in Helsinki
        \item Terminalemulation zum Zugriff auf Unix-Server der Uni (1991)
        \item hardwarenah für 80386
        \item Grundlage Minix und GNU Compiler
        \pause
        \item irgendwann \enquote{bemerkte} Torvalds, dass er ein
            Betriebssystem geschrieben hatte …
    \end{itemize}
\end{frame}

\begin{frame}{Die ersten 15 Jahre}
    \begin{itemize}
        \item v0.99 unter GPL (1992)
        \item Verbreitung auf Disketten und mit Dokumentation in \LaTeX
        \item Nutzung zusammen mit GNU am häufigsten: GNU/Linux
        \item Entwicklung von Distributionen (Slackware und Debian 1993, RedHat und SuSE 1994)
        \item v1.0 mit Netzwerk (1994)
        \item Portierung auf andere Plattformen (1995)
        \item Multiprozessorunterstützung (1996)
        \item große Desktopumgebungen KDE (1998) und Gnome (1999)
        \item Unterstützung großer Firmen wie IBM, Compaq, Oracle
        \item immer mehr Anwendungen
        \item Linux-Foundation (2007)
    \end{itemize}
\end{frame}

\section{Linux}

\subsection{Kernel}

\begin{frame}{Linux ist …}
    \begin{itemize}
        \item ein in C geschriebener Betriebssystemkern
            \begin{itemize}
                \item Treiber
                \item Scheduler
                \item Speicherverwaltung
                \item Netzwerkstack
            \end{itemize}
        \pause
        \item portabel
            \begin{itemize}
                \item verschiedene Architekturen (ARM, Intel, Mips,
                    Motorola, PowerPC, SPARC, …)
                \item unterschiedliche Geräte (Smartphone,
                    Industrierechner, NAS, Router, Server, Desktop,
                    Supercomputer, Kassensysteme, embedded Devices, …)
            \end{itemize}
        \pause
        \item die landläufige Bezeichnung für den Kernel und das \enquote{drumrum}
        \pause
        \item freie Software
    \end{itemize}
\end{frame}

\subsection{Distributionen}

\begin{frame}{Distributionen}
    \begin{itemize}
        \item Zusammenstellung von Software
            \begin{itemize}
                \item Kernel
                \item Systemprogramme
                \item Bibliotheken
                \item Anwendungsprogramme (Browser, Office, 
                    Entwicklungsumgebung, Spiele, Multimedia, Chat, …)
            \end{itemize}
        \pause
        \item Paketmanager (.deb, .rpm, portage, …)
        \pause
        \item Installer
        \item Dokumentation
        \item Community und/oder kommerzieller Support
    \end{itemize}
\end{frame}

\begin{frame}{Distributoren}
    \begin{itemize}
        \item Debianartige: Debian, Ubuntu, Mint
        \pause
        \item RPM-basierte: (Open) SuSE, Red Hat, Fedora, CentOS,
            Mandriva, Mageia
        \pause
        \item andere: Slackware, Gentoo, Sabayon, Arch
        \pause
        \item embedded: OpenWRT, fli4l, ptxdist, Android
        \pause
        \item Live-CD: Grml, Knoppix, Parted Magic, Aptosid
        \pause
        \item spezielle Distributionen für Musik, Gamer, Router, …
        \item siehe auch \url{http://distrowatch.com/}
    \end{itemize}
\end{frame}

\subsection{Oberfläche}

\begin{frame}{Grafische Oberfläche}
    \begin{block}{Desktopumgebung}
        \begin{itemize}
            \item KDE (Qt)
            \item Gnome (GTK)
            \item Unity
            \item Xfce  
            \item LXDE
        \end{itemize}
    \end{block}
    \pause
    \begin{block}{Window-Manager}
        \begin{itemize}
            \item Fluxbox
            \item Enlightenment
            \item awesome
        \end{itemize}
    \end{block}
\end{frame}

\begin{frame}{Shell}
    \begin{itemize}
        \item Kommandozeileninterpreter
        \item unerlässlisch für Systeme ohne grafische Oberfläche 
            (Server, embedded)
        \item Unix-Philosophie
        \item Skripte
        \item Komfortfunktionen (completion, history, job control, …)
    \end{itemize}
\end{frame}

\section{Hilfe}
\begin{frame}{Hilfe}
    \begin{itemize}
        \item \texttt{alex@hal9000:\textasciitilde\$ man woman}
        \item /usr/share/doc
        \item Wikis
        \item Mailinglisten
        \item IRC
        \item Foren
        \item regionale Nutzergruppen und Stammtische
        \item your local Hackerspace …
    \end{itemize}
\end{frame}

\section{Kontakt}

\begin{frame}{Kontakt}
    \begin{center}
        \begin{description}[Twitter/identi.ca]
            \item[WWW] \url{http://www.netz39.de/}
            \item[Twitter/identi.ca] @netz39
            \item[E-Mail] kontakt@netz39.de
            \item[Mailingliste] list@netz39.de
            \item[IRC] \#netz39 auf freenode
        \end{description}
    \end{center}
\end{frame}

\appendix

\section{Lizenz}

\begin{frame}{Lizenz}
    % \begin{block}{Bilder}
        % Die Bilder sind unter folgender Creative Commons-Lizenz
        % veröffentlicht: \emph{Namensnennung-Keine kommerzielle
        % Nutzung-Weitergabe unter gleichen Bedingungen 3.0}, (CC-BY-NC-SA
        % 3.0).
    % \end{block}
    \begin{block}{Folien}
        Die Folien sind freigegeben unter \emph{Creative Commons
        Namensnennung-Weitergabe unter gleichen Bedingungen 3.0
        Deutschland Lizenz}, (CC-BY-SA 3.0). Download unter:
        \url{https://github.com/netz39/Talks}
    \end{block}
\end{frame}

\end{document}
