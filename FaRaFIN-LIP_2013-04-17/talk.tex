%%%%%%%%%%%%%%%%%%%%%%%%%%%%%%%%%%%%%%%%%%%%%%%%%%%%%%%%%%%%%%%%%%%%%%%%
%%% documentclass and packages
%%%%%%%%%%%%%%%%%%%%%%%%%%%%%%%%%%%%%%%%%%%%%%%%%%%%%%%%%%%%%%%%%%%%%%%%
\RequirePackage{atbegshi}           % workaround for newer PGF versions
\documentclass{beamer}
% https://sourceforge.net/tracker/index.php?func=detail&aid=1848912&group_id=92412&atid=600660
\usepackage{lmodern}
\usepackage[T1]{fontenc}
\usepackage[utf8]{inputenc}
\usepackage{textcomp}
\usepackage[ngerman]{babel}
\usepackage[babel,english=american,german=guillemets]{csquotes}	% french
\usepackage{microtype}

% colors for listings
\definecolor{lightergray}{gray}{.95}
\definecolor{darkblue}{rgb}{0,0,0.5}
\definecolor{darkgreen}{rgb}{0,0.5,0}
\definecolor{darkred}{rgb}{0.5,0,0}
\definecolor{darkerblue}{rgb}{0,0,0.4}
\definecolor{darkergreen}{rgb}{0,0.4,0}
\definecolor{darkerred}{rgb}{0.4,0,0}

%\usepackage{listings}
%\lstloadlanguages{HTML,XML}
%\lstset{
%    basicstyle=\ttfamily\small\mdseries,
%    keywordstyle=\bfseries\color{darkblue},
%    identifierstyle=,
%    commentstyle=\color{darkgray},
%    stringstyle=\itshape\color{darkred},
%    frame=none,
%    showstringspaces=false,
%    tabsize=4,
%    backgroundcolor=\color{lightergray},
%}

%%%%%%%%%%%%%%%%%%%%%%%%%%%%%%%%%%%%%%%%%%%%%%%%%%%%%%%%%%%%%%%%%%%%%%%%
%%% preparations for beamer
%%%%%%%%%%%%%%%%%%%%%%%%%%%%%%%%%%%%%%%%%%%%%%%%%%%%%%%%%%%%%%%%%%%%%%%%
\useinnertheme{default}
\useoutertheme{infolines}
%\usecolortheme[rgb={0.28,0.37,0.52}]{structure}
\usecolortheme[rgb={0.18,0.23,0.33}]{structure}
%\usecolortheme{beaver}
\usefonttheme{structurebold}

%%% Ränder vergrößern für's Café Central
\setbeamersize{text margin left=1.2cm}
\setbeamersize{text margin right=1.2cm}

%%%%%%%%%%%%%%%%%%%%%%%%%%%%%%%%%%%%%%%%%%%%%%%%%%%%%%%%%%%%%%%%%%%%%%%%
%%% images
%%%%%%%%%%%%%%%%%%%%%%%%%%%%%%%%%%%%%%%%%%%%%%%%%%%%%%%%%%%%%%%%%%%%%%%%
%\pgfdeclareimage[height=0.75\paperheight]{strasse}{strasse}
%\pgfdeclareimage[height=0.75\paperheight]{arbeitsraum}{arbeitsraum}
%\pgfdeclareimage[height=0.75\paperheight]{lounge}{lounge}
%\pgfdeclareimage[height=0.75\paperheight]{mate}{mate}
%\pgfdeclareimage[height=0.75\paperheight]{regal}{regal}
%\pgfdeclareimage[height=0.75\paperheight]{werkstatt}{werkstatt}

%%%%%%%%%%%%%%%%%%%%%%%%%%%%%%%%%%%%%%%%%%%%%%%%%%%%%%%%%%%%%%%%%%%%%%%%
%%% title, author, date
%%%%%%%%%%%%%%%%%%%%%%%%%%%%%%%%%%%%%%%%%%%%%%%%%%%%%%%%%%%%%%%%%%%%%%%%
\title{Linux Installation Party}
\subtitle{powered by FaRaFIN and Netz39 e.\,V.}
\author{Alexander Dahl}
\institute[alex@netz39.de]{\url{http://www.netz39.de/}}
\date{2013-04-17}
\subject{subj}
\keywords{Linux, Installation}

%%%%%%%%%%%%%%%%%%%%%%%%%%%%%%%%%%%%%%%%%%%%%%%%%%%%%%%%%%%%%%%%%%%%%%%%
%%% document
%%%%%%%%%%%%%%%%%%%%%%%%%%%%%%%%%%%%%%%%%%%%%%%%%%%%%%%%%%%%%%%%%%%%%%%%
\begin{document}

\begin{frame}
	\titlepage
\end{frame}

%\begin{frame}{Überblick}
    %\tableofcontents
%\end{frame}

\section{Geschichte}

\subsection{GNU is Not Unix}

\begin{frame}{UNIX}
    \begin{itemize}
        \item Bell Laboratories, später AT\&T (1969)
        \item Unixartige Systeme
            \begin{itemize}
                \item HP-UX (Hewlett-Packard)
                \item AIX (IBM)
                \item IRIX (Silicon Graphics)
                \item Solaris (Sun/Oracle)
                \item BSD, Berkeley Software Distribution
                \item Mac OS X (Apple)
            \end{itemize}
        \item Kernel, Dateisystem, Netzwerkstack
        \item \enquote{alles} ist eine Datei
        \item Shell
    \end{itemize}
\end{frame}

\begin{frame}{GNU und die GPL}
    \begin{itemize}
        \item Unix bis Ende der 70er frei verteilt
        \item Richard Stallman (1983)
        \item Free Software Foundation (1985)
        \item POSIX
        \item gcc, gdb, coreutils, Emacs, …
        \item General Public License (v1 1989, v2 1991, v3 2007)
        \item GNU/Hurd (1990)
        \item Free Software Foundation Europe (2001)
    \end{itemize}
\end{frame}

\subsection{Linux}

\begin{frame}{Linus Torvalds}
    \begin{itemize}
        \item Student in Helsinki
        \item Terminalemulation zum Zugriff auf Unix-Server der Uni (1991)
        \item hardwarenah für 80386
        \item Grundlage Minix und GNU Compiler
        \item irgendwann \enquote{bemerkte} Torvalds, dass er ein
            Betriebssystem geschrieben hatte …
    \end{itemize}
\end{frame}

\begin{frame}{Die ersten 15 Jahre}
    \begin{itemize}
        \item v0.99 unter GPL (1992)
        \item Verbreitung auf Disketten und mit Dokumentation in \LaTeX
        \item Nutzung zusammen mit GNU am häufigsten: GNU/Linux
        \item Entwicklung von Distributionen (Slackware und Debian 1993, RedHat und SuSE 1994)
        \item v1.0 mit Netzwerk (1994)
        \item Portierung auf andere Plattformen (1995)
        \item Multiprozessorunterstützung (1996)
        \item große Desktopumgebungen KDE (1998) und Gnome (1999)
        \item Unterstützung großer Firmen wie IBM, Compaq, Oracle
        \item immer mehr Anwendungen
        \item Linux-Foundation (2007)
    \end{itemize}
\end{frame}

\section{Linux}

\subsection{Distributionen}

\begin{frame}{deb}
    \begin{itemize}
        \item Debian
        \item Ubuntu
        \item Mint
    \end{itemize}
\end{frame}

\begin{frame}{rpm}
    \begin{itemize}
        \item Open SuSE
        \item Red Hat
        \item Fedora
        \item CentOS
    \end{itemize}
\end{frame}

\begin{frame}{other}
    \begin{itemize}
        \item Slackware
        \item Gentoo
        \item Sabayon
        \item Arch
    \end{itemize}
\end{frame}

\begin{frame}{embedded}
    \begin{itemize}
        \item OpenWRT
        \item fli4l
        \item ptxdist
        \item Android
    \end{itemize}
\end{frame}

\section{Kontakt}

\begin{frame}{Kontakt}
    \begin{center}
        \begin{description}[Twitter/identi.ca]
            \item[WWW] \url{http://www.netz39.de/}
            \item[Twitter/identi.ca] @netz39
            \item[E-Mail] kontakt@netz39.de
            \item[Mailingliste] list@netz39.de
            \item[IRC] \#netz39 auf freenode
        \end{description}
    \end{center}
\end{frame}

\appendix

\section{Lizenz}

\begin{frame}{Lizenz}
    % \begin{block}{Bilder}
        % Die Bilder sind unter folgender Creative Commons-Lizenz
        % veröffentlicht: \emph{Namensnennung-Keine kommerzielle
        % Nutzung-Weitergabe unter gleichen Bedingungen 3.0}, (CC-BY-NC-SA
        % 3.0).
    % \end{block}
    \begin{block}{Folien}
        Die Folien sind freigegeben unter \emph{Creative Commons
        Namensnennung-Weitergabe unter gleichen Bedingungen 3.0
        Deutschland Lizenz}, (CC-BY-SA 3.0). Download unter:
        \url{https://github.com/netz39/Talks}
    \end{block}
\end{frame}

\end{document}
