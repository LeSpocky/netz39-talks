%%%%%%%%%%%%%%%%%%%%%%%%%%%%%%%%%%%%%%%%%%%%%%%%%%%%%%%%%%%%%%%%%%%%%%%%
%%% documentclass and packages
%%%%%%%%%%%%%%%%%%%%%%%%%%%%%%%%%%%%%%%%%%%%%%%%%%%%%%%%%%%%%%%%%%%%%%%%
%\RequirePackage{atbegshi}           % workaround for newer PGF versions
%\documentclass[hyperref={pdfpagelabels=false}]{beamer}
\documentclass{beamer}
% https://sourceforge.net/tracker/index.php?func=detail&aid=1848912&group_id=92412&atid=600660
\usepackage{lmodern}
\usepackage[T1]{fontenc}
\usepackage[utf8]{inputenc}
\usepackage{textcomp}
\usepackage[ngerman]{babel}
\usepackage[babel,english=american,german=guillemets]{csquotes}	% french
%\usepackage{microtype}

% colors for listings
\definecolor{lightergray}{gray}{.95}
\definecolor{darkblue}{rgb}{0,0,0.5}
\definecolor{darkgreen}{rgb}{0,0.5,0}
\definecolor{darkred}{rgb}{0.5,0,0}
\definecolor{darkerblue}{rgb}{0,0,0.4}
\definecolor{darkergreen}{rgb}{0,0.4,0}
\definecolor{darkerred}{rgb}{0.4,0,0}

\usepackage{listings}
\lstloadlanguages{HTML,XML}
\lstset{
    basicstyle=\ttfamily\small\mdseries,
    keywordstyle=\bfseries\color{darkblue},
    identifierstyle=,
    commentstyle=\color{darkgray},
    stringstyle=\itshape\color{darkred},
    frame=none,
    showstringspaces=false,
    tabsize=4,
    backgroundcolor=\color{lightergray},
}

%%%%%%%%%%%%%%%%%%%%%%%%%%%%%%%%%%%%%%%%%%%%%%%%%%%%%%%%%%%%%%%%%%%%%%%%
%%% preparations for beamer
%%%%%%%%%%%%%%%%%%%%%%%%%%%%%%%%%%%%%%%%%%%%%%%%%%%%%%%%%%%%%%%%%%%%%%%%
\useinnertheme{default}
\useoutertheme{infolines}
%\usecolortheme[rgb={0.28,0.37,0.52}]{structure}
\usecolortheme[rgb={0.18,0.23,0.33}]{structure}
%\usecolortheme{beaver}
\usefonttheme{structurebold}

%%%%%%%%%%%%%%%%%%%%%%%%%%%%%%%%%%%%%%%%%%%%%%%%%%%%%%%%%%%%%%%%%%%%%%%%
%%% images
%%%%%%%%%%%%%%%%%%%%%%%%%%%%%%%%%%%%%%%%%%%%%%%%%%%%%%%%%%%%%%%%%%%%%%%%
\pgfdeclareimage[width=0.65\textwidth]{tweetlili}{tweet_lili_2009}
\pgfdeclareimage[width=0.42\textwidth]{baellebad}{baellebad}
\pgfdeclareimage[width=0.48\textwidth]{ddr}{DDR_US_1st}
\pgfdeclareimage[height=0.75\paperheight]{s0matetags}{stratum0-matetags}
\pgfdeclareimage[height=0.75\paperheight]{s0pony}{stratum0-pony.jpg}
\pgfdeclareimage[height=0.75\paperheight]{s0reprap}{stratum0-reprap.jpg}
\pgfdeclareimage[height=0.75\paperheight]{s0invaders}{stratum0-space-invaders}
\pgfdeclareimage[height=0.75\paperheight]{gruendung}{gruppenfoto-gruendung}
\pgfdeclareimage[height=0.75\paperheight]{map}{hackerspaces-map}

%%%%%%%%%%%%%%%%%%%%%%%%%%%%%%%%%%%%%%%%%%%%%%%%%%%%%%%%%%%%%%%%%%%%%%%%
%%% title, author, date
%%%%%%%%%%%%%%%%%%%%%%%%%%%%%%%%%%%%%%%%%%%%%%%%%%%%%%%%%%%%%%%%%%%%%%%%
\title[Netz39]{Netz39}
\subtitle{Real Life Virtuality -- Ein Hackerspace für Magdeburg}
\author{Alexander Dahl}
\institute[netz39.de]{\url{http://www.netz39.de/}}
\date{2012-05-14}
\subject{subj}
\keywords{foo, bar}

%%%%%%%%%%%%%%%%%%%%%%%%%%%%%%%%%%%%%%%%%%%%%%%%%%%%%%%%%%%%%%%%%%%%%%%%
%%% document
%%%%%%%%%%%%%%%%%%%%%%%%%%%%%%%%%%%%%%%%%%%%%%%%%%%%%%%%%%%%%%%%%%%%%%%%
\begin{document}

\begin{frame}
	\titlepage
\end{frame}

%\begin{frame}{Überblick}
    %\tableofcontents
%\end{frame}

\section{Hackerspaces}

\subsection{Definition}

\begin{frame}[label=secinvaders]{All your base are belong to us}
    \begin{figure}
        \pgfuseimage{s0invaders}
    \end{figure}
\end{frame}

\begin{frame}{Ein was?}
    \begin{block}{hackerspaces.org:}
        \begin{quote}
            Hackerspaces are community-operated physical places, where 
            people can meet and work on their projects.
        \end{quote}
    \end{block}
    \pause
    \begin{block}{de.wikipedia.org:}
        \begin{quote}
            Ein Hackerspace (von Hacker und Space, engl. für Raum) oder
            Hackspace ist ein physischer, häufig offener Raum, in dem 
            sich Hacker und Interessierte treffen und austauschen 
            können. Mitglieder mit Interessen an Wissenschaft, 
            Technologie und digitaler Kunst organisieren sich meist in 
            Vereinen.
        \end{quote}
    \end{block}
\end{frame}

\begin{frame}[label=secmap]{Karte}
    \begin{figure}
        \pgfuseimage{map}
    \end{figure}
\end{frame}

\subsection{Geschichte}

\begin{frame}[label=secgeschichte]{Geschichte}
    \begin{block}{World}
        \begin{itemize}
            \item 80er Jahre: Chaos Computer Club
            \item 2007: Chaos Communication Camp und 24C3
                \cite{Ohlig2007}
            \item 2009: CRE 134 \cite{Pritlove2009}
        \end{itemize}
    \end{block}
    \pause
    \begin{block}{Hello}
        %\vspace{0.5em}
        \begin{center}
            \pgfuseimage{tweetlili}
        \end{center}
    \end{block}
\end{frame}

\subsection{Bilder}

\begin{frame}[label=secreprap]{Basteln}
    \begin{figure}
        \pgfuseimage{s0reprap}
    \end{figure}
\end{frame}

\begin{frame}[label=secmatetags]{Mate}
    \begin{figure}
        \pgfuseimage{s0matetags}
    \end{figure}
\end{frame}

\section{Netz39}

\subsection{past, present, future}

\begin{frame}[label=secpony]{Netz39}
    \begin{columns}
        \begin{column}{0.6\textwidth}
            \begin{quote}
                According to David Wells in \textbf{The Penguin Dictionary of Curious and Interesting Numbers}, 39 is the smallest mathematically uninteresting number. The book claims that it is also the first number that is simultaneously both interesting and uninteresting, thereby avoiding the paradox. \cite{Wikipedia2010}
            \end{quote}
        \end{column}
        \begin{column}{0.4\textwidth}
            \begin{figure}
                \pgfuseimage{s0pony}
            \end{figure}
        \end{column}
    \end{columns}
\end{frame}

\begin{frame}[label=secddr]{Gestern, heute, morgen \dots}
    \begin{columns}
        \begin{column}{0.5\textwidth}<1->
            \begin{block}{past}
                \begin{itemize}
                    \item FSA 2009
                    \item Twitter, Wiki, Mailingliste
                    \item Stammtische
                \end{itemize}
            \end{block}
            \begin{block}{present}<2->
                \begin{itemize}
                    \item Verein gegründet
                    \item Räume
                    \item Logo, Ideen, träumen \dots
                \end{itemize}
            \end{block}
            \begin{block}{future}<3->
                \begin{itemize}
                    \item Gemeinnützigkeit
                    \item Projekte
                \end{itemize}
            \end{block}
        \end{column}
        \begin{column}{0.5\textwidth}<1->
            \begin{figure}
                \pgfuseimage{ddr}
            \end{figure}
        \end{column}
    \end{columns}
\end{frame}

\subsection{Verein}

\begin{frame}[label=secgruendung]{Gründung}
    \begin{figure}
        \pgfuseimage{gruendung}
    \end{figure}
\end{frame}

\begin{frame}[label=secbaellebad]{Verein}
    \begin{columns}[b]
        \begin{column}{0.55\textwidth}
            \begin{block}{How to participate}<1->
                \begin{itemize}
                    \item neue Mitglieder willkommen
                    \item Mitgliedsantrag online
                    \item Mitgliedsbeitrag pro Monat: 30 € oder 8 €
                    \pause
                    \item Fördermitgliedschaft
                    \item Spenden
                \end{itemize}
            \end{block}
            \begin{block}{What you get}<3->
                \begin{itemize}
                    \item Raum
                    \item Leute
                    \item Know-How
                \end{itemize}
            \end{block}
        \end{column}
        \begin{column}{0.45\textwidth}<4->
            \begin{figure}
                \pgfuseimage{baellebad}
            \end{figure}
        \end{column}
    \end{columns}            
\end{frame}

\section{Kontakt}

\begin{frame}{Kontakt}
    \begin{center}
        \begin{description}[Twitter/identi.ca]
            \item[WWW] \url{http://www.netz39.de/}
            \item[Twitter/identi.ca] @netz39
            \item[E-Mail] kontakt@netz39.de
            \item[Mailingliste] list@netz39.de
            \item[IRC] \#netz39 auf freenode
        \end{description}
    \end{center}
\end{frame}

\appendix

\section{Referenzen}

\subsection{Literatur}

\begin{frame}{Weiterführende Links}
    \begin{thebibliography}{XX}
        %\beamertemplatebookbibitems
        %\beamertemplatearticlebibitems
        \beamertemplatetextbibitems
        \bibitem[Ohlig 2007]{Ohlig2007}
            Jens Ohlig, Lars Weiler.
            \newblock {\em Building a Hacker Space}.
            \footnotesize
            \newblock \url{http://events.ccc.de/congress/2007/Fahrplan/events/2133.en.html}
            \normalsize
        \bibitem[Pritlove 2009]{Pritlove2009}
            Tim Pritlove, Astera, Johannes Grenzfurthner.
            \newblock {\em CRE134 Hackerspaces}.
            \newblock \url{http://cre.fm/cre134}
        \bibitem[Wikipedia 2010]{Wikipedia2010}
            Wikipedia
            \newblock {\em 39 (number)}
            \scriptsize
            \newblock \url{https://en.wikipedia.org/w/index.php?title=39\_\%28number\%29\&oldid=344011024}
            \normalsize
    \end{thebibliography}
\end{frame}

\subsection{Abbildungen}

\begin{frame}{Bildnachweis}
    \footnotesize
    \begin{description}[Folie XX]
        \item[Folie \ref{secinvaders}] Roland Hieber (@daniel\_bohrer), CC-BY-SA 3.0, {\tiny \url{https://stratum0.org/wiki/Datei:Space\_Invaders\_fertig\_gro\%C3\%9F.jpg}}
        \item[Folie \ref{secmap}] hackerspaces.org, {\tiny \url{http://hackerspaces.org/wiki/Europe}}
        \item[Folie \ref{secgeschichte}] @lili\_quasselt auf Twitter,
            {\tiny \url{https://twitter.com/\#!/lili\_quasselt/status/3379026688}}
        \item[Folie \ref{secreprap}] Lars Andresen (@larsan), CC-BY-SA 3.0, {\tiny \url{https://stratum0.org/wiki/Datei:20120506-matetags-06.jpg}}
        \item[Folie \ref{secmatetags}] Lars Andresen (@larsan), CC-BY-SA 3.0, {\tiny \url{https://stratum0.org/wiki/Datei:20120506-matetags-01.jpg}}
        \item[Folie \ref{secpony}] Lars Andresen (@larsan), CC-BY-SA 3.0, {\tiny \url{https://stratum0.org/wiki/Datei:20120326-pony.jpg}}
        \item[Folie \ref{secddr}] gemeinfrei, {\tiny \url{https://de.wikipedia.org/w/index.php?title=Datei:DDR\_US\_1st.jpg}}
        \item[Folie \ref{secgruendung}] Andreas Pfohl (@andreaspfohl)
        \item[Folie \ref{secbaellebad}] Daniel Neumann (@The\_DanielSan), CC-BY-SA 3.0, {\tiny \url{http://bpt2012-1.deplaced.net/}}
    \end{description}
    \normalsize
\end{frame}

\subsection{Lizenz}

\begin{frame}{Lizenz}
    Die Folien sind freigegeben unter \emph{Creative Commons
    Namensnennung-Weitergabe unter gleichen Bedingungen 3.0 Deutschland
    Lizenz.} (CC-BY-SA 3.0).
\end{frame}

\end{document}
