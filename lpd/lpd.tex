%%%%%%%%%%%%%%%%%%%%%%%%%%%%%%%%%%%%%%%%%%%%%%%%%%%%%%%%%%%%%%%%%%%%%%%%
%%% documentclass and packages
%%%%%%%%%%%%%%%%%%%%%%%%%%%%%%%%%%%%%%%%%%%%%%%%%%%%%%%%%%%%%%%%%%%%%%%%
%\RequirePackage{atbegshi}           % workaround for newer PGF versions
\documentclass{beamer}
% https://sourceforge.net/tracker/index.php?func=detail&aid=1848912&group_id=92412&atid=600660
\usepackage{lmodern}
\usepackage[T1]{fontenc}
\usepackage[utf8]{inputenc}
\usepackage{textcomp}
\usepackage[ngerman]{babel}

\usepackage[babel,english=american,german=guillemets]{csquotes}	% french
\usepackage{microtype}
\usepackage{tikz}

% colors for listings
\definecolor{lightergray}{gray}{.95}
\definecolor{darkblue}{rgb}{0,0,0.5}
\definecolor{darkgreen}{rgb}{0,0.5,0}
\definecolor{darkred}{rgb}{0.5,0,0}
\definecolor{darkerblue}{rgb}{0,0,0.4}
\definecolor{darkergreen}{rgb}{0,0.4,0}
\definecolor{darkerred}{rgb}{0.4,0,0}

%\usepackage{listings}
%\lstloadlanguages{HTML,XML}
%\lstset{
%    basicstyle=\ttfamily\small\mdseries,
%    keywordstyle=\bfseries\color{darkblue},
%    identifierstyle=,
%    commentstyle=\color{darkgray},
%    stringstyle=\itshape\color{darkred},
%    frame=none,
%    showstringspaces=false,
%    tabsize=4,
%    backgroundcolor=\color{lightergray},
%}

%%%%%%%%%%%%%%%%%%%%%%%%%%%%%%%%%%%%%%%%%%%%%%%%%%%%%%%%%%%%%%%%%%%%%%%%
%%% macros
%%%%%%%%%%%%%%%%%%%%%%%%%%%%%%%%%%%%%%%%%%%%%%%%%%%%%%%%%%%%%%%%%%%%%%%%
\makeatletter
\newcommand{\strong}[1]{\@strong{#1}}
\newcommand{\@@strong}[1]{\textbf{\let\@strong\@@@strong#1}}
\newcommand{\@@@strong}[1]{\textnormal{\let\@strong\@@strong#1}}
\let\@strong\@@strong
\makeatother

%%%%%%%%%%%%%%%%%%%%%%%%%%%%%%%%%%%%%%%%%%%%%%%%%%%%%%%%%%%%%%%%%%%%%%%%
%%% preparations for beamer
%%%%%%%%%%%%%%%%%%%%%%%%%%%%%%%%%%%%%%%%%%%%%%%%%%%%%%%%%%%%%%%%%%%%%%%%
\useinnertheme{default}
\useoutertheme{infolines}
%\usecolortheme[rgb={0.28,0.37,0.52}]{structure}
\usecolortheme[rgb={0.18,0.23,0.33}]{structure}
%\usecolortheme{beaver}
\usefonttheme{structurebold}

%%% Ränder vergrößern, sieht dann nicht so an den Rand gequetscht aus
\setbeamersize{text margin left=0.5cm}
\setbeamersize{text margin right=0.5cm}

%%% let hyperlinks look like hyperlinks
\hypersetup{%
    colorlinks=true,
    linkcolor=black,
    urlcolor=darkblue
}

%%%%%%%%%%%%%%%%%%%%%%%%%%%%%%%%%%%%%%%%%%%%%%%%%%%%%%%%%%%%%%%%%%%%%%%%
%%% images
%%%%%%%%%%%%%%%%%%%%%%%%%%%%%%%%%%%%%%%%%%%%%%%%%%%%%%%%%%%%%%%%%%%%%%%%
\pgfdeclareimage[width=\textwidth]{officecalc1}{screenshot_libreoffice_calc_1.png}
\pgfdeclareimage[width=\textwidth]{firefox1}{screenshot_firefox_1.png}
\pgfdeclareimage[width=\textwidth]{firefox2}{screenshot_firefox_2.png}
\pgfdeclareimage[width=\textwidth]{inkscape1}{screenshot_inkscape_1.png}
\pgfdeclareimage[height=0.8\paperheight]{vlc1}{screenshot_vlc_1.png}
\pgfdeclareimage[height=0.7\paperheight]{gimp1}{screenshot_gimp_1.png}
\pgfdeclareimage[width=\textwidth]{okular1}{screenshot_okular_1.png}
\pgfdeclareimage[width=\textwidth]{thunderbird1}{screenshot_thunderbird_1.png}

\titlegraphic{%
    \begin{tikzpicture}[
    y=0.8pt, 
    x=0.8pt, 
    yscale=-1, 
    inner sep=0pt, 
    outer sep=0pt, 
    scale=0.2
]
\begin{scope}[shift={(-448.05262,-67.70133)}]
  \path[color=black,fill=black,even odd rule,line width=16.000pt]
    (534.1205,161.8816) -- (534.1205,172.2877) -- (581.3732,172.2877) .. controls
    (571.8452,181.8308) and (562.3036,191.3603) .. (552.7463,200.8741) --
    (565.3389,200.8741) .. controls (585.4511,200.8741) and (601.7400,217.2036) ..
    (601.7400,237.3158) .. controls (601.7400,257.4280) and (585.4511,273.7170) ..
    (565.3389,273.7170) .. controls (545.2267,273.7170) and (528.8972,257.4280) ..
    (528.8972,237.3158) -- (518.4911,237.3158) .. controls (518.4911,263.1743) and
    (539.4804,284.1231) .. (565.3389,284.1231) .. controls (591.1974,284.1231) and
    (612.1462,263.1743) .. (612.1462,237.3158) .. controls (612.1462,215.3233) and
    (596.9939,196.8355) .. (576.5548,191.8042) -- (606.5180,161.8816) -- cycle;
  \begin{scope}[cm={{-1.0,0.0,0.0,-1.0,(1130.6373,475.18252)}}]
    \path[color=black,fill=black,even odd rule,line width=16.000pt]
      (565.3389,208.6889) .. controls (549.5365,208.6889) and (536.7120,221.5134) ..
      (536.7120,237.3158) .. controls (536.7120,249.4467) and (544.2722,259.8031) ..
      (554.9328,263.9587) -- (554.9328,252.2164) .. controls (550.2090,248.9346) and
      (547.1181,243.5056) .. (547.1181,237.3158) .. controls (547.1181,227.2597) and
      (555.2828,219.0950) .. (565.3389,219.0950) .. controls (575.3950,219.0950) and
      (583.5192,227.2597) .. (583.5192,237.3158) .. controls (583.5192,243.4955) and
      (580.4441,248.9328) .. (575.7450,252.2164) -- (575.7450,263.9587) .. controls
      (586.3945,259.8031) and (593.9253,249.4467) .. (593.9253,237.3158) .. controls
      (593.9253,221.5134) and (581.1413,208.6889) .. (565.3389,208.6889) -- cycle;
    \path[color=black,fill=black,even odd rule,line width=16.000pt]
      (565.3389,232.0925) .. controls (562.4657,232.0925) and (560.1156,234.4426) ..
      (560.1156,237.3158) -- (560.1156,268.4937) -- (570.5217,268.4937) --
      (570.5217,237.3158) .. controls (570.5217,234.4426) and (568.2121,232.0925) ..
      (565.3389,232.0925) -- cycle;
  \end{scope}
  \path[fill=black] (630.8529,149.4914) .. controls (625.1306,149.5458) and
    (620.4890,150.3447) .. (619.3535,151.0301) .. controls (617.0825,152.4009) and
    (616.5461,155.6877) .. (617.2480,156.4558) .. controls (617.9500,157.2237) and
    (628.6785,164.1143) .. (633.3633,166.3761) .. controls (645.9778,172.4665) and
    (632.0756,195.7720) .. (621.2161,191.6018) .. controls (613.5199,188.6464) and
    (604.7892,179.8446) .. (603.3192,179.7785) .. controls (601.8492,179.7137) and
    (599.9557,180.7048) .. (599.4726,183.2607) .. controls (598.9895,185.8166) and
    (602.4810,198.1576) .. (611.2554,207.7576) .. controls (621.7846,219.2778) and
    (638.1817,206.5491) .. (638.1817,217.0704) --
    (638.1817,265.9427)arc(179.961:0.039:13.018) -- (664.2172,203.5060) --
    (664.2172,188.9294) .. controls (664.2172,183.4623) and (670.9201,161.5406) ..
    (648.7498,152.3663) .. controls (643.3754,150.1424) and (636.5752,149.4370) ..
    (630.8529,149.4914) -- cycle;
\end{scope}
\path[color=black,fill=black,even odd rule,line width=8.333pt]
  (150.0000,23.1213) .. controls (79.9291,23.1213) and (23.1213,79.9291) ..
  (23.1213,150.0000) .. controls (23.1213,220.0708) and (79.9291,276.8787) ..
  (150.0000,276.8787) .. controls (220.0708,276.8787) and (276.8787,220.0708) ..
  (276.8787,150.0000) .. controls (276.8787,79.9291) and (220.0708,23.1213) ..
  (150.0000,23.1213) -- cycle(150.0000,40.5068) .. controls (210.4603,40.5068)
  and (259.4540,89.5397) .. (259.4540,150.0000) .. controls (259.4540,210.4603)
  and (210.4603,259.4540) .. (150.0000,259.4540) .. controls (89.5278,259.8428)
  and (41.3547,203.5210) .. (40.5068,150.0000) .. controls (40.5068,89.5397) and
  (89.5397,40.5068) .. (150.0001,40.5068) -- cycle;
\end{tikzpicture}

% vim: set ft=tex :

}

%%%%%%%%%%%%%%%%%%%%%%%%%%%%%%%%%%%%%%%%%%%%%%%%%%%%%%%%%%%%%%%%%%%%%%%%
%%% title, author, date
%%%%%%%%%%%%%%%%%%%%%%%%%%%%%%%%%%%%%%%%%%%%%%%%%%%%%%%%%%%%%%%%%%%%%%%%
\title{Linux Presentation Day 2018.2}
\subtitle{Linux und Freie Software für Anwender}
\author{Netz39 e.\,V.}
\institute{\url{http://www.netz39.de/}}
\date{2018-11-10}
\subject{subj}
\keywords{Linux, Installation}

%%%%%%%%%%%%%%%%%%%%%%%%%%%%%%%%%%%%%%%%%%%%%%%%%%%%%%%%%%%%%%%%%%%%%%%%
%%% document
%%%%%%%%%%%%%%%%%%%%%%%%%%%%%%%%%%%%%%%%%%%%%%%%%%%%%%%%%%%%%%%%%%%%%%%%
\begin{document}

\begin{frame}
	\titlepage
\end{frame}

\logo{\begin{tikzpicture}[
    y=0.8pt, 
    x=0.8pt, 
    yscale=-1, 
    inner sep=0pt, 
    outer sep=0pt, 
    scale=0.1,
    opacity=0.5
]
\begin{scope}[shift={(-448.05262,-67.70133)}]
  \path[color=black,fill=black,even odd rule,line width=16.000pt]
    (534.1205,161.8816) -- (534.1205,172.2877) -- (581.3732,172.2877) .. controls
    (571.8452,181.8308) and (562.3036,191.3603) .. (552.7463,200.8741) --
    (565.3389,200.8741) .. controls (585.4511,200.8741) and (601.7400,217.2036) ..
    (601.7400,237.3158) .. controls (601.7400,257.4280) and (585.4511,273.7170) ..
    (565.3389,273.7170) .. controls (545.2267,273.7170) and (528.8972,257.4280) ..
    (528.8972,237.3158) -- (518.4911,237.3158) .. controls (518.4911,263.1743) and
    (539.4804,284.1231) .. (565.3389,284.1231) .. controls (591.1974,284.1231) and
    (612.1462,263.1743) .. (612.1462,237.3158) .. controls (612.1462,215.3233) and
    (596.9939,196.8355) .. (576.5548,191.8042) -- (606.5180,161.8816) -- cycle;
  \begin{scope}[cm={{-1.0,0.0,0.0,-1.0,(1130.6373,475.18252)}}]
    \path[color=black,fill=black,even odd rule,line width=16.000pt]
      (565.3389,208.6889) .. controls (549.5365,208.6889) and (536.7120,221.5134) ..
      (536.7120,237.3158) .. controls (536.7120,249.4467) and (544.2722,259.8031) ..
      (554.9328,263.9587) -- (554.9328,252.2164) .. controls (550.2090,248.9346) and
      (547.1181,243.5056) .. (547.1181,237.3158) .. controls (547.1181,227.2597) and
      (555.2828,219.0950) .. (565.3389,219.0950) .. controls (575.3950,219.0950) and
      (583.5192,227.2597) .. (583.5192,237.3158) .. controls (583.5192,243.4955) and
      (580.4441,248.9328) .. (575.7450,252.2164) -- (575.7450,263.9587) .. controls
      (586.3945,259.8031) and (593.9253,249.4467) .. (593.9253,237.3158) .. controls
      (593.9253,221.5134) and (581.1413,208.6889) .. (565.3389,208.6889) -- cycle;
    \path[color=black,fill=black,even odd rule,line width=16.000pt]
      (565.3389,232.0925) .. controls (562.4657,232.0925) and (560.1156,234.4426) ..
      (560.1156,237.3158) -- (560.1156,268.4937) -- (570.5217,268.4937) --
      (570.5217,237.3158) .. controls (570.5217,234.4426) and (568.2121,232.0925) ..
      (565.3389,232.0925) -- cycle;
  \end{scope}
  \path[fill=black] (630.8529,149.4914) .. controls (625.1306,149.5458) and
    (620.4890,150.3447) .. (619.3535,151.0301) .. controls (617.0825,152.4009) and
    (616.5461,155.6877) .. (617.2480,156.4558) .. controls (617.9500,157.2237) and
    (628.6785,164.1143) .. (633.3633,166.3761) .. controls (645.9778,172.4665) and
    (632.0756,195.7720) .. (621.2161,191.6018) .. controls (613.5199,188.6464) and
    (604.7892,179.8446) .. (603.3192,179.7785) .. controls (601.8492,179.7137) and
    (599.9557,180.7048) .. (599.4726,183.2607) .. controls (598.9895,185.8166) and
    (602.4810,198.1576) .. (611.2554,207.7576) .. controls (621.7846,219.2778) and
    (638.1817,206.5491) .. (638.1817,217.0704) --
    (638.1817,265.9427)arc(179.961:0.039:13.018) -- (664.2172,203.5060) --
    (664.2172,188.9294) .. controls (664.2172,183.4623) and (670.9201,161.5406) ..
    (648.7498,152.3663) .. controls (643.3754,150.1424) and (636.5752,149.4370) ..
    (630.8529,149.4914) -- cycle;
\end{scope}
\path[color=black,fill=black,even odd rule,line width=8.333pt]
  (150.0000,23.1213) .. controls (79.9291,23.1213) and (23.1213,79.9291) ..
  (23.1213,150.0000) .. controls (23.1213,220.0708) and (79.9291,276.8787) ..
  (150.0000,276.8787) .. controls (220.0708,276.8787) and (276.8787,220.0708) ..
  (276.8787,150.0000) .. controls (276.8787,79.9291) and (220.0708,23.1213) ..
  (150.0000,23.1213) -- cycle(150.0000,40.5068) .. controls (210.4603,40.5068)
  and (259.4540,89.5397) .. (259.4540,150.0000) .. controls (259.4540,210.4603)
  and (210.4603,259.4540) .. (150.0000,259.4540) .. controls (89.5278,259.8428)
  and (41.3547,203.5210) .. (40.5068,150.0000) .. controls (40.5068,89.5397) and
  (89.5397,40.5068) .. (150.0001,40.5068) -- cycle;
\end{tikzpicture}

% vim: set ft=tex :
}

\begin{frame}{Überblick}
    \begin{columns}[T]
        \begin{column}{0.5\textwidth}
            \tableofcontents
        \end{column}
        \begin{column}{0.5\textwidth}
            \definecolor{c666666}{RGB}{102,102,102}
\definecolor{c6d6d6d}{RGB}{109,109,109}
\definecolor{c757575}{RGB}{117,117,117}
\definecolor{c7c7c7c}{RGB}{124,124,124}
\definecolor{c848484}{RGB}{132,132,132}
\definecolor{c8c8c8c}{RGB}{140,140,140}
\definecolor{c939393}{RGB}{147,147,147}
\definecolor{c9b9b9b}{RGB}{155,155,155}
\definecolor{ca3a3a3}{RGB}{163,163,163}
\definecolor{caaaaaa}{RGB}{170,170,170}
\definecolor{cb2b2b2}{RGB}{178,178,178}
\definecolor{cbababa}{RGB}{186,186,186}
\definecolor{cc1c1c1}{RGB}{193,193,193}
\definecolor{cc9c9c9}{RGB}{201,201,201}
\definecolor{cd1d1d1}{RGB}{209,209,209}
\definecolor{cd8d8d8}{RGB}{216,216,216}
\definecolor{ce0e0e0}{RGB}{224,224,224}
\definecolor{ce8e8e8}{RGB}{232,232,232}
\definecolor{cefefef}{RGB}{239,239,239}
\definecolor{cf7f7f7}{RGB}{247,247,247}
\definecolor{cffffff}{RGB}{255,255,255}
\definecolor{c070707}{RGB}{7,7,7}
\definecolor{c0f0f0f}{RGB}{15,15,15}
\definecolor{c161616}{RGB}{22,22,22}
\definecolor{c1e1e1e}{RGB}{30,30,30}
\definecolor{c262626}{RGB}{38,38,38}
\definecolor{c2d2d2d}{RGB}{45,45,45}
\definecolor{c353535}{RGB}{53,53,53}
\definecolor{c3d3d3d}{RGB}{61,61,61}
\definecolor{c444444}{RGB}{68,68,68}
\definecolor{c4c4c4c}{RGB}{76,76,76}
\definecolor{c545454}{RGB}{84,84,84}
\definecolor{c5b5b5b}{RGB}{91,91,91}
\definecolor{c636363}{RGB}{99,99,99}
\definecolor{c6b6b6b}{RGB}{107,107,107}
\definecolor{c727272}{RGB}{114,114,114}
\definecolor{c7a7a7a}{RGB}{122,122,122}
\definecolor{c828282}{RGB}{130,130,130}
\definecolor{c898989}{RGB}{137,137,137}
\definecolor{c919191}{RGB}{145,145,145}
\definecolor{c999999}{RGB}{153,153,153}
\definecolor{c995900}{RGB}{153,89,0}
\definecolor{c9e5e00}{RGB}{158,94,0}
\definecolor{ca36400}{RGB}{163,100,0}
\definecolor{ca86a00}{RGB}{168,106,0}
\definecolor{cad7000}{RGB}{173,112,0}
\definecolor{cb27500}{RGB}{178,117,0}
\definecolor{cb77b00}{RGB}{183,123,0}
\definecolor{cbc8100}{RGB}{188,129,0}
\definecolor{cc18700}{RGB}{193,135,0}
\definecolor{cc68c00}{RGB}{198,140,0}
\definecolor{ccc9200}{RGB}{204,146,0}
\definecolor{cd19800}{RGB}{209,152,0}
\definecolor{cd69e00}{RGB}{214,158,0}
\definecolor{cdba300}{RGB}{219,163,0}
\definecolor{ce0a900}{RGB}{224,169,0}
\definecolor{ce5af00}{RGB}{229,175,0}
\definecolor{ceab500}{RGB}{234,181,0}
\definecolor{cefba00}{RGB}{239,186,0}
\definecolor{cf4c000}{RGB}{244,192,0}
\definecolor{cf9c600}{RGB}{249,198,0}
\definecolor{cffcc00}{RGB}{255,204,0}
\definecolor{cffcc02}{RGB}{255,204,2}
\definecolor{cffcc05}{RGB}{255,204,5}
\definecolor{cffcc07}{RGB}{255,204,7}
\definecolor{cffcd0a}{RGB}{255,205,10}
\definecolor{cffcd0c}{RGB}{255,205,12}
\definecolor{cffcd0f}{RGB}{255,205,15}
\definecolor{cffcd11}{RGB}{255,205,17}
\definecolor{cffce14}{RGB}{255,206,20}
\definecolor{cffce16}{RGB}{255,206,22}
\definecolor{cffce19}{RGB}{255,206,25}
\definecolor{cffce1c}{RGB}{255,206,28}
\definecolor{cffcf1e}{RGB}{255,207,30}
\definecolor{cffcf21}{RGB}{255,207,33}
\definecolor{cffcf23}{RGB}{255,207,35}
\definecolor{cffcf26}{RGB}{255,207,38}
\definecolor{cffd028}{RGB}{255,208,40}
\definecolor{cffd02b}{RGB}{255,208,43}
\definecolor{cffd02d}{RGB}{255,208,45}
\definecolor{cffd030}{RGB}{255,208,48}
\definecolor{cffd133}{RGB}{255,209,51}
\definecolor{cb27f19}{RGB}{178,127,25}
\definecolor{caf7c19}{RGB}{175,124,25}
\definecolor{caa7716}{RGB}{170,119,22}
\definecolor{ca87516}{RGB}{168,117,22}
\definecolor{ca37014}{RGB}{163,112,20}
\definecolor{ca06d14}{RGB}{160,109,20}
\definecolor{c9b6811}{RGB}{155,104,17}
\definecolor{c996611}{RGB}{153,102,17}
\definecolor{c966311}{RGB}{150,99,17}
\definecolor{c915e0f}{RGB}{145,94,15}
\definecolor{c8e5b0f}{RGB}{142,91,15}
\definecolor{c89560a}{RGB}{137,86,10}
\definecolor{c87540a}{RGB}{135,84,10}
\definecolor{c824f07}{RGB}{130,79,7}
\definecolor{c7f4c07}{RGB}{127,76,7}
\definecolor{c7c4907}{RGB}{124,73,7}
\definecolor{c774405}{RGB}{119,68,5}
\definecolor{c754205}{RGB}{117,66,5}
\definecolor{c703d02}{RGB}{112,61,2}
\definecolor{c6d3a02}{RGB}{109,58,2}
\definecolor{c683500}{RGB}{104,53,0}
\definecolor{c663300}{RGB}{102,51,0}
\definecolor{ccc9933}{RGB}{204,153,51}
\definecolor{cc69330}{RGB}{198,147,48}
\definecolor{cc18e2d}{RGB}{193,142,45}
\definecolor{cbc892b}{RGB}{188,137,43}
\definecolor{cb78428}{RGB}{183,132,40}
\definecolor{cb27f26}{RGB}{178,127,38}
\definecolor{caf7c26}{RGB}{175,124,38}
\definecolor{caa7723}{RGB}{170,119,35}
\definecolor{ca57221}{RGB}{165,114,33}
\definecolor{ca06d1e}{RGB}{160,109,30}
\definecolor{c9b681c}{RGB}{155,104,28}
\definecolor{c966316}{RGB}{150,99,22}
\definecolor{c915e14}{RGB}{145,94,20}
\definecolor{c8c5911}{RGB}{140,89,17}
\definecolor{c87540f}{RGB}{135,84,15}
\definecolor{c824f0c}{RGB}{130,79,12}
\definecolor{c7f4c0c}{RGB}{127,76,12}
\definecolor{c7a470a}{RGB}{122,71,10}
\definecolor{c754207}{RGB}{117,66,7}
\definecolor{c703d05}{RGB}{112,61,5}
\definecolor{c6b3802}{RGB}{107,56,2}
\definecolor{cf7c400}{RGB}{247,196,0}
\definecolor{cefbc00}{RGB}{239,188,0}
\definecolor{ce8b500}{RGB}{232,181,0}
\definecolor{ce2af00}{RGB}{226,175,0}
\definecolor{cdba800}{RGB}{219,168,0}
\definecolor{cd3a000}{RGB}{211,160,0}
\definecolor{ccc9900}{RGB}{204,153,0}
\definecolor{cc49100}{RGB}{196,145,0}
\definecolor{cbc8900}{RGB}{188,137,0}
\definecolor{cb58200}{RGB}{181,130,0}
\definecolor{caf7c00}{RGB}{175,124,0}
\definecolor{ca87500}{RGB}{168,117,0}
\definecolor{ca06d00}{RGB}{160,109,0}
\definecolor{c996600}{RGB}{153,102,0}
\definecolor{c915e00}{RGB}{145,94,0}
\definecolor{c895600}{RGB}{137,86,0}
\definecolor{c824f00}{RGB}{130,79,0}
\definecolor{c7c4900}{RGB}{124,73,0}
\definecolor{c754200}{RGB}{117,66,0}
\definecolor{c6d3a00}{RGB}{109,58,0}
\definecolor{cf9f9f9}{RGB}{249,249,249}
\definecolor{cf4f4f4}{RGB}{244,244,244}
\definecolor{ceaeaea}{RGB}{234,234,234}
\definecolor{ce5e5e5}{RGB}{229,229,229}
\definecolor{cdbdbdb}{RGB}{219,219,219}
\definecolor{cd6d6d6}{RGB}{214,214,214}
\definecolor{ccccccc}{RGB}{204,204,204}
\definecolor{cc6c6c6}{RGB}{198,198,198}
\definecolor{cbcbcbc}{RGB}{188,188,188}
\definecolor{cb7b7b7}{RGB}{183,183,183}
\definecolor{cadadad}{RGB}{173,173,173}
\definecolor{ca8a8a8}{RGB}{168,168,168}
\definecolor{c9e9e9e}{RGB}{158,158,158}
\definecolor{cfbfbfb}{RGB}{251,251,251}
\definecolor{cf8f8f8}{RGB}{248,248,248}
\definecolor{cf5f5f5}{RGB}{245,245,245}
\definecolor{cf2f2f2}{RGB}{242,242,242}
\definecolor{cebebeb}{RGB}{235,235,235}
\definecolor{ce2e2e2}{RGB}{226,226,226}
\definecolor{cdfdfdf}{RGB}{223,223,223}
\definecolor{cd5d5d5}{RGB}{213,213,213}
\definecolor{cd2d2d2}{RGB}{210,210,210}
\definecolor{ccfcfcf}{RGB}{207,207,207}
\definecolor{cc8c8c8}{RGB}{200,200,200}
\definecolor{cc5c5c5}{RGB}{197,197,197}
\definecolor{cc2c2c2}{RGB}{194,194,194}
\definecolor{cbfbfbf}{RGB}{191,191,191}
\definecolor{c050505}{RGB}{5,5,5}
\definecolor{c0a0a0a}{RGB}{10,10,10}
\definecolor{c141414}{RGB}{20,20,20}
\definecolor{c191919}{RGB}{25,25,25}
\definecolor{c232323}{RGB}{35,35,35}
\definecolor{c282828}{RGB}{40,40,40}
\definecolor{c333333}{RGB}{51,51,51}
\definecolor{c383838}{RGB}{56,56,56}
\definecolor{c424242}{RGB}{66,66,66}
\definecolor{c474747}{RGB}{71,71,71}
\definecolor{c515151}{RGB}{81,81,81}
\definecolor{c565656}{RGB}{86,86,86}
\definecolor{c606060}{RGB}{96,96,96}
\definecolor{c060606}{RGB}{6,6,6}
\definecolor{c0c0c0c}{RGB}{12,12,12}
\definecolor{c131313}{RGB}{19,19,19}
\definecolor{c1f1f1f}{RGB}{31,31,31}
\definecolor{c2c2c2c}{RGB}{44,44,44}
\definecolor{c393939}{RGB}{57,57,57}
\definecolor{c3f3f3f}{RGB}{63,63,63}
\definecolor{c464646}{RGB}{70,70,70}
\definecolor{c525252}{RGB}{82,82,82}
\definecolor{c595959}{RGB}{89,89,89}
\definecolor{c5f5f5f}{RGB}{95,95,95}
\definecolor{c6c6c6c}{RGB}{108,108,108}
\definecolor{c797979}{RGB}{121,121,121}
\definecolor{c7f7f7f}{RGB}{127,127,127}
\definecolor{c030303}{RGB}{3,3,3}
\definecolor{c0b0b0b}{RGB}{11,11,11}
\definecolor{c1a1a1a}{RGB}{26,26,26}
\definecolor{c222222}{RGB}{34,34,34}
\definecolor{c2a2a2a}{RGB}{42,42,42}
\definecolor{c313131}{RGB}{49,49,49}
\definecolor{c414141}{RGB}{65,65,65}
\definecolor{c484848}{RGB}{72,72,72}
\definecolor{c020202}{RGB}{2,2,2}
\definecolor{c111111}{RGB}{17,17,17}
\definecolor{c1c1c1c}{RGB}{28,28,28}
\definecolor{c212121}{RGB}{33,33,33}
\definecolor{c2b2b2b}{RGB}{43,43,43}
\definecolor{c303030}{RGB}{48,48,48}
\definecolor{c010101}{RGB}{1,1,1}
\definecolor{c090909}{RGB}{9,9,9}
\definecolor{c0d0d0d}{RGB}{13,13,13}
\definecolor{c151515}{RGB}{21,21,21}
\definecolor{c181818}{RGB}{24,24,24}
\definecolor{c202020}{RGB}{32,32,32}
\definecolor{c242424}{RGB}{36,36,36}
\definecolor{c505050}{RGB}{80,80,80}
\definecolor{c575757}{RGB}{87,87,87}
\definecolor{c676767}{RGB}{103,103,103}
\definecolor{c6e6e6e}{RGB}{110,110,110}
\definecolor{c767676}{RGB}{118,118,118}
\definecolor{c7e7e7e}{RGB}{126,126,126}
\definecolor{c858585}{RGB}{133,133,133}
\definecolor{c8d8d8d}{RGB}{141,141,141}
\definecolor{c959595}{RGB}{149,149,149}

\begin{tikzpicture}[%
    y=0.80pt,
    x=0.80pt,
    xscale=0.5,
    yscale=-0.5, 
    inner sep=0pt, 
    outer sep=0pt
]

  \path[fill=black] (281.0820,246.6060) .. controls (276.3300,266.1900) and
    (252.2100,307.0860) .. (239.3940,325.0860) .. controls (226.5780,343.1580) and
    (228.1620,359.4300) .. (204.4740,353.0940) .. controls (180.8580,346.7580) and
    (174.2340,347.9100) .. (149.8260,349.3500) .. controls (125.5620,350.7900) and
    (130.8180,348.6300) .. (115.6260,355.4700) .. controls (100.5060,362.3100) and
    (49.7461,272.5260) .. (45.6421,255.8220) .. controls (41.6101,239.1180) and
    (39.6661,241.1340) .. (50.1781,223.0620) .. controls (60.6901,204.9900) and
    (62.2021,187.1340) .. (76.0981,165.2460) .. controls (89.9941,143.2860) and
    (106.0500,132.1260) .. (104.8980,115.3500) .. controls (100.3620,53.0705) and
    (96.7621,21.9665) .. (124.4100,7.5665) .. controls (150.7620,-6.1136) and
    (172.7940,2.0225) .. (181.5060,6.7024) .. controls (185.2500,8.7184) and
    (192.8820,12.6064) .. (198.5700,19.4464) .. controls (204.2580,26.1425) and
    (209.3700,36.2945) .. (212.2500,49.1104) .. controls (218.1540,74.8145) and
    (209.8020,66.3185) .. (216.4980,95.7664) .. controls (223.1220,125.1420) and
    (236.5860,139.5420) .. (253.0020,162.7980) .. controls (269.4180,186.0540) and
    (286.5540,224.4300) .. (281.0820,246.6060) -- cycle;
    \path[fill=c666666] (126.0660,95.0464) .. controls (130.3860,93.3185) and
      (130.5300,93.4625) .. (133.4100,87.1985) .. controls (135.7140,82.3745) and
      (136.8660,79.9265) .. (136.7940,72.5825) .. controls (136.7940,65.3824) and
      (134.5620,62.9344) .. (131.1780,58.2545) .. controls (127.9380,53.7905) and
      (122.7540,53.5745) .. (119.5140,54.1505) .. controls (117.6420,54.4384) and
      (115.1940,56.8145) .. (113.5380,60.3424) .. controls (112.4580,62.7184) and
      (111.5940,65.7425) .. (111.5220,68.9105) .. controls (111.3060,77.4065) and
      (112.0260,80.6465) .. (113.9700,86.4065) .. controls (116.2740,93.1744) and
      (121.8900,96.7025) .. (126.0660,95.0464) -- cycle;
      \path[fill=c6d6d6d] (126.0660,95.0197) .. controls (130.3590,93.3029) and
        (130.5020,93.4460) .. (133.3630,87.2223) .. controls (135.6520,82.4294) and
        (136.7970,79.9971) .. (136.7250,72.7004) .. controls (136.7250,65.5468) and
        (134.4730,62.9640) .. (131.1460,58.4647) .. controls (127.8510,54.1204) and
        (122.8730,53.8462) .. (119.6570,54.4028) .. controls (117.7410,54.6858) and
        (115.3990,57.1312) .. (113.7160,60.5298) .. controls (112.5560,62.8403) and
        (111.6560,65.8669) .. (111.5850,69.0489) .. controls (111.3730,77.4871) and
        (112.1170,80.7156) .. (114.0480,86.4354) .. controls (116.3340,93.1598) and
        (121.9170,96.6651) .. (126.0660,95.0197) -- cycle;
      \path[fill=c757575] (126.0670,94.9930) .. controls (130.3310,93.2873) and
        (130.4730,93.4295) .. (133.3160,87.2462) .. controls (135.5900,82.4843) and
        (136.7280,80.0677) .. (136.6570,72.8184) .. controls (136.6570,65.7112) and
        (134.3840,62.9936) .. (131.1130,58.6750) .. controls (127.7640,54.4504) and
        (122.9920,54.1180) .. (119.8000,54.6552) .. controls (117.8390,54.9331) and
        (115.6040,57.4478) .. (113.8950,60.7172) .. controls (112.6530,62.9622) and
        (111.7180,65.9913) .. (111.6470,69.1873) .. controls (111.4400,77.5677) and
        (112.2080,80.7848) .. (114.1270,86.4644) .. controls (116.3950,93.1451) and
        (121.9440,96.6277) .. (126.0670,94.9930) -- cycle;
      \path[fill=c7c7c7c] (126.0670,94.9663) .. controls (130.3030,93.2716) and
        (130.4450,93.4130) .. (133.2690,87.2700) .. controls (135.5290,82.5392) and
        (136.6580,80.1384) .. (136.5880,72.9363) .. controls (136.5880,65.8755) and
        (134.2950,63.0231) .. (131.0800,58.8852) .. controls (127.6770,54.7803) and
        (123.1110,54.3897) .. (119.9430,54.9075) .. controls (117.9370,55.1805) and
        (115.8100,57.7646) .. (114.0730,60.9046) .. controls (112.7500,63.0841) and
        (111.7800,66.1157) .. (111.7100,69.3258) .. controls (111.5070,77.6483) and
        (112.2980,80.8539) .. (114.2050,86.4933) .. controls (116.4550,93.1305) and
        (121.9720,96.5904) .. (126.0670,94.9663) -- cycle;
      \path[fill=c848484] (126.0670,94.9396) .. controls (130.2760,93.2560) and
        (130.4160,93.3965) .. (133.2220,87.2939) .. controls (135.4670,82.5941) and
        (136.5890,80.2090) .. (136.5190,73.0543) .. controls (136.5190,66.0399) and
        (134.2060,63.0526) .. (131.0470,59.0955) .. controls (127.5900,55.1103) and
        (123.2290,54.6615) .. (120.0850,55.1599) .. controls (118.0360,55.4279) and
        (116.0150,58.0812) .. (114.2510,61.0920) .. controls (112.8470,63.2061) and
        (111.8420,66.2401) .. (111.7720,69.4643) .. controls (111.5740,77.7289) and
        (112.3890,80.9231) .. (114.2830,86.5222) .. controls (116.5150,93.1158) and
        (121.9990,96.5530) .. (126.0670,94.9396) -- cycle;
      \path[fill=c8c8c8c] (126.0670,94.9129) .. controls (130.2480,93.2404) and
        (130.3880,93.3800) .. (133.1750,87.3177) .. controls (135.4050,82.6490) and
        (136.5200,80.2797) .. (136.4500,73.1722) .. controls (136.4500,66.2042) and
        (134.1170,63.0822) .. (131.0150,59.3057) .. controls (127.5030,55.4402) and
        (123.3480,54.9332) .. (120.2280,55.4122) .. controls (118.1340,55.6752) and
        (116.2200,58.3979) .. (114.4290,61.2794) .. controls (112.9450,63.3279) and
        (111.9040,66.3645) .. (111.8350,69.6027) .. controls (111.6410,77.8095) and
        (112.4790,80.9922) .. (114.3610,86.5512) .. controls (116.5750,93.1012) and
        (122.0260,96.5157) .. (126.0670,94.9129) -- cycle;
      \path[fill=c939393] (126.0680,94.8863) .. controls (130.2210,93.2249) and
        (130.3590,93.3635) .. (133.1280,87.3416) .. controls (135.3430,82.7039) and
        (136.4500,80.3503) .. (136.3810,73.2902) .. controls (136.3810,66.3686) and
        (134.0280,63.1118) .. (130.9820,59.5160) .. controls (127.4160,55.7702) and
        (123.4670,55.2050) .. (120.3710,55.6646) .. controls (118.2330,55.9225) and
        (116.4250,58.7147) .. (114.6070,61.4669) .. controls (113.0420,63.4499) and
        (111.9670,66.4889) .. (111.8980,69.7411) .. controls (111.7090,77.8901) and
        (112.5700,81.0613) .. (114.4390,86.5802) .. controls (116.6350,93.0865) and
        (122.0530,96.4784) .. (126.0680,94.8863) -- cycle;
      \path[fill=c9b9b9b] (126.0680,94.8596) .. controls (130.1930,93.2093) and
        (130.3310,93.3470) .. (133.0810,87.3654) .. controls (135.2810,82.7588) and
        (136.3810,80.4210) .. (136.3120,73.4081) .. controls (136.3120,66.5329) and
        (133.9390,63.1413) .. (130.9490,59.7262) .. controls (127.3290,56.1001) and
        (123.5860,55.4767) .. (120.5140,55.9169) .. controls (118.3310,56.1699) and
        (116.6300,59.0313) .. (114.7850,61.6542) .. controls (113.1390,63.5717) and
        (112.0290,66.6133) .. (111.9600,69.8796) .. controls (111.7760,77.9707) and
        (112.6610,81.1305) .. (114.5170,86.6091) .. controls (116.6950,93.0719) and
        (122.0800,96.4410) .. (126.0680,94.8596) -- cycle;
      \path[fill=ca3a3a3] (126.0680,94.8329) .. controls (130.1650,93.1936) and
        (130.3020,93.3304) .. (133.0340,87.3893) .. controls (135.2190,82.8137) and
        (136.3120,80.4916) .. (136.2430,73.5261) .. controls (136.2430,66.6973) and
        (133.8500,63.1708) .. (130.9170,59.9365) .. controls (127.2410,56.4301) and
        (123.7050,55.7485) .. (120.6570,56.1693) .. controls (118.4290,56.4173) and
        (116.8350,59.3481) .. (114.9630,61.8416) .. controls (113.2370,63.6937) and
        (112.0910,66.7376) .. (112.0230,70.0181) .. controls (111.8430,78.0513) and
        (112.7510,81.1996) .. (114.5950,86.6381) .. controls (116.7550,93.0573) and
        (122.1070,96.4037) .. (126.0680,94.8329) -- cycle;
      \path[fill=caaaaaa] (126.0680,94.8062) .. controls (130.1380,93.1780) and
        (130.2740,93.3140) .. (132.9870,87.4131) .. controls (135.1570,82.8686) and
        (136.2430,80.5623) .. (136.1750,73.6440) .. controls (136.1750,66.8616) and
        (133.7610,63.2004) .. (130.8840,60.1467) .. controls (127.1540,56.7600) and
        (123.8230,56.0202) .. (120.7990,56.4216) .. controls (118.5280,56.6646) and
        (117.0400,59.6647) .. (115.1420,62.0291) .. controls (113.3340,63.8156) and
        (112.1530,66.8621) .. (112.0850,70.1565) .. controls (111.9100,78.1319) and
        (112.8420,81.2688) .. (114.6740,86.6670) .. controls (116.8160,93.0426) and
        (122.1340,96.3663) .. (126.0680,94.8062) -- cycle;
      \path[fill=cb2b2b2] (126.0690,94.7794) .. controls (130.1100,93.1624) and
        (130.2450,93.2975) .. (132.9400,87.4370) .. controls (135.0950,82.9234) and
        (136.1730,80.6330) .. (136.1060,73.7620) .. controls (136.1060,67.0260) and
        (133.6720,63.2299) .. (130.8510,60.3570) .. controls (127.0670,57.0900) and
        (123.9420,56.2920) .. (120.9420,56.6740) .. controls (118.6260,56.9120) and
        (117.2450,59.9814) .. (115.3200,62.2165) .. controls (113.4310,63.9375) and
        (112.2150,66.9865) .. (112.1480,70.2950) .. controls (111.9770,78.2125) and
        (112.9330,81.3380) .. (114.7520,86.6960) .. controls (116.8760,93.0279) and
        (122.1610,96.3289) .. (126.0690,94.7794) -- cycle;
      \path[fill=cbababa] (126.0690,94.7527) .. controls (130.0830,93.1469) and
        (130.2170,93.2809) .. (132.8930,87.4608) .. controls (135.0330,82.9783) and
        (136.1040,80.7036) .. (136.0370,73.8799) .. controls (136.0370,67.1903) and
        (133.5830,63.2595) .. (130.8190,60.5672) .. controls (126.9800,57.4199) and
        (124.0610,56.5637) .. (121.0850,56.9263) .. controls (118.7250,57.1593) and
        (117.4500,60.2982) .. (115.4980,62.4038) .. controls (113.5290,64.0594) and
        (112.2770,67.1108) .. (112.2100,70.4334) .. controls (112.0440,78.2931) and
        (113.0230,81.4071) .. (114.8300,86.7249) .. controls (116.9360,93.0133) and
        (122.1880,96.2916) .. (126.0690,94.7527) -- cycle;
      \path[fill=cc1c1c1] (126.0690,94.7260) .. controls (130.0550,93.1313) and
        (130.1880,93.2645) .. (132.8460,87.4846) .. controls (134.9710,83.0332) and
        (136.0350,80.7743) .. (135.9680,73.9979) .. controls (135.9680,67.3546) and
        (133.4940,63.2891) .. (130.7860,60.7774) .. controls (126.8930,57.7499) and
        (124.1800,56.8354) .. (121.2280,57.1786) .. controls (118.8230,57.4066) and
        (117.6550,60.6148) .. (115.6760,62.5912) .. controls (113.6260,64.1812) and
        (112.3390,67.2353) .. (112.2730,70.5718) .. controls (112.1110,78.3737) and
        (113.1140,81.4763) .. (114.9080,86.7538) .. controls (116.9960,92.9986) and
        (122.2150,96.2542) .. (126.0690,94.7260) -- cycle;
      \path[fill=cc9c9c9] (126.0690,94.6993) .. controls (130.0270,93.1156) and
        (130.1600,93.2480) .. (132.7990,87.5085) .. controls (134.9100,83.0881) and
        (135.9650,80.8449) .. (135.8990,74.1158) .. controls (135.8990,67.5190) and
        (133.4050,63.3186) .. (130.7530,60.9877) .. controls (126.8060,58.0798) and
        (124.2990,57.1072) .. (121.3710,57.4310) .. controls (118.9210,57.6540) and
        (117.8610,60.9315) .. (115.8540,62.7787) .. controls (113.7230,64.3032) and
        (112.4010,67.3596) .. (112.3350,70.7103) .. controls (112.1780,78.4543) and
        (113.2050,81.5454) .. (114.9860,86.7828) .. controls (117.0560,92.9840) and
        (122.2430,96.2169) .. (126.0690,94.6993) -- cycle;
      \path[fill=cd1d1d1] (126.0700,94.6726) .. controls (130.0000,93.1000) and
        (130.1310,93.2314) .. (132.7520,87.5323) .. controls (134.8480,83.1430) and
        (135.8960,80.9156) .. (135.8300,74.2338) .. controls (135.8300,67.6833) and
        (133.3160,63.3481) .. (130.7200,61.1979) .. controls (126.7190,58.4098) and
        (124.4170,57.3790) .. (121.5130,57.6833) .. controls (119.0200,57.9014) and
        (118.0660,61.2482) .. (116.0320,62.9661) .. controls (113.8200,64.4250) and
        (112.4630,67.4840) .. (112.3980,70.8488) .. controls (112.2450,78.5349) and
        (113.2950,81.6146) .. (115.0640,86.8117) .. controls (117.1160,92.9693) and
        (122.2700,96.1795) .. (126.0700,94.6726) -- cycle;
      \path[fill=cd8d8d8] (126.0700,94.6460) .. controls (129.9720,93.0844) and
        (130.1030,93.2149) .. (132.7040,87.5562) .. controls (134.7860,83.1979) and
        (135.8270,80.9862) .. (135.7610,74.3517) .. controls (135.7610,67.8477) and
        (133.2270,63.3777) .. (130.6880,61.4082) .. controls (126.6320,58.7397) and
        (124.5360,57.6507) .. (121.6560,57.9357) .. controls (119.1180,58.1487) and
        (118.2710,61.5649) .. (116.2100,63.1534) .. controls (113.9180,64.5470) and
        (112.5250,67.6084) .. (112.4600,70.9872) .. controls (112.3120,78.6154) and
        (113.3860,81.6837) .. (115.1420,86.8407) .. controls (117.1760,92.9547) and
        (122.2970,96.1422) .. (126.0700,94.6460) -- cycle;
      \path[fill=ce0e0e0] (126.0700,94.6193) .. controls (129.9450,93.0688) and
        (130.0740,93.1985) .. (132.6570,87.5800) .. controls (134.7240,83.2528) and
        (135.7570,81.0569) .. (135.6930,74.4697) .. controls (135.6930,68.0121) and
        (133.1380,63.4073) .. (130.6550,61.6184) .. controls (126.5450,59.0697) and
        (124.6550,57.9225) .. (121.7990,58.1880) .. controls (119.2170,58.3961) and
        (118.4760,61.8817) .. (116.3890,63.3409) .. controls (114.0150,64.6689) and
        (112.5870,67.7328) .. (112.5230,71.1256) .. controls (112.3790,78.6961) and
        (113.4770,81.7528) .. (115.2210,86.8697) .. controls (117.2370,92.9400) and
        (122.3240,96.1049) .. (126.0700,94.6193) -- cycle;
      \path[fill=ce8e8e8] (126.0700,94.5926) .. controls (129.9170,93.0533) and
        (130.0460,93.1819) .. (132.6100,87.6039) .. controls (134.6620,83.3077) and
        (135.6880,81.1275) .. (135.6240,74.5876) .. controls (135.6240,68.1764) and
        (133.0490,63.4368) .. (130.6220,61.8287) .. controls (126.4580,59.3996) and
        (124.7740,58.1942) .. (121.9420,58.4404) .. controls (119.3150,58.6434) and
        (118.6810,62.1983) .. (116.5670,63.5283) .. controls (114.1120,64.7908) and
        (112.6490,67.8572) .. (112.5860,71.2641) .. controls (112.4460,78.7767) and
        (113.5670,81.8220) .. (115.2990,86.8986) .. controls (117.2970,92.9254) and
        (122.3510,96.0675) .. (126.0700,94.5926) -- cycle;
      \path[fill=cefefef] (126.0710,94.5659) .. controls (129.8890,93.0376) and
        (130.0170,93.1654) .. (132.5630,87.6277) .. controls (134.6000,83.3626) and
        (135.6190,81.1982) .. (135.5550,74.7056) .. controls (135.5550,68.3408) and
        (132.9600,63.4663) .. (130.5900,62.0389) .. controls (126.3700,59.7296) and
        (124.8930,58.4660) .. (122.0850,58.6927) .. controls (119.4130,58.8908) and
        (118.8860,62.5150) .. (116.7450,63.7156) .. controls (114.2100,64.9127) and
        (112.7110,67.9816) .. (112.6480,71.4026) .. controls (112.5130,78.8573) and
        (113.6580,81.8911) .. (115.3770,86.9276) .. controls (117.3570,92.9107) and
        (122.3780,96.0302) .. (126.0710,94.5659) -- cycle;
      \path[fill=cf7f7f7] (126.0710,94.5392) .. controls (129.8620,93.0220) and
        (129.9890,93.1490) .. (132.5160,87.6516) .. controls (134.5380,83.4175) and
        (135.5500,81.2688) .. (135.4860,74.8235) .. controls (135.4860,68.5051) and
        (132.8710,63.4959) .. (130.5570,62.2492) .. controls (126.2830,60.0595) and
        (125.0110,58.7377) .. (122.2270,58.9451) .. controls (119.5120,59.1381) and
        (119.0910,62.8317) .. (116.9230,63.9030) .. controls (114.3070,65.0346) and
        (112.7730,68.1060) .. (112.7110,71.5410) .. controls (112.5800,78.9379) and
        (113.7490,81.9603) .. (115.4550,86.9565) .. controls (117.4170,92.8961) and
        (122.4050,95.9928) .. (126.0710,94.5392) -- cycle;
    \path[fill=cffffff] (126.0710,94.5125) .. controls (129.8340,93.0064) and
      (129.9600,93.1324) .. (132.4690,87.6754) .. controls (134.4760,83.4724) and
      (135.4800,81.3394) .. (135.4170,74.9415) .. controls (135.4170,68.6695) and
      (132.7820,63.5255) .. (130.5240,62.4594) .. controls (126.1960,60.3895) and
      (125.1300,59.0095) .. (122.3700,59.1974) .. controls (119.6100,59.3855) and
      (119.2960,63.1484) .. (117.1010,64.0905) .. controls (114.4040,65.1565) and
      (112.8350,68.2304) .. (112.7730,71.6794) .. controls (112.6470,79.0185) and
      (113.8390,82.0294) .. (115.5330,86.9854) .. controls (117.4770,92.8814) and
      (122.4320,95.9554) .. (126.0710,94.5125) -- cycle;
  \path[fill=black] (122.0340,65.7425) .. controls (124.1940,65.7425) and
    (126.9300,67.1825) .. (128.2260,69.1265) .. controls (129.5940,71.0705) and
    (130.6020,73.8065) .. (130.6020,76.9025) .. controls (130.6020,81.5105) and
    (130.0980,86.6225) .. (127.3620,88.2065) .. controls (126.4980,88.7104) and
    (124.6260,89.1425) .. (123.5460,89.1425) .. controls (121.0980,89.1425) and
    (120.8820,87.5584) .. (118.5780,85.1825) .. controls (117.7860,84.3185) and
    (115.4100,80.1425) .. (115.4100,76.6865) .. controls (115.4100,74.5265) and
    (114.9060,71.4305) .. (116.7780,68.6945) .. controls (118.0740,66.6785) and
    (119.7300,65.7425) .. (122.0340,65.7425) -- cycle;
    \path[fill=black] (121.4580,69.6054) .. controls (122.3020,68.3004) and
      (125.6800,68.9155) .. (126.9080,71.6014) .. controls (128.1370,74.2885) and
      (127.9060,80.1234) .. (127.0620,80.4305) .. controls (124.8360,81.1215) and
      (125.5260,77.8965) .. (123.6070,74.9795) .. controls (121.6880,72.2165) and
      (120.6130,70.9114) .. (121.4580,69.6054) -- cycle;
      \path[fill=c070707] (121.5120,69.6606) .. controls (122.3460,68.3721) and
        (125.6810,68.9793) .. (126.8930,71.6313) .. controls (128.1070,74.2842) and
        (127.8790,80.0453) .. (127.0450,80.3484) .. controls (124.8470,81.0306) and
        (125.5290,77.8465) .. (123.6340,74.9665) .. controls (121.7390,72.2385) and
        (120.6780,70.9500) .. (121.5120,69.6606) -- cycle;
      \path[fill=c0f0f0f] (121.5660,69.7156) .. controls (122.3890,68.4438) and
        (125.6810,69.0431) .. (126.8780,71.6610) .. controls (128.0760,74.2799) and
        (127.8510,79.9670) .. (127.0280,80.2663) .. controls (124.8590,80.9398) and
        (125.5310,77.7964) .. (123.6610,74.9535) .. controls (121.7900,72.2605) and
        (120.7430,70.9884) .. (121.5660,69.7156) -- cycle;
      \path[fill=c161616] (121.6200,69.7708) .. controls (122.4320,68.5154) and
        (125.6820,69.1069) .. (126.8630,71.6909) .. controls (128.0450,74.2757) and
        (127.8230,79.8889) .. (127.0110,80.1841) .. controls (124.8700,80.8489) and
        (125.5340,77.7465) .. (123.6880,74.9405) .. controls (121.8420,72.2825) and
        (120.8070,71.0269) .. (121.6200,69.7708) -- cycle;
      \path[fill=c1e1e1e] (121.6740,69.8258) .. controls (122.4750,68.5870) and
        (125.6820,69.1707) .. (126.8480,71.7206) .. controls (128.0150,74.2715) and
        (127.7950,79.8106) .. (126.9940,80.1021) .. controls (124.8810,80.7581) and
        (125.5360,77.6964) .. (123.7140,74.9275) .. controls (121.8930,72.3045) and
        (120.8720,71.0654) .. (121.6740,69.8258) -- cycle;
      \path[fill=c262626] (121.7280,69.8810) .. controls (122.5190,68.6587) and
        (125.6830,69.2345) .. (126.8330,71.7505) .. controls (127.9840,74.2672) and
        (127.7680,79.7325) .. (126.9770,80.0200) .. controls (124.8920,80.6672) and
        (125.5390,77.6465) .. (123.7410,74.9145) .. controls (121.9440,72.3264) and
        (120.9370,71.1039) .. (121.7280,69.8810) -- cycle;
      \path[fill=c2d2d2d] (121.7820,69.9360) .. controls (122.5620,68.7303) and
        (125.6830,69.2983) .. (126.8180,71.7802) .. controls (127.9530,74.2630) and
        (127.7400,79.6542) .. (126.9600,79.9378) .. controls (124.9030,80.5764) and
        (125.5410,77.5965) .. (123.7680,74.9015) .. controls (121.9950,72.3484) and
        (121.0020,71.1425) .. (121.7820,69.9360) -- cycle;
      \path[fill=c353535] (121.8360,69.9911) .. controls (122.6050,68.8020) and
        (125.6840,69.3621) .. (126.8030,71.8100) .. controls (127.9230,74.2587) and
        (127.7120,79.5760) .. (126.9430,79.8557) .. controls (124.9150,80.4855) and
        (125.5440,77.5464) .. (123.7950,74.8885) .. controls (122.0460,72.3705) and
        (121.0660,71.1809) .. (121.8360,69.9911) -- cycle;
      \path[fill=c3d3d3d] (121.8900,70.0462) .. controls (122.6490,68.8737) and
        (125.6840,69.4259) .. (126.7880,71.8398) .. controls (127.8920,74.2545) and
        (127.6850,79.4978) .. (126.9260,79.7737) .. controls (124.9260,80.3947) and
        (125.5460,77.4965) .. (123.8220,74.8755) .. controls (122.0970,72.3925) and
        (121.1310,71.2195) .. (121.8900,70.0462) -- cycle;
      \path[fill=c444444] (121.9440,70.1013) .. controls (122.6920,68.9453) and
        (125.6850,69.4897) .. (126.7730,71.8696) .. controls (127.8610,74.2502) and
        (127.6570,79.4196) .. (126.9090,79.6916) .. controls (124.9370,80.3038) and
        (125.5490,77.4464) .. (123.8480,74.8625) .. controls (122.1480,72.4145) and
        (121.1960,71.2579) .. (121.9440,70.1013) -- cycle;
      \path[fill=c4c4c4c] (121.9980,70.1565) .. controls (122.7350,69.0169) and
        (125.6850,69.5535) .. (126.7580,71.8994) .. controls (127.8310,74.2459) and
        (127.6290,79.3415) .. (126.8920,79.6095) .. controls (124.9480,80.2130) and
        (125.5510,77.3965) .. (123.8750,74.8495) .. controls (122.1990,72.4365) and
        (121.2610,71.2964) .. (121.9980,70.1565) -- cycle;
      \path[fill=c545454] (122.0520,70.2115) .. controls (122.7790,69.0886) and
        (125.6860,69.6173) .. (126.7430,71.9292) .. controls (127.8000,74.2417) and
        (127.6020,79.2632) .. (126.8750,79.5273) .. controls (124.9590,80.1221) and
        (125.5540,77.3465) .. (123.9020,74.8365) .. controls (122.2500,72.4585) and
        (121.3250,71.3350) .. (122.0520,70.2115) -- cycle;
      \path[fill=c5b5b5b] (122.1060,70.2667) .. controls (122.8220,69.1602) and
        (125.6860,69.6811) .. (126.7280,71.9590) .. controls (127.7690,74.2375) and
        (127.5740,79.1851) .. (126.8580,79.4453) .. controls (124.9710,80.0312) and
        (125.5560,77.2964) .. (123.9290,74.8235) .. controls (122.3010,72.4805) and
        (121.3900,71.3734) .. (122.1060,70.2667) -- cycle;
      \path[fill=c636363] (122.1600,70.3217) .. controls (122.8650,69.2319) and
        (125.6870,69.7449) .. (126.7130,71.9889) .. controls (127.7390,74.2332) and
        (127.5460,79.1068) .. (126.8410,79.3632) .. controls (124.9820,79.9404) and
        (125.5590,77.2465) .. (123.9560,74.8105) .. controls (122.3530,72.5025) and
        (121.4550,71.4120) .. (122.1600,70.3217) -- cycle;
      \path[fill=c6b6b6b] (122.2140,70.3769) .. controls (122.9080,69.3036) and
        (125.6870,69.8087) .. (126.6980,72.0186) .. controls (127.7080,74.2290) and
        (127.5180,79.0287) .. (126.8240,79.2811) .. controls (124.9930,79.8495) and
        (125.5610,77.1964) .. (123.9820,74.7975) .. controls (122.4040,72.5245) and
        (121.5200,71.4504) .. (122.2140,70.3769) -- cycle;
      \path[fill=c727272] (122.2680,70.4319) .. controls (122.9520,69.3752) and
        (125.6880,69.8725) .. (126.6820,72.0484) .. controls (127.6770,74.2247) and
        (127.4910,78.9504) .. (126.8070,79.1989) .. controls (125.0040,79.7587) and
        (125.5640,77.1465) .. (124.0090,74.7845) .. controls (122.4550,72.5465) and
        (121.5840,71.4890) .. (122.2680,70.4319) -- cycle;
      \path[fill=c7a7a7a] (122.3220,70.4871) .. controls (122.9950,69.4469) and
        (125.6880,69.9363) .. (126.6670,72.0782) .. controls (127.6470,74.2205) and
        (127.4630,78.8723) .. (126.7900,79.1169) .. controls (125.0150,79.6678) and
        (125.5660,77.0965) .. (124.0360,74.7715) .. controls (122.5060,72.5685) and
        (121.6490,71.5275) .. (122.3220,70.4871) -- cycle;
      \path[fill=c828282] (122.3760,70.5421) .. controls (123.0380,69.5185) and
        (125.6890,70.0001) .. (126.6520,72.1081) .. controls (127.6160,74.2162) and
        (127.4350,78.7940) .. (126.7730,79.0348) .. controls (125.0270,79.5770) and
        (125.5690,77.0464) .. (124.0630,74.7585) .. controls (122.5570,72.5905) and
        (121.7140,71.5659) .. (122.3760,70.5421) -- cycle;
      \path[fill=c898989] (122.4300,70.5973) .. controls (123.0820,69.5901) and
        (125.6890,70.0639) .. (126.6370,72.1378) .. controls (127.5860,74.2119) and
        (127.4080,78.7159) .. (126.7560,78.9526) .. controls (125.0380,79.4861) and
        (125.5710,76.9965) .. (124.0900,74.7455) .. controls (122.6080,72.6125) and
        (121.7790,71.6045) .. (122.4300,70.5973) -- cycle;
      \path[fill=c919191] (122.4840,70.6523) .. controls (123.1250,69.6618) and
        (125.6900,70.1277) .. (126.6220,72.1677) .. controls (127.5550,74.2077) and
        (127.3800,78.6376) .. (126.7390,78.8705) .. controls (125.0490,79.3953) and
        (125.5740,76.9464) .. (124.1160,74.7325) .. controls (122.6590,72.6345) and
        (121.8430,71.6429) .. (122.4840,70.6523) -- cycle;
    \path[fill=c999999] (122.5380,70.7075) .. controls (123.1680,69.7335) and
      (125.6900,70.1915) .. (126.6070,72.1974) .. controls (127.5240,74.2035) and
      (127.3520,78.5594) .. (126.7220,78.7885) .. controls (125.0600,79.3044) and
      (125.5760,76.8965) .. (124.1430,74.7195) .. controls (122.7100,72.6565) and
      (121.9080,71.6815) .. (122.5380,70.7075) -- cycle;
    \path[fill=c666666] (170.7060,95.9825) .. controls (181.5060,94.8304) and
      (185.3220,90.6544) .. (187.2660,83.3824) .. controls (188.9940,76.9025) and
      (189.0660,69.7025) .. (184.2420,61.2784) .. controls (179.7060,53.2144) and
      (177.1140,51.9185) .. (170.5620,51.4145) .. controls (160.4820,50.5504) and
      (155.7300,57.6064) .. (153.4980,62.6465) .. controls (151.1220,68.1184) and
      (151.6260,67.3264) .. (151.7700,74.2385) .. controls (151.9140,81.5105) and
      (154.1460,84.5345) .. (156.6660,89.0705) .. controls (159.1860,93.5345) and
      (169.4820,96.0544) .. (170.7060,95.9825) -- cycle;
      \path[fill=c6d6d6d] (170.7110,95.9410) .. controls (181.4700,94.8098) and
        (185.3290,90.5872) .. (187.2260,83.3721) .. controls (188.9270,76.8469) and
        (188.9170,69.6987) .. (184.1300,61.4722) .. controls (179.6970,53.6753) and
        (177.0360,52.3606) .. (170.5620,51.8606) .. controls (160.5130,50.9998) and
        (155.8930,57.7740) .. (153.6430,62.7345) .. controls (151.3290,68.0081) and
        (151.7510,67.4196) .. (151.9000,74.2488) .. controls (152.0540,81.5205) and
        (154.2160,84.5122) .. (156.7340,89.0498) .. controls (159.2540,93.5241) and
        (169.4870,96.0139) .. (170.7110,95.9410) -- cycle;
      \path[fill=c757575] (170.7170,95.8996) .. controls (181.4340,94.7890) and
        (185.3360,90.5201) .. (187.1850,83.3618) .. controls (188.8600,76.7914) and
        (188.7690,69.6949) .. (184.0180,61.6660) .. controls (179.6870,54.1360) and
        (176.9590,52.8026) .. (170.5620,52.3068) .. controls (160.5450,51.4490) and
        (156.0560,57.9417) .. (153.7880,62.8225) .. controls (151.5370,67.8978) and
        (151.8750,67.5128) .. (152.0290,74.2592) .. controls (152.1930,81.5305) and
        (154.2870,84.4900) .. (156.8010,89.0291) .. controls (159.3210,93.5138) and
        (169.4930,95.9734) .. (170.7170,95.8996) -- cycle;
      \path[fill=c7c7c7c] (170.7220,95.8581) .. controls (181.3970,94.7683) and
        (185.3430,90.4529) .. (187.1450,83.3514) .. controls (188.7930,76.7358) and
        (188.6200,69.6912) .. (183.9060,61.8598) .. controls (179.6770,54.5969) and
        (176.8810,53.2448) .. (170.5620,52.7529) .. controls (160.5760,51.8983) and
        (156.2200,58.1093) .. (153.9330,62.9106) .. controls (151.7440,67.7875) and
        (151.9990,67.6060) .. (152.1590,74.2695) .. controls (152.3320,81.5405) and
        (154.3570,84.4677) .. (156.8680,89.0084) .. controls (159.3880,93.5034) and
        (169.4980,95.9330) .. (170.7220,95.8581) -- cycle;
      \path[fill=c848484] (170.7270,95.8167) .. controls (181.3610,94.7477) and
        (185.3500,90.3857) .. (187.1040,83.3410) .. controls (188.7260,76.6803) and
        (188.4710,69.6874) .. (183.7940,62.0536) .. controls (179.6680,55.0576) and
        (176.8030,53.6869) .. (170.5620,53.1991) .. controls (160.6070,52.3477) and
        (156.3830,58.2769) .. (154.0780,62.9987) .. controls (151.9510,67.6772) and
        (152.1230,67.6992) .. (152.2880,74.2799) .. controls (152.4720,81.5505) and
        (154.4270,84.4455) .. (156.9350,88.9877) .. controls (159.4550,93.4930) and
        (169.5030,95.8925) .. (170.7270,95.8167) -- cycle;
      \path[fill=c8c8c8c] (170.7320,95.7752) .. controls (181.3250,94.7270) and
        (185.3570,90.3185) .. (187.0640,83.3307) .. controls (188.6590,76.6247) and
        (188.3220,69.6837) .. (183.6820,62.2474) .. controls (179.6580,55.5185) and
        (176.7250,54.1290) .. (170.5620,53.6452) .. controls (160.6380,52.7969) and
        (156.5460,58.4445) .. (154.2230,63.0867) .. controls (152.1580,67.5669) and
        (152.2480,67.7924) .. (152.4170,74.2902) .. controls (152.6110,81.5605) and
        (154.4970,84.4232) .. (157.0030,88.9669) .. controls (159.5230,93.4827) and
        (169.5080,95.8520) .. (170.7320,95.7752) -- cycle;
      \path[fill=c939393] (170.7370,95.7337) .. controls (181.2890,94.7062) and
        (185.3650,90.2513) .. (187.0240,83.3203) .. controls (188.5920,76.5692) and
        (188.1730,69.6800) .. (183.5710,62.4413) .. controls (179.6490,55.9792) and
        (176.6480,54.5711) .. (170.5620,54.0914) .. controls (160.6700,53.2462) and
        (156.7090,58.6121) .. (154.3680,63.1747) .. controls (152.3650,67.4566) and
        (152.3720,67.8857) .. (152.5470,74.3006) .. controls (152.7500,81.5705) and
        (154.5670,84.4009) .. (157.0700,88.9463) .. controls (159.5900,93.4724) and
        (169.5140,95.8114) .. (170.7370,95.7337) -- cycle;
      \path[fill=c9b9b9b] (170.7420,95.6923) .. controls (181.2520,94.6855) and
        (185.3720,90.1841) .. (186.9830,83.3100) .. controls (188.5250,76.5136) and
        (188.0240,69.6762) .. (183.4590,62.6350) .. controls (179.6390,56.4400) and
        (176.5700,55.0132) .. (170.5620,54.5375) .. controls (160.7010,53.6956) and
        (156.8720,58.7797) .. (154.5130,63.2628) .. controls (152.5720,67.3463) and
        (152.4960,67.9789) .. (152.6760,74.3109) .. controls (152.8900,81.5805) and
        (154.6370,84.3787) .. (157.1370,88.9256) .. controls (159.6570,93.4620) and
        (169.5190,95.7709) .. (170.7420,95.6923) -- cycle;
      \path[fill=ca3a3a3] (170.7470,95.6508) .. controls (181.2160,94.6649) and
        (185.3790,90.1169) .. (186.9430,83.2997) .. controls (188.4580,76.4581) and
        (187.8750,69.6725) .. (183.3470,62.8289) .. controls (179.6290,56.9008) and
        (176.4920,55.4553) .. (170.5620,54.9837) .. controls (160.7320,54.1448) and
        (157.0350,58.9473) .. (154.6580,63.3509) .. controls (152.7790,67.2361) and
        (152.6210,68.0721) .. (152.8060,74.3213) .. controls (153.0290,81.5905) and
        (154.7070,84.3564) .. (157.2050,88.9049) .. controls (159.7250,93.4517) and
        (169.5240,95.7304) .. (170.7470,95.6508) -- cycle;
      \path[fill=caaaaaa] (170.7530,95.6094) .. controls (181.1800,94.6441) and
        (185.3860,90.0497) .. (186.9020,83.2893) .. controls (188.3910,76.4025) and
        (187.7270,69.6687) .. (183.2350,63.0226) .. controls (179.6200,57.3616) and
        (176.4140,55.8974) .. (170.5620,55.4298) .. controls (160.7630,54.5941) and
        (157.1980,59.1149) .. (154.8030,63.4389) .. controls (152.9870,67.1257) and
        (152.7450,68.1653) .. (152.9350,74.3316) .. controls (153.1680,81.6005) and
        (154.7780,84.3342) .. (157.2720,88.8842) .. controls (159.7920,93.4413) and
        (169.5290,95.6899) .. (170.7530,95.6094) -- cycle;
      \path[fill=cb2b2b2] (170.7580,95.5680) .. controls (181.1440,94.6234) and
        (185.3930,89.9825) .. (186.8620,83.2790) .. controls (188.3240,76.3470) and
        (187.5780,69.6649) .. (183.1230,63.2165) .. controls (179.6100,57.8224) and
        (176.3370,56.3394) .. (170.5620,55.8759) .. controls (160.7950,55.0434) and
        (157.3610,59.2824) .. (154.9480,63.5269) .. controls (153.1940,67.0154) and
        (152.8690,68.2585) .. (153.0650,74.3419) .. controls (153.3080,81.6104) and
        (154.8480,84.3120) .. (157.3390,88.8635) .. controls (159.8590,93.4309) and
        (169.5350,95.6494) .. (170.7580,95.5680) -- cycle;
      \path[fill=cbababa] (170.7630,95.5265) .. controls (181.1070,94.6028) and
        (185.4000,89.9153) .. (186.8210,83.2686) .. controls (188.2570,76.2914) and
        (187.4290,69.6612) .. (183.0110,63.4102) .. controls (179.6010,58.2832) and
        (176.2590,56.7816) .. (170.5620,56.3221) .. controls (160.8260,55.4927) and
        (157.5240,59.4500) .. (155.0930,63.6150) .. controls (153.4010,66.9052) and
        (152.9940,68.3517) .. (153.1940,74.3523) .. controls (153.4470,81.6205) and
        (154.9180,84.2897) .. (157.4070,88.8428) .. controls (159.9270,93.4206) and
        (169.5400,95.6089) .. (170.7630,95.5265) -- cycle;
      \path[fill=cc1c1c1] (170.7680,95.4850) .. controls (181.0710,94.5820) and
        (185.4070,89.8481) .. (186.7810,83.2582) .. controls (188.1900,76.2358) and
        (187.2800,69.6574) .. (182.8990,63.6040) .. controls (179.5910,58.7440) and
        (176.1810,57.2237) .. (170.5620,56.7682) .. controls (160.8570,55.9420) and
        (157.6870,59.6176) .. (155.2380,63.7030) .. controls (153.6080,66.7948) and
        (153.1180,68.4449) .. (153.3240,74.3627) .. controls (153.5860,81.6305) and
        (154.9880,84.2675) .. (157.4740,88.8221) .. controls (159.9940,93.4102) and
        (169.5450,95.5685) .. (170.7680,95.4850) -- cycle;
      \path[fill=cc9c9c9] (170.7730,95.4436) .. controls (181.0350,94.5613) and
        (185.4140,89.7809) .. (186.7400,83.2479) .. controls (188.1230,76.1803) and
        (187.1310,69.6537) .. (182.7870,63.7979) .. controls (179.5810,59.2049) and
        (176.1030,57.6658) .. (170.5620,57.2144) .. controls (160.8880,56.3913) and
        (157.8510,59.7852) .. (155.3830,63.7911) .. controls (153.8150,66.6845) and
        (153.2420,68.5381) .. (153.4530,74.3730) .. controls (153.7260,81.6404) and
        (155.0580,84.2452) .. (157.5410,88.8014) .. controls (160.0610,93.3999) and
        (169.5500,95.5280) .. (170.7730,95.4436) -- cycle;
      \path[fill=cd1d1d1] (170.7780,95.4022) .. controls (180.9990,94.5406) and
        (185.4210,89.7137) .. (186.7000,83.2375) .. controls (188.0560,76.1248) and
        (186.9820,69.6499) .. (182.6750,63.9916) .. controls (179.5720,59.6656) and
        (176.0260,58.1079) .. (170.5620,57.6606) .. controls (160.9200,56.8406) and
        (158.0140,59.9529) .. (155.5280,63.8792) .. controls (154.0220,66.5742) and
        (153.3660,68.6313) .. (153.5830,74.3834) .. controls (153.8650,81.6505) and
        (155.1280,84.2230) .. (157.6080,88.7807) .. controls (160.1280,93.3896) and
        (169.5560,95.4875) .. (170.7780,95.4022) -- cycle;
      \path[fill=cd8d8d8] (170.7830,95.3607) .. controls (180.9620,94.5199) and
        (185.4280,89.6465) .. (186.6590,83.2272) .. controls (187.9890,76.0692) and
        (186.8330,69.6462) .. (182.5630,64.1855) .. controls (179.5620,60.1264) and
        (175.9480,58.5500) .. (170.5620,58.1067) .. controls (160.9510,57.2899) and
        (158.1770,60.1205) .. (155.6730,63.9672) .. controls (154.2290,66.4640) and
        (153.4910,68.7245) .. (153.7120,74.3937) .. controls (154.0040,81.6605) and
        (155.1980,84.2007) .. (157.6760,88.7600) .. controls (160.1960,93.3792) and
        (169.5610,95.4469) .. (170.7830,95.3607) -- cycle;
      \path[fill=ce0e0e0] (170.7890,95.3192) .. controls (180.9260,94.4992) and
        (185.4350,89.5793) .. (186.6190,83.2169) .. controls (187.9220,76.0136) and
        (186.6850,69.6425) .. (182.4510,64.3792) .. controls (179.5530,60.5872) and
        (175.8700,58.9921) .. (170.5620,58.5529) .. controls (160.9820,57.7392) and
        (158.3400,60.2881) .. (155.8180,64.0552) .. controls (154.4370,66.3536) and
        (153.6150,68.8177) .. (153.8410,74.4041) .. controls (154.1440,81.6704) and
        (155.2690,84.1784) .. (157.7430,88.7393) .. controls (160.2630,93.3689) and
        (169.5660,95.4064) .. (170.7890,95.3192) -- cycle;
      \path[fill=ce8e8e8] (170.7940,95.2778) .. controls (180.8900,94.4785) and
        (185.4420,89.5121) .. (186.5790,83.2065) .. controls (187.8550,75.9581) and
        (186.5360,69.6387) .. (182.3390,64.5731) .. controls (179.5430,61.0481) and
        (175.7920,59.4342) .. (170.5620,58.9990) .. controls (161.0130,58.1885) and
        (158.5030,60.4557) .. (155.9630,64.1433) .. controls (154.6440,66.2433) and
        (153.7390,68.9109) .. (153.9710,74.4144) .. controls (154.2830,81.6805) and
        (155.3390,84.1562) .. (157.8100,88.7186) .. controls (160.3300,93.3585) and
        (169.5710,95.3659) .. (170.7940,95.2778) -- cycle;
      \path[fill=cefefef] (170.7990,95.2363) .. controls (180.8540,94.4579) and
        (185.4490,89.4449) .. (186.5380,83.1961) .. controls (187.7880,75.9026) and
        (186.3870,69.6349) .. (182.2270,64.7668) .. controls (179.5330,61.5089) and
        (175.7150,59.8763) .. (170.5620,59.4452) .. controls (161.0450,58.6378) and
        (158.6660,60.6233) .. (156.1080,64.2314) .. controls (154.8510,66.1331) and
        (153.8640,69.0041) .. (154.1000,74.4248) .. controls (154.4230,81.6905) and
        (155.4090,84.1339) .. (157.8780,88.6979) .. controls (160.3980,93.3481) and
        (169.5770,95.3254) .. (170.7990,95.2363) -- cycle;
      \path[fill=cf7f7f7] (170.8040,95.1949) .. controls (180.8170,94.4371) and
        (185.4560,89.3777) .. (186.4980,83.1858) .. controls (187.7210,75.8470) and
        (186.2380,69.6312) .. (182.1150,64.9607) .. controls (179.5240,61.9697) and
        (175.6370,60.3184) .. (170.5620,59.8913) .. controls (161.0760,59.0871) and
        (158.8290,60.7909) .. (156.2530,64.3194) .. controls (155.0580,66.0227) and
        (153.9880,69.0973) .. (154.2300,74.4351) .. controls (154.5620,81.7004) and
        (155.4790,84.1117) .. (157.9450,88.6772) .. controls (160.4650,93.3378) and
        (169.5820,95.2849) .. (170.8040,95.1949) -- cycle;
    \path[fill=cffffff] (170.8090,95.1534) .. controls (180.7810,94.4164) and
      (185.4630,89.3105) .. (186.4570,83.1754) .. controls (187.6540,75.7914) and
      (186.0890,69.6274) .. (182.0030,65.1544) .. controls (179.5140,62.4305) and
      (175.5590,60.7605) .. (170.5620,60.3375) .. controls (161.1070,59.5364) and
      (158.9920,60.9585) .. (156.3980,64.4074) .. controls (155.2650,65.9124) and
      (154.1120,69.1905) .. (154.3590,74.4455) .. controls (154.7010,81.7104) and
      (155.5490,84.0894) .. (158.0120,88.6565) .. controls (160.5320,93.3275) and
      (169.5870,95.2444) .. (170.8090,95.1534) -- cycle;
  \path[fill=black] (169.8420,65.4544) .. controls (175.2420,65.4544) and
    (178.4100,70.2784) .. (179.4900,76.4705) .. controls (179.9220,79.2784) and
    (179.2740,82.5185) .. (177.5460,84.7505) .. controls (175.6020,87.3424) and
    (172.1460,88.9265) .. (169.3380,88.9265) .. controls (166.6740,88.9265) and
    (163.6500,89.3585) .. (162.0660,87.1985) .. controls (160.4820,84.9665) and
    (160.1220,79.9984) .. (160.1220,76.4705) .. controls (160.1220,72.5105) and
    (161.2740,69.7025) .. (163.2900,67.4705) .. controls (164.8020,65.8145) and
    (167.5380,65.4544) .. (169.8420,65.4544) -- cycle;
    \path[fill=black] (170.1870,67.6765) .. controls (171.1670,67.0234) and
      (172.7990,67.6765) .. (174.6760,69.7985) .. controls (176.7150,72.0835) and
      (177.6130,73.8784) .. (175.1650,75.1834) .. controls (173.2880,76.1635) and
      (172.7170,73.2255) .. (171.5750,72.0015) .. controls (169.7800,70.0435) and
      (168.2290,68.9825) .. (170.1870,67.6765) -- cycle;
      \path[fill=c070707] (170.2270,67.7351) .. controls (171.1870,67.0860) and
        (172.8000,67.7351) .. (174.6490,69.8256) .. controls (176.6580,72.0767) and
        (177.5140,73.8451) .. (175.1310,75.1307) .. controls (173.2940,76.1077) and
        (172.7280,73.2018) .. (171.6060,71.9960) .. controls (169.8380,70.0699) and
        (168.3100,69.0303) .. (170.2270,67.7351) -- cycle;
      \path[fill=c0f0f0f] (170.2670,67.7938) .. controls (171.2060,67.1487) and
        (172.8010,67.7938) .. (174.6230,69.8528) .. controls (176.6010,72.0700) and
        (177.4150,73.8116) .. (175.0970,75.0779) .. controls (173.2990,76.0518) and
        (172.7390,73.1781) .. (171.6370,71.9904) .. controls (169.8950,70.0963) and
        (168.3900,69.0782) .. (170.2670,67.7938) -- cycle;
      \path[fill=c161616] (170.3070,67.8524) .. controls (171.2260,67.2112) and
        (172.8030,67.8524) .. (174.5960,69.8799) .. controls (176.5440,72.0632) and
        (177.3160,73.7783) .. (175.0630,75.0252) .. controls (173.3040,75.9961) and
        (172.7500,73.1544) .. (171.6670,71.9850) .. controls (169.9520,70.1227) and
        (168.4700,69.1260) .. (170.3070,67.8524) -- cycle;
      \path[fill=c1e1e1e] (170.3470,67.9110) .. controls (171.2460,67.2738) and
        (172.8040,67.9110) .. (174.5690,69.9071) .. controls (176.4870,72.0565) and
        (177.2170,73.7448) .. (175.0290,74.9724) .. controls (173.3100,75.9402) and
        (172.7610,73.1307) .. (171.6980,71.9795) .. controls (170.0100,70.1490) and
        (168.5510,69.1739) .. (170.3470,67.9110) -- cycle;
      \path[fill=c262626] (170.3870,67.9697) .. controls (171.2650,67.3365) and
        (172.8050,67.9697) .. (174.5420,69.9342) .. controls (176.4300,72.0497) and
        (177.1180,73.7115) .. (174.9950,74.9197) .. controls (173.3150,75.8845) and
        (172.7720,73.1070) .. (171.7290,71.9740) .. controls (170.0670,70.1754) and
        (168.6310,69.2217) .. (170.3870,67.9697) -- cycle;
      \path[fill=c2d2d2d] (170.4270,68.0284) .. controls (171.2850,67.3990) and
        (172.8060,68.0284) .. (174.5160,69.9614) .. controls (176.3730,72.0430) and
        (177.0190,73.6780) .. (174.9610,74.8669) .. controls (173.3200,75.8286) and
        (172.7830,73.0833) .. (171.7600,71.9685) .. controls (170.1240,70.2018) and
        (168.7120,69.2696) .. (170.4270,68.0284) -- cycle;
      \path[fill=c353535] (170.4670,68.0870) .. controls (171.3040,67.4616) and
        (172.8070,68.0870) .. (174.4890,69.9885) .. controls (176.3160,72.0362) and
        (176.9200,73.6447) .. (174.9270,74.8142) .. controls (173.3260,75.7729) and
        (172.7940,73.0596) .. (171.7900,71.9630) .. controls (170.1820,70.2282) and
        (168.7920,69.3174) .. (170.4670,68.0870) -- cycle;
      \path[fill=c3d3d3d] (170.5060,68.1457) .. controls (171.3240,67.5242) and
        (172.8080,68.1457) .. (174.4620,70.0157) .. controls (176.2590,72.0294) and
        (176.8210,73.6112) .. (174.8930,74.7614) .. controls (173.3310,75.7170) and
        (172.8050,73.0359) .. (171.8210,71.9575) .. controls (170.2390,70.2546) and
        (168.8720,69.3653) .. (170.5060,68.1457) -- cycle;
      \path[fill=c444444] (170.5460,68.2043) .. controls (171.3440,67.5869) and
        (172.8090,68.2043) .. (174.4350,70.0428) .. controls (176.2020,72.0227) and
        (176.7220,73.5779) .. (174.8590,74.7087) .. controls (173.3360,75.6613) and
        (172.8160,73.0122) .. (171.8520,71.9520) .. controls (170.2960,70.2810) and
        (168.9530,69.4131) .. (170.5460,68.2043) -- cycle;
      \path[fill=c4c4c4c] (170.5860,68.2630) .. controls (171.3630,67.6494) and
        (172.8100,68.2630) .. (174.4090,70.0700) .. controls (176.1450,72.0160) and
        (176.6230,73.5445) .. (174.8250,74.6559) .. controls (173.3420,75.6054) and
        (172.8270,72.9885) .. (171.8830,71.9464) .. controls (170.3540,70.3075) and
        (169.0330,69.4610) .. (170.5860,68.2630) -- cycle;
      \path[fill=c545454] (170.6260,68.3216) .. controls (171.3830,67.7120) and
        (172.8110,68.3216) .. (174.3820,70.0971) .. controls (176.0880,72.0092) and
        (176.5240,73.5110) .. (174.7910,74.6032) .. controls (173.3470,75.5497) and
        (172.8380,72.9648) .. (171.9130,71.9409) .. controls (170.4110,70.3339) and
        (169.1140,69.5088) .. (170.6260,68.3216) -- cycle;
      \path[fill=c5b5b5b] (170.6660,68.3802) .. controls (171.4020,67.7747) and
        (172.8120,68.3802) .. (174.3550,70.1243) .. controls (176.0310,72.0025) and
        (176.4250,73.4777) .. (174.7570,74.5504) .. controls (173.3520,75.4939) and
        (172.8490,72.9411) .. (171.9440,71.9355) .. controls (170.4680,70.3603) and
        (169.1940,69.5566) .. (170.6660,68.3802) -- cycle;
      \path[fill=c636363] (170.7060,68.4389) .. controls (171.4220,67.8372) and
        (172.8140,68.4389) .. (174.3280,70.1514) .. controls (175.9740,71.9957) and
        (176.3260,73.4442) .. (174.7230,74.4977) .. controls (173.3580,75.4380) and
        (172.8600,72.9174) .. (171.9750,71.9300) .. controls (170.5260,70.3867) and
        (169.2740,69.6045) .. (170.7060,68.4389) -- cycle;
      \path[fill=c6b6b6b] (170.7460,68.4976) .. controls (171.4420,67.8998) and
        (172.8150,68.4976) .. (174.3020,70.1786) .. controls (175.9170,71.9890) and
        (176.2270,73.4109) .. (174.6890,74.4449) .. controls (173.3630,75.3823) and
        (172.8710,72.8937) .. (172.0060,71.9245) .. controls (170.5830,70.4131) and
        (169.3550,69.6523) .. (170.7460,68.4976) -- cycle;
      \path[fill=c727272] (170.7860,68.5562) .. controls (171.4610,67.9624) and
        (172.8160,68.5562) .. (174.2750,70.2057) .. controls (175.8600,71.9822) and
        (176.1280,73.3774) .. (174.6550,74.3922) .. controls (173.3680,75.3264) and
        (172.8810,72.8700) .. (172.0360,71.9189) .. controls (170.6400,70.4395) and
        (169.4350,69.7002) .. (170.7860,68.5562) -- cycle;
      \path[fill=c7a7a7a] (170.8260,68.6149) .. controls (171.4810,68.0251) and
        (172.8170,68.6149) .. (174.2480,70.2328) .. controls (175.8030,71.9755) and
        (176.0290,73.3441) .. (174.6210,74.3394) .. controls (173.3740,75.2707) and
        (172.8920,72.8463) .. (172.0670,71.9135) .. controls (170.6980,70.4659) and
        (169.5160,69.7480) .. (170.8260,68.6149) -- cycle;
      \path[fill=c828282] (170.8660,68.6735) .. controls (171.5000,68.0876) and
        (172.8180,68.6735) .. (174.2210,70.2600) .. controls (175.7460,71.9687) and
        (175.9300,73.3106) .. (174.5870,74.2867) .. controls (173.3790,75.2148) and
        (172.9030,72.8226) .. (172.0980,71.9080) .. controls (170.7550,70.4922) and
        (169.5960,69.7959) .. (170.8660,68.6735) -- cycle;
      \path[fill=c898989] (170.9050,68.7321) .. controls (171.5200,68.1502) and
        (172.8190,68.7321) .. (174.1950,70.2872) .. controls (175.6890,71.9619) and
        (175.8310,73.2773) .. (174.5530,74.2339) .. controls (173.3850,75.1591) and
        (172.9140,72.7989) .. (172.1290,71.9024) .. controls (170.8130,70.5186) and
        (169.6760,69.8438) .. (170.9050,68.7321) -- cycle;
      \path[fill=c919191] (170.9450,68.7908) .. controls (171.5400,68.2129) and
        (172.8200,68.7908) .. (174.1680,70.3143) .. controls (175.6320,71.9552) and
        (175.7320,73.2438) .. (174.5190,74.1812) .. controls (173.3900,75.1032) and
        (172.9250,72.7752) .. (172.1590,71.8969) .. controls (170.8700,70.5450) and
        (169.7570,69.8916) .. (170.9450,68.7908) -- cycle;
    \path[fill=c999999] (170.9850,68.8495) .. controls (171.5590,68.2755) and
      (172.8210,68.8495) .. (174.1410,70.3415) .. controls (175.5750,71.9485) and
      (175.6330,73.2104) .. (174.4850,74.1284) .. controls (173.3950,75.0475) and
      (172.9360,72.7515) .. (172.1900,71.8914) .. controls (170.9270,70.5714) and
      (169.8370,69.9395) .. (170.9850,68.8495) -- cycle;
    \path[fill=c666666] (137.5860,129.2460) .. controls (128.2260,129.6060) and
      (113.3940,103.3980) .. (113.0340,114.2700) .. controls (112.7460,123.4860) and
      (113.2500,123.3420) .. (113.2500,132.2700) .. controls (113.2500,138.2460) and
      (110.5140,138.6780) .. (104.6100,147.6780) .. controls (101.5860,152.4300) and
      (99.2101,157.5420) .. (97.3381,162.7260) .. controls (96.1861,165.8220) and
      (95.1061,169.0620) .. (94.2421,172.2300) .. controls (93.8821,173.8140) and
      (93.2341,175.4700) .. (92.8741,177.0540) .. controls (89.9221,187.9260) and
      (79.4101,201.3180) .. (76.9621,212.1900) .. controls (74.5141,222.9900) and
      (71.6341,229.9020) .. (71.9941,244.3740) .. controls (72.3541,258.8460) and
      (72.4981,254.6700) .. (76.8901,258.2700) .. controls (81.2101,261.8700) and
      (85.6741,265.2540) .. (92.5141,271.3020) .. controls (99.7141,277.5660) and
      (114.6900,288.5100) .. (116.7060,291.8940) .. controls (118.8660,295.3500) and
      (118.7940,303.1260) .. (117.4980,305.6460) .. controls (116.2020,308.0940) and
      (104.8980,309.4620) .. (104.9700,309.4620) .. controls (104.8980,309.4620) and
      (114.8340,323.1420) .. (116.7780,325.0860) .. controls (118.6500,326.9580) and
      (126.7140,335.9580) .. (159.5460,329.8380) .. controls (178.0500,326.3820) and
      (192.3780,316.0140) .. (202.7460,306.0060) .. controls (216.1380,292.9740) and
      (209.3700,289.2300) .. (211.0980,282.6780) .. controls (213.6180,273.2460) and
      (221.8260,269.7180) .. (223.6980,259.0620) .. controls (223.9140,257.5500) and
      (224.4180,256.3980) .. (225.7860,254.1660) .. controls (227.8740,250.9980) and
      (227.3700,244.7340) .. (227.3700,238.9740) .. controls (227.3700,223.9980) and
      (225.6420,208.7340) .. (222.1860,197.5020) .. controls (219.0180,186.9900) and
      (213.9780,179.6460) .. (209.6580,170.1420) .. controls (201.0180,151.2060) and
      (201.4500,142.7100) .. (193.7460,130.6140) .. controls (184.9620,116.6460) and
      (189.2820,107.3580) .. (177.6180,107.9340) .. controls (163.0740,108.7260) and
      (151.3380,128.6700) .. (137.5860,129.2460) -- cycle;
      \path[fill=c6d6d6d] (137.5690,130.3310) .. controls (128.2880,130.6840) and
        (113.8230,104.8200) .. (113.3270,115.1520) .. controls (113.0110,123.9070) and
        (113.4270,123.8300) .. (113.3570,132.3990) .. controls (113.2070,138.2690) and
        (110.4040,139.0360) .. (104.6310,147.8800) .. controls (101.6180,152.6430) and
        (99.3572,157.6920) .. (97.6384,162.7570) .. controls (96.6700,166.0100) and
        (96.0782,169.1790) .. (95.2084,172.2830) .. controls (94.7929,173.9250) and
        (93.7115,175.4700) .. (93.0229,177.3250) .. controls (89.7703,188.1050) and
        (79.4793,201.5070) .. (77.0626,212.2030) .. controls (74.5970,223.0120) and
        (71.6929,229.8970) .. (72.1014,244.3430) .. controls (72.4679,258.3430) and
        (72.4962,254.5200) .. (76.8752,258.1600) .. controls (81.1578,261.7760) and
        (85.7137,265.2290) .. (92.5373,271.2620) .. controls (99.7205,277.5100) and
        (114.7740,288.4780) .. (116.7800,291.8500) .. controls (118.9300,295.2940) and
        (118.9110,303.1680) .. (117.6240,305.6740) .. controls (116.3400,308.1120) and
        (105.0450,309.5480) .. (105.1160,309.5480) .. controls (105.0450,309.5480) and
        (114.9320,323.1150) .. (116.8730,325.0560) .. controls (118.7430,326.9260) and
        (126.7750,335.8640) .. (159.5510,329.7600) .. controls (178.0760,326.3050) and
        (192.6750,316.0070) .. (202.6290,305.9040) .. controls (215.4180,293.1420) and
        (208.7360,289.1300) .. (210.4550,282.6120) .. controls (212.9680,273.1960) and
        (221.7320,269.6520) .. (223.5980,259.0110) .. controls (223.8140,257.5040) and
        (224.3520,256.3570) .. (225.7110,254.1350) .. controls (227.8070,250.9330) and
        (227.2720,244.6880) .. (227.2930,238.9500) .. controls (227.3600,223.9240) and
        (225.5880,208.7160) .. (222.1340,197.5170) .. controls (218.9630,187.0340) and
        (213.9300,179.7000) .. (209.6190,170.2120) .. controls (200.9950,151.3060) and
        (201.3980,142.7970) .. (193.6870,130.7380) .. controls (185.1010,117.1250) and
        (189.0860,108.1550) .. (177.6760,108.7470) .. controls (163.2990,109.5730) and
        (151.2990,129.7630) .. (137.5690,130.3310) -- cycle;
      \path[fill=c757575] (137.5510,131.4160) .. controls (128.3500,131.7620) and
        (114.2520,106.2410) .. (113.6200,116.0330) .. controls (113.2770,124.3280) and
        (113.6040,124.3170) .. (113.4640,132.5270) .. controls (113.1630,138.2910) and
        (110.2940,139.3940) .. (104.6520,148.0820) .. controls (101.6500,152.8560) and
        (99.5043,157.8420) .. (97.9386,162.7880) .. controls (97.1539,166.1980) and
        (97.0503,169.2960) .. (96.1746,172.3360) .. controls (95.7036,174.0350) and
        (94.1889,175.4690) .. (93.1717,177.5950) .. controls (89.6185,188.2830) and
        (79.5484,201.6950) .. (77.1631,212.2150) .. controls (74.6798,223.0340) and
        (71.7516,229.8920) .. (72.2087,244.3110) .. controls (72.5817,257.8400) and
        (72.4942,254.3690) .. (76.8602,258.0500) .. controls (81.1054,261.6820) and
        (85.7533,265.2040) .. (92.5605,271.2210) .. controls (99.7268,277.4540) and
        (114.8580,288.4450) .. (116.8550,291.8060) .. controls (118.9950,295.2380) and
        (119.0280,303.2100) .. (117.7490,305.7020) .. controls (116.4780,308.1300) and
        (105.1910,309.6340) .. (105.2630,309.6340) .. controls (105.1910,309.6340) and
        (115.0300,323.0870) .. (116.9680,325.0260) .. controls (118.8350,326.8930) and
        (126.8350,335.7700) .. (159.5560,329.6810) .. controls (178.1010,326.2280) and
        (192.9710,316.0000) .. (202.5120,305.8020) .. controls (214.6980,293.3100) and
        (208.1030,289.0300) .. (209.8120,282.5450) .. controls (212.3180,273.1460) and
        (221.6380,269.5850) .. (223.4980,258.9590) .. controls (223.7130,257.4580) and
        (224.2860,256.3160) .. (225.6360,254.1040) .. controls (227.7400,250.8680) and
        (227.1740,244.6420) .. (227.2160,238.9260) .. controls (227.3490,223.8500) and
        (225.5350,208.6980) .. (222.0810,197.5320) .. controls (218.9080,187.0770) and
        (213.8820,179.7530) .. (209.5800,170.2820) .. controls (200.9710,151.4050) and
        (201.3460,142.8830) .. (193.6280,130.8620) .. controls (185.2390,117.6040) and
        (188.8890,108.9510) .. (177.7330,109.5610) .. controls (163.5230,110.4200) and
        (151.2590,130.8560) .. (137.5510,131.4160) -- cycle;
      \path[fill=c7c7c7c] (137.5340,132.5010) .. controls (128.4120,132.8390) and
        (114.6810,107.6630) .. (113.9120,116.9150) .. controls (113.5420,124.7480) and
        (113.7810,124.8050) .. (113.5710,132.6560) .. controls (113.1190,138.3130) and
        (110.1840,139.7520) .. (104.6720,148.2840) .. controls (101.6820,153.0690) and
        (99.6515,157.9920) .. (98.2389,162.8190) .. controls (97.6378,166.3850) and
        (98.0224,169.4130) .. (97.1409,172.3890) .. controls (96.6144,174.1450) and
        (94.6664,175.4690) .. (93.3204,177.8660) .. controls (89.4666,188.4620) and
        (79.6176,201.8830) .. (77.2636,212.2270) .. controls (74.7626,223.0560) and
        (71.8104,229.8870) .. (72.3160,244.2790) .. controls (72.6955,257.3370) and
        (72.4923,254.2180) .. (76.8452,257.9390) .. controls (81.0530,261.5880) and
        (85.7928,265.1780) .. (92.5838,271.1810) .. controls (99.7332,277.3980) and
        (114.9420,288.4130) .. (116.9290,291.7620) .. controls (119.0590,295.1820) and
        (119.1450,303.2520) .. (117.8750,305.7300) .. controls (116.6160,308.1470) and
        (105.3370,309.7200) .. (105.4090,309.7200) .. controls (105.3370,309.7200) and
        (115.1270,323.0600) .. (117.0630,324.9960) .. controls (118.9280,326.8600) and
        (126.8960,335.6760) .. (159.5620,329.6030) .. controls (178.1260,326.1500) and
        (193.2670,315.9930) .. (202.3950,305.7000) .. controls (213.9780,293.4780) and
        (207.4690,288.9300) .. (209.1690,282.4780) .. controls (211.6680,273.0960) and
        (221.5440,269.5190) .. (223.3980,258.9070) .. controls (223.6130,257.4120) and
        (224.2200,256.2750) .. (225.5610,254.0730) .. controls (227.6720,250.8020) and
        (227.0760,244.5960) .. (227.1390,238.9010) .. controls (227.3380,223.7760) and
        (225.4810,208.6800) .. (222.0290,197.5470) .. controls (218.8530,187.1200) and
        (213.8330,179.8060) .. (209.5420,170.3520) .. controls (200.9470,151.5040) and
        (201.2940,142.9690) .. (193.5690,130.9860) .. controls (185.3770,118.0830) and
        (188.6920,109.7470) .. (177.7900,110.3740) .. controls (163.7470,111.2670) and
        (151.2190,131.9490) .. (137.5340,132.5010) -- cycle;
      \path[fill=c848484] (137.5170,133.5860) .. controls (128.4740,133.9170) and
        (115.1100,109.0840) .. (114.2050,117.7960) .. controls (113.8070,125.1690) and
        (113.9580,125.2920) .. (113.6780,132.7840) .. controls (113.0760,138.3350) and
        (110.0740,140.1100) .. (104.6930,148.4860) .. controls (101.7130,153.2820) and
        (99.7986,158.1420) .. (98.5392,162.8500) .. controls (98.1218,166.5730) and
        (98.9946,169.5290) .. (98.1072,172.4420) .. controls (97.5252,174.2550) and
        (95.1438,175.4680) .. (93.4692,178.1360) .. controls (89.3148,188.6400) and
        (79.6867,202.0710) .. (77.3641,212.2390) .. controls (74.8455,223.0780) and
        (71.8691,229.8820) .. (72.4233,244.2470) .. controls (72.8093,256.8330) and
        (72.4903,254.0670) .. (76.8303,257.8290) .. controls (81.0006,261.4930) and
        (85.8324,265.1530) .. (92.6070,271.1400) .. controls (99.7395,277.3420) and
        (115.0260,288.3800) .. (117.0030,291.7170) .. controls (119.1230,295.1260) and
        (119.2620,303.2940) .. (118.0010,305.7580) .. controls (116.7540,308.1650) and
        (105.4840,309.8060) .. (105.5550,309.8060) .. controls (105.4840,309.8060) and
        (115.2250,323.0330) .. (117.1580,324.9660) .. controls (119.0200,326.8280) and
        (126.9560,335.5820) .. (159.5670,329.5240) .. controls (178.1510,326.0730) and
        (193.5640,315.9860) .. (202.2780,305.5980) .. controls (213.2580,293.6460) and
        (206.8350,288.8300) .. (208.5260,282.4120) .. controls (211.0180,273.0450) and
        (221.4500,269.4530) .. (223.2980,258.8550) .. controls (223.5130,257.3650) and
        (224.1540,256.2330) .. (225.4860,254.0420) .. controls (227.6050,250.7370) and
        (226.9780,244.5490) .. (227.0620,238.8770) .. controls (227.3280,223.7020) and
        (225.4270,208.6620) .. (221.9760,197.5620) .. controls (218.7970,187.1630) and
        (213.7850,179.8590) .. (209.5030,170.4210) .. controls (200.9240,151.6030) and
        (201.2410,143.0550) .. (193.5090,131.1090) .. controls (185.5160,118.5620) and
        (188.4960,110.5440) .. (177.8480,111.1870) .. controls (163.9720,112.1140) and
        (151.1790,133.0420) .. (137.5170,133.5860) -- cycle;
      \path[fill=c8c8c8c] (137.4990,134.6710) .. controls (128.5360,134.9940) and
        (115.5380,110.5060) .. (114.4980,118.6770) .. controls (114.0720,125.5890) and
        (114.1350,125.7790) .. (113.7850,132.9130) .. controls (113.0320,138.3580) and
        (109.9640,140.4670) .. (104.7140,148.6880) .. controls (101.7450,153.4950) and
        (99.9457,158.2920) .. (98.8394,162.8800) .. controls (98.6057,166.7610) and
        (99.9667,169.6460) .. (99.0734,172.4950) .. controls (98.4359,174.3650) and
        (95.6212,175.4670) .. (93.6180,178.4060) .. controls (89.1630,188.8190) and
        (79.7558,202.2590) .. (77.4646,212.2510) .. controls (74.9283,223.1000) and
        (71.9279,229.8770) .. (72.5305,244.2150) .. controls (72.9231,256.3300) and
        (72.4884,253.9160) .. (76.8153,257.7190) .. controls (80.9482,261.3990) and
        (85.8719,265.1280) .. (92.6302,271.0990) .. controls (99.7459,277.2860) and
        (115.1100,288.3470) .. (117.0770,291.6730) .. controls (119.1870,295.0700) and
        (119.3780,303.3360) .. (118.1260,305.7860) .. controls (116.8910,308.1830) and
        (105.6300,309.8910) .. (105.7010,309.8910) .. controls (105.6300,309.8910) and
        (115.3230,323.0050) .. (117.2530,324.9360) .. controls (119.1130,326.7950) and
        (127.0160,335.4880) .. (159.5720,329.4450) .. controls (178.1770,325.9950) and
        (193.8600,315.9780) .. (202.1610,305.4960) .. controls (212.5380,293.8140) and
        (206.2010,288.7300) .. (207.8820,282.3450) .. controls (210.3680,272.9950) and
        (221.3560,269.3860) .. (223.1980,258.8030) .. controls (223.4120,257.3190) and
        (224.0880,256.1920) .. (225.4110,254.0100) .. controls (227.5370,250.6720) and
        (226.8790,244.5030) .. (226.9840,238.8520) .. controls (227.3170,223.6280) and
        (225.3730,208.6440) .. (221.9230,197.5760) .. controls (218.7420,187.2060) and
        (213.7370,179.9120) .. (209.4640,170.4910) .. controls (200.9000,151.7020) and
        (201.1890,143.1410) .. (193.4500,131.2330) .. controls (185.6540,119.0410) and
        (188.2990,111.3400) .. (177.9050,112.0000) .. controls (164.1960,112.9610) and
        (151.1400,134.1340) .. (137.4990,134.6710) -- cycle;
      \path[fill=c939393] (137.4820,135.7560) .. controls (128.5980,136.0720) and
        (115.9670,111.9270) .. (114.7910,119.5590) .. controls (114.3370,126.0100) and
        (114.3120,126.2670) .. (113.8920,133.0410) .. controls (112.9880,138.3800) and
        (109.8550,140.8250) .. (104.7340,148.8900) .. controls (101.7770,153.7080) and
        (100.0930,158.4420) .. (99.1397,162.9110) .. controls (99.0896,166.9480) and
        (100.9390,169.7630) .. (100.0400,172.5480) .. controls (99.3467,174.4760) and
        (96.0986,175.4670) .. (93.7667,178.6770) .. controls (89.0111,188.9970) and
        (79.8250,202.4470) .. (77.5651,212.2640) .. controls (75.0111,223.1220) and
        (71.9866,229.8720) .. (72.6378,244.1840) .. controls (73.0369,255.8270) and
        (72.4864,253.7650) .. (76.8003,257.6080) .. controls (80.8958,261.3050) and
        (85.9115,265.1020) .. (92.6534,271.0590) .. controls (99.7522,277.2290) and
        (115.1940,288.3150) .. (117.1510,291.6290) .. controls (119.2510,295.0150) and
        (119.4950,303.3780) .. (118.2520,305.8140) .. controls (117.0290,308.2000) and
        (105.7760,309.9770) .. (105.8480,309.9770) .. controls (105.7760,309.9770) and
        (115.4210,322.9780) .. (117.3490,324.9060) .. controls (119.2050,326.7620) and
        (127.0770,335.3940) .. (159.5770,329.3670) .. controls (178.2020,325.9180) and
        (194.1560,315.9710) .. (202.0440,305.3940) .. controls (211.8180,293.9820) and
        (205.5670,288.6300) .. (207.2390,282.2780) .. controls (209.7180,272.9450) and
        (221.2620,269.3200) .. (223.0980,258.7510) .. controls (223.3120,257.2730) and
        (224.0220,256.1510) .. (225.3370,253.9790) .. controls (227.4700,250.6060) and
        (226.7810,244.4570) .. (226.9070,238.8280) .. controls (227.3070,223.5540) and
        (225.3200,208.6260) .. (221.8710,197.5910) .. controls (218.6870,187.2490) and
        (213.6880,179.9660) .. (209.4250,170.5610) .. controls (200.8770,151.8010) and
        (201.1370,143.2280) .. (193.3910,131.3570) .. controls (185.7930,119.5200) and
        (188.1020,112.1360) .. (177.9630,112.8130) .. controls (164.4200,113.8080) and
        (151.1000,135.2270) .. (137.4820,135.7560) -- cycle;
      \path[fill=c9b9b9b] (137.4640,136.8410) .. controls (128.6600,137.1490) and
        (116.3960,113.3490) .. (115.0830,120.4400) .. controls (114.6020,126.4310) and
        (114.4890,126.7540) .. (113.9990,133.1700) .. controls (112.9450,138.4020) and
        (109.7450,141.1830) .. (104.7550,149.0920) .. controls (101.8090,153.9210) and
        (100.2400,158.5920) .. (99.4399,162.9420) .. controls (99.5735,167.1360) and
        (101.9110,169.8790) .. (101.0060,172.6000) .. controls (100.2570,174.5860) and
        (96.5760,175.4660) .. (93.9155,178.9470) .. controls (88.8593,189.1760) and
        (79.8941,202.6350) .. (77.6656,212.2760) .. controls (75.0939,223.1440) and
        (72.0454,229.8670) .. (72.7451,244.1520) .. controls (73.1507,255.3230) and
        (72.4845,253.6140) .. (76.7854,257.4980) .. controls (80.8434,261.2110) and
        (85.9510,265.0770) .. (92.6766,271.0180) .. controls (99.7586,277.1730) and
        (115.2780,288.2820) .. (117.2250,291.5850) .. controls (119.3160,294.9590) and
        (119.6120,303.4200) .. (118.3770,305.8420) .. controls (117.1670,308.2180) and
        (105.9230,310.0630) .. (105.9940,310.0630) .. controls (105.9230,310.0630) and
        (115.5180,322.9500) .. (117.4440,324.8760) .. controls (119.2980,326.7300) and
        (127.1370,335.3000) .. (159.5820,329.2880) .. controls (178.2270,325.8410) and
        (194.4530,315.9640) .. (201.9270,305.2920) .. controls (211.0980,294.1500) and
        (204.9340,288.5300) .. (206.5960,282.2120) .. controls (209.0680,272.8950) and
        (221.1680,269.2530) .. (222.9980,258.6990) .. controls (223.2110,257.2260) and
        (223.9560,256.1100) .. (225.2620,253.9480) .. controls (227.4030,250.5410) and
        (226.6830,244.4100) .. (226.8300,238.8040) .. controls (227.2960,223.4800) and
        (225.2660,208.6080) .. (221.8180,197.6060) .. controls (218.6320,187.2920) and
        (213.6400,180.0190) .. (209.3860,170.6300) .. controls (200.8530,151.9000) and
        (201.0850,143.3140) .. (193.3320,131.4810) .. controls (185.9310,119.9990) and
        (187.9060,112.9330) .. (178.0200,113.6260) .. controls (164.6450,114.6550) and
        (151.0600,136.3200) .. (137.4640,136.8410) -- cycle;
      \path[fill=ca3a3a3] (137.4470,137.9260) .. controls (128.7210,138.2270) and
        (116.8250,114.7700) .. (115.3760,121.3220) .. controls (114.8670,126.8510) and
        (114.6660,127.2410) .. (114.1050,133.2980) .. controls (112.9010,138.4240) and
        (109.6350,141.5410) .. (104.7760,149.2940) .. controls (101.8410,154.1340) and
        (100.3870,158.7420) .. (99.7402,162.9730) .. controls (100.0570,167.3240) and
        (102.8830,169.9960) .. (101.9720,172.6530) .. controls (101.1680,174.6960) and
        (97.0534,175.4650) .. (94.0643,179.2170) .. controls (88.7075,189.3540) and
        (79.9633,202.8240) .. (77.7661,212.2880) .. controls (75.1768,223.1660) and
        (72.1041,229.8620) .. (72.8524,244.1200) .. controls (73.2645,254.8200) and
        (72.4825,253.4630) .. (76.7704,257.3880) .. controls (80.7911,261.1160) and
        (85.9906,265.0520) .. (92.6998,270.9780) .. controls (99.7649,277.1170) and
        (115.3610,288.2490) .. (117.2990,291.5400) .. controls (119.3800,294.9030) and
        (119.7290,303.4620) .. (118.5030,305.8700) .. controls (117.3050,308.2360) and
        (106.0690,310.1490) .. (106.1400,310.1490) .. controls (106.0690,310.1490) and
        (115.6160,322.9230) .. (117.5390,324.8460) .. controls (119.3900,326.6970) and
        (127.1980,335.2060) .. (159.5870,329.2100) .. controls (178.2530,325.7630) and
        (194.7490,315.9570) .. (201.8100,305.1900) .. controls (210.3780,294.3170) and
        (204.3000,288.4290) .. (205.9530,282.1450) .. controls (208.4170,272.8440) and
        (221.0740,269.1870) .. (222.8970,258.6470) .. controls (223.1110,257.1800) and
        (223.8890,256.0680) .. (225.1870,253.9170) .. controls (227.3350,250.4760) and
        (226.5850,244.3640) .. (226.7530,238.7790) .. controls (227.2850,223.4060) and
        (225.2120,208.5900) .. (221.7660,197.6210) .. controls (218.5770,187.3360) and
        (213.5920,180.0720) .. (209.3470,170.7000) .. controls (200.8290,151.9990) and
        (201.0330,143.4000) .. (193.2730,131.6040) .. controls (186.0690,120.4780) and
        (187.7090,113.7290) .. (178.0770,114.4390) .. controls (164.8690,115.5020) and
        (151.0210,137.4130) .. (137.4470,137.9260) -- cycle;
      \path[fill=caaaaaa] (137.4300,139.0110) .. controls (128.7830,139.3040) and
        (117.2540,116.1920) .. (115.6690,122.2030) .. controls (115.1330,127.2720) and
        (114.8430,127.7290) .. (114.2120,133.4260) .. controls (112.8570,138.4470) and
        (109.5250,141.8990) .. (104.7970,149.4960) .. controls (101.8720,154.3470) and
        (100.5340,158.8920) .. (100.0400,163.0040) .. controls (100.5410,167.5110) and
        (103.8550,170.1130) .. (102.9380,172.7060) .. controls (102.0790,174.8060) and
        (97.5308,175.4650) .. (94.2130,179.4880) .. controls (88.5556,189.5330) and
        (80.0324,203.0120) .. (77.8665,212.3000) .. controls (75.2596,223.1880) and
        (72.1629,229.8570) .. (72.9597,244.0880) .. controls (73.3783,254.3170) and
        (72.4805,253.3120) .. (76.7555,257.2770) .. controls (80.7387,261.0220) and
        (86.0302,265.0260) .. (92.7230,270.9370) .. controls (99.7713,277.0610) and
        (115.4450,288.2170) .. (117.3740,291.4960) .. controls (119.4440,294.8470) and
        (119.8460,303.5040) .. (118.6290,305.8980) .. controls (117.4430,308.2530) and
        (106.2150,310.2350) .. (106.2860,310.2350) .. controls (106.2150,310.2350) and
        (115.7140,322.8950) .. (117.6340,324.8160) .. controls (119.4830,326.6640) and
        (127.2580,335.1120) .. (159.5920,329.1310) .. controls (178.2780,325.6860) and
        (195.0450,315.9500) .. (201.6930,305.0880) .. controls (209.6580,294.4850) and
        (203.6660,288.3290) .. (205.3100,282.0780) .. controls (207.7670,272.7940) and
        (220.9800,269.1200) .. (222.7970,258.5950) .. controls (223.0110,257.1340) and
        (223.8230,256.0270) .. (225.1120,253.8860) .. controls (227.2680,250.4100) and
        (226.4870,244.3180) .. (226.6760,238.7550) .. controls (227.2750,223.3320) and
        (225.1580,208.5720) .. (221.7130,197.6360) .. controls (218.5210,187.3790) and
        (213.5440,180.1250) .. (209.3080,170.7700) .. controls (200.8060,152.0980) and
        (200.9800,143.4860) .. (193.2130,131.7280) .. controls (186.2080,120.9570) and
        (187.5120,114.5250) .. (178.1350,115.2520) .. controls (165.0930,116.3490) and
        (150.9810,138.5060) .. (137.4300,139.0110) -- cycle;
      \path[fill=cb2b2b2] (137.4120,140.0960) .. controls (128.8450,140.3820) and
        (117.6830,117.6130) .. (115.9620,123.0840) .. controls (115.3980,127.6920) and
        (115.0200,128.2160) .. (114.3190,133.5550) .. controls (112.8140,138.4690) and
        (109.4150,142.2560) .. (104.8170,149.6980) .. controls (101.9040,154.5600) and
        (100.6810,159.0420) .. (100.3410,163.0340) .. controls (101.0250,167.6990) and
        (104.8270,170.2290) .. (103.9050,172.7590) .. controls (102.9900,174.9160) and
        (98.0082,175.4640) .. (94.3618,179.7580) .. controls (88.4038,189.7110) and
        (80.1015,203.2000) .. (77.9670,212.3120) .. controls (75.3424,223.2100) and
        (72.2216,229.8520) .. (73.0669,244.0560) .. controls (73.4921,253.8130) and
        (72.4786,253.1610) .. (76.7405,257.1670) .. controls (80.6863,260.9280) and
        (86.0697,265.0010) .. (92.7462,270.8960) .. controls (99.7776,277.0050) and
        (115.5290,288.1840) .. (117.4480,291.4520) .. controls (119.5080,294.7910) and
        (119.9630,303.5460) .. (118.7540,305.9260) .. controls (117.5810,308.2710) and
        (106.3620,310.3200) .. (106.4330,310.3200) .. controls (106.3620,310.3200) and
        (115.8120,322.8680) .. (117.7290,324.7850) .. controls (119.5750,326.6320) and
        (127.3190,335.0180) .. (159.5970,329.0520) .. controls (178.3030,325.6080) and
        (195.3420,315.9420) .. (201.5760,304.9860) .. controls (208.9380,294.6530) and
        (203.0320,288.2290) .. (204.6670,282.0120) .. controls (207.1170,272.7440) and
        (220.8860,269.0540) .. (222.6970,258.5430) .. controls (222.9100,257.0870) and
        (223.7570,255.9860) .. (225.0370,253.8540) .. controls (227.2010,250.3450) and
        (226.3890,244.2710) .. (226.5990,238.7300) .. controls (227.2640,223.2580) and
        (225.1050,208.5530) .. (221.6610,197.6500) .. controls (218.4660,187.4220) and
        (213.4950,180.1780) .. (209.2690,170.8390) .. controls (200.7820,152.1970) and
        (200.9280,143.5720) .. (193.1540,131.8520) .. controls (186.3460,121.4360) and
        (187.3160,115.3210) .. (178.1920,116.0650) .. controls (165.3180,117.1960) and
        (150.9410,139.5980) .. (137.4120,140.0960) -- cycle;
      \path[fill=cbababa] (137.3950,141.1810) .. controls (128.9070,141.4600) and
        (118.1120,119.0350) .. (116.2540,123.9660) .. controls (115.6630,128.1130) and
        (115.1970,128.7030) .. (114.4260,133.6830) .. controls (112.7700,138.4910) and
        (109.3050,142.6140) .. (104.8380,149.9000) .. controls (101.9360,154.7730) and
        (100.8280,159.1920) .. (100.6410,163.0650) .. controls (101.5090,167.8870) and
        (105.7990,170.3460) .. (104.8710,172.8120) .. controls (103.9000,175.0270) and
        (98.4856,175.4630) .. (94.5106,180.0280) .. controls (88.2520,189.8890) and
        (80.1707,203.3880) .. (78.0675,212.3250) .. controls (75.4253,223.2320) and
        (72.2804,229.8470) .. (73.1742,244.0250) .. controls (73.6059,253.3100) and
        (72.4766,253.0110) .. (76.7255,257.0570) .. controls (80.6339,260.8340) and
        (86.1093,264.9760) .. (92.7694,270.8560) .. controls (99.7840,276.9480) and
        (115.6130,288.1520) .. (117.5220,291.4080) .. controls (119.5720,294.7350) and
        (120.0800,303.5880) .. (118.8800,305.9540) .. controls (117.7190,308.2880) and
        (106.5080,310.4060) .. (106.5790,310.4060) .. controls (106.5080,310.4060) and
        (115.9090,322.8410) .. (117.8240,324.7550) .. controls (119.6680,326.5990) and
        (127.3790,334.9240) .. (159.6020,328.9740) .. controls (178.3290,325.5310) and
        (195.6380,315.9350) .. (201.4590,304.8840) .. controls (208.2180,294.8210) and
        (202.3980,288.1290) .. (204.0240,281.9450) .. controls (206.4670,272.6930) and
        (220.7920,268.9870) .. (222.5970,258.4920) .. controls (222.8100,257.0410) and
        (223.6910,255.9450) .. (224.9620,253.8230) .. controls (227.1330,250.2800) and
        (226.2910,244.2250) .. (226.5220,238.7060) .. controls (227.2540,223.1840) and
        (225.0510,208.5350) .. (221.6080,197.6650) .. controls (218.4110,187.4650) and
        (213.4470,180.2320) .. (209.2300,170.9090) .. controls (200.7590,152.2970) and
        (200.8760,143.6590) .. (193.0950,131.9760) .. controls (186.4850,121.9150) and
        (187.1190,116.1180) .. (178.2500,116.8780) .. controls (165.5420,118.0430) and
        (150.9020,140.6910) .. (137.3950,141.1810) -- cycle;
      \path[fill=cc1c1c1] (137.3770,142.2660) .. controls (128.9690,142.5370) and
        (118.5400,120.4560) .. (116.5470,124.8470) .. controls (115.9280,128.5340) and
        (115.3740,129.1910) .. (114.5330,133.8120) .. controls (112.7260,138.5130) and
        (109.1950,142.9720) .. (104.8590,150.1020) .. controls (101.9680,154.9860) and
        (100.9750,159.3420) .. (100.9410,163.0960) .. controls (101.9930,168.0740) and
        (106.7710,170.4630) .. (105.8370,172.8650) .. controls (104.8110,175.1370) and
        (98.9630,175.4630) .. (94.6593,180.2990) .. controls (88.1001,190.0680) and
        (80.2398,203.5760) .. (78.1680,212.3370) .. controls (75.5081,223.2540) and
        (72.3391,229.8420) .. (73.2815,243.9930) .. controls (73.7197,252.8070) and
        (72.4747,252.8600) .. (76.7106,256.9460) .. controls (80.5815,260.7390) and
        (86.1488,264.9500) .. (92.7926,270.8150) .. controls (99.7903,276.8920) and
        (115.6970,288.1190) .. (117.5960,291.3640) .. controls (119.6370,294.6790) and
        (120.1960,303.6300) .. (119.0050,305.9820) .. controls (117.8560,308.3060) and
        (106.6540,310.4920) .. (106.7250,310.4920) .. controls (106.6540,310.4920) and
        (116.0070,322.8130) .. (117.9190,324.7250) .. controls (119.7600,326.5660) and
        (127.4400,334.8310) .. (159.6070,328.8950) .. controls (178.3540,325.4540) and
        (195.9340,315.9280) .. (201.3420,304.7820) .. controls (207.4980,294.9890) and
        (201.7650,288.0290) .. (203.3800,281.8780) .. controls (205.8170,272.6430) and
        (220.6980,268.9210) .. (222.4970,258.4400) .. controls (222.7090,256.9950) and
        (223.6250,255.9030) .. (224.8870,253.7920) .. controls (227.0660,250.2140) and
        (226.1920,244.1790) .. (226.4440,238.6820) .. controls (227.2430,223.1100) and
        (224.9970,208.5170) .. (221.5560,197.6800) .. controls (218.3560,187.5080) and
        (213.3990,180.2850) .. (209.1910,170.9790) .. controls (200.7350,152.3960) and
        (200.8240,143.7450) .. (193.0360,132.0990) .. controls (186.6230,122.3940) and
        (186.9220,116.9140) .. (178.3070,117.6910) .. controls (165.7660,118.8900) and
        (150.8620,141.7840) .. (137.3770,142.2660) -- cycle;
      \path[fill=cc9c9c9] (137.3600,143.3510) .. controls (129.0310,143.6150) and
        (118.9690,121.8780) .. (116.8400,125.7290) .. controls (116.1930,128.9540) and
        (115.5510,129.6780) .. (114.6400,133.9400) .. controls (112.6830,138.5360) and
        (109.0850,143.3300) .. (104.8790,150.3040) .. controls (102.0000,155.1990) and
        (101.1230,159.4920) .. (101.2410,163.1270) .. controls (102.4770,168.2620) and
        (107.7430,170.5800) .. (106.8030,172.9170) .. controls (105.7220,175.2470) and
        (99.4404,175.4620) .. (94.8081,180.5690) .. controls (87.9483,190.2460) and
        (80.3090,203.7640) .. (78.2685,212.3490) .. controls (75.5909,223.2760) and
        (72.3979,229.8370) .. (73.3888,243.9610) .. controls (73.8335,252.3040) and
        (72.4727,252.7090) .. (76.6956,256.8360) .. controls (80.5291,260.6450) and
        (86.1884,264.9250) .. (92.8158,270.7750) .. controls (99.7967,276.8360) and
        (115.7810,288.0860) .. (117.6700,291.3190) .. controls (119.7010,294.6230) and
        (120.3130,303.6720) .. (119.1310,306.0100) .. controls (117.9940,308.3240) and
        (106.8010,310.5780) .. (106.8710,310.5780) .. controls (106.8010,310.5780) and
        (116.1050,322.7860) .. (118.0140,324.6950) .. controls (119.8530,326.5340) and
        (127.5000,334.7370) .. (159.6130,328.8160) .. controls (178.3790,325.3760) and
        (196.2310,315.9210) .. (201.2250,304.6800) .. controls (206.7780,295.1570) and
        (201.1310,287.9290) .. (202.7370,281.8120) .. controls (205.1670,272.5930) and
        (220.6040,268.8550) .. (222.3970,258.3880) .. controls (222.6090,256.9490) and
        (223.5590,255.8620) .. (224.8120,253.7610) .. controls (226.9990,250.1490) and
        (226.0940,244.1330) .. (226.3670,238.6570) .. controls (227.2320,223.0360) and
        (224.9430,208.4990) .. (221.5030,197.6950) .. controls (218.3010,187.5510) and
        (213.3500,180.3380) .. (209.1530,171.0490) .. controls (200.7110,152.4950) and
        (200.7720,143.8310) .. (192.9770,132.2230) .. controls (186.7610,122.8730) and
        (186.7260,117.7100) .. (178.3640,118.5040) .. controls (165.9910,119.7370) and
        (150.8220,142.8770) .. (137.3600,143.3510) -- cycle;
      \path[fill=cd1d1d1] (137.3430,144.4360) .. controls (129.0930,144.6920) and
        (119.3980,123.2990) .. (117.1330,126.6100) .. controls (116.4580,129.3750) and
        (115.7280,130.1650) .. (114.7470,134.0690) .. controls (112.6390,138.5580) and
        (108.9750,143.6880) .. (104.9000,150.5060) .. controls (102.0310,155.4120) and
        (101.2700,159.6420) .. (101.5420,163.1580) .. controls (102.9610,168.4500) and
        (108.7160,170.6960) .. (107.7700,172.9700) .. controls (106.6330,175.3570) and
        (99.9178,175.4610) .. (94.9568,180.8390) .. controls (87.7964,190.4250) and
        (80.3781,203.9530) .. (78.3690,212.3610) .. controls (75.6738,223.2980) and
        (72.4566,229.8320) .. (73.4961,243.9290) .. controls (73.9473,251.8000) and
        (72.4708,252.5580) .. (76.6806,256.7260) .. controls (80.4767,260.5510) and
        (86.2280,264.9000) .. (92.8390,270.7340) .. controls (99.8030,276.7800) and
        (115.8650,288.0540) .. (117.7440,291.2750) .. controls (119.7650,294.5670) and
        (120.4300,303.7140) .. (119.2570,306.0380) .. controls (118.1320,308.3410) and
        (106.9470,310.6640) .. (107.0180,310.6640) .. controls (106.9470,310.6640) and
        (116.2030,322.7580) .. (118.1090,324.6650) .. controls (119.9450,326.5010) and
        (127.5610,334.6430) .. (159.6180,328.7380) .. controls (178.4040,325.2990) and
        (196.5270,315.9140) .. (201.1080,304.5780) .. controls (206.0580,295.3250) and
        (200.4970,287.8290) .. (202.0940,281.7450) .. controls (204.5170,272.5430) and
        (220.5100,268.7880) .. (222.2970,258.3360) .. controls (222.5090,256.9020) and
        (223.4930,255.8210) .. (224.7370,253.7300) .. controls (226.9310,250.0840) and
        (225.9960,244.0860) .. (226.2900,238.6330) .. controls (227.2220,222.9620) and
        (224.8900,208.4810) .. (221.4510,197.7100) .. controls (218.2450,187.5950) and
        (213.3020,180.3910) .. (209.1140,171.1180) .. controls (200.6880,152.5940) and
        (200.7190,143.9170) .. (192.9170,132.3470) .. controls (186.9000,123.3520) and
        (186.5290,118.5070) .. (178.4220,119.3170) .. controls (166.2150,120.5840) and
        (150.7820,143.9700) .. (137.3430,144.4360) -- cycle;
      \path[fill=cd8d8d8] (137.3250,145.5210) .. controls (129.1550,145.7700) and
        (119.8270,124.7210) .. (117.4250,127.4910) .. controls (116.7230,129.7950) and
        (115.9040,130.6530) .. (114.8540,134.1970) .. controls (112.5950,138.5800) and
        (108.8650,144.0450) .. (104.9210,150.7080) .. controls (102.0630,155.6250) and
        (101.4170,159.7920) .. (101.8420,163.1880) .. controls (103.4450,168.6370) and
        (109.6880,170.8130) .. (108.7360,173.0230) .. controls (107.5430,175.4670) and
        (100.3950,175.4610) .. (95.1056,181.1100) .. controls (87.6446,190.6030) and
        (80.4472,204.1410) .. (78.4695,212.3730) .. controls (75.7566,223.3200) and
        (72.5154,229.8270) .. (73.6033,243.8970) .. controls (74.0611,251.2970) and
        (72.4688,252.4070) .. (76.6657,256.6150) .. controls (80.4244,260.4570) and
        (86.2675,264.8740) .. (92.8622,270.6930) .. controls (99.8094,276.7240) and
        (115.9490,288.0210) .. (117.8180,291.2310) .. controls (119.8290,294.5110) and
        (120.5470,303.7560) .. (119.3820,306.0660) .. controls (118.2700,308.3590) and
        (107.0930,310.7490) .. (107.1640,310.7490) .. controls (107.0930,310.7490) and
        (116.3000,322.7310) .. (118.2040,324.6350) .. controls (120.0380,326.4690) and
        (127.6210,334.5490) .. (159.6230,328.6590) .. controls (178.4300,325.2210) and
        (196.8230,315.9060) .. (200.9910,304.4760) .. controls (205.3370,295.4930) and
        (199.8630,287.7290) .. (201.4510,281.6780) .. controls (203.8670,272.4920) and
        (220.4150,268.7220) .. (222.1970,258.2840) .. controls (222.4080,256.8560) and
        (223.4270,255.7800) .. (224.6620,253.6980) .. controls (226.8640,250.0180) and
        (225.8980,244.0400) .. (226.2130,238.6080) .. controls (227.2110,222.8880) and
        (224.8360,208.4630) .. (221.3980,197.7240) .. controls (218.1900,187.6380) and
        (213.2540,180.4440) .. (209.0750,171.1880) .. controls (200.6640,152.6930) and
        (200.6670,144.0030) .. (192.8580,132.4710) .. controls (187.0380,123.8310) and
        (186.3320,119.3030) .. (178.4790,120.1300) .. controls (166.4390,121.4310) and
        (150.7430,145.0620) .. (137.3250,145.5210) -- cycle;
      \path[fill=ce0e0e0] (137.3080,146.6060) .. controls (129.2170,146.8470) and
        (120.2560,126.1420) .. (117.7180,128.3730) .. controls (116.9890,130.2160) and
        (116.0810,131.1400) .. (114.9610,134.3260) .. controls (112.5520,138.6020) and
        (108.7550,144.4030) .. (104.9410,150.9100) .. controls (102.0950,155.8380) and
        (101.5640,159.9420) .. (102.1420,163.2190) .. controls (103.9290,168.8250) and
        (110.6600,170.9300) .. (109.7020,173.0760) .. controls (108.4540,175.5780) and
        (100.8730,175.4600) .. (95.2544,181.3800) .. controls (87.4928,190.7820) and
        (80.5164,204.3290) .. (78.5700,212.3860) .. controls (75.8394,223.3420) and
        (72.5741,229.8220) .. (73.7106,243.8660) .. controls (74.1749,250.7940) and
        (72.4669,252.2560) .. (76.6507,256.5050) .. controls (80.3720,260.3620) and
        (86.3071,264.8490) .. (92.8854,270.6530) .. controls (99.8157,276.6680) and
        (116.0330,287.9890) .. (117.8930,291.1870) .. controls (119.8930,294.4550) and
        (120.6640,303.7980) .. (119.5080,306.0940) .. controls (118.4080,308.3770) and
        (107.2400,310.8350) .. (107.3100,310.8350) .. controls (107.2400,310.8350) and
        (116.3980,322.7030) .. (118.2990,324.6050) .. controls (120.1300,326.4360) and
        (127.6810,334.4550) .. (159.6280,328.5810) .. controls (178.4550,325.1440) and
        (197.1200,315.8990) .. (200.8740,304.3740) .. controls (204.6170,295.6600) and
        (199.2290,287.6280) .. (200.8080,281.6120) .. controls (203.2170,272.4420) and
        (220.3210,268.6550) .. (222.0970,258.2320) .. controls (222.3080,256.8100) and
        (223.3610,255.7380) .. (224.5870,253.6670) .. controls (226.7970,249.9530) and
        (225.8000,243.9940) .. (226.1360,238.5840) .. controls (227.2010,222.8140) and
        (224.7820,208.4450) .. (221.3450,197.7390) .. controls (218.1350,187.6810) and
        (213.2050,180.4980) .. (209.0360,171.2580) .. controls (200.6410,152.7920) and
        (200.6150,144.0900) .. (192.7990,132.5940) .. controls (187.1770,124.3100) and
        (186.1360,120.0990) .. (178.5370,120.9430) .. controls (166.6640,122.2780) and
        (150.7030,146.1550) .. (137.3080,146.6060) -- cycle;
      \path[fill=ce8e8e8] (137.2900,147.6910) .. controls (129.2790,147.9250) and
        (120.6850,127.5640) .. (118.0110,129.2540) .. controls (117.2540,130.6370) and
        (116.2580,131.6270) .. (115.0680,134.4540) .. controls (112.5080,138.6250) and
        (108.6450,144.7610) .. (104.9620,151.1120) .. controls (102.1270,156.0510) and
        (101.7110,160.0920) .. (102.4420,163.2500) .. controls (104.4130,169.0120) and
        (111.6320,171.0460) .. (110.6680,173.1290) .. controls (109.3650,175.6880) and
        (101.3500,175.4590) .. (95.4031,181.6500) .. controls (87.3409,190.9600) and
        (80.5855,204.5170) .. (78.6705,212.3980) .. controls (75.9222,223.3640) and
        (72.6329,229.8170) .. (73.8179,243.8340) .. controls (74.2887,250.2900) and
        (72.4649,252.1050) .. (76.6357,256.3950) .. controls (80.3196,260.2680) and
        (86.3466,264.8230) .. (92.9086,270.6120) .. controls (99.8221,276.6110) and
        (116.1170,287.9560) .. (117.9670,291.1420) .. controls (119.9580,294.3990) and
        (120.7810,303.8400) .. (119.6330,306.1220) .. controls (118.5460,308.3940) and
        (107.3860,310.9210) .. (107.4560,310.9210) .. controls (107.3860,310.9210) and
        (116.4960,322.6760) .. (118.3940,324.5750) .. controls (120.2230,326.4030) and
        (127.7420,334.3610) .. (159.6330,328.5020) .. controls (178.4800,325.0670) and
        (197.4160,315.8920) .. (200.7570,304.2720) .. controls (203.8970,295.8280) and
        (198.5960,287.5280) .. (200.1650,281.5450) .. controls (202.5670,272.3920) and
        (220.2270,268.5890) .. (221.9970,258.1800) .. controls (222.2070,256.7630) and
        (223.2950,255.6970) .. (224.5120,253.6360) .. controls (226.7290,249.8880) and
        (225.7020,243.9470) .. (226.0590,238.5600) .. controls (227.1900,222.7400) and
        (224.7280,208.4270) .. (221.2930,197.7540) .. controls (218.0800,187.7240) and
        (213.1570,180.5510) .. (208.9970,171.3270) .. controls (200.6170,152.8910) and
        (200.5630,144.1760) .. (192.7400,132.7180) .. controls (187.3150,124.7890) and
        (185.9390,120.8960) .. (178.5940,121.7560) .. controls (166.8880,123.1250) and
        (150.6630,147.2480) .. (137.2900,147.6910) -- cycle;
      \path[fill=cefefef] (137.2730,148.7760) .. controls (129.3400,149.0020) and
        (121.1140,128.9850) .. (118.3040,130.1360) .. controls (117.5190,131.0570) and
        (116.4350,132.1150) .. (115.1740,134.5830) .. controls (112.4650,138.6470) and
        (108.5350,145.1190) .. (104.9830,151.3140) .. controls (102.1590,156.2640) and
        (101.8580,160.2420) .. (102.7430,163.2810) .. controls (104.8960,169.2000) and
        (112.6040,171.1630) .. (111.6350,173.1820) .. controls (110.2760,175.7980) and
        (101.8270,175.4590) .. (95.5519,181.9210) .. controls (87.1891,191.1390) and
        (80.6547,204.7050) .. (78.7710,212.4100) .. controls (76.0051,223.3860) and
        (72.6916,229.8120) .. (73.9252,243.8020) .. controls (74.4025,249.7870) and
        (72.4630,251.9540) .. (76.6208,256.2840) .. controls (80.2672,260.1740) and
        (86.3862,264.7980) .. (92.9318,270.5720) .. controls (99.8284,276.5550) and
        (116.2000,287.9230) .. (118.0410,291.0980) .. controls (120.0220,294.3430) and
        (120.8980,303.8810) .. (119.7590,306.1500) .. controls (118.6840,308.4120) and
        (107.5330,311.0070) .. (107.6030,311.0070) .. controls (107.5330,311.0070) and
        (116.5940,322.6480) .. (118.4890,324.5450) .. controls (120.3150,326.3710) and
        (127.8020,334.2670) .. (159.6380,328.4230) .. controls (178.5060,324.9890) and
        (197.7130,315.8850) .. (200.6400,304.1690) .. controls (203.1770,295.9960) and
        (197.9620,287.4280) .. (199.5220,281.4780) .. controls (201.9160,272.3420) and
        (220.1330,268.5220) .. (221.8960,258.1280) .. controls (222.1070,256.7170) and
        (223.2280,255.6560) .. (224.4370,253.6050) .. controls (226.6620,249.8220) and
        (225.6040,243.9010) .. (225.9820,238.5350) .. controls (227.1790,222.6660) and
        (224.6750,208.4090) .. (221.2400,197.7690) .. controls (218.0250,187.7670) and
        (213.1090,180.6040) .. (208.9580,171.3970) .. controls (200.5930,152.9900) and
        (200.5110,144.2620) .. (192.6810,132.8420) .. controls (187.4530,125.2680) and
        (185.7430,121.6920) .. (178.6510,122.5690) .. controls (167.1130,123.9720) and
        (150.6240,148.3410) .. (137.2730,148.7760) -- cycle;
      \path[fill=cf7f7f7] (137.2560,149.8610) .. controls (129.4020,150.0800) and
        (121.5420,130.4070) .. (118.5960,131.0170) .. controls (117.7840,131.4780) and
        (116.6120,132.6020) .. (115.2810,134.7110) .. controls (112.4210,138.6690) and
        (108.4250,145.4770) .. (105.0040,151.5160) .. controls (102.1900,156.4770) and
        (102.0050,160.3920) .. (103.0430,163.3120) .. controls (105.3800,169.3880) and
        (113.5760,171.2800) .. (112.6010,173.2350) .. controls (111.1860,175.9080) and
        (102.3050,175.4580) .. (95.7007,182.1910) .. controls (87.0373,191.3170) and
        (80.7238,204.8930) .. (78.8714,212.4220) .. controls (76.0879,223.4080) and
        (72.7504,229.8070) .. (74.0325,243.7700) .. controls (74.5163,249.2840) and
        (72.4610,251.8030) .. (76.6058,256.1740) .. controls (80.2148,260.0800) and
        (86.4258,264.7730) .. (92.9550,270.5310) .. controls (99.8348,276.4990) and
        (116.2840,287.8910) .. (118.1150,291.0540) .. controls (120.0860,294.2870) and
        (121.0140,303.9230) .. (119.8850,306.1780) .. controls (118.8210,308.4300) and
        (107.6790,311.0930) .. (107.7490,311.0930) .. controls (107.6790,311.0930) and
        (116.6910,322.6210) .. (118.5840,324.5150) .. controls (120.4080,326.3380) and
        (127.8630,334.1730) .. (159.6430,328.3450) .. controls (178.5310,324.9120) and
        (198.0090,315.8780) .. (200.5230,304.0670) .. controls (202.4570,296.1640) and
        (197.3280,287.3280) .. (198.8780,281.4120) .. controls (201.2660,272.2910) and
        (220.0390,268.4560) .. (221.7960,258.0760) .. controls (222.0070,256.6710) and
        (223.1620,255.6150) .. (224.3620,253.5740) .. controls (226.5950,249.7570) and
        (225.5050,243.8550) .. (225.9040,238.5110) .. controls (227.1690,222.5920) and
        (224.6210,208.3910) .. (221.1880,197.7840) .. controls (217.9690,187.8100) and
        (213.0610,180.6570) .. (208.9190,171.4670) .. controls (200.5700,153.0890) and
        (200.4580,144.3480) .. (192.6210,132.9660) .. controls (187.5920,125.7470) and
        (185.5460,122.4880) .. (178.7090,123.3820) .. controls (167.3370,124.8180) and
        (150.5840,149.4340) .. (137.2560,149.8610) -- cycle;
    \path[fill=cffffff] (137.2380,150.9460) .. controls (129.4640,151.1570) and
      (121.9710,131.8280) .. (118.8890,131.8980) .. controls (118.0490,131.8980) and
      (116.7890,133.0890) .. (115.3880,134.8390) .. controls (112.3770,138.6910) and
      (108.3150,145.8340) .. (105.0240,151.7170) .. controls (102.2220,156.6890) and
      (102.1520,160.5410) .. (103.3430,163.3420) .. controls (105.8640,169.5750) and
      (114.5480,171.3960) .. (113.5670,173.2870) .. controls (112.0970,176.0180) and
      (102.7820,175.4570) .. (95.8494,182.4610) .. controls (86.8854,191.4950) and
      (80.7929,205.0810) .. (78.9719,212.4340) .. controls (76.1707,223.4290) and
      (72.8091,229.8020) .. (74.1397,243.7380) .. controls (74.6301,248.7800) and
      (72.4590,251.6520) .. (76.5908,256.0630) .. controls (80.1624,259.9850) and
      (86.4653,264.7470) .. (92.9782,270.4900) .. controls (99.8411,276.4430) and
      (116.3680,287.8580) .. (118.1890,291.0100) .. controls (120.1500,294.2310) and
      (121.1310,303.9650) .. (120.0100,306.2060) .. controls (118.9590,308.4470) and
      (107.8250,311.1780) .. (107.8950,311.1780) .. controls (107.8250,311.1780) and
      (116.7890,322.5940) .. (118.6790,324.4850) .. controls (120.5000,326.3050) and
      (127.9230,334.0790) .. (159.6480,328.2660) .. controls (178.5560,324.8350) and
      (198.3050,315.8710) .. (200.4060,303.9650) .. controls (201.7370,296.3320) and
      (196.6940,287.2280) .. (198.2350,281.3450) .. controls (200.6160,272.2410) and
      (219.9450,268.3890) .. (221.6960,258.0240) .. controls (221.9060,256.6240) and
      (223.0960,255.5730) .. (224.2870,253.5420) .. controls (226.5270,249.6910) and
      (225.4070,243.8080) .. (225.8270,238.4860) .. controls (227.1580,222.5180) and
      (224.5670,208.3720) .. (221.1350,197.7980) .. controls (217.9140,187.8530) and
      (213.0120,180.7100) .. (208.8800,171.5360) .. controls (200.5460,153.1880) and
      (200.4060,144.4340) .. (192.5620,133.0890) .. controls (187.7300,126.2260) and
      (185.3490,123.2840) .. (178.7660,124.1950) .. controls (167.5610,125.6650) and
      (150.5440,150.5260) .. (137.2380,150.9460) -- cycle;
    \path[fill=c995900] (142.5540,79.8545) .. controls (147.8100,79.2065) and
      (155.0100,80.5024) .. (158.3220,82.9504) .. controls (161.4180,85.2545) and
      (163.5780,86.4785) .. (166.3860,87.4145) .. controls (175.8180,90.5105) and
      (188.2020,91.9504) .. (187.6260,100.4460) .. controls (186.9780,110.5980) and
      (184.0260,115.1340) .. (175.6020,117.7980) .. controls (168.8340,119.8860) and
      (156.7380,131.6220) .. (147.3780,131.6220) .. controls (143.2020,131.6220) and
      (137.3700,131.8380) .. (133.9860,130.6140) .. controls (130.7460,129.4620) and
      (126.2100,123.9900) .. (120.8820,119.5980) .. controls (115.5540,115.2780) and
      (110.5860,110.6700) .. (110.4420,104.6220) .. controls (110.2260,98.2144) and
      (114.4020,96.1265) .. (120.3060,91.0145) .. controls (123.4020,88.2784) and
      (129.0180,83.7425) .. (132.9060,81.7264) .. controls (136.5060,79.9265) and
      (138.8100,80.2864) .. (142.5540,79.8545) -- cycle;
      \path[fill=c9e5e00] (142.6640,79.9478) .. controls (147.8900,79.3035) and
        (155.0480,80.5920) .. (158.3410,83.0259) .. controls (161.4190,85.3167) and
        (163.5670,86.5369) .. (166.3590,87.4643) .. controls (175.7330,90.5425) and
        (188.0870,91.9933) .. (187.5180,100.4370) .. controls (186.8770,110.5280) and
        (183.8490,115.0180) .. (175.4770,117.6610) .. controls (168.7540,119.7340) and
        (156.7090,131.2490) .. (147.4060,131.3000) .. controls (143.1970,131.3320) and
        (137.4810,131.5340) .. (134.1200,130.3230) .. controls (130.9050,129.1810) and
        (126.3820,123.7150) .. (121.0880,119.3550) .. controls (115.7970,115.0630) and
        (110.8290,110.5800) .. (110.7720,104.5890) .. controls (110.6170,98.3359) and
        (114.6830,96.1772) .. (120.5440,91.1009) .. controls (123.6160,88.3902) and
        (129.1390,83.7975) .. (133.0080,81.7675) .. controls (136.5710,79.9652) and
        (138.9410,80.3772) .. (142.6640,79.9478) -- cycle;
      \path[fill=ca36400] (142.7730,80.0410) .. controls (147.9690,79.4005) and
        (155.0860,80.6816) .. (158.3600,83.1014) .. controls (161.4200,85.3790) and
        (163.5560,86.5952) .. (166.3310,87.5142) .. controls (175.6480,90.5745) and
        (187.9730,92.0361) .. (187.4100,100.4280) .. controls (186.7760,110.4570) and
        (183.6730,114.9020) .. (175.3520,117.5230) .. controls (168.6750,119.5810) and
        (156.6800,130.8760) .. (147.4340,130.9780) .. controls (143.1910,131.0420) and
        (137.5920,131.2300) .. (134.2530,130.0320) .. controls (131.0630,128.9000) and
        (126.5540,123.4400) .. (121.2930,119.1110) .. controls (116.0390,114.8470) and
        (111.0710,110.4900) .. (111.1010,104.5560) .. controls (111.0080,98.4575) and
        (114.9640,96.2279) .. (120.7810,91.1873) .. controls (123.8290,88.5020) and
        (129.2590,83.8524) .. (133.1090,81.8087) .. controls (136.6360,80.0039) and
        (139.0720,80.4680) .. (142.7730,80.0410) -- cycle;
      \path[fill=ca86a00] (142.8830,80.1344) .. controls (148.0480,79.4976) and
        (155.1240,80.7712) .. (158.3790,83.1769) .. controls (161.4210,85.4412) and
        (163.5440,86.6537) .. (166.3040,87.5640) .. controls (175.5630,90.6065) and
        (187.8580,92.0790) .. (187.3010,100.4190) .. controls (186.6740,110.3860) and
        (183.4960,114.7860) .. (175.2270,117.3850) .. controls (168.5950,119.4280) and
        (156.6500,130.5030) .. (147.4610,130.6550) .. controls (143.1850,130.7510) and
        (137.7020,130.9250) .. (134.3860,129.7410) .. controls (131.2210,128.6190) and
        (126.7250,123.1650) .. (121.4990,118.8680) .. controls (116.2820,114.6320) and
        (111.3140,110.4000) .. (111.4300,104.5230) .. controls (111.4000,98.5789) and
        (115.2450,96.2786) .. (121.0190,91.2738) .. controls (124.0420,88.6137) and
        (129.3800,83.9075) .. (133.2100,81.8497) .. controls (136.7000,80.0427) and
        (139.2040,80.5588) .. (142.8830,80.1344) -- cycle;
      \path[fill=cad7000] (142.9920,80.2277) .. controls (148.1270,79.5947) and
        (155.1620,80.8608) .. (158.3980,83.2524) .. controls (161.4220,85.5034) and
        (163.5330,86.7121) .. (166.2760,87.6139) .. controls (175.4780,90.6385) and
        (187.7430,92.1219) .. (187.1930,100.4100) .. controls (186.5730,110.3150) and
        (183.3190,114.6700) .. (175.1020,117.2480) .. controls (168.5150,119.2750) and
        (156.6210,130.1290) .. (147.4890,130.3330) .. controls (143.1800,130.4610) and
        (137.8130,130.6210) .. (134.5200,129.4500) .. controls (131.3800,128.3370) and
        (126.8970,122.8890) .. (121.7040,118.6240) .. controls (116.5240,114.4160) and
        (111.5560,110.3090) .. (111.7590,104.4890) .. controls (111.7910,98.7004) and
        (115.5260,96.3293) .. (121.2560,91.3603) .. controls (124.2560,88.7254) and
        (129.5000,83.9625) .. (133.3120,81.8908) .. controls (136.7650,80.0815) and
        (139.3350,80.6496) .. (142.9920,80.2277) -- cycle;
      \path[fill=cb27500] (143.1020,80.3210) .. controls (148.2070,79.6917) and
        (155.2000,80.9504) .. (158.4160,83.3279) .. controls (161.4230,85.5657) and
        (163.5210,86.7704) .. (166.2480,87.6637) .. controls (175.3930,90.6705) and
        (187.6280,92.1647) .. (187.0850,100.4000) .. controls (186.4710,110.2440) and
        (183.1420,114.5540) .. (174.9760,117.1100) .. controls (168.4350,119.1220) and
        (156.5910,129.7560) .. (147.5160,130.0110) .. controls (143.1740,130.1700) and
        (137.9240,130.3160) .. (134.6530,129.1590) .. controls (131.5380,128.0560) and
        (127.0690,122.6140) .. (121.9100,118.3800) .. controls (116.7670,114.2000) and
        (111.7980,110.2190) .. (112.0890,104.4560) .. controls (112.1820,98.8219) and
        (115.8070,96.3800) .. (121.4940,91.4467) .. controls (124.4690,88.8372) and
        (129.6210,84.0175) .. (133.4130,81.9319) .. controls (136.8300,80.1202) and
        (139.4660,80.7404) .. (143.1020,80.3210) -- cycle;
      \path[fill=cb77b00] (143.2120,80.4143) .. controls (148.2860,79.7888) and
        (155.2380,81.0400) .. (158.4350,83.4034) .. controls (161.4250,85.6280) and
        (163.5100,86.8289) .. (166.2210,87.7136) .. controls (175.3090,90.7025) and
        (187.5130,92.2076) .. (186.9770,100.3910) .. controls (186.3700,110.1740) and
        (182.9660,114.4380) .. (174.8510,116.9720) .. controls (168.3550,118.9690) and
        (156.5620,129.3830) .. (147.5440,129.6880) .. controls (143.1680,129.8800) and
        (138.0340,130.0120) .. (134.7860,128.8680) .. controls (131.6960,127.7750) and
        (127.2400,122.3390) .. (122.1160,118.1370) .. controls (117.0100,113.9850) and
        (112.0410,110.1290) .. (112.4180,104.4230) .. controls (112.5730,98.9435) and
        (116.0890,96.4307) .. (121.7320,91.5332) .. controls (124.6820,88.9489) and
        (129.7410,84.0724) .. (133.5140,81.9731) .. controls (136.8940,80.1590) and
        (139.5970,80.8312) .. (143.2120,80.4143) -- cycle;
      \path[fill=cbc8100] (143.3210,80.5076) .. controls (148.3650,79.8858) and
        (155.2760,81.1296) .. (158.4540,83.4789) .. controls (161.4260,85.6902) and
        (163.4990,86.8873) .. (166.1930,87.7634) .. controls (175.2240,90.7345) and
        (187.3990,92.2504) .. (186.8680,100.3820) .. controls (186.2690,110.1030) and
        (182.7890,114.3220) .. (174.7260,116.8350) .. controls (168.2750,118.8160) and
        (156.5330,129.0090) .. (147.5720,129.3660) .. controls (143.1630,129.5890) and
        (138.1450,129.7070) .. (134.9200,128.5770) .. controls (131.8550,127.4940) and
        (127.4120,122.0640) .. (122.3210,117.8930) .. controls (117.2520,113.7690) and
        (112.2830,110.0380) .. (112.7470,104.3900) .. controls (112.9640,99.0649) and
        (116.3700,96.4814) .. (121.9690,91.6196) .. controls (124.8960,89.0607) and
        (129.8620,84.1274) .. (133.6160,82.0142) .. controls (136.9590,80.1977) and
        (139.7280,80.9220) .. (143.3210,80.5076) -- cycle;
      \path[fill=cc18700] (143.4310,80.6009) .. controls (148.4450,79.9828) and
        (155.3130,81.2192) .. (158.4730,83.5544) .. controls (161.4270,85.7524) and
        (163.4870,86.9456) .. (166.1660,87.8133) .. controls (175.1390,90.7664) and
        (187.2840,92.2932) .. (186.7600,100.3730) .. controls (186.1670,110.0320) and
        (182.6120,114.2060) .. (174.6010,116.6970) .. controls (168.1950,118.6630) and
        (156.5030,128.6360) .. (147.5990,129.0440) .. controls (143.1570,129.2990) and
        (138.2560,129.4030) .. (135.0530,128.2860) .. controls (132.0130,127.2120) and
        (127.5840,121.7880) .. (122.5270,117.6500) .. controls (117.4950,113.5540) and
        (112.5260,109.9480) .. (113.0770,104.3560) .. controls (113.3550,99.1864) and
        (116.6510,96.5320) .. (122.2070,91.7061) .. controls (125.1090,89.1725) and
        (129.9820,84.1825) .. (133.7170,82.0552) .. controls (137.0240,80.2365) and
        (139.8590,81.0128) .. (143.4310,80.6009) -- cycle;
      \path[fill=cc68c00] (143.5400,80.6942) .. controls (148.5240,80.0799) and
        (155.3510,81.3088) .. (158.4920,83.6299) .. controls (161.4280,85.8147) and
        (163.4760,87.0041) .. (166.1380,87.8631) .. controls (175.0540,90.7985) and
        (187.1690,92.3361) .. (186.6520,100.3640) .. controls (186.0660,109.9610) and
        (182.4350,114.0900) .. (174.4760,116.5590) .. controls (168.1160,118.5100) and
        (156.4740,128.2630) .. (147.6270,128.7210) .. controls (143.1510,129.0080) and
        (138.3670,129.0980) .. (135.1860,127.9950) .. controls (132.1710,126.9310) and
        (127.7560,121.5130) .. (122.7320,117.4060) .. controls (117.7370,113.3380) and
        (112.7680,109.8580) .. (113.4060,104.3230) .. controls (113.7460,99.3080) and
        (116.9320,96.5828) .. (122.4440,91.7925) .. controls (125.3220,89.2842) and
        (130.1030,84.2375) .. (133.8180,82.0963) .. controls (137.0890,80.2752) and
        (139.9900,81.1036) .. (143.5400,80.6942) -- cycle;
      \path[fill=ccc9200] (143.6500,80.7874) .. controls (148.6030,80.1770) and
        (155.3890,81.3985) .. (158.5110,83.7054) .. controls (161.4290,85.8770) and
        (163.4650,87.0625) .. (166.1110,87.9129) .. controls (174.9690,90.8305) and
        (187.0540,92.3789) .. (186.5440,100.3540) .. controls (185.9650,109.8900) and
        (182.2590,113.9740) .. (174.3510,116.4210) .. controls (168.0360,118.3570) and
        (156.4450,127.8890) .. (147.6550,128.3990) .. controls (143.1460,128.7180) and
        (138.4770,128.7940) .. (135.3200,127.7040) .. controls (132.3300,126.6500) and
        (127.9270,121.2380) .. (122.9380,117.1620) .. controls (117.9800,113.1220) and
        (113.0110,109.7670) .. (113.7350,104.2900) .. controls (114.1370,99.4294) and
        (117.2130,96.6335) .. (122.6820,91.8789) .. controls (125.5360,89.3960) and
        (130.2230,84.2924) .. (133.9200,82.1375) .. controls (137.1530,80.3139) and
        (140.1210,81.1944) .. (143.6500,80.7874) -- cycle;
      \path[fill=cd19800] (143.7590,80.8808) .. controls (148.6830,80.2740) and
        (155.4270,81.4881) .. (158.5300,83.7809) .. controls (161.4300,85.9392) and
        (163.4530,87.1208) .. (166.0830,87.9628) .. controls (174.8840,90.8625) and
        (186.9390,92.4218) .. (186.4350,100.3450) .. controls (185.8630,109.8200) and
        (182.0820,113.8580) .. (174.2260,116.2840) .. controls (167.9560,118.2050) and
        (156.4150,127.5160) .. (147.6820,128.0770) .. controls (143.1400,128.4280) and
        (138.5880,128.4900) .. (135.4530,127.4130) .. controls (132.4880,126.3690) and
        (128.0990,120.9630) .. (123.1430,116.9190) .. controls (118.2220,112.9070) and
        (113.2530,109.6770) .. (114.0650,104.2570) .. controls (114.5280,99.5509) and
        (117.4940,96.6841) .. (122.9190,91.9654) .. controls (125.7490,89.5077) and
        (130.3440,84.3474) .. (134.0210,82.1786) .. controls (137.2180,80.3527) and
        (140.2520,81.2852) .. (143.7590,80.8808) -- cycle;
      \path[fill=cd69e00] (143.8690,80.9741) .. controls (148.7620,80.3711) and
        (155.4650,81.5777) .. (158.5480,83.8564) .. controls (161.4310,86.0015) and
        (163.4420,87.1793) .. (166.0560,88.0126) .. controls (174.7990,90.8945) and
        (186.8250,92.4647) .. (186.3270,100.3360) .. controls (185.7620,109.7490) and
        (181.9050,113.7420) .. (174.1000,116.1460) .. controls (167.8760,118.0520) and
        (156.3860,127.1430) .. (147.7100,127.7540) .. controls (143.1340,128.1370) and
        (138.6990,128.1850) .. (135.5860,127.1220) .. controls (132.6460,126.0870) and
        (128.2710,120.6870) .. (123.3490,116.6750) .. controls (118.4650,112.6910) and
        (113.4960,109.5870) .. (114.3940,104.2230) .. controls (114.9190,99.6725) and
        (117.7750,96.7349) .. (123.1570,92.0518) .. controls (125.9620,89.6194) and
        (130.4640,84.4024) .. (134.1220,82.2197) .. controls (137.2830,80.3914) and
        (140.3830,81.3760) .. (143.8690,80.9741) -- cycle;
      \path[fill=cdba300] (143.9780,81.0674) .. controls (148.8410,80.4681) and
        (155.5030,81.6673) .. (158.5670,83.9319) .. controls (161.4320,86.0637) and
        (163.4310,87.2376) .. (166.0280,88.0625) .. controls (174.7140,90.9265) and
        (186.7100,92.5075) .. (186.2190,100.3270) .. controls (185.6610,109.6780) and
        (181.7280,113.6260) .. (173.9750,116.0080) .. controls (167.7960,117.8990) and
        (156.3570,126.7700) .. (147.7380,127.4320) .. controls (143.1290,127.8470) and
        (138.8090,127.8810) .. (135.7200,126.8310) .. controls (132.8050,125.8060) and
        (128.4420,120.4120) .. (123.5540,116.4320) .. controls (118.7070,112.4760) and
        (113.7380,109.4970) .. (114.7230,104.1900) .. controls (115.3110,99.7939) and
        (118.0560,96.7856) .. (123.3940,92.1383) .. controls (126.1760,89.7312) and
        (130.5850,84.4575) .. (134.2240,82.2607) .. controls (137.3470,80.4302) and
        (140.5150,81.4669) .. (143.9780,81.0674) -- cycle;
      \path[fill=ce0a900] (144.0880,81.1607) .. controls (148.9200,80.5652) and
        (155.5410,81.7569) .. (158.5860,84.0074) .. controls (161.4330,86.1259) and
        (163.4190,87.2961) .. (166.0010,88.1123) .. controls (174.6290,90.9585) and
        (186.5950,92.5504) .. (186.1110,100.3180) .. controls (185.5590,109.6070) and
        (181.5520,113.5100) .. (173.8500,115.8710) .. controls (167.7160,117.7460) and
        (156.3270,126.3960) .. (147.7650,127.1100) .. controls (143.1230,127.5560) and
        (138.9200,127.5760) .. (135.8530,126.5400) .. controls (132.9630,125.5250) and
        (128.6140,120.1370) .. (123.7600,116.1880) .. controls (118.9500,112.2600) and
        (113.9810,109.4060) .. (115.0520,104.1570) .. controls (115.7020,99.9154) and
        (118.3370,96.8362) .. (123.6320,92.2248) .. controls (126.3890,89.8430) and
        (130.7050,84.5125) .. (134.3250,82.3018) .. controls (137.4120,80.4690) and
        (140.6460,81.5576) .. (144.0880,81.1607) -- cycle;
      \path[fill=ce5af00] (144.1970,81.2540) .. controls (149.0000,80.6622) and
        (155.5790,81.8465) .. (158.6050,84.0829) .. controls (161.4340,86.1882) and
        (163.4080,87.3545) .. (165.9730,88.1622) .. controls (174.5440,90.9904) and
        (186.4800,92.5932) .. (186.0020,100.3080) .. controls (185.4580,109.5360) and
        (181.3750,113.3940) .. (173.7250,115.7330) .. controls (167.6360,117.5930) and
        (156.2980,126.0230) .. (147.7930,126.7870) .. controls (143.1170,127.2660) and
        (139.0310,127.2720) .. (135.9860,126.2490) .. controls (133.1210,125.2440) and
        (128.7860,119.8620) .. (123.9650,115.9440) .. controls (119.1920,112.0440) and
        (114.2230,109.3160) .. (115.3820,104.1240) .. controls (116.0930,100.0370) and
        (118.6180,96.8869) .. (123.8690,92.3112) .. controls (126.6020,89.9547) and
        (130.8260,84.5674) .. (134.4260,82.3430) .. controls (137.4770,80.5077) and
        (140.7770,81.6484) .. (144.1970,81.2540) -- cycle;
      \path[fill=ceab500] (144.3070,81.3473) .. controls (149.0790,80.7592) and
        (155.6170,81.9361) .. (158.6240,84.1584) .. controls (161.4350,86.2505) and
        (163.3970,87.4128) .. (165.9450,88.2120) .. controls (174.4590,91.0225) and
        (186.3650,92.6360) .. (185.8940,100.2990) .. controls (185.3570,109.4660) and
        (181.1980,113.2780) .. (173.6000,115.5950) .. controls (167.5570,117.4400) and
        (156.2690,125.6500) .. (147.8210,126.4650) .. controls (143.1120,126.9750) and
        (139.1410,126.9670) .. (136.1200,125.9580) .. controls (133.2800,124.9620) and
        (128.9570,119.5860) .. (124.1710,115.7010) .. controls (119.4350,111.8290) and
        (114.4650,109.2260) .. (115.7110,104.0900) .. controls (116.4840,100.1580) and
        (118.8990,96.9377) .. (124.1070,92.3976) .. controls (126.8160,90.0665) and
        (130.9460,84.6224) .. (134.5280,82.3841) .. controls (137.5410,80.5464) and
        (140.9080,81.7393) .. (144.3070,81.3473) -- cycle;
      \path[fill=cefba00] (144.4170,81.4406) .. controls (149.1580,80.8563) and
        (155.6550,82.0257) .. (158.6430,84.2339) .. controls (161.4360,86.3127) and
        (163.3850,87.4713) .. (165.9180,88.2619) .. controls (174.3740,91.0544) and
        (186.2510,92.6789) .. (185.7860,100.2900) .. controls (185.2550,109.3950) and
        (181.0210,113.1620) .. (173.4750,115.4580) .. controls (167.4770,117.2870) and
        (156.2390,125.2760) .. (147.8480,126.1430) .. controls (143.1060,126.6850) and
        (139.2520,126.6630) .. (136.2530,125.6670) .. controls (133.4380,124.6810) and
        (129.1290,119.3110) .. (124.3770,115.4570) .. controls (119.6780,111.6130) and
        (114.7080,109.1350) .. (116.0400,104.0570) .. controls (116.8750,100.2800) and
        (119.1800,96.9883) .. (124.3450,92.4841) .. controls (127.0290,90.1782) and
        (131.0670,84.6774) .. (134.6290,82.4252) .. controls (137.6060,80.5852) and
        (141.0390,81.8300) .. (144.4170,81.4406) -- cycle;
      \path[fill=cf4c000] (144.5260,81.5338) .. controls (149.2380,80.9534) and
        (155.6920,82.1153) .. (158.6620,84.3094) .. controls (161.4370,86.3749) and
        (163.3740,87.5297) .. (165.8900,88.3117) .. controls (174.2890,91.0865) and
        (186.1360,92.7218) .. (185.6780,100.2810) .. controls (185.1540,109.3240) and
        (180.8450,113.0460) .. (173.3500,115.3200) .. controls (167.3970,117.1340) and
        (156.2100,124.9030) .. (147.8760,125.8200) .. controls (143.1010,126.3940) and
        (139.3630,126.3580) .. (136.3870,125.3760) .. controls (133.5970,124.4000) and
        (129.3010,119.0360) .. (124.5820,115.2140) .. controls (119.9200,111.3980) and
        (114.9500,109.0450) .. (116.3700,104.0240) .. controls (117.2660,100.4010) and
        (119.4610,97.0390) .. (124.5820,92.5705) .. controls (127.2430,90.2899) and
        (131.1870,84.7325) .. (134.7310,82.4662) .. controls (137.6710,80.6239) and
        (141.1700,81.9208) .. (144.5260,81.5338) -- cycle;
      \path[fill=cf9c600] (144.6360,81.6272) .. controls (149.3170,81.0504) and
        (155.7300,82.2049) .. (158.6800,84.3849) .. controls (161.4380,86.4372) and
        (163.3630,87.5880) .. (165.8630,88.3616) .. controls (174.2040,91.1184) and
        (186.0210,92.7646) .. (185.5690,100.2720) .. controls (185.0530,109.2530) and
        (180.6680,112.9300) .. (173.2240,115.1820) .. controls (167.3170,116.9810) and
        (156.1810,124.5300) .. (147.9040,125.4980) .. controls (143.0950,126.1040) and
        (139.4740,126.0540) .. (136.5200,125.0850) .. controls (133.7550,124.1190) and
        (129.4730,118.7610) .. (124.7880,114.9700) .. controls (120.1630,111.1820) and
        (115.1930,108.9550) .. (116.6990,103.9910) .. controls (117.6570,100.5230) and
        (119.7420,97.0898) .. (124.8200,92.6570) .. controls (127.4560,90.4017) and
        (131.3080,84.7874) .. (134.8320,82.5074) .. controls (137.7360,80.6627) and
        (141.3010,82.0117) .. (144.6360,81.6272) -- cycle;
    \path[fill=cffcc00] (144.7450,81.7205) .. controls (149.3960,81.1475) and
      (155.7680,82.2945) .. (158.6990,84.4604) .. controls (161.4390,86.4995) and
      (163.3510,87.6465) .. (165.8350,88.4114) .. controls (174.1190,91.1505) and
      (185.9060,92.8075) .. (185.4610,100.2620) .. controls (184.9510,109.1820) and
      (180.4910,112.8140) .. (173.0990,115.0440) .. controls (167.2370,116.8280) and
      (156.1510,124.1560) .. (147.9310,125.1750) .. controls (143.0890,125.8130) and
      (139.5840,125.7490) .. (136.6530,124.7930) .. controls (133.9130,123.8370) and
      (129.6440,118.4850) .. (124.9930,114.7260) .. controls (120.4050,110.9660) and
      (115.4350,108.8640) .. (117.0280,103.9570) .. controls (118.0480,100.6440) and
      (120.0230,97.1404) .. (125.0570,92.7434) .. controls (127.6690,90.5135) and
      (131.4280,84.8424) .. (134.9330,82.5485) .. controls (137.8000,80.7014) and
      (141.4320,82.1024) .. (144.7450,81.7205) -- cycle;
    \path[fill=cffcc00] (145.5780,84.6064) .. controls (146.5860,86.6945) and
      (149.1780,86.9825) .. (150.9060,87.9185) .. controls (152.5620,88.8545) and
      (153.4980,89.0705) .. (154.1460,88.7104) .. controls (155.5860,87.9185) and
      (154.5060,85.3264) .. (153.0660,84.3904) .. controls (151.6980,83.4544) and
      (145.0020,83.2385) .. (145.5780,84.6064) -- cycle;
      \path[fill=cf9c600] (145.7210,84.6281) .. controls (146.7030,86.6631) and
        (149.2290,86.9438) .. (150.9140,87.8562) .. controls (152.5280,88.7685) and
        (153.4400,88.9790) .. (154.0720,88.6281) .. controls (155.4750,87.8562) and
        (154.4220,85.3298) .. (153.0190,84.4175) .. controls (151.6860,83.5052) and
        (145.1590,83.2947) .. (145.7210,84.6281) -- cycle;
      \path[fill=cf4c000] (145.8630,84.6497) .. controls (146.8200,86.6319) and
        (149.2810,86.9052) .. (150.9210,87.7939) .. controls (152.4930,88.6825) and
        (153.3820,88.8875) .. (153.9970,88.5457) .. controls (155.3640,87.7939) and
        (154.3390,85.3332) .. (152.9720,84.4445) .. controls (151.6730,83.5560) and
        (145.3160,83.3510) .. (145.8630,84.6497) -- cycle;
      \path[fill=cefba00] (146.0050,84.6712) .. controls (146.9370,86.6006) and
        (149.3320,86.8666) .. (150.9290,87.7316) .. controls (152.4590,88.5965) and
        (153.3230,88.7960) .. (153.9220,88.4633) .. controls (155.2530,87.7316) and
        (154.2550,85.3365) .. (152.9240,84.4716) .. controls (151.6600,83.6068) and
        (145.4730,83.4072) .. (146.0050,84.6712) -- cycle;
      \path[fill=ceab500] (146.1480,84.6928) .. controls (147.0540,86.5692) and
        (149.3830,86.8281) .. (150.9360,87.6693) .. controls (152.4240,88.5105) and
        (153.2650,88.7045) .. (153.8480,88.3809) .. controls (155.1420,87.6693) and
        (154.1710,85.3398) .. (152.8770,84.4987) .. controls (151.6480,83.6577) and
        (145.6300,83.4635) .. (146.1480,84.6928) -- cycle;
      \path[fill=ce5af00] (146.2900,84.7144) .. controls (147.1710,86.5380) and
        (149.4340,86.7895) .. (150.9430,87.6070) .. controls (152.3900,88.4245) and
        (153.2070,88.6129) .. (153.7730,88.2984) .. controls (155.0300,87.6070) and
        (154.0870,85.3432) .. (152.8300,84.5257) .. controls (151.6350,83.7084) and
        (145.7870,83.5197) .. (146.2900,84.7144) -- cycle;
      \path[fill=ce0a900] (146.4330,84.7361) .. controls (147.2880,86.5067) and
        (149.4850,86.7509) .. (150.9510,87.5446) .. controls (152.3550,88.3385) and
        (153.1490,88.5215) .. (153.6980,88.2160) .. controls (154.9190,87.5446) and
        (154.0040,85.3466) .. (152.7820,84.5527) .. controls (151.6220,83.7592) and
        (145.9440,83.5760) .. (146.4330,84.7361) -- cycle;
      \path[fill=cdba300] (146.5750,84.7577) .. controls (147.4050,86.4753) and
        (149.5370,86.7122) .. (150.9580,87.4824) .. controls (152.3210,88.2525) and
        (153.0900,88.4300) .. (153.6240,88.1336) .. controls (154.8080,87.4824) and
        (153.9200,85.3499) .. (152.7350,84.5798) .. controls (151.6100,83.8101) and
        (146.1010,83.6322) .. (146.5750,84.7577) -- cycle;
      \path[fill=cd69e00] (146.7170,84.7792) .. controls (147.5210,86.4441) and
        (149.5880,86.6736) .. (150.9660,87.4200) .. controls (152.2860,88.1665) and
        (153.0320,88.3385) .. (153.5490,88.0512) .. controls (154.6970,87.4200) and
        (153.8360,85.3532) .. (152.6880,84.6068) .. controls (151.5970,83.8608) and
        (146.2580,83.6885) .. (146.7170,84.7792) -- cycle;
      \path[fill=cd19800] (146.8600,84.8008) .. controls (147.6380,86.4128) and
        (149.6390,86.6350) .. (150.9730,87.3578) .. controls (152.2520,88.0805) and
        (152.9740,88.2469) .. (153.4740,87.9688) .. controls (154.5860,87.3578) and
        (153.7520,85.3566) .. (152.6410,84.6339) .. controls (151.5840,83.9117) and
        (146.4150,83.7447) .. (146.8600,84.8008) -- cycle;
      \path[fill=ccc9200] (147.0020,84.8224) .. controls (147.7550,86.3814) and
        (149.6900,86.5965) .. (150.9810,87.2954) .. controls (152.2170,87.9945) and
        (152.9160,88.1555) .. (153.4000,87.8864) .. controls (154.4750,87.2954) and
        (153.6690,85.3600) .. (152.5930,84.6609) .. controls (151.5720,83.9624) and
        (146.5720,83.8010) .. (147.0020,84.8224) -- cycle;
      \path[fill=cc68c00] (147.1450,84.8441) .. controls (147.8720,86.3502) and
        (149.7410,86.5579) .. (150.9880,87.2332) .. controls (152.1830,87.9084) and
        (152.8570,88.0640) .. (153.3250,87.8040) .. controls (154.3640,87.2332) and
        (153.5850,85.3633) .. (152.5460,84.6880) .. controls (151.5590,84.0132) and
        (146.7290,83.8572) .. (147.1450,84.8441) -- cycle;
      \path[fill=cc18700] (147.2870,84.8657) .. controls (147.9890,86.3188) and
        (149.7930,86.5193) .. (150.9960,87.1709) .. controls (152.1480,87.8224) and
        (152.7990,87.9724) .. (153.2500,87.7216) .. controls (154.2520,87.1709) and
        (153.5010,85.3667) .. (152.4990,84.7151) .. controls (151.5460,84.0641) and
        (146.8860,83.9135) .. (147.2870,84.8657) -- cycle;
      \path[fill=cbc8100] (147.4290,84.8872) .. controls (148.1060,86.2876) and
        (149.8440,86.4807) .. (151.0030,87.1086) .. controls (152.1140,87.7365) and
        (152.7410,87.8810) .. (153.1760,87.6393) .. controls (154.1410,87.1086) and
        (153.4170,85.3700) .. (152.4510,84.7421) .. controls (151.5340,84.1148) and
        (147.0430,83.9697) .. (147.4290,84.8872) -- cycle;
      \path[fill=cb77b00] (147.5720,84.9088) .. controls (148.2230,86.2563) and
        (149.8950,86.4420) .. (151.0110,87.0463) .. controls (152.0790,87.6505) and
        (152.6830,87.7895) .. (153.1010,87.5569) .. controls (154.0300,87.0463) and
        (153.3340,85.3734) .. (152.4040,84.7692) .. controls (151.5210,84.1656) and
        (147.2000,84.0260) .. (147.5720,84.9088) -- cycle;
      \path[fill=cb27500] (147.7140,84.9305) .. controls (148.3400,86.2249) and
        (149.9460,86.4034) .. (151.0180,86.9839) .. controls (152.0450,87.5645) and
        (152.6240,87.6980) .. (153.0260,87.4745) .. controls (153.9190,86.9839) and
        (153.2500,85.3767) .. (152.3570,84.7962) .. controls (151.5080,84.2165) and
        (147.3570,84.0822) .. (147.7140,84.9305) -- cycle;
      \path[fill=cad7000] (147.8570,84.9521) .. controls (148.4570,86.1937) and
        (149.9970,86.3648) .. (151.0250,86.9217) .. controls (152.0100,87.4785) and
        (152.5660,87.6064) .. (152.9520,87.3920) .. controls (153.8080,86.9217) and
        (153.1660,85.3801) .. (152.3090,84.8232) .. controls (151.4960,84.2672) and
        (147.5140,84.1385) .. (147.8570,84.9521) -- cycle;
      \path[fill=ca86a00] (147.9990,84.9737) .. controls (148.5740,86.1624) and
        (150.0490,86.3263) .. (151.0330,86.8593) .. controls (151.9760,87.3925) and
        (152.5080,87.5150) .. (152.8770,87.3096) .. controls (153.6970,86.8593) and
        (153.0820,85.3834) .. (152.2620,84.8503) .. controls (151.4830,84.3181) and
        (147.6710,84.1947) .. (147.9990,84.9737) -- cycle;
      \path[fill=ca36400] (148.1410,84.9952) .. controls (148.6900,86.1310) and
        (150.1000,86.2877) .. (151.0400,86.7971) .. controls (151.9410,87.3065) and
        (152.4500,87.4235) .. (152.8030,87.2272) .. controls (153.5860,86.7971) and
        (152.9990,85.3867) .. (152.2150,84.8773) .. controls (151.4710,84.3689) and
        (147.8280,84.2510) .. (148.1410,84.9952) -- cycle;
      \path[fill=c9e5e00] (148.2840,85.0168) .. controls (148.8070,86.0998) and
        (150.1510,86.2491) .. (151.0480,86.7347) .. controls (151.9070,87.2205) and
        (152.3910,87.3319) .. (152.7280,87.1448) .. controls (153.4740,86.7347) and
        (152.9150,85.3901) .. (152.1680,84.9044) .. controls (151.4580,84.4196) and
        (147.9850,84.3072) .. (148.2840,85.0168) -- cycle;
    \path[fill=c995900] (148.4260,85.0385) .. controls (148.9240,86.0685) and
      (150.2020,86.2104) .. (151.0550,86.6725) .. controls (151.8720,87.1345) and
      (152.3330,87.2404) .. (152.6530,87.0624) .. controls (153.3630,86.6725) and
      (152.8310,85.3935) .. (152.1200,84.9315) .. controls (151.4450,84.4705) and
      (148.1420,84.3635) .. (148.4260,85.0385) -- cycle;
    \path[fill=cffcc00] (130.4710,87.2864) .. controls (130.1460,88.0995) and
      (132.4230,89.6454) .. (133.5620,88.5875) .. controls (134.7830,87.5305) and
      (136.2470,86.5544) .. (136.7360,86.2285) .. controls (138.9320,84.7635) and
      (138.1190,83.7065) .. (134.4580,84.3575) .. controls (130.7960,85.0085) and
      (130.7960,86.4724) .. (130.4710,87.2864) -- cycle;
      \path[fill=cf9c600] (130.5300,87.2762) .. controls (130.2130,88.0703) and
        (132.4370,89.5805) .. (133.5490,88.5471) .. controls (134.7420,87.5145) and
        (136.1720,86.5610) .. (136.6500,86.2426) .. controls (138.7950,84.8116) and
        (138.0010,83.7790) .. (134.4250,84.4149) .. controls (130.8470,85.0509) and
        (130.8470,86.4810) .. (130.5300,87.2762) -- cycle;
      \path[fill=cf4c000] (130.5890,87.2658) .. controls (130.2790,88.0412) and
        (132.4500,89.5157) .. (133.5370,88.5067) .. controls (134.7010,87.4986) and
        (136.0970,86.5677) .. (136.5640,86.2569) .. controls (138.6580,84.8596) and
        (137.8830,83.8516) .. (134.3910,84.4724) .. controls (130.8990,85.0934) and
        (130.8990,86.4895) .. (130.5890,87.2658) -- cycle;
      \path[fill=cefba00] (130.6470,87.2555) .. controls (130.3450,88.0120) and
        (132.4640,89.4507) .. (133.5240,88.4662) .. controls (134.6600,87.4826) and
        (136.0220,86.5742) .. (136.4770,86.2711) .. controls (138.5210,84.9077) and
        (137.7640,83.9241) .. (134.3580,84.5299) .. controls (130.9500,85.1358) and
        (130.9500,86.4981) .. (130.6470,87.2555) -- cycle;
      \path[fill=ceab500] (130.7060,87.2452) .. controls (130.4110,87.9828) and
        (132.4780,89.3858) .. (133.5110,88.4259) .. controls (134.6190,87.4666) and
        (135.9470,86.5808) .. (136.3910,86.2852) .. controls (138.3840,84.9559) and
        (137.6460,83.9966) .. (134.3240,84.5875) .. controls (131.0010,85.1783) and
        (131.0010,86.5067) .. (130.7060,87.2452) -- cycle;
      \path[fill=ce5af00] (130.7650,87.2350) .. controls (130.4780,87.9537) and
        (132.4910,89.3210) .. (133.4980,88.3855) .. controls (134.5780,87.4507) and
        (135.8720,86.5874) .. (136.3050,86.2995) .. controls (138.2470,85.0040) and
        (137.5280,84.0692) .. (134.2900,84.6450) .. controls (131.0520,85.2207) and
        (131.0520,86.5152) .. (130.7650,87.2350) -- cycle;
      \path[fill=ce0a900] (130.8240,87.2246) .. controls (130.5440,87.9246) and
        (132.5050,89.2560) .. (133.4850,88.3451) .. controls (134.5370,87.4348) and
        (135.7980,86.5941) .. (136.2180,86.3137) .. controls (138.1100,85.0521) and
        (137.4090,84.1418) .. (134.2570,84.7025) .. controls (131.1040,85.2632) and
        (131.1040,86.5237) .. (130.8240,87.2246) -- cycle;
      \path[fill=cdba300] (130.8820,87.2143) .. controls (130.6100,87.8954) and
        (132.5180,89.1911) .. (133.4730,88.3047) .. controls (134.4960,87.4188) and
        (135.7230,86.6006) .. (136.1320,86.3279) .. controls (137.9730,85.1002) and
        (137.2910,84.2143) .. (134.2230,84.7599) .. controls (131.1550,85.3056) and
        (131.1550,86.5323) .. (130.8820,87.2143) -- cycle;
      \path[fill=cd69e00] (130.9410,87.2040) .. controls (130.6770,87.8662) and
        (132.5320,89.1263) .. (133.4600,88.2643) .. controls (134.4550,87.4029) and
        (135.6480,86.6072) .. (136.0460,86.3420) .. controls (137.8350,85.1483) and
        (137.1730,84.2868) .. (134.1900,84.8174) .. controls (131.2060,85.3481) and
        (131.2060,86.5409) .. (130.9410,87.2040) -- cycle;
      \path[fill=cd19800] (131.0000,87.1938) .. controls (130.7430,87.8371) and
        (132.5460,89.0613) .. (133.4470,88.2238) .. controls (134.4140,87.3869) and
        (135.5730,86.6139) .. (135.9600,86.3563) .. controls (137.6980,85.1964) and
        (137.0550,84.3594) .. (134.1560,84.8749) .. controls (131.2570,85.3905) and
        (131.2570,86.5494) .. (131.0000,87.1938) -- cycle;
      \path[fill=ccc9200] (131.0590,87.1834) .. controls (130.8090,87.8080) and
        (132.5590,88.9964) .. (133.4340,88.1834) .. controls (134.3730,87.3709) and
        (135.4980,86.6205) .. (135.8730,86.3705) .. controls (137.5610,85.2445) and
        (136.9360,84.4319) .. (134.1230,84.9325) .. controls (131.3090,85.4330) and
        (131.3090,86.5580) .. (131.0590,87.1834) -- cycle;
      \path[fill=cc68c00] (131.1170,87.1731) .. controls (130.8760,87.7788) and
        (132.5730,88.9315) .. (133.4210,88.1431) .. controls (134.3320,87.3550) and
        (135.4230,86.6270) .. (135.7870,86.3846) .. controls (137.4240,85.2926) and
        (136.8180,84.5045) .. (134.0890,84.9900) .. controls (131.3600,85.4754) and
        (131.3600,86.5665) .. (131.1170,87.1731) -- cycle;
      \path[fill=cc18700] (131.1760,87.1628) .. controls (130.9420,87.7497) and
        (132.5860,88.8666) .. (133.4090,88.1027) .. controls (134.2910,87.3391) and
        (135.3480,86.6337) .. (135.7010,86.3989) .. controls (137.2870,85.3407) and
        (136.7000,84.5771) .. (134.0560,85.0475) .. controls (131.4110,85.5179) and
        (131.4110,86.5750) .. (131.1760,87.1628) -- cycle;
      \path[fill=cbc8100] (131.2350,87.1526) .. controls (131.0080,87.7205) and
        (132.6000,88.8018) .. (133.3960,88.0623) .. controls (134.2500,87.3231) and
        (135.2730,86.6403) .. (135.6140,86.4131) .. controls (137.1500,85.3888) and
        (136.5810,84.6496) .. (134.0220,85.1049) .. controls (131.4620,85.5603) and
        (131.4620,86.5836) .. (131.2350,87.1526) -- cycle;
      \path[fill=cb77b00] (131.2940,87.1422) .. controls (131.0740,87.6913) and
        (132.6140,88.7368) .. (133.3830,88.0219) .. controls (134.2090,87.3072) and
        (135.1980,86.6469) .. (135.5280,86.4272) .. controls (137.0130,85.4369) and
        (136.4630,84.7221) .. (133.9890,85.1624) .. controls (131.5140,85.6028) and
        (131.5140,86.5922) .. (131.2940,87.1422) -- cycle;
      \path[fill=cb27500] (131.3520,87.1319) .. controls (131.1410,87.6622) and
        (132.6270,88.6719) .. (133.3700,87.9814) .. controls (134.1670,87.2912) and
        (135.1230,86.6535) .. (135.4420,86.4414) .. controls (136.8760,85.4850) and
        (136.3450,84.7947) .. (133.9550,85.2199) .. controls (131.5650,85.6452) and
        (131.5650,86.6007) .. (131.3520,87.1319) -- cycle;
      \path[fill=cad7000] (131.4110,87.1216) .. controls (131.2070,87.6331) and
        (132.6410,88.6071) .. (133.3570,87.9410) .. controls (134.1260,87.2753) and
        (135.0480,86.6601) .. (135.3550,86.4557) .. controls (136.7390,85.5331) and
        (136.2260,84.8672) .. (133.9210,85.2774) .. controls (131.6160,85.6877) and
        (131.6160,86.6093) .. (131.4110,87.1216) -- cycle;
      \path[fill=ca86a00] (131.4700,87.1114) .. controls (131.2730,87.6039) and
        (132.6540,88.5421) .. (133.3450,87.9007) .. controls (134.0850,87.2593) and
        (134.9730,86.6667) .. (135.2690,86.4698) .. controls (136.6020,85.5812) and
        (136.1080,84.9398) .. (133.8880,85.3350) .. controls (131.6670,85.7301) and
        (131.6670,86.6178) .. (131.4700,87.1114) -- cycle;
      \path[fill=ca36400] (131.5290,87.1010) .. controls (131.3400,87.5747) and
        (132.6680,88.4772) .. (133.3320,87.8603) .. controls (134.0440,87.2433) and
        (134.8980,86.6732) .. (135.1830,86.4840) .. controls (136.4640,85.6293) and
        (135.9900,85.0123) .. (133.8540,85.3925) .. controls (131.7190,85.7726) and
        (131.7190,86.6264) .. (131.5290,87.1010) -- cycle;
      \path[fill=c9e5e00] (131.5870,87.0908) .. controls (131.4060,87.5456) and
        (132.6820,88.4124) .. (133.3190,87.8199) .. controls (134.0030,87.2274) and
        (134.8230,86.6799) .. (135.0970,86.4983) .. controls (136.3270,85.6774) and
        (135.8720,85.0849) .. (133.8210,85.4500) .. controls (131.7700,85.8150) and
        (131.7700,86.6349) .. (131.5870,87.0908) -- cycle;
    \path[fill=c995900] (131.6460,87.0804) .. controls (131.4720,87.5164) and
      (132.6950,88.3474) .. (133.3060,87.7794) .. controls (133.9620,87.2115) and
      (134.7480,86.6865) .. (135.0100,86.5125) .. controls (136.1900,85.7255) and
      (135.7530,85.1574) .. (133.7870,85.5074) .. controls (131.8210,85.8575) and
      (131.8210,86.6435) .. (131.6460,87.0804) -- cycle;
    \path[fill=cffcc00] (134.5470,108.8840) .. controls (134.3180,111.7420) and
      (140.8900,104.5980) .. (141.2900,103.9690) .. controls (142.1470,102.4260) and
      (145.0050,98.0835) .. (145.4620,96.2545) .. controls (146.3190,93.0544) and
      (147.8630,90.7115) .. (146.8910,87.3394) .. controls (146.5480,86.2534) and
      (144.1480,85.9674) .. (143.0620,86.6534) .. controls (139.9760,88.4825) and
      (140.4330,90.7115) .. (140.0900,92.7685) .. controls (138.9470,98.5974) and
      (134.9470,104.4840) .. (134.5470,108.8840) -- cycle;
      \path[fill=cffcc02] (134.8160,108.4380) .. controls (134.5970,111.2250) and
        (140.9940,104.2500) .. (141.3830,103.6360) .. controls (142.2160,102.1300) and
        (144.9970,97.8912) .. (145.4400,96.1071) .. controls (146.2700,92.9858) and
        (147.7720,90.6990) .. (146.8200,87.4127) .. controls (146.4840,86.3542) and
        (144.1430,86.0790) .. (143.0850,86.7495) .. controls (140.0790,88.5374) and
        (140.5280,90.7101) .. (140.1960,92.7162) .. controls (139.0910,98.4013) and
        (135.1990,104.1470) .. (134.8160,108.4380) -- cycle;
      \path[fill=cffcc05] (135.0840,107.9920) .. controls (134.8750,110.7070) and
        (141.0980,103.9010) .. (141.4760,103.3020) .. controls (142.2860,101.8330) and
        (144.9880,97.6989) .. (145.4170,95.9599) .. controls (146.2220,92.9171) and
        (147.6810,90.6866) .. (146.7480,87.4858) .. controls (146.4190,86.4550) and
        (144.1380,86.1906) .. (143.1080,86.8454) .. controls (140.1820,88.5923) and
        (140.6220,90.7086) .. (140.3030,92.6640) .. controls (139.2340,98.2052) and
        (135.4510,103.8100) .. (135.0840,107.9920) -- cycle;
      \path[fill=cffcc07] (135.3530,107.5450) .. controls (135.1540,110.1900) and
        (141.2020,103.5520) .. (141.5690,102.9680) .. controls (142.3550,101.5370) and
        (144.9790,97.5066) .. (145.3940,95.8126) .. controls (146.1730,92.8485) and
        (147.5910,90.6741) .. (146.6760,87.5591) .. controls (146.3540,86.5558) and
        (144.1330,86.3021) .. (143.1310,86.9414) .. controls (140.2840,88.6472) and
        (140.7170,90.7072) .. (140.4090,92.6117) .. controls (139.3780,98.0092) and
        (135.7040,103.4730) .. (135.3530,107.5450) -- cycle;
      \path[fill=cffcd0a] (135.6220,107.0990) .. controls (135.4330,109.6720) and
        (141.3060,103.2030) .. (141.6620,102.6340) .. controls (142.4240,101.2400) and
        (144.9710,97.3143) .. (145.3710,95.6653) .. controls (146.1240,92.7798) and
        (147.5000,90.6617) .. (146.6050,87.6323) .. controls (146.2890,86.6566) and
        (144.1270,86.4136) .. (143.1540,87.0374) .. controls (140.3870,88.7021) and
        (140.8120,90.7058) .. (140.5150,92.5594) .. controls (139.5210,97.8130) and
        (135.9560,103.1360) .. (135.6220,107.0990) -- cycle;
      \path[fill=cffcd0c] (135.8900,106.6520) .. controls (135.7110,109.1550) and
        (141.4090,102.8540) .. (141.7550,102.3010) .. controls (142.4930,100.9440) and
        (144.9620,97.1220) .. (145.3490,95.5179) .. controls (146.0750,92.7112) and
        (147.4090,90.6492) .. (146.5330,87.7054) .. controls (146.2250,86.7574) and
        (144.1220,86.5252) .. (143.1770,87.1335) .. controls (140.4900,88.7570) and
        (140.9060,90.7044) .. (140.6210,92.5072) .. controls (139.6640,97.6169) and
        (136.2080,102.7980) .. (135.8900,106.6520) -- cycle;
      \path[fill=cffcd0f] (136.1590,106.2060) .. controls (135.9900,108.6370) and
        (141.5130,102.5050) .. (141.8480,101.9670) .. controls (142.5630,100.6470) and
        (144.9540,96.9297) .. (145.3260,95.3706) .. controls (146.0260,92.6425) and
        (147.3180,90.6367) .. (146.4610,87.7787) .. controls (146.1600,86.8582) and
        (144.1170,86.6367) .. (143.2000,87.2295) .. controls (140.5920,88.8119) and
        (141.0010,90.7031) .. (140.7280,92.4550) .. controls (139.8080,97.4208) and
        (136.4600,102.4610) .. (136.1590,106.2060) -- cycle;
      \path[fill=cffcd11] (136.4270,105.7590) .. controls (136.2680,108.1200) and
        (141.6170,102.1560) .. (141.9410,101.6330) .. controls (142.6320,100.3510) and
        (144.9450,96.7374) .. (145.3030,95.2234) .. controls (145.9770,92.5739) and
        (147.2270,90.6243) .. (146.3900,87.8519) .. controls (146.0950,86.9590) and
        (144.1120,86.7483) .. (143.2230,87.3254) .. controls (140.6950,88.8668) and
        (141.0950,90.7017) .. (140.8340,92.4027) .. controls (139.9510,97.2247) and
        (136.7120,102.1240) .. (136.4270,105.7590) -- cycle;
      \path[fill=cffce14] (136.6960,105.3130) .. controls (136.5470,107.6020) and
        (141.7210,101.8080) .. (142.0340,101.2990) .. controls (142.7010,100.0540) and
        (144.9360,96.5451) .. (145.2810,95.0760) .. controls (145.9280,92.5052) and
        (147.1360,90.6118) .. (146.3180,87.9250) .. controls (146.0310,87.0598) and
        (144.1070,86.8598) .. (143.2460,87.4214) .. controls (140.7980,88.9217) and
        (141.1900,90.7003) .. (140.9400,92.3505) .. controls (140.0950,97.0287) and
        (136.9640,101.7870) .. (136.6960,105.3130) -- cycle;
      \path[fill=cffce16] (136.9650,104.8660) .. controls (136.8260,107.0850) and
        (141.8250,101.4590) .. (142.1270,100.9660) .. controls (142.7700,99.7580) and
        (144.9280,96.3528) .. (145.2580,94.9287) .. controls (145.8800,92.4366) and
        (147.0450,90.5994) .. (146.2460,87.9983) .. controls (145.9660,87.1606) and
        (144.1010,86.9714) .. (143.2690,87.5175) .. controls (140.9010,88.9766) and
        (141.2850,90.6989) .. (141.0460,92.2982) .. controls (140.2380,96.8326) and
        (137.2160,101.4500) .. (136.9650,104.8660) -- cycle;
      \path[fill=cffce19] (137.2330,104.4200) .. controls (137.1040,106.5670) and
        (141.9290,101.1100) .. (142.2200,100.6320) .. controls (142.8400,99.4615) and
        (144.9190,96.1605) .. (145.2350,94.7815) .. controls (145.8310,92.3680) and
        (146.9540,90.5869) .. (146.1750,88.0714) .. controls (145.9010,87.2614) and
        (144.0960,87.0829) .. (143.2920,87.6134) .. controls (141.0030,89.0315) and
        (141.3790,90.6974) .. (141.1530,92.2459) .. controls (140.3820,96.6364) and
        (137.4680,101.1120) .. (137.2330,104.4200) -- cycle;
      \path[fill=cffce1c] (137.5020,103.9740) .. controls (137.3830,106.0500) and
        (142.0330,100.7610) .. (142.3130,100.2980) .. controls (142.9090,99.1649) and
        (144.9110,95.9682) .. (145.2130,94.6342) .. controls (145.7820,92.2993) and
        (146.8630,90.5745) .. (146.1030,88.1447) .. controls (145.8370,87.3622) and
        (144.0910,87.1945) .. (143.3150,87.7094) .. controls (141.1060,89.0863) and
        (141.4740,90.6960) .. (141.2590,92.1937) .. controls (140.5250,96.4403) and
        (137.7200,100.7750) .. (137.5020,103.9740) -- cycle;
      \path[fill=cffcf1e] (137.7700,103.5270) .. controls (137.6610,105.5320) and
        (142.1360,100.4120) .. (142.4060,99.9644) .. controls (142.9780,98.8685) and
        (144.9020,95.7758) .. (145.1900,94.4868) .. controls (145.7330,92.2307) and
        (146.7720,90.5621) .. (146.0310,88.2179) .. controls (145.7720,87.4630) and
        (144.0860,87.3061) .. (143.3380,87.8055) .. controls (141.2090,89.1412) and
        (141.5680,90.6946) .. (141.3650,92.1414) .. controls (140.6690,96.2442) and
        (137.9720,100.4380) .. (137.7700,103.5270) -- cycle;
      \path[fill=cffcf21] (138.0390,103.0810) .. controls (137.9400,105.0150) and
        (142.2400,100.0630) .. (142.4990,99.6307) .. controls (143.0470,98.5720) and
        (144.8930,95.5836) .. (145.1670,94.3395) .. controls (145.6840,92.1620) and
        (146.6820,90.5496) .. (145.9600,88.2910) .. controls (145.7070,87.5638) and
        (144.0810,87.4176) .. (143.3610,87.9015) .. controls (141.3110,89.1961) and
        (141.6630,90.6932) .. (141.4710,92.0892) .. controls (140.8120,96.0482) and
        (138.2250,100.1010) .. (138.0390,103.0810) -- cycle;
      \path[fill=cffcf23] (138.3080,102.6340) .. controls (138.2190,104.4970) and
        (142.3440,99.7146) .. (142.5920,99.2970) .. controls (143.1170,98.2755) and
        (144.8850,95.3913) .. (145.1440,94.1923) .. controls (145.6350,92.0934) and
        (146.5910,90.5372) .. (145.8880,88.3643) .. controls (145.6420,87.6646) and
        (144.0750,87.5291) .. (143.3840,87.9975) .. controls (141.4140,89.2510) and
        (141.7580,90.6918) .. (141.5780,92.0370) .. controls (140.9560,95.8521) and
        (138.4770,99.7636) .. (138.3080,102.6340) -- cycle;
      \path[fill=cffcf26] (138.5760,102.1880) .. controls (138.4970,103.9800) and
        (142.4480,99.3657) .. (142.6840,98.9632) .. controls (143.1860,97.9790) and
        (144.8760,95.1989) .. (145.1220,94.0450) .. controls (145.5860,92.0247) and
        (146.5000,90.5247) .. (145.8160,88.4375) .. controls (145.5780,87.7654) and
        (144.0700,87.6407) .. (143.4060,88.0934) .. controls (141.5170,89.3059) and
        (141.8520,90.6905) .. (141.6840,91.9847) .. controls (141.0990,95.6559) and
        (138.7290,99.4265) .. (138.5760,102.1880) -- cycle;
      \path[fill=cffd028] (138.8450,101.7410) .. controls (138.7760,103.4620) and
        (142.5520,99.0168) .. (142.7770,98.6295) .. controls (143.2550,97.6825) and
        (144.8680,95.0067) .. (145.0990,93.8976) .. controls (145.5380,91.9561) and
        (146.4090,90.5123) .. (145.7450,88.5107) .. controls (145.5130,87.8662) and
        (144.0650,87.7523) .. (143.4290,88.1895) .. controls (141.6190,89.3608) and
        (141.9470,90.6891) .. (141.7900,91.9325) .. controls (141.2420,95.4598) and
        (138.9810,99.0893) .. (138.8450,101.7410) -- cycle;
      \path[fill=cffd02b] (139.1130,101.2950) .. controls (139.0540,102.9450) and
        (142.6560,98.6680) .. (142.8700,98.2957) .. controls (143.3240,97.3860) and
        (144.8590,94.8144) .. (145.0760,93.7503) .. controls (145.4890,91.8874) and
        (146.3180,90.4998) .. (145.6730,88.5839) .. controls (145.4480,87.9670) and
        (144.0600,87.8638) .. (143.4520,88.2855) .. controls (141.7220,89.4157) and
        (142.0410,90.6877) .. (141.8960,91.8802) .. controls (141.3860,95.2637) and
        (139.2330,98.7520) .. (139.1130,101.2950) -- cycle;
      \path[fill=cffd02d] (139.3820,100.8480) .. controls (139.3330,102.4270) and
        (142.7600,98.3192) .. (142.9630,97.9620) .. controls (143.3940,97.0894) and
        (144.8500,94.6221) .. (145.0540,93.6030) .. controls (145.4400,91.8188) and
        (146.2270,90.4874) .. (145.6020,88.6571) .. controls (145.3840,88.0678) and
        (144.0550,87.9753) .. (143.4750,88.3814) .. controls (141.8250,89.4706) and
        (142.1360,90.6862) .. (142.0030,91.8279) .. controls (141.5290,95.0677) and
        (139.4850,98.4149) .. (139.3820,100.8480) -- cycle;
      \path[fill=cffd030] (139.6510,100.4020) .. controls (139.6120,101.9100) and
        (142.8630,97.9703) .. (143.0560,97.6282) .. controls (143.4630,96.7930) and
        (144.8420,94.4297) .. (145.0310,93.4557) .. controls (145.3910,91.7501) and
        (146.1360,90.4749) .. (145.5300,88.7303) .. controls (145.3190,88.1686) and
        (144.0490,88.0869) .. (143.4980,88.4774) .. controls (141.9280,89.5255) and
        (142.2310,90.6848) .. (142.1090,91.7757) .. controls (141.6730,94.8716) and
        (139.7370,98.0776) .. (139.6510,100.4020) -- cycle;
    \path[fill=cffd133] (139.9190,99.9554) .. controls (139.8900,101.3920) and
      (142.9670,97.6215) .. (143.1490,97.2945) .. controls (143.5320,96.4965) and
      (144.8330,94.2375) .. (145.0080,93.3084) .. controls (145.3420,91.6815) and
      (146.0450,90.4625) .. (145.4580,88.8035) .. controls (145.2540,88.2694) and
      (144.0440,88.1985) .. (143.5210,88.5735) .. controls (142.0300,89.5804) and
      (142.3250,90.6834) .. (142.2150,91.7234) .. controls (141.8160,94.6754) and
      (139.9890,97.7404) .. (139.9190,99.9554) -- cycle;
    \path[fill=cffcc00] (157.2420,96.9904) .. controls (154.0740,100.4460) and
      (151.8420,103.7580) .. (150.0420,105.9900) .. controls (148.1700,108.2940) and
      (143.5620,111.0300) .. (145.8660,113.6940) .. controls (147.8100,116.0700) and
      (155.8020,111.7500) .. (161.9940,107.1420) .. controls (168.1140,102.5340) and
      (177.6900,98.4305) .. (173.0100,93.1744) .. controls (170.5620,90.5105) and
      (164.8020,91.0865) .. (162.5700,92.5265) .. controls (160.8420,93.6064) and
      (159.8340,94.1825) .. (157.2420,96.9904) -- cycle;
      \path[fill=cffcc02] (157.3780,97.1439) .. controls (154.2600,100.5220) and
        (152.0650,103.7720) .. (150.2920,105.9530) .. controls (148.4510,108.2020) and
        (143.9170,110.8980) .. (146.1620,113.4870) .. controls (148.0550,115.7970) and
        (155.8860,111.5440) .. (161.9570,107.0180) .. controls (167.9580,102.4930) and
        (177.3370,98.4438) .. (172.7770,93.3364) .. controls (170.3910,90.7481) and
        (164.7550,91.3387) .. (162.5680,92.7525) .. controls (160.8760,93.8141) and
        (159.9030,94.4223) .. (157.3780,97.1439) -- cycle;
      \path[fill=cffcc05] (157.5150,97.2973) .. controls (154.4460,100.5970) and
        (152.2890,103.7850) .. (150.5420,105.9160) .. controls (148.7320,108.1090) and
        (144.2710,110.7660) .. (146.4570,113.2800) .. controls (148.3000,115.5230) and
        (155.9700,111.3370) .. (161.9200,106.8940) .. controls (167.8020,102.4520) and
        (176.9840,98.4572) .. (172.5430,93.4984) .. controls (170.2200,90.9857) and
        (164.7080,91.5911) .. (162.5650,92.9785) .. controls (160.9100,94.0218) and
        (159.9720,94.6620) .. (157.5150,97.2973) -- cycle;
      \path[fill=cffcc07] (157.6510,97.4508) .. controls (154.6320,100.6720) and
        (152.5120,103.7980) .. (150.7920,105.8790) .. controls (149.0120,108.0170) and
        (144.6260,110.6330) .. (146.7520,113.0730) .. controls (148.5460,115.2490) and
        (156.0540,111.1310) .. (161.8830,106.7700) .. controls (167.6450,102.4110) and
        (176.6310,98.4706) .. (172.3100,93.6603) .. controls (170.0490,91.2233) and
        (164.6610,91.8434) .. (162.5630,93.2046) .. controls (160.9440,94.2294) and
        (160.0410,94.9019) .. (157.6510,97.4508) -- cycle;
      \path[fill=cffcd0a] (157.7870,97.6042) .. controls (154.8170,100.7470) and
        (152.7350,103.8110) .. (151.0420,105.8420) .. controls (149.2930,107.9240) and
        (144.9800,110.5010) .. (147.0480,112.8650) .. controls (148.7910,114.9760) and
        (156.1380,110.9240) .. (161.8460,106.6460) .. controls (167.4890,102.3700) and
        (176.2780,98.4840) .. (172.0770,93.8222) .. controls (169.8780,91.4609) and
        (164.6140,92.0956) .. (162.5600,93.4307) .. controls (160.9780,94.4370) and
        (160.1090,95.1417) .. (157.7870,97.6042) -- cycle;
      \path[fill=cffcd0c] (157.9230,97.7577) .. controls (155.0030,100.8230) and
        (152.9580,103.8240) .. (151.2920,105.8050) .. controls (149.5740,107.8320) and
        (145.3350,110.3690) .. (147.3430,112.6580) .. controls (149.0360,114.7020) and
        (156.2210,110.7180) .. (161.8090,106.5220) .. controls (167.3330,102.3280) and
        (175.9250,98.4975) .. (171.8430,93.9842) .. controls (169.7060,91.6985) and
        (164.5670,92.3480) .. (162.5580,93.6567) .. controls (161.0120,94.6447) and
        (160.1780,95.3815) .. (157.9230,97.7577) -- cycle;
      \path[fill=cffcd0f] (158.0600,97.9112) .. controls (155.1890,100.8980) and
        (153.1810,103.8370) .. (151.5420,105.7680) .. controls (149.8550,107.7390) and
        (145.6890,110.2360) .. (147.6390,112.4510) .. controls (149.2810,114.4290) and
        (156.3050,110.5110) .. (161.7720,106.3980) .. controls (167.1760,102.2870) and
        (175.5730,98.5109) .. (171.6100,94.1461) .. controls (169.5350,91.9361) and
        (164.5210,92.6003) .. (162.5560,93.8828) .. controls (161.0460,94.8524) and
        (160.2470,95.6213) .. (158.0600,97.9112) -- cycle;
      \path[fill=cffcd11] (158.1960,98.0646) .. controls (155.3750,100.9730) and
        (153.4040,103.8500) .. (151.7920,105.7310) .. controls (150.1350,107.6470) and
        (146.0440,110.1040) .. (147.9340,112.2430) .. controls (149.5260,114.1550) and
        (156.3890,110.3050) .. (161.7350,106.2740) .. controls (167.0200,102.2460) and
        (175.2200,98.5243) .. (171.3760,94.3081) .. controls (169.3640,92.1737) and
        (164.4740,92.8525) .. (162.5530,94.1088) .. controls (161.0800,95.0600) and
        (160.3160,95.8611) .. (158.1960,98.0646) -- cycle;
      \path[fill=cffce14] (158.3320,98.2180) .. controls (155.5610,101.0480) and
        (153.6270,103.8630) .. (152.0420,105.6940) .. controls (150.4160,107.5540) and
        (146.3980,109.9720) .. (148.2290,112.0360) .. controls (149.7710,113.8810) and
        (156.4730,110.0980) .. (161.6980,106.1500) .. controls (166.8640,102.2050) and
        (174.8670,98.5377) .. (171.1430,94.4700) .. controls (169.1930,92.4113) and
        (164.4270,93.1049) .. (162.5510,94.3349) .. controls (161.1140,95.2676) and
        (160.3850,96.1009) .. (158.3320,98.2180) -- cycle;
      \path[fill=cffce16] (158.4680,98.3715) .. controls (155.7460,101.1240) and
        (153.8510,103.8760) .. (152.2920,105.6570) .. controls (150.6970,107.4620) and
        (146.7530,109.8390) .. (148.5250,111.8290) .. controls (150.0160,113.6080) and
        (156.5570,109.8920) .. (161.6610,106.0260) .. controls (166.7080,102.1640) and
        (174.5140,98.5511) .. (170.9100,94.6320) .. controls (169.0220,92.6489) and
        (164.3800,93.3571) .. (162.5480,94.5609) .. controls (161.1480,95.4753) and
        (160.4530,96.3407) .. (158.4680,98.3715) -- cycle;
      \path[fill=cffce19] (158.6050,98.5249) .. controls (155.9320,101.1990) and
        (154.0740,103.8890) .. (152.5420,105.6200) .. controls (150.9780,107.3690) and
        (147.1070,109.7070) .. (148.8200,111.6210) .. controls (150.2610,113.3340) and
        (156.6410,109.6850) .. (161.6240,105.9020) .. controls (166.5510,102.1220) and
        (174.1610,98.5645) .. (170.6760,94.7939) .. controls (168.8510,92.8864) and
        (164.3330,93.6095) .. (162.5460,94.7870) .. controls (161.1820,95.6830) and
        (160.5220,96.5805) .. (158.6050,98.5249) -- cycle;
      \path[fill=cffce1c] (158.7410,98.6784) .. controls (156.1180,101.2740) and
        (154.2970,103.9030) .. (152.7920,105.5830) .. controls (151.2580,107.2760) and
        (147.4620,109.5750) .. (149.1160,111.4140) .. controls (150.5060,113.0600) and
        (156.7250,109.4790) .. (161.5870,105.7780) .. controls (166.3950,102.0810) and
        (173.8080,98.5779) .. (170.4430,94.9559) .. controls (168.6800,93.1241) and
        (164.2860,93.8618) .. (162.5430,95.0130) .. controls (161.2160,95.8906) and
        (160.5910,96.8203) .. (158.7410,98.6784) -- cycle;
      \path[fill=cffcf1e] (158.8770,98.8318) .. controls (156.3040,101.3490) and
        (154.5200,103.9160) .. (153.0420,105.5460) .. controls (151.5390,107.1840) and
        (147.8160,109.4420) .. (149.4110,111.2070) .. controls (150.7510,112.7870) and
        (156.8080,109.2720) .. (161.5500,105.6540) .. controls (166.2390,102.0400) and
        (173.4550,98.5912) .. (170.2090,95.1179) .. controls (168.5080,93.3617) and
        (164.2390,94.1140) .. (162.5410,95.2390) .. controls (161.2500,96.0983) and
        (160.6600,97.0601) .. (158.8770,98.8318) -- cycle;
      \path[fill=cffcf21] (159.0130,98.9853) .. controls (156.4900,101.4250) and
        (154.7430,103.9290) .. (153.2920,105.5090) .. controls (151.8200,107.0910) and
        (148.1710,109.3100) .. (149.7060,111.0000) .. controls (150.9970,112.5130) and
        (156.8920,109.0660) .. (161.5130,105.5300) .. controls (166.0820,101.9990) and
        (173.1020,98.6046) .. (169.9760,95.2798) .. controls (168.3370,93.5992) and
        (164.1920,94.3664) .. (162.5380,95.4651) .. controls (161.2840,96.3059) and
        (160.7290,97.2999) .. (159.0130,98.9853) -- cycle;
      \path[fill=cffcf23] (159.1500,99.1388) .. controls (156.6750,101.5000) and
        (154.9660,103.9420) .. (153.5420,105.4720) .. controls (152.1010,106.9990) and
        (148.5250,109.1780) .. (150.0020,110.7920) .. controls (151.2420,112.2390) and
        (156.9760,108.8590) .. (161.4760,105.4060) .. controls (165.9260,101.9580) and
        (172.7490,98.6180) .. (169.7430,95.4417) .. controls (168.1660,93.8369) and
        (164.1450,94.6187) .. (162.5360,95.6912) .. controls (161.3180,96.5135) and
        (160.7970,97.5397) .. (159.1500,99.1388) -- cycle;
      \path[fill=cffcf26] (159.2860,99.2922) .. controls (156.8610,101.5750) and
        (155.1890,103.9550) .. (153.7920,105.4350) .. controls (152.3810,106.9060) and
        (148.8800,109.0450) .. (150.2970,110.5850) .. controls (151.4870,111.9660) and
        (157.0600,108.6530) .. (161.4390,105.2820) .. controls (165.7700,101.9160) and
        (172.3960,98.6314) .. (169.5090,95.6037) .. controls (167.9950,94.0745) and
        (164.0980,94.8709) .. (162.5330,95.9172) .. controls (161.3510,96.7212) and
        (160.8660,97.7794) .. (159.2860,99.2922) -- cycle;
      \path[fill=cffd028] (159.4220,99.4456) .. controls (157.0470,101.6500) and
        (155.4130,103.9680) .. (154.0420,105.3980) .. controls (152.6620,106.8140) and
        (149.2340,108.9130) .. (150.5930,110.3780) .. controls (151.7320,111.6920) and
        (157.1440,108.4460) .. (161.4020,105.1580) .. controls (165.6130,101.8750) and
        (172.0430,98.6449) .. (169.2760,95.7657) .. controls (167.8240,94.3120) and
        (164.0510,95.1233) .. (162.5310,96.1432) .. controls (161.3850,96.9289) and
        (160.9350,98.0193) .. (159.4220,99.4456) -- cycle;
      \path[fill=cffd02b] (159.5580,99.5991) .. controls (157.2330,101.7260) and
        (155.6360,103.9810) .. (154.2920,105.3610) .. controls (152.9430,106.7210) and
        (149.5890,108.7810) .. (150.8880,110.1700) .. controls (151.9770,111.4180) and
        (157.2280,108.2400) .. (161.3650,105.0340) .. controls (165.4570,101.8340) and
        (171.6900,98.6583) .. (169.0420,95.9276) .. controls (167.6530,94.5497) and
        (164.0040,95.3755) .. (162.5290,96.3693) .. controls (161.4190,97.1365) and
        (161.0040,98.2591) .. (159.5580,99.5991) -- cycle;
      \path[fill=cffd02d] (159.6950,99.7525) .. controls (157.4190,101.8010) and
        (155.8590,103.9940) .. (154.5420,105.3240) .. controls (153.2240,106.6290) and
        (149.9430,108.6480) .. (151.1830,109.9630) .. controls (152.2220,111.1450) and
        (157.3120,108.0330) .. (161.3280,104.9100) .. controls (165.3010,101.7930) and
        (171.3370,98.6717) .. (168.8090,96.0895) .. controls (167.4820,94.7873) and
        (163.9570,95.6278) .. (162.5260,96.5954) .. controls (161.4530,97.3441) and
        (161.0730,98.4988) .. (159.6950,99.7525) -- cycle;
      \path[fill=cffd030] (159.8310,99.9060) .. controls (157.6040,101.8760) and
        (156.0820,104.0070) .. (154.7920,105.2870) .. controls (153.5040,106.5360) and
        (150.2980,108.5160) .. (151.4790,109.7560) .. controls (152.4670,110.8710) and
        (157.3950,107.8270) .. (161.2910,104.7860) .. controls (165.1450,101.7520) and
        (170.9840,98.6851) .. (168.5760,96.2515) .. controls (167.3100,95.0248) and
        (163.9100,95.8802) .. (162.5240,96.8214) .. controls (161.4870,97.5518) and
        (161.1410,98.7386) .. (159.8310,99.9060) -- cycle;
    \path[fill=cffd133] (159.9670,100.0590) .. controls (157.7900,101.9510) and
      (156.3050,104.0200) .. (155.0420,105.2500) .. controls (153.7850,106.4430) and
      (150.6520,108.3830) .. (151.7740,109.5480) .. controls (152.7120,110.5970) and
      (157.4790,107.6200) .. (161.2540,104.6620) .. controls (164.9880,101.7100) and
      (170.6310,98.6985) .. (168.3420,96.4135) .. controls (167.1390,95.2625) and
      (163.8630,96.1324) .. (162.5210,97.0475) .. controls (161.5210,97.7595) and
      (161.2100,98.9785) .. (159.9670,100.0590) -- cycle;
  \path[fill=cffcc00] (131.1060,119.9580) .. controls (134.6340,120.3180) and
    (137.8740,120.4620) .. (140.0340,120.4620) .. controls (145.5060,120.4620) and
    (152.8500,118.1580) .. (160.0500,114.4860) .. controls (167.1060,110.8140) and
    (169.2660,109.3740) .. (174.3060,106.4220) .. controls (179.1300,103.5420) and
    (184.8180,99.8705) .. (182.6580,98.7905) .. controls (180.4980,97.6385) and
    (177.9780,98.8625) .. (170.9940,103.2540) .. controls (164.2260,107.5020) and
    (160.4100,109.5180) .. (152.9220,112.8300) .. controls (150.0420,114.1260) and
    (145.5780,115.9260) .. (142.5540,116.0700) .. controls (139.6740,116.2140) and
    (135.7860,116.1420) .. (133.5540,116.1420) .. controls (130.8900,116.1420) and
    (127.4340,115.0620) .. (124.6260,113.8380) .. controls (124.6260,113.7660) and
    (131.1060,119.9580) .. (131.1060,119.9580) -- cycle;
  \path[fill=cb27f19] (124.9140,118.0140) .. controls (120.0900,115.4220) and
    (113.6820,112.0380) .. (112.3860,107.0700) .. controls (112.3140,106.7100) and
    (113.3220,106.1340) .. (113.7540,106.2780) -- (124.9140,118.0140) -- cycle;
    \path[fill=caf7c19] (124.9140,117.9420) .. controls (120.0180,115.3500) and
      (113.7540,111.9660) .. (112.5300,107.2140) .. controls (112.4580,106.8540) and
      (113.3940,106.2780) .. (113.8980,106.4940) .. controls (117.5700,110.1660) and
      (116.7060,109.2300) .. (124.9140,117.9420) -- cycle;
    \path[fill=caa7716] (124.8420,117.9420) .. controls (119.9460,115.2780) and
      (113.8260,111.8940) .. (112.6740,107.2860) .. controls (112.6020,106.9260) and
      (113.5380,106.3500) .. (114.1140,106.6380) .. controls (117.5700,110.1660) and
      (117.1380,109.6620) .. (124.8420,117.9420) -- cycle;
    \path[fill=ca87516] (124.8420,117.8700) .. controls (119.8740,115.2780) and
      (113.8980,111.8220) .. (112.7460,107.4300) .. controls (112.7460,107.0700) and
      (113.6100,106.4940) .. (114.2580,106.8540) .. controls (117.4980,110.1660) and
      (117.4980,110.0220) .. (124.8420,117.8700) -- cycle;
    \path[fill=ca37014] (124.7700,117.7980) .. controls (119.8740,115.2060) and
      (113.8980,111.8220) .. (112.8900,107.5020) .. controls (112.8180,107.2140) and
      (113.7540,106.6380) .. (114.4020,106.9980) .. controls (117.4980,110.1660) and
      (117.8580,110.3820) .. (124.7700,117.7980) -- cycle;
    \path[fill=ca06d14] (124.7700,117.7260) .. controls (119.8020,115.1340) and
      (113.9700,111.7500) .. (113.0340,107.6460) .. controls (112.9620,107.2860) and
      (113.8260,106.7100) .. (114.5460,107.2140) .. controls (117.4980,110.1660) and
      (118.2180,110.8140) .. (124.7700,117.7260) -- cycle;
    \path[fill=c9b6811] (124.7700,117.7260) .. controls (119.7300,115.0620) and
      (114.0420,111.6780) .. (113.1780,107.7180) .. controls (113.1060,107.4300) and
      (113.9700,106.8540) .. (114.7620,107.3580) .. controls (117.4980,110.1660) and
      (118.6500,111.1740) .. (124.7700,117.7260) -- cycle;
    \path[fill=c996611] (124.6980,117.6540) .. controls (119.6580,114.9900) and
      (114.1140,111.6060) .. (113.3220,107.8620) .. controls (113.2500,107.5740) and
      (114.0420,106.9980) .. (114.9060,107.5740) .. controls (117.4980,110.1660) and
      (119.0100,111.5340) .. (124.6980,117.6540) -- cycle;
    \path[fill=c966311] (124.6980,117.5820) .. controls (119.5860,114.9900) and
      (114.1860,111.5340) .. (113.3940,108.0060) .. controls (113.3940,107.6460) and
      (114.1140,107.0700) .. (115.0500,107.7180) .. controls (117.4260,110.1660) and
      (119.3700,111.9660) .. (124.6980,117.5820) -- cycle;
    \path[fill=c915e0f] (124.6980,117.5820) .. controls (119.5140,114.9180) and
      (114.2580,111.4620) .. (113.5380,108.0780) .. controls (113.5380,107.7900) and
      (114.2580,107.2140) .. (115.2660,107.9340) .. controls (117.4260,110.1660) and
      (119.8020,112.3260) .. (124.6980,117.5820) -- cycle;
    \path[fill=c8e5b0f] (124.6260,117.5100) .. controls (119.4420,114.8460) and
      (114.3300,111.3900) .. (113.6820,108.2220) .. controls (113.6820,107.9340) and
      (114.3300,107.3580) .. (115.4100,108.0780) .. controls (117.4260,110.1660) and
      (120.1620,112.6860) .. (124.6260,117.5100) -- cycle;
    \path[fill=c89560a] (124.6260,117.4380) .. controls (119.4420,114.7740) and
      (114.3300,111.3900) .. (113.8260,108.2940) .. controls (113.7540,108.0060) and
      (114.4740,107.4300) .. (115.5540,108.2940) .. controls (117.4260,110.0940) and
      (120.5220,113.1180) .. (124.6260,117.4380) -- cycle;
    \path[fill=c87540a] (124.5540,117.3660) .. controls (119.3700,114.7020) and
      (114.4020,111.3180) .. (113.9700,108.4380) .. controls (113.8980,108.1500) and
      (114.5460,107.5740) .. (115.6980,108.4380) .. controls (117.4260,110.0940) and
      (120.8820,113.4780) .. (124.5540,117.3660) -- cycle;
    \path[fill=c824f07] (124.5540,117.3660) .. controls (119.2980,114.6300) and
      (114.4740,111.2460) .. (114.1140,108.5100) .. controls (114.0420,108.2940) and
      (114.6900,107.7180) .. (115.9140,108.6540) .. controls (117.4260,110.0940) and
      (121.3140,113.8380) .. (124.5540,117.3660) -- cycle;
    \path[fill=c7f4c07] (124.5540,117.2940) .. controls (119.2260,114.6300) and
      (114.5460,111.1740) .. (114.1860,108.6540) .. controls (114.1860,108.3660) and
      (114.7620,107.7900) .. (116.0580,108.7980) .. controls (117.3540,110.0940) and
      (121.6740,114.2700) .. (124.5540,117.2940) -- cycle;
    \path[fill=c7c4907] (124.4820,117.2220) .. controls (119.1540,114.5580) and
      (114.6180,111.1020) .. (114.3300,108.7980) .. controls (114.3300,108.5100) and
      (114.8340,107.9340) .. (116.2020,109.0140) .. controls (117.3540,110.0940) and
      (122.0340,114.6300) .. (124.4820,117.2220) -- cycle;
    \path[fill=c774405] (124.4820,117.2220) .. controls (119.0820,114.4860) and
      (114.6900,111.0300) .. (114.4740,108.8700) .. controls (114.4740,108.6540) and
      (114.9780,108.0780) .. (116.4180,109.1580) .. controls (117.3540,110.0940) and
      (122.4660,114.9900) .. (124.4820,117.2220) -- cycle;
    \path[fill=c754205] (124.4820,117.1500) .. controls (119.0100,114.4140) and
      (114.7620,110.9580) .. (114.6180,109.0140) .. controls (114.6180,108.7260) and
      (115.0500,108.1500) .. (116.5620,109.3740) .. controls (117.3540,110.0940) and
      (122.8260,115.4220) .. (124.4820,117.1500) -- cycle;
    \path[fill=c703d02] (124.4100,117.0780) .. controls (119.0100,114.3420) and
      (114.7620,110.9580) .. (114.7620,109.0860) .. controls (114.6900,108.8700) and
      (115.1940,108.2940) .. (116.7060,109.5180) .. controls (117.3540,110.0940) and
      (123.1860,115.7820) .. (124.4100,117.0780) -- cycle;
    \path[fill=c6d3a02] (124.4100,117.0060) .. controls (118.9380,114.3420) and
      (114.8340,110.8860) .. (114.8340,109.2300) .. controls (114.8340,109.0140) and
      (115.2660,108.4380) .. (116.8500,109.7340) .. controls (117.2820,110.0940) and
      (123.5460,116.1420) .. (124.4100,117.0060) -- cycle;
    \path[fill=c683500] (124.3380,117.0060) .. controls (118.8660,114.2700) and
      (114.9060,110.8140) .. (114.9780,109.3020) .. controls (114.9780,109.0860) and
      (115.4100,108.5100) .. (117.0660,109.8780) .. controls (117.2820,110.0940) and
      (123.9780,116.5740) .. (124.3380,117.0060) -- cycle;
  \path[fill=c663300] (124.3380,116.9340) .. controls (118.7940,114.1980) and
    (114.9780,110.7420) .. (115.1220,109.4460) .. controls (115.1220,109.2300) and
    (115.4820,108.6540) .. (117.2100,110.0940) .. controls (117.2820,110.0940) and
    (124.3380,116.9340) .. (124.3380,116.9340) -- cycle;
  \path[fill=ccc9933] (130.6020,119.5980) .. controls (130.6020,119.5980) and
    (126.8580,118.7340) .. (125.2020,118.2300) .. controls (125.2020,118.2300) and
    (113.7540,106.2780) .. (113.8260,106.2780) .. controls (115.4820,106.9260) and
    (119.5860,110.8140) .. (121.3860,112.3980) .. controls (122.0340,112.9740) and
    (123.5460,113.8380) .. (125.2740,114.3420) .. controls (125.3460,114.3420) and
    (130.6020,119.5980) .. (130.6020,119.5980) -- cycle;
    \path[fill=cc69330] (130.6020,119.5260) .. controls (130.6020,119.5260) and
      (126.7140,118.6620) .. (125.0580,118.0860) .. controls (125.0580,118.0860) and
      (113.8260,106.4220) .. (113.8980,106.4220) .. controls (115.5540,107.0700) and
      (119.6580,110.9580) .. (121.4580,112.4700) .. controls (122.1060,113.0460) and
      (123.6900,113.9100) .. (125.3460,114.4140) .. controls (125.4180,114.4140) and
      (130.6020,119.5260) .. (130.6020,119.5260) -- cycle;
    \path[fill=cc18e2d] (130.5300,119.5260) .. controls (130.5300,119.5260) and
      (126.5700,118.5900) .. (124.9140,117.9420) .. controls (124.9140,117.9420) and
      (113.9700,106.5660) .. (114.0420,106.5660) .. controls (115.6260,107.2140) and
      (119.7300,111.0300) .. (121.5300,112.5420) .. controls (122.1780,113.1180) and
      (123.7620,114.0540) .. (125.4900,114.4860) .. controls (125.5620,114.4860) and
      (130.5300,119.5260) .. (130.5300,119.5260) -- cycle;
    \path[fill=cbc892b] (130.5300,119.4540) .. controls (130.5300,119.4540) and
      (126.4260,118.5180) .. (124.7700,117.8700) .. controls (124.7700,117.8700) and
      (114.0420,106.6380) .. (114.1140,106.6380) .. controls (115.6980,107.3580) and
      (119.8020,111.1740) .. (121.5300,112.6140) .. controls (122.2500,113.1900) and
      (123.9060,114.1260) .. (125.5620,114.6300) .. controls (125.6340,114.6300) and
      (130.5300,119.4540) .. (130.5300,119.4540) -- cycle;
    \path[fill=cb78428] (130.4580,119.4540) .. controls (130.4580,119.4540) and
      (126.3540,118.4460) .. (124.6260,117.7260) .. controls (124.6260,117.7260) and
      (114.1140,106.7820) .. (114.1860,106.7820) .. controls (115.7700,107.5020) and
      (119.8740,111.3180) .. (121.6020,112.7580) .. controls (122.3220,113.2620) and
      (124.0500,114.1980) .. (125.7060,114.7020) .. controls (125.7780,114.7020) and
      (130.4580,119.4540) .. (130.4580,119.4540) -- cycle;
    \path[fill=cb27f26] (130.4580,119.3820) .. controls (130.4580,119.3820) and
      (126.2100,118.3740) .. (124.4820,117.5820) .. controls (124.4820,117.5820) and
      (114.2580,106.9260) .. (114.3300,106.9260) .. controls (115.8420,107.6460) and
      (119.9460,111.4620) .. (121.6740,112.8300) .. controls (122.3940,113.3340) and
      (124.1220,114.3420) .. (125.7780,114.7740) .. controls (125.8500,114.7740) and
      (130.4580,119.3820) .. (130.4580,119.3820) -- cycle;
    \path[fill=caf7c26] (130.4580,119.3820) .. controls (130.4580,119.3820) and
      (126.0660,118.3740) .. (124.3380,117.4380) .. controls (124.3380,117.4380) and
      (114.3300,107.0700) .. (114.4020,107.0700) .. controls (115.9140,107.7900) and
      (120.0180,111.5340) .. (121.7460,112.9020) .. controls (122.4660,113.4060) and
      (124.2660,114.4140) .. (125.9220,114.8460) .. controls (125.9940,114.8460) and
      (130.4580,119.3820) .. (130.4580,119.3820) -- cycle;
    \path[fill=caa7723] (130.3860,119.3100) .. controls (130.3860,119.3100) and
      (125.9220,118.3020) .. (124.1940,117.3660) .. controls (124.1940,117.3660) and
      (114.4020,107.2140) .. (114.4740,107.2140) .. controls (115.9860,107.9340) and
      (120.0900,111.6780) .. (121.8180,112.9740) .. controls (122.5380,113.4780) and
      (124.4100,114.4860) .. (125.9940,114.9180) .. controls (126.0660,114.9180) and
      (130.3860,119.3100) .. (130.3860,119.3100) -- cycle;
    \path[fill=ca57221] (130.3860,119.2380) .. controls (130.3860,119.2380) and
      (125.7780,118.2300) .. (124.0500,117.2220) .. controls (124.0500,117.2220) and
      (114.5460,107.2860) .. (114.6180,107.2860) .. controls (116.0580,108.0780) and
      (120.1620,111.8220) .. (121.8180,113.0460) .. controls (122.6100,113.5500) and
      (124.4820,114.6300) .. (126.1380,115.0620) .. controls (126.2100,115.0620) and
      (130.3860,119.2380) .. (130.3860,119.2380) -- cycle;
    \path[fill=ca06d1e] (130.3860,119.2380) .. controls (130.3860,119.2380) and
      (125.6340,118.1580) .. (123.9060,117.0780) .. controls (123.9060,117.0780) and
      (114.6180,107.4300) .. (114.6900,107.4300) .. controls (116.1300,108.2220) and
      (120.2340,111.8940) .. (121.8900,113.1180) .. controls (122.6820,113.6220) and
      (124.6260,114.7020) .. (126.2100,115.1340) .. controls (126.2820,115.1340) and
      (130.3860,119.2380) .. (130.3860,119.2380) -- cycle;
    \path[fill=c9b681c] (130.3140,119.1660) .. controls (130.3140,119.1660) and
      (125.4900,118.0860) .. (123.7620,116.9340) .. controls (123.7620,116.9340) and
      (114.6900,107.5740) .. (114.7620,107.5740) .. controls (116.2020,108.3660) and
      (120.3060,112.0380) .. (121.9620,113.1900) .. controls (122.7540,113.6940) and
      (124.7700,114.7740) .. (126.3540,115.2060) .. controls (126.4260,115.2060) and
      (130.3140,119.1660) .. (130.3140,119.1660) -- cycle;
    \path[fill=c966316] (130.3140,119.1660) .. controls (130.3140,119.1660) and
      (125.4180,118.0140) .. (123.5460,116.8620) .. controls (123.5460,116.8620) and
      (114.8340,107.7180) .. (114.9060,107.7180) .. controls (116.3460,108.5820) and
      (120.3780,112.1820) .. (122.0340,113.3340) .. controls (122.8260,113.7660) and
      (124.8420,114.9180) .. (126.4260,115.2780) .. controls (126.4980,115.2780) and
      (130.3140,119.1660) .. (130.3140,119.1660) -- cycle;
    \path[fill=c915e14] (130.2420,119.0940) .. controls (130.2420,119.0940) and
      (125.2740,117.9420) .. (123.4020,116.7180) .. controls (123.4020,116.7180) and
      (114.9060,107.8620) .. (114.9780,107.8620) .. controls (116.4180,108.7260) and
      (120.4500,112.3260) .. (122.1060,113.4060) .. controls (122.8980,113.8380) and
      (124.9860,114.9900) .. (126.5700,115.3500) .. controls (126.6420,115.3500) and
      (130.2420,119.0940) .. (130.2420,119.0940) -- cycle;
    \path[fill=c8c5911] (130.2420,119.0940) .. controls (130.2420,119.0940) and
      (125.1300,117.8700) .. (123.2580,116.5740) .. controls (123.2580,116.5740) and
      (114.9780,108.0060) .. (115.0500,108.0060) .. controls (116.4900,108.8700) and
      (120.5220,112.3980) .. (122.1780,113.4780) .. controls (122.9700,113.9100) and
      (125.1300,115.0620) .. (126.6420,115.4220) .. controls (126.7140,115.4220) and
      (130.2420,119.0940) .. (130.2420,119.0940) -- cycle;
    \path[fill=c87540f] (130.2420,119.0220) .. controls (130.2420,119.0220) and
      (124.9860,117.7980) .. (123.1140,116.4300) .. controls (123.1140,116.4300) and
      (115.1220,108.0780) .. (115.1940,108.0780) .. controls (116.5620,109.0140) and
      (120.5940,112.5420) .. (122.1780,113.5500) .. controls (123.0420,113.9820) and
      (125.2020,115.2060) .. (126.7860,115.5660) .. controls (126.8580,115.5660) and
      (130.2420,119.0220) .. (130.2420,119.0220) -- cycle;
    \path[fill=c824f0c] (130.1700,118.9500) .. controls (130.1700,118.9500) and
      (124.8420,117.7260) .. (122.9700,116.3580) .. controls (122.9700,116.3580) and
      (115.1940,108.2220) .. (115.2660,108.2220) .. controls (116.6340,109.1580) and
      (120.6660,112.6860) .. (122.2500,113.6220) .. controls (123.1140,114.0540) and
      (125.3460,115.2780) .. (126.8580,115.6380) .. controls (126.9300,115.6380) and
      (130.1700,118.9500) .. (130.1700,118.9500) -- cycle;
    \path[fill=c7f4c0c] (130.1700,118.9500) .. controls (130.1700,118.9500) and
      (124.6980,117.7260) .. (122.8260,116.2140) .. controls (122.8260,116.2140) and
      (115.2660,108.3660) .. (115.3380,108.3660) .. controls (116.7060,109.3020) and
      (120.7380,112.7580) .. (122.3220,113.6940) .. controls (123.1860,114.1260) and
      (125.4900,115.3500) .. (127.0020,115.7100) .. controls (127.0740,115.7100) and
      (130.1700,118.9500) .. (130.1700,118.9500) -- cycle;
    \path[fill=c7a470a] (130.1700,118.8780) .. controls (130.1700,118.8780) and
      (124.5540,117.6540) .. (122.6820,116.0700) .. controls (122.6820,116.0700) and
      (115.4100,108.5100) .. (115.4820,108.5100) .. controls (116.7780,109.4460) and
      (120.8100,112.9020) .. (122.3940,113.7660) .. controls (123.2580,114.1980) and
      (125.5620,115.4940) .. (127.0740,115.7820) .. controls (127.1460,115.7820) and
      (130.1700,118.8780) .. (130.1700,118.8780) -- cycle;
    \path[fill=c754207] (130.0980,118.8780) .. controls (130.0980,118.8780) and
      (124.4820,117.5820) .. (122.5380,115.9260) .. controls (122.5380,115.9260) and
      (115.4820,108.6540) .. (115.5540,108.6540) .. controls (116.8500,109.5900) and
      (120.8820,113.0460) .. (122.4660,113.9100) .. controls (123.3300,114.2700) and
      (125.7060,115.5660) .. (127.2180,115.8540) .. controls (127.2900,115.8540) and
      (130.0980,118.8780) .. (130.0980,118.8780) -- cycle;
    \path[fill=c703d05] (130.0980,118.8060) .. controls (130.0980,118.8060) and
      (124.3380,117.5100) .. (122.3940,115.8540) .. controls (122.3940,115.8540) and
      (115.5540,108.7260) .. (115.6260,108.7260) .. controls (116.9220,109.7340) and
      (120.9540,113.1900) .. (122.4660,113.9820) .. controls (123.4020,114.3420) and
      (125.8500,115.6380) .. (127.2900,115.9980) .. controls (127.3620,115.9980) and
      (130.0980,118.8060) .. (130.0980,118.8060) -- cycle;
    \path[fill=c6b3802] (130.0260,118.8060) .. controls (130.0260,118.8060) and
      (124.1940,117.4380) .. (122.2500,115.7100) .. controls (122.2500,115.7100) and
      (115.6980,108.8700) .. (115.7700,108.8700) .. controls (116.9940,109.8780) and
      (121.0260,113.2620) .. (122.5380,114.0540) .. controls (123.4740,114.4140) and
      (125.9220,115.7820) .. (127.4340,116.0700) .. controls (127.5060,116.0700) and
      (130.0260,118.8060) .. (130.0260,118.8060) -- cycle;
  \path[fill=c663300] (130.0260,118.7340) .. controls (130.0260,118.7340) and
    (124.0500,117.3660) .. (122.1060,115.5660) .. controls (122.1060,115.5660) and
    (115.7700,109.0140) .. (115.8420,109.0140) .. controls (117.0660,110.0220) and
    (121.0980,113.4060) .. (122.6100,114.1260) .. controls (123.5460,114.4860) and
    (126.0660,115.8540) .. (127.5060,116.1420) .. controls (127.5780,116.1420) and
    (130.0260,118.7340) .. (130.0260,118.7340) -- cycle;
    \path[fill=cf7c400] (131.0340,119.8860) .. controls (134.5620,120.2460) and
      (137.8740,120.3900) .. (140.0340,120.3900) .. controls (145.5060,120.3900) and
      (152.7780,118.0860) .. (159.9780,114.4860) .. controls (166.9620,110.8140) and
      (169.1220,109.3740) .. (174.0900,106.4220) .. controls (178.9140,103.6140) and
      (184.6740,99.9424) .. (182.5140,98.8625) .. controls (180.3540,97.7104) and
      (177.9060,99.0064) .. (170.9220,103.3260) .. controls (164.1540,107.5740) and
      (160.2660,109.6620) .. (152.8500,112.9020) .. controls (149.9700,114.1980) and
      (145.5780,115.9980) .. (142.5540,116.1420) .. controls (139.6740,116.2860) and
      (135.7860,116.2140) .. (133.5540,116.2140) .. controls (130.8900,116.2140) and
      (127.5060,115.1340) .. (124.6980,113.9100) .. controls (124.6980,113.8380) and
      (131.0340,119.8860) .. (131.0340,119.8860) -- cycle;
    \path[fill=cefbc00] (130.8900,119.8140) .. controls (134.4900,120.2460) and
      (137.8740,120.3180) .. (140.1060,120.3180) .. controls (145.5060,120.3180) and
      (152.7060,118.0140) .. (159.8340,114.4140) .. controls (166.8900,110.8140) and
      (168.9060,109.3740) .. (173.9460,106.4940) .. controls (178.7700,103.6140) and
      (184.5300,100.0140) .. (182.3700,98.9344) .. controls (180.2100,97.8545) and
      (177.8340,99.1505) .. (170.8500,103.4700) .. controls (164.0820,107.6460) and
      (160.1220,109.8060) .. (152.8500,112.9740) .. controls (149.8980,114.2700) and
      (145.5780,116.0700) .. (142.5540,116.2140) .. controls (139.6740,116.3580) and
      (135.7860,116.2860) .. (133.6260,116.2860) .. controls (130.9620,116.2860) and
      (127.5780,115.2060) .. (124.7700,114.0540) .. controls (124.7700,113.9820) and
      (130.8900,119.8140) .. (130.8900,119.8140) -- cycle;
    \path[fill=ce8b500] (130.8180,119.7420) .. controls (134.4180,120.1740) and
      (137.8740,120.2460) .. (140.1060,120.2460) .. controls (145.5060,120.2460) and
      (152.6340,117.9420) .. (159.7620,114.4140) .. controls (166.7460,110.8140) and
      (168.7620,109.3740) .. (173.7300,106.4940) .. controls (178.5540,103.6860) and
      (184.3140,100.0860) .. (182.2260,99.0785) .. controls (180.0660,97.9265) and
      (177.6900,99.2945) .. (170.7780,103.5420) .. controls (164.0820,107.7180) and
      (159.9780,109.9500) .. (152.7780,113.0460) .. controls (149.8980,114.3420) and
      (145.5780,116.0700) .. (142.5540,116.2860) .. controls (139.6740,116.4300) and
      (135.8580,116.3580) .. (133.6260,116.3580) .. controls (130.9620,116.3580) and
      (127.6500,115.2780) .. (124.9140,114.1260) .. controls (124.9140,114.0540) and
      (130.8180,119.7420) .. (130.8180,119.7420) -- cycle;
    \path[fill=ce2af00] (130.7460,119.7420) .. controls (134.3460,120.1740) and
      (137.8740,120.1740) .. (140.1060,120.1740) .. controls (145.5060,120.1740) and
      (152.5620,117.9420) .. (159.6900,114.3420) .. controls (166.6740,110.8140) and
      (168.5460,109.3740) .. (173.5860,106.5660) .. controls (178.3380,103.7580) and
      (184.1700,100.2300) .. (182.0820,99.1505) .. controls (179.9220,98.0705) and
      (177.6180,99.4384) .. (170.7060,103.6860) .. controls (164.0100,107.7900) and
      (159.8340,110.0220) .. (152.7780,113.1180) .. controls (149.8260,114.4140) and
      (145.5780,116.1420) .. (142.5540,116.2860) .. controls (139.6740,116.5020) and
      (135.8580,116.4300) .. (133.6260,116.4300) .. controls (131.0340,116.4300) and
      (127.7220,115.3500) .. (124.9860,114.1980) .. controls (124.9860,114.1260) and
      (130.7460,119.7420) .. (130.7460,119.7420) -- cycle;
    \path[fill=cdba800] (130.6020,119.6700) .. controls (134.2740,120.1020) and
      (137.8740,120.1020) .. (140.1060,120.1020) .. controls (145.5060,120.1020) and
      (152.4900,117.8700) .. (159.6180,114.3420) .. controls (166.5300,110.8140) and
      (168.4020,109.3740) .. (173.3700,106.5660) .. controls (178.1220,103.8300) and
      (184.0260,100.3020) .. (181.9380,99.2224) .. controls (179.7780,98.1425) and
      (177.5460,99.5825) .. (170.6340,103.7580) .. controls (163.9380,107.8620) and
      (159.6900,110.1660) .. (152.7060,113.1900) .. controls (149.7540,114.4860) and
      (145.5780,116.2140) .. (142.5540,116.3580) .. controls (139.6740,116.5740) and
      (135.8580,116.5020) .. (133.6260,116.5020) .. controls (131.0340,116.5020) and
      (127.7940,115.4220) .. (125.0580,114.2700) .. controls (125.0580,114.1980) and
      (130.6020,119.6700) .. (130.6020,119.6700) -- cycle;
    \path[fill=cd3a000] (130.5300,119.5980) .. controls (134.2740,120.1020) and
      (137.8740,120.0300) .. (140.1780,120.0300) .. controls (145.5780,120.0300) and
      (152.4180,117.7980) .. (159.4740,114.2700) .. controls (166.4580,110.8140) and
      (168.2580,109.4460) .. (173.2260,106.6380) .. controls (177.9780,103.8300) and
      (183.8820,100.3740) .. (181.7220,99.2945) .. controls (179.6340,98.2144) and
      (177.4740,99.6544) .. (170.5620,103.9020) .. controls (163.8660,107.9340) and
      (159.5460,110.3100) .. (152.7060,113.3340) .. controls (149.6820,114.6300) and
      (145.5780,116.2860) .. (142.5540,116.4300) .. controls (139.6740,116.6460) and
      (135.8580,116.5740) .. (133.6980,116.5740) .. controls (131.1060,116.5740) and
      (127.8660,115.5660) .. (125.1300,114.4140) .. controls (125.1300,114.3420) and
      (130.5300,119.5980) .. (130.5300,119.5980) -- cycle;
    \path[fill=ccc9900] (130.4580,119.5260) .. controls (134.2020,120.0300) and
      (137.8740,119.9580) .. (140.1780,119.9580) .. controls (145.5780,119.9580) and
      (152.3460,117.7260) .. (159.4020,114.2700) .. controls (166.3140,110.8140) and
      (168.0420,109.4460) .. (173.0100,106.6380) .. controls (177.7620,103.9020) and
      (183.7380,100.4460) .. (181.5780,99.4384) .. controls (179.4900,98.3585) and
      (177.3300,99.7985) .. (170.4900,103.9740) .. controls (163.7940,108.0060) and
      (159.4020,110.4540) .. (152.6340,113.4060) .. controls (149.6820,114.7020) and
      (145.5780,116.2860) .. (142.5540,116.5020) .. controls (139.6740,116.7180) and
      (135.8580,116.6460) .. (133.6980,116.6460) .. controls (131.1060,116.6460) and
      (127.9380,115.6380) .. (125.2020,114.4860) .. controls (125.2020,114.4140) and
      (130.4580,119.5260) .. (130.4580,119.5260) -- cycle;
    \path[fill=cc49100] (130.3140,119.4540) .. controls (134.1300,119.9580) and
      (137.8740,119.8860) .. (140.1780,119.8860) .. controls (145.5780,119.8860) and
      (152.2740,117.6540) .. (159.3300,114.2700) .. controls (166.2420,110.8140) and
      (167.8980,109.4460) .. (172.7940,106.6380) .. controls (177.5460,103.9740) and
      (183.5220,100.5180) .. (181.4340,99.5105) .. controls (179.3460,98.4305) and
      (177.2580,99.9424) .. (170.4180,104.0460) .. controls (163.7940,108.0780) and
      (159.2580,110.5980) .. (152.6340,113.4780) .. controls (149.6100,114.7740) and
      (145.5780,116.3580) .. (142.5540,116.5740) .. controls (139.6740,116.7900) and
      (135.9300,116.7180) .. (133.6980,116.7180) .. controls (131.1060,116.7180) and
      (128.0100,115.7100) .. (125.3460,114.5580) .. controls (125.3460,114.4860) and
      (130.3140,119.4540) .. (130.3140,119.4540) -- cycle;
    \path[fill=cbc8900] (130.2420,119.3820) .. controls (134.0580,119.9580) and
      (137.8740,119.8140) .. (140.2500,119.8140) .. controls (145.5780,119.8140) and
      (152.2020,117.5820) .. (159.1860,114.1980) .. controls (166.0980,110.8140) and
      (167.7540,109.4460) .. (172.6500,106.7100) .. controls (177.4020,103.9740) and
      (183.3780,100.5900) .. (181.2900,99.5825) .. controls (179.2020,98.5024) and
      (177.1860,100.0860) .. (170.3460,104.1900) .. controls (163.7220,108.1500) and
      (159.1140,110.7420) .. (152.5620,113.5500) .. controls (149.5380,114.8460) and
      (145.5780,116.4300) .. (142.5540,116.6460) .. controls (139.6740,116.8620) and
      (135.9300,116.7900) .. (133.7700,116.7900) .. controls (131.1780,116.7900) and
      (128.0820,115.7820) .. (125.4180,114.7020) .. controls (125.4180,114.6300) and
      (130.2420,119.3820) .. (130.2420,119.3820) -- cycle;
    \path[fill=cb58200] (130.1700,119.3100) .. controls (133.9860,119.8860) and
      (137.8740,119.7420) .. (140.2500,119.7420) .. controls (145.5780,119.7420) and
      (152.1300,117.5100) .. (159.1140,114.1980) .. controls (166.0260,110.8140) and
      (167.5380,109.4460) .. (172.4340,106.7100) .. controls (177.1860,104.0460) and
      (183.2340,100.6620) .. (181.1460,99.6544) .. controls (179.0580,98.6465) and
      (177.1140,100.2300) .. (170.2740,104.2620) .. controls (163.6500,108.2220) and
      (158.9700,110.8860) .. (152.5620,113.6220) .. controls (149.4660,114.9180) and
      (145.5780,116.5020) .. (142.5540,116.7180) .. controls (139.6740,116.9340) and
      (135.9300,116.8620) .. (133.7700,116.8620) .. controls (131.1780,116.8620) and
      (128.1540,115.8540) .. (125.4900,114.7740) .. controls (125.4900,114.7020) and
      (130.1700,119.3100) .. (130.1700,119.3100) -- cycle;
    \path[fill=caf7c00] (130.0260,119.3100) .. controls (133.9140,119.8860) and
      (137.8020,119.5980) .. (140.2500,119.5980) .. controls (145.5780,119.5980) and
      (152.1300,117.5100) .. (159.0420,114.1260) .. controls (165.8820,110.8140) and
      (167.3940,109.4460) .. (172.2900,106.7820) .. controls (176.9700,104.1180) and
      (183.0900,100.8060) .. (181.0020,99.7985) .. controls (178.9140,98.7184) and
      (176.9700,100.3740) .. (170.2740,104.4060) .. controls (163.5780,108.3660) and
      (158.8980,110.9580) .. (152.4900,113.6940) .. controls (149.4660,114.9900) and
      (145.6500,116.5020) .. (142.6260,116.7180) .. controls (139.7460,116.9340) and
      (135.9300,116.8620) .. (133.7700,116.8620) .. controls (131.2500,116.8620) and
      (128.2260,115.9260) .. (125.5620,114.8460) .. controls (125.5620,114.7740) and
      (130.0260,119.3100) .. (130.0260,119.3100) -- cycle;
    \path[fill=ca87500] (129.9540,119.2380) .. controls (133.8420,119.8140) and
      (137.8020,119.5260) .. (140.2500,119.5260) .. controls (145.5780,119.5260) and
      (152.0580,117.4380) .. (158.9700,114.1260) .. controls (165.8100,110.8140) and
      (167.1780,109.4460) .. (172.0740,106.7820) .. controls (176.7540,104.1900) and
      (182.9460,100.8780) .. (180.8580,99.8705) .. controls (178.7700,98.8625) and
      (176.8980,100.5180) .. (170.2020,104.4780) .. controls (163.5060,108.4380) and
      (158.7540,111.1020) .. (152.4900,113.7660) .. controls (149.3940,115.0620) and
      (145.6500,116.5740) .. (142.6260,116.7900) .. controls (139.7460,117.0060) and
      (135.9300,116.9340) .. (133.7700,116.9340) .. controls (131.2500,116.9340) and
      (128.2980,115.9980) .. (125.6340,114.9180) .. controls (125.6340,114.8460) and
      (129.9540,119.2380) .. (129.9540,119.2380) -- cycle;
    \path[fill=ca06d00] (129.8820,119.1660) .. controls (133.7700,119.8140) and
      (137.8020,119.4540) .. (140.3220,119.4540) .. controls (145.5780,119.4540) and
      (151.9860,117.3660) .. (158.8260,114.0540) .. controls (165.6660,110.8140) and
      (167.0340,109.4460) .. (171.9300,106.8540) .. controls (176.6100,104.1900) and
      (182.8020,100.9500) .. (180.7140,99.9424) .. controls (178.6260,98.9344) and
      (176.8260,100.6620) .. (170.1300,104.6220) .. controls (163.4340,108.5100) and
      (158.6100,111.2460) .. (152.4180,113.8380) .. controls (149.3220,115.1340) and
      (145.6500,116.6460) .. (142.6260,116.8620) .. controls (139.7460,117.0780) and
      (135.9300,117.0060) .. (133.8420,117.0060) .. controls (131.3220,117.0060) and
      (128.3700,116.0700) .. (125.7060,115.0620) .. controls (125.7060,114.9900) and
      (129.8820,119.1660) .. (129.8820,119.1660) -- cycle;
    \path[fill=c996600] (129.7380,119.0940) .. controls (133.6980,119.7420) and
      (137.8020,119.3820) .. (140.3220,119.3820) .. controls (145.5780,119.3820) and
      (151.9140,117.2940) .. (158.7540,114.0540) .. controls (165.5940,110.8140) and
      (166.8900,109.4460) .. (171.7140,106.8540) .. controls (176.3940,104.2620) and
      (182.5860,101.0220) .. (180.5700,100.0140) .. controls (178.4820,99.0064) and
      (176.7540,100.8060) .. (170.0580,104.6940) .. controls (163.4340,108.5820) and
      (158.4660,111.3900) .. (152.4180,113.9100) .. controls (149.2500,115.2060) and
      (145.6500,116.7180) .. (142.6260,116.9340) .. controls (139.7460,117.1500) and
      (136.0020,117.0780) .. (133.8420,117.0780) .. controls (131.3220,117.0780) and
      (128.4420,116.1420) .. (125.8500,115.1340) .. controls (125.8500,115.0620) and
      (129.7380,119.0940) .. (129.7380,119.0940) -- cycle;
    \path[fill=c915e00] (129.6660,119.0220) .. controls (133.6260,119.6700) and
      (137.8020,119.3100) .. (140.3220,119.3100) .. controls (145.5780,119.3100) and
      (151.8420,117.2220) .. (158.6820,114.0540) .. controls (165.4500,110.8140) and
      (166.6740,109.4460) .. (171.4980,106.8540) .. controls (176.1780,104.3340) and
      (182.4420,101.0940) .. (180.4260,100.1580) .. controls (178.3380,99.1505) and
      (176.6100,100.9500) .. (169.9860,104.7660) .. controls (163.3620,108.6540) and
      (158.3220,111.5340) .. (152.3460,113.9820) .. controls (149.2500,115.2780) and
      (145.6500,116.7180) .. (142.6260,117.0060) .. controls (139.7460,117.2220) and
      (136.0020,117.1500) .. (133.8420,117.1500) .. controls (131.3220,117.1500) and
      (128.5140,116.2140) .. (125.9220,115.2060) .. controls (125.9220,115.1340) and
      (129.6660,119.0220) .. (129.6660,119.0220) -- cycle;
    \path[fill=c895600] (129.5940,118.9500) .. controls (133.6260,119.6700) and
      (137.8020,119.2380) .. (140.3940,119.2380) .. controls (145.6500,119.2380) and
      (151.7700,117.1500) .. (158.5380,113.9820) .. controls (165.3780,110.8140) and
      (166.5300,109.5180) .. (171.3540,106.9260) .. controls (176.0340,104.3340) and
      (182.2980,101.1660) .. (180.2100,100.2300) .. controls (178.1940,99.2224) and
      (176.5380,101.0220) .. (169.9140,104.9100) .. controls (163.2900,108.7260) and
      (158.1780,111.6780) .. (152.3460,114.1260) .. controls (149.1780,115.4220) and
      (145.6500,116.7900) .. (142.6260,117.0780) .. controls (139.7460,117.2940) and
      (136.0020,117.2220) .. (133.9140,117.2220) .. controls (131.3940,117.2220) and
      (128.5860,116.3580) .. (125.9940,115.3500) .. controls (125.9940,115.2780) and
      (129.5940,118.9500) .. (129.5940,118.9500) -- cycle;
    \path[fill=c824f00] (129.4500,118.8780) .. controls (133.5540,119.5980) and
      (137.8020,119.1660) .. (140.3940,119.1660) .. controls (145.6500,119.1660) and
      (151.6980,117.0780) .. (158.4660,113.9820) .. controls (165.2340,110.8140) and
      (166.3860,109.5180) .. (171.1380,106.9260) .. controls (175.8180,104.4060) and
      (182.1540,101.2380) .. (180.0660,100.3020) .. controls (178.0500,99.2945) and
      (176.4660,101.1660) .. (169.8420,104.9820) .. controls (163.2180,108.7980) and
      (158.0340,111.8220) .. (152.2740,114.1980) .. controls (149.1060,115.4940) and
      (145.6500,116.8620) .. (142.6260,117.1500) .. controls (139.7460,117.3660) and
      (136.0020,117.2940) .. (133.9140,117.2940) .. controls (131.3940,117.2940) and
      (128.6580,116.4300) .. (126.0660,115.4220) .. controls (126.0660,115.3500) and
      (129.4500,118.8780) .. (129.4500,118.8780) -- cycle;
    \path[fill=c7c4900] (129.3780,118.8780) .. controls (133.4820,119.5980) and
      (137.8020,119.0940) .. (140.3940,119.0940) .. controls (145.6500,119.0940) and
      (151.6260,117.0780) .. (158.3940,113.9100) .. controls (165.1620,110.8140) and
      (166.1700,109.5180) .. (170.9940,106.9980) .. controls (175.6020,104.4780) and
      (182.0100,101.3820) .. (179.9220,100.3740) .. controls (177.9060,99.4384) and
      (176.3940,101.3100) .. (169.7700,105.1260) .. controls (163.1460,108.8700) and
      (157.8900,111.8940) .. (152.2740,114.2700) .. controls (149.0340,115.5660) and
      (145.6500,116.9340) .. (142.6260,117.1500) .. controls (139.7460,117.4380) and
      (136.0020,117.3660) .. (133.9140,117.3660) .. controls (131.4660,117.3660) and
      (128.7300,116.5020) .. (126.1380,115.4940) .. controls (126.1380,115.4220) and
      (129.3780,118.8780) .. (129.3780,118.8780) -- cycle;
    \path[fill=c754200] (129.3060,118.8060) .. controls (133.4100,119.5260) and
      (137.8020,119.0220) .. (140.3940,119.0220) .. controls (145.6500,119.0220) and
      (151.5540,117.0060) .. (158.3220,113.9100) .. controls (165.0180,110.8140) and
      (166.0260,109.5180) .. (170.7780,106.9980) .. controls (175.3860,104.5500) and
      (181.7940,101.4540) .. (179.7780,100.5180) .. controls (177.7620,99.5105) and
      (176.2500,101.4540) .. (169.6980,105.1980) .. controls (163.1460,108.9420) and
      (157.7460,112.0380) .. (152.2020,114.3420) .. controls (149.0340,115.6380) and
      (145.6500,116.9340) .. (142.6260,117.2220) .. controls (139.7460,117.5100) and
      (136.0740,117.4380) .. (133.9140,117.4380) .. controls (131.4660,117.4380) and
      (128.8020,116.5740) .. (126.2820,115.5660) .. controls (126.2820,115.4940) and
      (129.3060,118.8060) .. (129.3060,118.8060) -- cycle;
    \path[fill=c6d3a00] (129.1620,118.7340) .. controls (133.3380,119.5260) and
      (137.8020,118.9500) .. (140.4660,118.9500) .. controls (145.6500,118.9500) and
      (151.4820,116.9340) .. (158.1780,113.8380) .. controls (164.9460,110.8140) and
      (165.8100,109.5180) .. (170.6340,107.0700) .. controls (175.2420,104.5500) and
      (181.6500,101.5260) .. (179.6340,100.5900) .. controls (177.6180,99.6544) and
      (176.1780,101.5980) .. (169.6260,105.3420) .. controls (163.0740,109.0140) and
      (157.6020,112.1820) .. (152.2020,114.4140) .. controls (148.9620,115.7100) and
      (145.6500,117.0060) .. (142.6260,117.2940) .. controls (139.7460,117.5820) and
      (136.0740,117.5100) .. (133.9860,117.5100) .. controls (131.5380,117.5100) and
      (128.8740,116.6460) .. (126.3540,115.7100) .. controls (126.3540,115.6380) and
      (129.1620,118.7340) .. (129.1620,118.7340) -- cycle;
  \path[fill=c663300] (129.0900,118.6620) .. controls (133.2660,119.4540) and
    (137.8020,118.8780) .. (140.4660,118.8780) .. controls (145.6500,118.8780) and
    (151.4100,116.8620) .. (158.1060,113.8380) .. controls (164.8020,110.8140) and
    (165.6660,109.5180) .. (170.4180,107.0700) .. controls (175.0260,104.6220) and
    (181.5060,101.5980) .. (179.4900,100.6620) .. controls (177.4740,99.7264) and
    (176.1060,101.7420) .. (169.5540,105.4140) .. controls (163.0020,109.0860) and
    (157.4580,112.3260) .. (152.1300,114.4860) .. controls (148.8900,115.7820) and
    (145.6500,117.0780) .. (142.6260,117.3660) .. controls (139.7460,117.6540) and
    (136.0740,117.5820) .. (133.9860,117.5820) .. controls (131.5380,117.5820) and
    (128.9460,116.7180) .. (126.4260,115.7820) .. controls (126.4260,115.7100) and
    (129.0900,118.6620) .. (129.0900,118.6620) -- cycle;
    \path[fill=cffffff] (164.0100,169.5660) .. controls (164.8740,166.7580) and
      (192.8820,160.1340) .. (197.4900,162.2940) .. controls (202.0260,164.4540) and
      (223.7700,196.0620) .. (219.8820,197.7900) .. controls (215.9940,199.4460) and
      (207.6420,187.2780) .. (195.4740,181.6620) .. controls (183.3060,176.0460) and
      (163.1460,172.4460) .. (164.0100,169.5660) -- cycle;
      \path[fill=cf9f9f9] (164.8350,169.6230) .. controls (165.6830,166.8620) and
        (192.9800,160.5470) .. (197.5250,162.6260) .. controls (202.0010,164.7050) and
        (223.2930,195.5230) .. (219.4690,197.2170) .. controls (215.6450,198.8420) and
        (207.5040,186.8710) .. (195.5280,181.3430) .. controls (183.5530,175.8160) and
        (163.9870,172.4530) .. (164.8350,169.6230) -- cycle;
      \path[fill=cf4f4f4] (165.6590,169.6800) .. controls (166.4910,166.9660) and
        (193.0770,160.9600) .. (197.5590,162.9580) .. controls (201.9760,164.9560) and
        (222.8170,194.9840) .. (219.0560,196.6430) .. controls (215.2960,198.2370) and
        (207.3660,186.4630) .. (195.5820,181.0240) .. controls (183.7990,175.5860) and
        (164.8270,172.4590) .. (165.6590,169.6800) -- cycle;
      \path[fill=cefefef] (166.4840,169.7370) .. controls (167.2990,167.0690) and
        (193.1750,161.3730) .. (197.5940,163.2900) .. controls (201.9520,165.2070) and
        (222.3400,194.4450) .. (218.6440,196.0690) .. controls (214.9470,197.6320) and
        (207.2270,186.0550) .. (195.6360,180.7050) .. controls (184.0450,175.3560) and
        (165.6680,172.4660) .. (166.4840,169.7370) -- cycle;
      \path[fill=ceaeaea] (167.3080,169.7940) .. controls (168.1080,167.1730) and
        (193.2720,161.7850) .. (197.6280,163.6210) .. controls (201.9270,165.4580) and
        (221.8630,193.9060) .. (218.2310,195.4950) .. controls (214.5980,197.0270) and
        (207.0890,185.6480) .. (195.6900,180.3860) .. controls (184.2910,175.1250) and
        (166.5090,172.4720) .. (167.3080,169.7940) -- cycle;
      \path[fill=ce5e5e5] (168.1330,169.8500) .. controls (168.9160,167.2770) and
        (193.3690,162.1980) .. (197.6630,163.9530) .. controls (201.9020,165.7090) and
        (221.3860,193.3670) .. (217.8180,194.9210) .. controls (214.2490,196.4220) and
        (206.9510,185.2400) .. (195.7440,180.0670) .. controls (184.5380,174.8950) and
        (167.3490,172.4780) .. (168.1330,169.8500) -- cycle;
      \path[fill=ce0e0e0] (168.9570,169.9070) .. controls (169.7250,167.3800) and
        (193.4670,162.6110) .. (197.6970,164.2850) .. controls (201.8770,165.9600) and
        (220.9090,192.8280) .. (217.4050,194.3480) .. controls (213.9000,195.8170) and
        (206.8130,184.8320) .. (195.7980,179.7480) .. controls (184.7840,174.6650) and
        (168.1900,172.4850) .. (168.9570,169.9070) -- cycle;
      \path[fill=cdbdbdb] (169.7820,169.9640) .. controls (170.5330,167.4840) and
        (193.5640,163.0230) .. (197.7320,164.6170) .. controls (201.8520,166.2100) and
        (220.4330,192.2880) .. (216.9920,193.7740) .. controls (213.5510,195.2120) and
        (206.6740,184.4250) .. (195.8520,179.4290) .. controls (185.0300,174.4350) and
        (169.0300,172.4910) .. (169.7820,169.9640) -- cycle;
      \path[fill=cd6d6d6] (170.6060,170.0210) .. controls (171.3410,167.5880) and
        (193.6620,163.4360) .. (197.7660,164.9480) .. controls (201.8270,166.4610) and
        (219.9560,191.7490) .. (216.5790,193.2000) .. controls (213.2020,194.6070) and
        (206.5360,184.0170) .. (195.9060,179.1100) .. controls (185.2770,174.2040) and
        (169.8710,172.4980) .. (170.6060,170.0210) -- cycle;
      \path[fill=cd1d1d1] (171.4310,170.0780) .. controls (172.1500,167.6910) and
        (193.7590,163.8490) .. (197.8010,165.2800) .. controls (201.8020,166.7120) and
        (219.4790,191.2100) .. (216.1660,192.6260) .. controls (212.8530,194.0020) and
        (206.3980,183.6090) .. (195.9600,178.7910) .. controls (185.5230,173.9740) and
        (170.7120,172.5040) .. (171.4310,170.0780) -- cycle;
      \path[fill=ccccccc] (172.2550,170.1340) .. controls (172.9580,167.7950) and
        (193.8570,164.2610) .. (197.8350,165.6120) .. controls (201.7770,166.9630) and
        (219.0020,190.6710) .. (215.7530,192.0520) .. controls (212.5040,193.3970) and
        (206.2600,183.2010) .. (196.0140,178.4720) .. controls (185.7690,173.7440) and
        (171.5520,172.5100) .. (172.2550,170.1340) -- cycle;
      \path[fill=cc6c6c6] (173.0800,170.1910) .. controls (173.7670,167.8990) and
        (193.9540,164.6740) .. (197.8700,165.9440) .. controls (201.7520,167.2140) and
        (218.5250,190.1320) .. (215.3400,191.4790) .. controls (212.1550,192.7930) and
        (206.1210,182.7940) .. (196.0680,178.1530) .. controls (186.0160,173.5140) and
        (172.3930,172.5170) .. (173.0800,170.1910) -- cycle;
      \path[fill=cc1c1c1] (173.9040,170.2480) .. controls (174.5750,168.0020) and
        (194.0520,165.0870) .. (197.9040,166.2750) .. controls (201.7270,167.4650) and
        (218.0490,189.5930) .. (214.9270,190.9050) .. controls (211.8060,192.1880) and
        (205.9830,182.3860) .. (196.1220,177.8340) .. controls (186.2620,173.2830) and
        (173.2330,172.5230) .. (173.9040,170.2480) -- cycle;
      \path[fill=cbcbcbc] (174.7290,170.3050) .. controls (175.3830,168.1060) and
        (194.1490,165.5000) .. (197.9390,166.6070) .. controls (201.7030,167.7150) and
        (217.5720,189.0530) .. (214.5150,190.3310) .. controls (211.4570,191.5830) and
        (205.8450,181.9780) .. (196.1760,177.5150) .. controls (186.5080,173.0530) and
        (174.0740,172.5300) .. (174.7290,170.3050) -- cycle;
      \path[fill=cb7b7b7] (175.5530,170.3620) .. controls (176.1920,168.2100) and
        (194.2470,165.9120) .. (197.9730,166.9390) .. controls (201.6780,167.9660) and
        (217.0950,188.5140) .. (214.1020,189.7570) .. controls (211.1080,190.9780) and
        (205.7070,181.5710) .. (196.2300,177.1960) .. controls (186.7540,172.8230) and
        (174.9150,172.5360) .. (175.5530,170.3620) -- cycle;
      \path[fill=cb2b2b2] (176.3780,170.4180) .. controls (177.0000,168.3130) and
        (194.3440,166.3250) .. (198.0080,167.2710) .. controls (201.6530,168.2170) and
        (216.6180,187.9750) .. (213.6890,189.1830) .. controls (210.7590,190.3730) and
        (205.5680,181.1630) .. (196.2840,176.8770) .. controls (187.0010,172.5930) and
        (175.7550,172.5420) .. (176.3780,170.4180) -- cycle;
      \path[fill=cadadad] (177.2020,170.4750) .. controls (177.8090,168.4170) and
        (194.4410,166.7380) .. (198.0420,167.6020) .. controls (201.6280,168.4680) and
        (216.1410,187.4360) .. (213.2760,188.6100) .. controls (210.4100,189.7680) and
        (205.4300,180.7550) .. (196.3380,176.5580) .. controls (187.2470,172.3620) and
        (176.5960,172.5490) .. (177.2020,170.4750) -- cycle;
      \path[fill=ca8a8a8] (178.0270,170.5320) .. controls (178.6170,168.5210) and
        (194.5390,167.1500) .. (198.0770,167.9340) .. controls (201.6030,168.7190) and
        (215.6650,186.8970) .. (212.8630,188.0360) .. controls (210.0610,189.1630) and
        (205.2920,180.3480) .. (196.3920,176.2390) .. controls (187.4930,172.1320) and
        (177.4360,172.5550) .. (178.0270,170.5320) -- cycle;
      \path[fill=ca3a3a3] (178.8510,170.5890) .. controls (179.4250,168.6240) and
        (194.6360,167.5630) .. (198.1110,168.2660) .. controls (201.5780,168.9700) and
        (215.1880,186.3580) .. (212.4500,187.4620) .. controls (209.7120,188.5580) and
        (205.1540,179.9400) .. (196.4460,175.9200) .. controls (187.7400,171.9020) and
        (178.2770,172.5620) .. (178.8510,170.5890) -- cycle;
      \path[fill=c9e9e9e] (179.6760,170.6460) .. controls (180.2340,168.7280) and
        (194.7340,167.9760) .. (198.1460,168.5980) .. controls (201.5530,169.2210) and
        (214.7110,185.8190) .. (212.0370,186.8880) .. controls (209.3630,187.9530) and
        (205.0150,179.5320) .. (196.5000,175.6010) .. controls (187.9860,171.6720) and
        (179.1180,172.5680) .. (179.6760,170.6460) -- cycle;
    \path[fill=c999999] (180.5000,170.7020) .. controls (181.0420,168.8310) and
      (194.8310,168.3880) .. (198.1800,168.9290) .. controls (201.5280,169.4710) and
      (214.2340,185.2790) .. (211.6240,186.3140) .. controls (209.0140,187.3480) and
      (204.8770,179.1240) .. (196.5540,175.2820) .. controls (188.2320,171.4410) and
      (179.9580,172.5740) .. (180.5000,170.7020) -- cycle;
    \path[fill=cffffff] (128.7300,212.6220) .. controls (131.6100,212.7660) and
      (128.8740,228.8940) .. (128.8740,245.5260) .. controls (128.8740,262.1580) and
      (131.1060,267.9900) .. (129.0180,270.0780) .. controls (126.9300,272.1660) and
      (123.3300,264.6780) .. (123.3300,248.0460) .. controls (123.3300,231.4140) and
      (125.8500,212.4780) .. (128.7300,212.6220) -- cycle;
      \path[fill=cf9f9f9] (128.7010,213.0130) .. controls (131.4840,213.1750) and
        (128.8110,229.0650) .. (128.7980,245.4320) .. controls (128.7860,261.7990) and
        (130.9850,267.6390) .. (128.9620,269.6700) .. controls (126.9420,271.6980) and
        (123.4010,264.2860) .. (123.4160,247.9170) .. controls (123.4280,231.5500) and
        (125.9180,212.8520) .. (128.7010,213.0130) -- cycle;
      \path[fill=cf4f4f4] (128.6710,213.4040) .. controls (131.3570,213.5830) and
        (128.7470,229.2360) .. (128.7220,245.3370) .. controls (128.6980,261.4390) and
        (130.8640,267.2890) .. (128.9060,269.2610) .. controls (126.9530,271.2290) and
        (123.4720,263.8940) .. (123.5010,247.7880) .. controls (123.5260,231.6860) and
        (125.9860,213.2250) .. (128.6710,213.4040) -- cycle;
      \path[fill=cefefef] (128.6420,213.7950) .. controls (131.2300,213.9920) and
        (128.6830,229.4070) .. (128.6460,245.2430) .. controls (128.6090,261.0790) and
        (130.7430,266.9380) .. (128.8500,268.8530) .. controls (126.9640,270.7600) and
        (123.5420,263.5020) .. (123.5870,247.6580) .. controls (123.6240,231.8220) and
        (126.0540,213.5990) .. (128.6420,213.7950) -- cycle;
      \path[fill=ceaeaea] (128.6130,214.1860) .. controls (131.1040,214.4000) and
        (128.6200,229.5770) .. (128.5700,245.1480) .. controls (128.5210,260.7190) and
        (130.6220,266.5870) .. (128.7940,268.4440) .. controls (126.9760,270.2920) and
        (123.6130,263.1100) .. (123.6720,247.5290) .. controls (123.7210,231.9580) and
        (126.1220,213.9720) .. (128.6130,214.1860) -- cycle;
      \path[fill=ce5e5e5] (128.5830,214.5770) .. controls (130.9770,214.8080) and
        (128.5560,229.7480) .. (128.4940,245.0540) .. controls (128.4330,260.3600) and
        (130.5010,266.2360) .. (128.7380,268.0360) .. controls (126.9870,269.8230) and
        (123.6840,262.7170) .. (123.7580,247.3990) .. controls (123.8190,232.0940) and
        (126.1890,214.3460) .. (128.5830,214.5770) -- cycle;
      \path[fill=ce0e0e0] (128.5540,214.9680) .. controls (130.8510,215.2170) and
        (128.4920,229.9190) .. (128.4190,244.9590) .. controls (128.3440,260.0000) and
        (130.3800,265.8850) .. (128.6820,267.6270) .. controls (126.9990,269.3540) and
        (123.7540,262.3250) .. (123.8430,247.2700) .. controls (123.9170,232.2300) and
        (126.2570,214.7190) .. (128.5540,214.9680) -- cycle;
      \path[fill=cdbdbdb] (128.5240,215.3590) .. controls (130.7240,215.6250) and
        (128.4290,230.0900) .. (128.3430,244.8650) .. controls (128.2560,259.6400) and
        (130.2590,265.5340) .. (128.6260,267.2190) .. controls (127.0100,268.8860) and
        (123.8250,261.9330) .. (123.9290,247.1410) .. controls (124.0150,232.3650) and
        (126.3250,215.0930) .. (128.5240,215.3590) -- cycle;
      \path[fill=cd6d6d6] (128.4950,215.7500) .. controls (130.5970,216.0330) and
        (128.3650,230.2600) .. (128.2670,244.7700) .. controls (128.1680,259.2800) and
        (130.1370,265.1830) .. (128.5690,266.8100) .. controls (127.0210,268.4170) and
        (123.8960,261.5410) .. (124.0140,247.0110) .. controls (124.1130,232.5010) and
        (126.3930,215.4660) .. (128.4950,215.7500) -- cycle;
      \path[fill=cd1d1d1] (128.4660,216.1410) .. controls (130.4710,216.4420) and
        (128.3010,230.4310) .. (128.1910,244.6760) .. controls (128.0800,258.9210) and
        (130.0160,264.8320) .. (128.5130,266.4010) .. controls (127.0330,267.9490) and
        (123.9670,261.1490) .. (124.1000,246.8820) .. controls (124.2100,232.6370) and
        (126.4610,215.8400) .. (128.4660,216.1410) -- cycle;
      \path[fill=ccccccc] (128.4360,216.5320) .. controls (130.3440,216.8500) and
        (128.2380,230.6020) .. (128.1150,244.5810) .. controls (127.9910,258.5610) and
        (129.8950,264.4810) .. (128.4570,265.9930) .. controls (127.0440,267.4800) and
        (124.0370,260.7560) .. (124.1850,246.7520) .. controls (124.3080,232.7730) and
        (126.5290,216.2130) .. (128.4360,216.5320) -- cycle;
      \path[fill=cc6c6c6] (128.4070,216.9230) .. controls (130.2180,217.2580) and
        (128.1740,230.7730) .. (128.0390,244.4870) .. controls (127.9030,258.2010) and
        (129.7740,264.1300) .. (128.4010,265.5840) .. controls (127.0560,267.0110) and
        (124.1080,260.3640) .. (124.2710,246.6230) .. controls (124.4060,232.9090) and
        (126.5970,216.5870) .. (128.4070,216.9230) -- cycle;
      \path[fill=cc1c1c1] (128.3770,217.3140) .. controls (130.0910,217.6670) and
        (128.1100,230.9430) .. (127.9630,244.3920) .. controls (127.8150,257.8410) and
        (129.6530,263.7790) .. (128.3450,265.1760) .. controls (127.0670,266.5430) and
        (124.1790,259.9720) .. (124.3560,246.4940) .. controls (124.5040,233.0450) and
        (126.6640,216.9600) .. (128.3770,217.3140) -- cycle;
      \path[fill=cbcbcbc] (128.3480,217.7050) .. controls (129.9640,218.0750) and
        (128.0470,231.1140) .. (127.8870,244.2980) .. controls (127.7260,257.4820) and
        (129.5320,263.4280) .. (128.2890,264.7670) .. controls (127.0780,266.0740) and
        (124.2490,259.5800) .. (124.4420,246.3640) .. controls (124.6020,233.1810) and
        (126.7320,217.3340) .. (128.3480,217.7050) -- cycle;
      \path[fill=cb7b7b7] (128.3190,218.0960) .. controls (129.8380,218.4830) and
        (127.9830,231.2850) .. (127.8110,244.2030) .. controls (127.6380,257.1220) and
        (129.4110,263.0770) .. (128.2330,264.3590) .. controls (127.0900,265.6050) and
        (124.3200,259.1880) .. (124.5270,246.2350) .. controls (124.6990,233.3160) and
        (126.8000,217.7070) .. (128.3190,218.0960) -- cycle;
      \path[fill=cb2b2b2] (128.2890,218.4870) .. controls (129.7110,218.8920) and
        (127.9190,231.4560) .. (127.7350,244.1090) .. controls (127.5500,256.7620) and
        (129.2900,262.7260) .. (128.1770,263.9500) .. controls (127.1010,265.1370) and
        (124.3910,258.7950) .. (124.6130,246.1050) .. controls (124.7970,233.4520) and
        (126.8680,218.0810) .. (128.2890,218.4870) -- cycle;
      \path[fill=cadadad] (128.2600,218.8780) .. controls (129.5850,219.3000) and
        (127.8560,231.6260) .. (127.6590,244.0140) .. controls (127.4610,256.4020) and
        (129.1690,262.3750) .. (128.1210,263.5420) .. controls (127.1130,264.6680) and
        (124.4610,258.4030) .. (124.6980,245.9760) .. controls (124.8950,233.5880) and
        (126.9360,218.4540) .. (128.2600,218.8780) -- cycle;
      \path[fill=ca8a8a8] (128.2300,219.2690) .. controls (129.4580,219.7080) and
        (127.7920,231.7970) .. (127.5830,243.9200) .. controls (127.3730,256.0430) and
        (129.0480,262.0240) .. (128.0650,263.1330) .. controls (127.1240,264.1990) and
        (124.5320,258.0110) .. (124.7840,245.8470) .. controls (124.9930,233.7240) and
        (127.0040,218.8280) .. (128.2300,219.2690) -- cycle;
      \path[fill=ca3a3a3] (128.2010,219.6600) .. controls (129.3310,220.1170) and
        (127.7290,231.9680) .. (127.5070,243.8250) .. controls (127.2850,255.6830) and
        (128.9260,261.6730) .. (128.0080,262.7250) .. controls (127.1350,263.7310) and
        (124.6030,257.6190) .. (124.8690,245.7170) .. controls (125.0910,233.8600) and
        (127.0720,219.2010) .. (128.2010,219.6600) -- cycle;
      \path[fill=c9e9e9e] (128.1720,220.0510) .. controls (129.2050,220.5250) and
        (127.6650,232.1390) .. (127.4310,243.7310) .. controls (127.1970,255.3230) and
        (128.8050,261.3220) .. (127.9520,262.3160) .. controls (127.1470,263.2620) and
        (124.6740,257.2270) .. (124.9550,245.5880) .. controls (125.1880,233.9960) and
        (127.1390,219.5750) .. (128.1720,220.0510) -- cycle;
    \path[fill=c999999] (128.1420,220.4410) .. controls (129.0780,220.9330) and
      (127.6010,232.3090) .. (127.3550,243.6360) .. controls (127.1080,254.9630) and
      (128.6840,260.9710) .. (127.8960,261.9070) .. controls (127.1580,262.7930) and
      (124.7440,256.8340) .. (125.0400,245.4580) .. controls (125.2860,234.1310) and
      (127.2070,219.9480) .. (128.1420,220.4410) -- cycle;
    \path[fill=cffffff] (135.0300,168.1580) .. controls (136.7660,168.3500) and
      (147.4670,165.9400) .. (147.8530,166.9040) .. controls (148.2380,167.6760) and
      (141.2010,169.7970) .. (138.8870,172.2070) .. controls (138.1160,172.9780) and
      (136.0910,174.8100) .. (134.8380,174.6170) .. controls (133.8740,174.5210) and
      (133.2950,172.4960) .. (131.8490,170.9530) .. controls (128.3780,167.4830) and
      (126.1610,167.7720) .. (126.8360,166.4220) .. controls (127.4140,165.3620) and
      (132.0420,167.8680) .. (135.0300,168.1580) -- cycle;
      \path[fill=cfbfbfb] (135.0750,168.1980) .. controls (136.7750,168.3860) and
        (147.2510,166.0270) .. (147.6290,166.9710) .. controls (148.0060,167.7260) and
        (141.1170,169.8030) .. (138.8510,172.1620) .. controls (138.0970,172.9170) and
        (136.1140,174.7110) .. (134.8870,174.5220) .. controls (133.9440,174.4280) and
        (133.3770,172.4450) .. (131.9610,170.9350) .. controls (128.5630,167.5380) and
        (126.3930,167.8200) .. (127.0540,166.4990) .. controls (127.6200,165.4610) and
        (132.1500,167.9140) .. (135.0750,168.1980) -- cycle;
      \path[fill=cf8f8f8] (135.1210,168.2380) .. controls (136.7840,168.4220) and
        (147.0350,166.1140) .. (147.4040,167.0370) .. controls (147.7730,167.7760) and
        (141.0320,169.8080) .. (138.8150,172.1170) .. controls (138.0770,172.8560) and
        (136.1370,174.6110) .. (134.9370,174.4260) .. controls (134.0130,174.3340) and
        (133.4590,172.3940) .. (132.0730,170.9160) .. controls (128.7480,167.5920) and
        (126.6250,167.8680) .. (127.2710,166.5750) .. controls (127.8250,165.5600) and
        (132.2580,167.9600) .. (135.1210,168.2380) -- cycle;
      \path[fill=cf5f5f5] (135.1660,168.2780) .. controls (136.7920,168.4580) and
        (146.8180,166.2000) .. (147.1800,167.1030) .. controls (147.5410,167.8260) and
        (140.9480,169.8140) .. (138.7800,172.0720) .. controls (138.0570,172.7940) and
        (136.1600,174.5110) .. (134.9860,174.3300) .. controls (134.0830,174.2400) and
        (133.5400,172.3430) .. (132.1850,170.8970) .. controls (128.9330,167.6460) and
        (126.8560,167.9160) .. (127.4890,166.6520) .. controls (128.0300,165.6590) and
        (132.3660,168.0060) .. (135.1660,168.2780) -- cycle;
      \path[fill=cf2f2f2] (135.2110,168.3180) .. controls (136.8010,168.4940) and
        (146.6020,166.2870) .. (146.9560,167.1700) .. controls (147.3080,167.8760) and
        (140.8630,169.8190) .. (138.7440,172.0260) .. controls (138.0370,172.7330) and
        (136.1830,174.4110) .. (135.0350,174.2340) .. controls (134.1520,174.1460) and
        (133.6220,172.2910) .. (132.2970,170.8780) .. controls (129.1180,167.7000) and
        (127.0880,167.9640) .. (127.7060,166.7280) .. controls (128.2350,165.7570) and
        (132.4740,168.0520) .. (135.2110,168.3180) -- cycle;
      \path[fill=cefefef] (135.2560,168.3580) .. controls (136.8100,168.5300) and
        (146.3860,166.3730) .. (146.7310,167.2360) .. controls (147.0760,167.9260) and
        (140.7780,169.8250) .. (138.7080,171.9810) .. controls (138.0180,172.6710) and
        (136.2050,174.3110) .. (135.0840,174.1380) .. controls (134.2210,174.0520) and
        (133.7030,172.2400) .. (132.4090,170.8590) .. controls (129.3030,167.7540) and
        (127.3190,168.0120) .. (127.9230,166.8040) .. controls (128.4410,165.8560) and
        (132.5820,168.0980) .. (135.2560,168.3580) -- cycle;
      \path[fill=cebebeb] (135.3010,168.3980) .. controls (136.8180,168.5660) and
        (146.1690,166.4600) .. (146.5070,167.3020) .. controls (146.8430,167.9760) and
        (140.6940,169.8300) .. (138.6720,171.9360) .. controls (137.9980,172.6100) and
        (136.2280,174.2110) .. (135.1330,174.0420) .. controls (134.2910,173.9580) and
        (133.7850,172.1890) .. (132.5220,170.8400) .. controls (129.4890,167.8080) and
        (127.5510,168.0600) .. (128.1410,166.8810) .. controls (128.6460,165.9550) and
        (132.6900,168.1440) .. (135.3010,168.3980) -- cycle;
      \path[fill=ce8e8e8] (135.3470,168.4380) .. controls (136.8270,168.6020) and
        (145.9530,166.5460) .. (146.2820,167.3690) .. controls (146.6110,168.0260) and
        (140.6090,169.8360) .. (138.6360,171.8910) .. controls (137.9780,172.5480) and
        (136.2510,174.1110) .. (135.1830,173.9460) .. controls (134.3600,173.8640) and
        (133.8670,172.1370) .. (132.6340,170.8220) .. controls (129.6740,167.8620) and
        (127.7830,168.1080) .. (128.3580,166.9570) .. controls (128.8510,166.0530) and
        (132.7980,168.1900) .. (135.3470,168.4380) -- cycle;
      \path[fill=ce5e5e5] (135.3920,168.4780) .. controls (136.8360,168.6380) and
        (145.7370,166.6330) .. (146.0580,167.4350) .. controls (146.3780,168.0760) and
        (140.5250,169.8410) .. (138.6000,171.8450) .. controls (137.9590,172.4870) and
        (136.2740,174.0110) .. (135.2320,173.8500) .. controls (134.4300,173.7700) and
        (133.9480,172.0860) .. (132.7460,170.8030) .. controls (129.8590,167.9160) and
        (128.0140,168.1560) .. (128.5760,167.0340) .. controls (129.0570,166.1520) and
        (132.9060,168.2360) .. (135.3920,168.4780) -- cycle;
      \path[fill=ce2e2e2] (135.4370,168.5180) .. controls (136.8450,168.6740) and
        (145.5210,166.7190) .. (145.8340,167.5010) .. controls (146.1460,168.1260) and
        (140.4400,169.8460) .. (138.5640,171.8000) .. controls (137.9390,172.4250) and
        (136.2970,173.9110) .. (135.2810,173.7540) .. controls (134.4990,173.6760) and
        (134.0300,172.0350) .. (132.8580,170.7840) .. controls (130.0440,167.9700) and
        (128.2460,168.2040) .. (128.7930,167.1100) .. controls (129.2620,166.2510) and
        (133.0140,168.2820) .. (135.4370,168.5180) -- cycle;
      \path[fill=cdfdfdf] (135.4820,168.5570) .. controls (136.8530,168.7090) and
        (145.3040,166.8060) .. (145.6090,167.5670) .. controls (145.9130,168.1760) and
        (140.3560,169.8520) .. (138.5280,171.7550) .. controls (137.9190,172.3640) and
        (136.3200,173.8110) .. (135.3300,173.6580) .. controls (134.5690,173.5820) and
        (134.1120,171.9830) .. (132.9700,170.7650) .. controls (130.2290,168.0240) and
        (128.4780,168.2520) .. (129.0110,167.1860) .. controls (129.4670,166.3490) and
        (133.1220,168.3280) .. (135.4820,168.5570) -- cycle;
      \path[fill=cdbdbdb] (135.5270,168.5970) .. controls (136.8620,168.7450) and
        (145.0880,166.8930) .. (145.3850,167.6340) .. controls (145.6810,168.2260) and
        (140.2710,169.8570) .. (138.4920,171.7100) .. controls (137.9000,172.3030) and
        (136.3430,173.7110) .. (135.3790,173.5630) .. controls (134.6380,173.4890) and
        (134.1930,171.9320) .. (133.0820,170.7460) .. controls (130.4140,168.0790) and
        (128.7090,168.3000) .. (129.2280,167.2630) .. controls (129.6730,166.4480) and
        (133.2300,168.3740) .. (135.5270,168.5970) -- cycle;
      \path[fill=cd8d8d8] (135.5730,168.6370) .. controls (136.8710,168.7810) and
        (144.8720,166.9790) .. (145.1600,167.7000) .. controls (145.4480,168.2760) and
        (140.1870,169.8630) .. (138.4560,171.6640) .. controls (137.8800,172.2410) and
        (136.3650,173.6110) .. (135.4290,173.4670) .. controls (134.7080,173.3950) and
        (134.2750,171.8810) .. (133.1940,170.7270) .. controls (130.5990,168.1330) and
        (128.9410,168.3480) .. (129.4460,167.3390) .. controls (129.8780,166.5470) and
        (133.3380,168.4200) .. (135.5730,168.6370) -- cycle;
      \path[fill=cd5d5d5] (135.6180,168.6770) .. controls (136.8790,168.8170) and
        (144.6550,167.0660) .. (144.9360,167.7660) .. controls (145.2160,168.3260) and
        (140.1020,169.8680) .. (138.4210,171.6190) .. controls (137.8600,172.1800) and
        (136.3880,173.5110) .. (135.4780,173.3710) .. controls (134.7770,173.3010) and
        (134.3570,171.8300) .. (133.3060,170.7080) .. controls (130.7840,168.1870) and
        (129.1730,168.3960) .. (129.6630,167.4160) .. controls (130.0830,166.6460) and
        (133.4460,168.4660) .. (135.6180,168.6770) -- cycle;
      \path[fill=cd2d2d2] (135.6630,168.7170) .. controls (136.8880,168.8530) and
        (144.4390,167.1520) .. (144.7120,167.8330) .. controls (144.9830,168.3760) and
        (140.0180,169.8740) .. (138.3850,171.5740) .. controls (137.8400,172.1180) and
        (136.4110,173.4110) .. (135.5270,173.2750) .. controls (134.8470,173.2070) and
        (134.4380,171.7780) .. (133.4180,170.6900) .. controls (130.9690,168.2410) and
        (129.4040,168.4440) .. (129.8810,167.4920) .. controls (130.2880,166.7440) and
        (133.5540,168.5120) .. (135.6630,168.7170) -- cycle;
      \path[fill=ccfcfcf] (135.7080,168.7570) .. controls (136.8970,168.8890) and
        (144.2230,167.2390) .. (144.4870,167.8990) .. controls (144.7510,168.4260) and
        (139.9330,169.8790) .. (138.3490,171.5290) .. controls (137.8210,172.0570) and
        (136.4340,173.3110) .. (135.5760,173.1790) .. controls (134.9160,173.1130) and
        (134.5200,171.7270) .. (133.5300,170.6710) .. controls (131.1540,168.2950) and
        (129.6360,168.4920) .. (130.0980,167.5680) .. controls (130.4940,166.8430) and
        (133.6620,168.5580) .. (135.7080,168.7570) -- cycle;
      \path[fill=ccccccc] (135.7530,168.7970) .. controls (136.9050,168.9250) and
        (144.0060,167.3250) .. (144.2630,167.9650) .. controls (144.5180,168.4760) and
        (139.8480,169.8850) .. (138.3130,171.4830) .. controls (137.8010,171.9950) and
        (136.4570,173.2110) .. (135.6250,173.0830) .. controls (134.9850,173.0190) and
        (134.6020,171.6760) .. (133.6420,170.6520) .. controls (131.3390,168.3490) and
        (129.8680,168.5400) .. (130.3150,167.6450) .. controls (130.6990,166.9420) and
        (133.7700,168.6040) .. (135.7530,168.7970) -- cycle;
      \path[fill=cc8c8c8] (135.7990,168.8370) .. controls (136.9140,168.9610) and
        (143.7900,167.4120) .. (144.0380,168.0320) .. controls (144.2860,168.5260) and
        (139.7640,169.8900) .. (138.2770,171.4380) .. controls (137.7810,171.9340) and
        (136.4800,173.1110) .. (135.6750,172.9870) .. controls (135.0550,172.9250) and
        (134.6830,171.6240) .. (133.7540,170.6330) .. controls (131.5240,168.4030) and
        (130.0990,168.5880) .. (130.5330,167.7210) .. controls (130.9040,167.0400) and
        (133.8780,168.6500) .. (135.7990,168.8370) -- cycle;
      \path[fill=cc5c5c5] (135.8440,168.8770) .. controls (136.9230,168.9970) and
        (143.5740,167.4980) .. (143.8140,168.0980) .. controls (144.0530,168.5760) and
        (139.6790,169.8960) .. (138.2410,171.3930) .. controls (137.7620,171.8720) and
        (136.5030,173.0110) .. (135.7240,172.8910) .. controls (135.1240,172.8310) and
        (134.7650,171.5730) .. (133.8660,170.6140) .. controls (131.7090,168.4570) and
        (130.3310,168.6360) .. (130.7500,167.7980) .. controls (131.1100,167.1390) and
        (133.9860,168.6960) .. (135.8440,168.8770) -- cycle;
      \path[fill=cc2c2c2] (135.8890,168.9170) .. controls (136.9320,169.0330) and
        (143.3580,167.5850) .. (143.5900,168.1640) .. controls (143.8210,168.6260) and
        (139.5950,169.9010) .. (138.2050,171.3480) .. controls (137.7420,171.8110) and
        (136.5250,172.9110) .. (135.7730,172.7950) .. controls (135.1940,172.7370) and
        (134.8470,171.5220) .. (133.9780,170.5950) .. controls (131.8940,168.5110) and
        (130.5630,168.6840) .. (130.9680,167.8740) .. controls (131.3150,167.2380) and
        (134.0940,168.7420) .. (135.8890,168.9170) -- cycle;
    \path[fill=cbfbfbf] (135.9340,168.9560) .. controls (136.9400,169.0680) and
      (143.1410,167.6710) .. (143.3650,168.2300) .. controls (143.5880,168.6760) and
      (139.5100,169.9060) .. (138.1690,171.3020) .. controls (137.7220,171.7490) and
      (136.5480,172.8110) .. (135.8220,172.6990) .. controls (135.2630,172.6430) and
      (134.9280,171.4700) .. (134.0900,170.5760) .. controls (132.0790,168.5650) and
      (130.7940,168.7320) .. (131.1850,167.9500) .. controls (131.5200,167.3360) and
      (134.2020,168.7880) .. (135.9340,168.9560) -- cycle;
    \path[fill=black] (170.5360,15.1695) .. controls (168.9390,17.3004) and
      (167.6960,32.1245) .. (171.6010,33.8115) .. controls (175.5070,35.4084) and
      (192.0180,30.1714) .. (191.4860,26.7985) .. controls (190.6870,21.3835) and
      (183.7630,15.7025) .. (180.1230,14.0154) .. controls (177.1050,12.5064) and
      (172.1340,13.0395) .. (170.5360,15.1695) -- cycle;
      \path[fill=c050505] (170.7460,15.4040) .. controls (169.1880,17.4832) and
        (167.9590,31.9290) .. (171.7720,33.5773) .. controls (175.5850,35.1401) and
        (191.5770,29.9891) .. (191.0550,26.6912) .. controls (190.2690,21.4054) and
        (183.6130,15.8920) .. (180.0660,14.2574) .. controls (177.1100,12.7941) and
        (172.3060,13.3257) .. (170.7460,15.4040) -- cycle;
      \path[fill=c0a0a0a] (170.9570,15.6385) .. controls (169.4370,17.6658) and
        (168.2230,31.7336) .. (171.9430,33.3432) .. controls (175.6630,34.8717) and
        (191.1350,29.8069) .. (190.6240,26.5840) .. controls (189.8510,21.4273) and
        (183.4640,16.0815) .. (180.0100,14.4993) .. controls (177.1150,13.0818) and
        (172.4770,13.6118) .. (170.9570,15.6385) -- cycle;
      \path[fill=c0f0f0f] (171.1670,15.8730) .. controls (169.6860,17.8485) and
        (168.4860,31.5381) .. (172.1130,33.1090) .. controls (175.7420,34.6034) and
        (190.6940,29.6245) .. (190.1940,26.4767) .. controls (189.4320,21.4493) and
        (183.3140,16.2711) .. (179.9530,14.7413) .. controls (177.1200,13.3695) and
        (172.6480,13.8981) .. (171.1670,15.8730) -- cycle;
      \path[fill=c141414] (171.3770,16.1075) .. controls (169.9350,18.0312) and
        (168.7490,31.3427) .. (172.2840,32.8748) .. controls (175.8200,34.3351) and
        (190.2520,29.4423) .. (189.7630,26.3694) .. controls (189.0140,21.4713) and
        (183.1640,16.4607) .. (179.8960,14.9832) .. controls (177.1250,13.6573) and
        (172.8200,14.1843) .. (171.3770,16.1075) -- cycle;
      \path[fill=c191919] (171.5870,16.3419) .. controls (170.1840,18.2140) and
        (169.0120,31.1472) .. (172.4540,32.6407) .. controls (175.8980,34.0667) and
        (189.8100,29.2599) .. (189.3320,26.2622) .. controls (188.5950,21.4932) and
        (183.0140,16.6502) .. (179.8390,15.2252) .. controls (177.1300,13.9449) and
        (172.9910,14.4705) .. (171.5870,16.3419) -- cycle;
      \path[fill=c1e1e1e] (171.7970,16.5765) .. controls (170.4330,18.3966) and
        (169.2750,30.9518) .. (172.6250,32.4066) .. controls (175.9760,33.7983) and
        (189.3690,29.0776) .. (188.9010,26.1550) .. controls (188.1770,21.5152) and
        (182.8640,16.8398) .. (179.7820,15.4672) .. controls (177.1350,14.2326) and
        (173.1630,14.7567) .. (171.7970,16.5765) -- cycle;
      \path[fill=c232323] (172.0080,16.8110) .. controls (170.6820,18.5793) and
        (169.5380,30.7563) .. (172.7960,32.1724) .. controls (176.0540,33.5300) and
        (188.9270,28.8954) .. (188.4700,26.0477) .. controls (187.7590,21.5371) and
        (182.7150,17.0293) .. (179.7250,15.7091) .. controls (177.1400,14.5204) and
        (173.3340,15.0428) .. (172.0080,16.8110) -- cycle;
      \path[fill=c282828] (172.2180,17.0455) .. controls (170.9310,18.7621) and
        (169.8010,30.5609) .. (172.9660,31.9383) .. controls (176.1320,33.2617) and
        (188.4860,28.7130) .. (188.0390,25.9405) .. controls (187.3400,21.5591) and
        (182.5650,17.2188) .. (179.6680,15.9510) .. controls (177.1450,14.8080) and
        (173.5050,15.3291) .. (172.2180,17.0455) -- cycle;
      \path[fill=c2d2d2d] (172.4280,17.2800) .. controls (171.1800,18.9447) and
        (170.0650,30.3654) .. (173.1370,31.7041) .. controls (176.2100,32.9933) and
        (188.0440,28.5308) .. (187.6080,25.8332) .. controls (186.9220,21.5810) and
        (182.4150,17.4084) .. (179.6120,16.1930) .. controls (177.1500,15.0958) and
        (173.6770,15.6153) .. (172.4280,17.2800) -- cycle;
      \path[fill=c333333] (172.6380,17.5145) .. controls (171.4290,19.1274) and
        (170.3280,30.1700) .. (173.3080,31.4699) .. controls (176.2880,32.7249) and
        (187.6030,28.3484) .. (187.1770,25.7260) .. controls (186.5040,21.6029) and
        (182.2650,17.5979) .. (179.5550,16.4350) .. controls (177.1550,15.3835) and
        (173.8480,15.9015) .. (172.6380,17.5145) -- cycle;
      \path[fill=c383838] (172.8480,17.7490) .. controls (171.6780,19.3102) and
        (170.5910,29.9745) .. (173.4780,31.2358) .. controls (176.3660,32.4566) and
        (187.1610,28.1661) .. (186.7460,25.6187) .. controls (186.0850,21.6249) and
        (182.1150,17.7875) .. (179.4980,16.6769) .. controls (177.1600,15.6711) and
        (174.0200,16.1877) .. (172.8480,17.7490) -- cycle;
      \path[fill=c3d3d3d] (173.0590,17.9835) .. controls (171.9270,19.4928) and
        (170.8540,29.7791) .. (173.6490,31.0016) .. controls (176.4440,32.1882) and
        (186.7200,27.9839) .. (186.3150,25.5114) .. controls (185.6670,21.6469) and
        (181.9660,17.9771) .. (179.4410,16.9189) .. controls (177.1650,15.9589) and
        (174.1910,16.4738) .. (173.0590,17.9835) -- cycle;
      \path[fill=c424242] (173.2690,18.2180) .. controls (172.1760,19.6755) and
        (171.1170,29.5836) .. (173.8200,30.7675) .. controls (176.5230,31.9199) and
        (186.2780,27.8015) .. (185.8850,25.4042) .. controls (185.2490,21.6688) and
        (181.8160,18.1666) .. (179.3840,17.1608) .. controls (177.1700,16.2466) and
        (174.3620,16.7601) .. (173.2690,18.2180) -- cycle;
      \path[fill=c474747] (173.4790,18.4525) .. controls (172.4250,19.8582) and
        (171.3800,29.3882) .. (173.9900,30.5334) .. controls (176.6010,31.6516) and
        (185.8370,27.6193) .. (185.4540,25.2969) .. controls (184.8300,21.6908) and
        (181.6660,18.3561) .. (179.3270,17.4028) .. controls (177.1750,16.5343) and
        (174.5340,17.0463) .. (173.4790,18.4525) -- cycle;
      \path[fill=c4c4c4c] (173.6890,18.6870) .. controls (172.6730,20.0409) and
        (171.6430,29.1927) .. (174.1610,30.2992) .. controls (176.6790,31.3832) and
        (185.3950,27.4370) .. (185.0230,25.1897) .. controls (184.4120,21.7127) and
        (181.5160,18.5457) .. (179.2700,17.6447) .. controls (177.1800,16.8220) and
        (174.7050,17.3325) .. (173.6890,18.6870) -- cycle;
      \path[fill=c515151] (173.8990,18.9215) .. controls (172.9220,20.2236) and
        (171.9070,28.9973) .. (174.3320,30.0651) .. controls (176.7570,31.1148) and
        (184.9530,27.2546) .. (184.5920,25.0825) .. controls (183.9940,21.7346) and
        (181.3660,18.7352) .. (179.2140,17.8867) .. controls (177.1850,17.1096) and
        (174.8770,17.6187) .. (173.8990,18.9215) -- cycle;
      \path[fill=c565656] (174.1100,19.1560) .. controls (173.1710,20.4063) and
        (172.1700,28.8018) .. (174.5020,29.8309) .. controls (176.8350,30.8465) and
        (184.5120,27.0724) .. (184.1610,24.9752) .. controls (183.5750,21.7566) and
        (181.2170,18.9248) .. (179.1570,18.1286) .. controls (177.1900,17.3974) and
        (175.0480,17.9048) .. (174.1100,19.1560) -- cycle;
      \path[fill=c5b5b5b] (174.3200,19.3905) .. controls (173.4200,20.5891) and
        (172.4330,28.6064) .. (174.6730,29.5967) .. controls (176.9130,30.5782) and
        (184.0700,26.8900) .. (183.7300,24.8680) .. controls (183.1570,21.7785) and
        (181.0670,19.1143) .. (179.1000,18.3705) .. controls (177.1950,17.6851) and
        (175.2190,18.1911) .. (174.3200,19.3905) -- cycle;
      \path[fill=c606060] (174.5300,19.6250) .. controls (173.6690,20.7717) and
        (172.6960,28.4109) .. (174.8440,29.3626) .. controls (176.9910,30.3098) and
        (183.6290,26.7078) .. (183.2990,24.7607) .. controls (182.7390,21.8005) and
        (180.9170,19.3039) .. (179.0430,18.6125) .. controls (177.2000,17.9728) and
        (175.3910,18.4773) .. (174.5300,19.6250) -- cycle;
    \path[fill=c666666] (174.7400,19.8595) .. controls (173.9180,20.9544) and
      (172.9590,28.2155) .. (175.0140,29.1284) .. controls (177.0690,30.0414) and
      (183.1870,26.5255) .. (182.8680,24.6534) .. controls (182.3200,21.8224) and
      (180.7670,19.4934) .. (178.9860,18.8545) .. controls (177.2050,18.2605) and
      (175.5620,18.7635) .. (174.7400,19.8595) -- cycle;
    \path[fill=black] (197.7480,108.5180) .. controls (197.0100,109.4200) and
      (198.3220,116.8830) .. (203.1600,121.8030) .. controls (207.9990,126.6410) and
      (210.9510,126.6410) .. (212.9190,124.5090) .. controls (216.6910,120.4910) and
      (213.6570,116.7180) .. (211.3610,114.0940) .. controls (209.0650,111.4700) and
      (206.1130,112.3720) .. (203.4060,109.7480) .. controls (200.7000,107.1240) and
      (198.8140,107.2880) .. (197.7480,108.5180) -- cycle;
      \path[fill=c050505] (197.8190,108.5850) .. controls (197.0860,109.4800) and
        (198.3880,116.8850) .. (203.1890,121.7670) .. controls (207.9900,126.5680) and
        (210.9190,126.5680) .. (212.8720,124.4520) .. controls (216.6140,120.4650) and
        (213.6040,116.7220) .. (211.3260,114.1180) .. controls (209.0480,111.5140) and
        (206.1190,112.4090) .. (203.4330,109.8060) .. controls (200.7480,107.2020) and
        (198.8760,107.3650) .. (197.8190,108.5850) -- cycle;
      \path[fill=c0a0a0a] (197.8890,108.6520) .. controls (197.1620,109.5400) and
        (198.4540,116.8870) .. (203.2170,121.7310) .. controls (207.9810,126.4940) and
        (210.8870,126.4940) .. (212.8240,124.3950) .. controls (216.5380,120.4390) and
        (213.5510,116.7250) .. (211.2900,114.1420) .. controls (209.0300,111.5580) and
        (206.1240,112.4460) .. (203.4590,109.8630) .. controls (200.7950,107.2800) and
        (198.9380,107.4410) .. (197.8890,108.6520) -- cycle;
      \path[fill=c0f0f0f] (197.9590,108.7190) .. controls (197.2380,109.6000) and
        (198.5200,116.8890) .. (203.2450,121.6950) .. controls (207.9710,126.4200) and
        (210.8540,126.4200) .. (212.7770,124.3370) .. controls (216.4610,120.4130) and
        (213.4970,116.7280) .. (211.2550,114.1650) .. controls (209.0120,111.6020) and
        (206.1290,112.4830) .. (203.4850,109.9210) .. controls (200.8420,107.3580) and
        (199.0000,107.5180) .. (197.9590,108.7190) -- cycle;
      \path[fill=c141414] (198.0290,108.7860) .. controls (197.3140,109.6600) and
        (198.5860,116.8910) .. (203.2730,121.6580) .. controls (207.9620,126.3460) and
        (210.8220,126.3460) .. (212.7290,124.2800) .. controls (216.3840,120.3870) and
        (213.4440,116.7310) .. (211.2190,114.1890) .. controls (208.9950,111.6460) and
        (206.1340,112.5200) .. (203.5120,109.9780) .. controls (200.8900,107.4350) and
        (199.0620,107.5940) .. (198.0290,108.7860) -- cycle;
      \path[fill=c191919] (198.1000,108.8530) .. controls (197.3900,109.7200) and
        (198.6510,116.8930) .. (203.3020,121.6220) .. controls (207.9520,126.2720) and
        (210.7900,126.2720) .. (212.6810,124.2230) .. controls (216.3070,120.3610) and
        (213.3910,116.7340) .. (211.1840,114.2120) .. controls (208.9770,111.6900) and
        (206.1400,112.5570) .. (203.5380,110.0350) .. controls (200.9370,107.5130) and
        (199.1240,107.6710) .. (198.1000,108.8530) -- cycle;
      \path[fill=c1e1e1e] (198.1700,108.9200) .. controls (197.4670,109.7800) and
        (198.7170,116.8950) .. (203.3300,121.5860) .. controls (207.9430,126.1980) and
        (210.7580,126.1980) .. (212.6340,124.1650) .. controls (216.2300,120.3350) and
        (213.3370,116.7380) .. (211.1490,114.2360) .. controls (208.9600,111.7340) and
        (206.1450,112.5940) .. (203.5640,110.0930) .. controls (200.9850,107.5910) and
        (199.1860,107.7470) .. (198.1700,108.9200) -- cycle;
      \path[fill=c232323] (198.2400,108.9870) .. controls (197.5430,109.8400) and
        (198.7830,116.8970) .. (203.3580,121.5490) .. controls (207.9340,126.1240) and
        (210.7250,126.1240) .. (212.5860,124.1080) .. controls (216.1530,120.3080) and
        (213.2840,116.7410) .. (211.1130,114.2600) .. controls (208.9420,111.7780) and
        (206.1500,112.6310) .. (203.5910,110.1500) .. controls (201.0320,107.6690) and
        (199.2480,107.8240) .. (198.2400,108.9870) -- cycle;
      \path[fill=c282828] (198.3110,109.0540) .. controls (197.6190,109.9000) and
        (198.8490,116.8990) .. (203.3870,121.5130) .. controls (207.9240,126.0500) and
        (210.6930,126.0500) .. (212.5390,124.0510) .. controls (216.0760,120.2820) and
        (213.2310,116.7440) .. (211.0780,114.2830) .. controls (208.9240,111.8220) and
        (206.1560,112.6680) .. (203.6170,110.2070) .. controls (201.0790,107.7460) and
        (199.3100,107.9000) .. (198.3110,109.0540) -- cycle;
      \path[fill=c2d2d2d] (198.3810,109.1210) .. controls (197.6950,109.9600) and
        (198.9150,116.9010) .. (203.4150,121.4770) .. controls (207.9150,125.9760) and
        (210.6610,125.9760) .. (212.4910,123.9930) .. controls (216.0000,120.2560) and
        (213.1780,116.7470) .. (211.0420,114.3070) .. controls (208.9070,111.8660) and
        (206.1610,112.7050) .. (203.6430,110.2650) .. controls (201.1270,107.8240) and
        (199.3720,107.9770) .. (198.3810,109.1210) -- cycle;
      \path[fill=c333333] (198.4510,109.1870) .. controls (197.7710,110.0190) and
        (198.9810,116.9020) .. (203.4430,121.4400) .. controls (207.9060,125.9020) and
        (210.6290,125.9020) .. (212.4440,123.9360) .. controls (215.9230,120.2300) and
        (213.1240,116.7500) .. (211.0070,114.3300) .. controls (208.8890,111.9100) and
        (206.1660,112.7420) .. (203.6700,110.3220) .. controls (201.1740,107.9020) and
        (199.4340,108.0530) .. (198.4510,109.1870) -- cycle;
      \path[fill=c383838] (198.5220,109.2540) .. controls (197.8470,110.0790) and
        (199.0470,116.9040) .. (203.4720,121.4040) .. controls (207.8960,125.8290) and
        (210.5960,125.8290) .. (212.3960,123.8790) .. controls (215.8460,120.2040) and
        (213.0710,116.7540) .. (210.9710,114.3540) .. controls (208.8720,111.9540) and
        (206.1720,112.7790) .. (203.6960,110.3790) .. controls (201.2220,107.9800) and
        (199.4960,108.1290) .. (198.5220,109.2540) -- cycle;
      \path[fill=c3d3d3d] (198.5920,109.3210) .. controls (197.9230,110.1390) and
        (199.1120,116.9060) .. (203.5000,121.3680) .. controls (207.8870,125.7550) and
        (210.5640,125.7550) .. (212.3490,123.8210) .. controls (215.7690,120.1780) and
        (213.0180,116.7570) .. (210.9360,114.3780) .. controls (208.8540,111.9980) and
        (206.1770,112.8160) .. (203.7220,110.4370) .. controls (201.2690,108.0570) and
        (199.5580,108.2060) .. (198.5920,109.3210) -- cycle;
      \path[fill=c424242] (198.6620,109.3880) .. controls (197.9990,110.1990) and
        (199.1780,116.9080) .. (203.5280,121.3320) .. controls (207.8780,125.6810) and
        (210.5320,125.6810) .. (212.3010,123.7640) .. controls (215.6920,120.1520) and
        (212.9640,116.7600) .. (210.9000,114.4010) .. controls (208.8360,112.0420) and
        (206.1820,112.8530) .. (203.7490,110.4940) .. controls (201.3160,108.1350) and
        (199.6200,108.2820) .. (198.6620,109.3880) -- cycle;
      \path[fill=c474747] (198.7320,109.4550) .. controls (198.0750,110.2590) and
        (199.2440,116.9100) .. (203.5560,121.2950) .. controls (207.8680,125.6070) and
        (210.5000,125.6070) .. (212.2540,123.7070) .. controls (215.6150,120.1250) and
        (212.9110,116.7630) .. (210.8650,114.4250) .. controls (208.8190,112.0860) and
        (206.1870,112.8900) .. (203.7750,110.5510) .. controls (201.3640,108.2130) and
        (199.6820,108.3590) .. (198.7320,109.4550) -- cycle;
      \path[fill=c4c4c4c] (198.8030,109.5220) .. controls (198.1510,110.3190) and
        (199.3100,116.9120) .. (203.5850,121.2590) .. controls (207.8590,125.5330) and
        (210.4670,125.5330) .. (212.2060,123.6490) .. controls (215.5380,120.0990) and
        (212.8580,116.7660) .. (210.8290,114.4480) .. controls (208.8010,112.1300) and
        (206.1930,112.9270) .. (203.8010,110.6090) .. controls (201.4110,108.2910) and
        (199.7440,108.4350) .. (198.8030,109.5220) -- cycle;
      \path[fill=c515151] (198.8730,109.5890) .. controls (198.2270,110.3790) and
        (199.3760,116.9140) .. (203.6130,121.2230) .. controls (207.8500,125.4590) and
        (210.4350,125.4590) .. (212.1580,123.5920) .. controls (215.4620,120.0730) and
        (212.8040,116.7700) .. (210.7940,114.4720) .. controls (208.7840,112.1740) and
        (206.1980,112.9640) .. (203.8280,110.6660) .. controls (201.4590,108.3680) and
        (199.8060,108.5120) .. (198.8730,109.5890) -- cycle;
      \path[fill=c565656] (198.9430,109.6560) .. controls (198.3030,110.4390) and
        (199.4420,116.9160) .. (203.6410,121.1860) .. controls (207.8400,125.3850) and
        (210.4030,125.3850) .. (212.1110,123.5350) .. controls (215.3850,120.0470) and
        (212.7510,116.7730) .. (210.7590,114.4960) .. controls (208.7660,112.2180) and
        (206.2030,113.0010) .. (203.8540,110.7230) .. controls (201.5060,108.4460) and
        (199.8680,108.5880) .. (198.9430,109.6560) -- cycle;
      \path[fill=c5b5b5b] (199.0140,109.7230) .. controls (198.3790,110.4990) and
        (199.5080,116.9180) .. (203.6700,121.1500) .. controls (207.8310,125.3110) and
        (210.3710,125.3110) .. (212.0630,123.4770) .. controls (215.3080,120.0210) and
        (212.6980,116.7760) .. (210.7230,114.5190) .. controls (208.7480,112.2620) and
        (206.2090,113.0380) .. (203.8810,110.7810) .. controls (201.5530,108.5240) and
        (199.9300,108.6650) .. (199.0140,109.7230) -- cycle;
      \path[fill=c606060] (199.0840,109.7900) .. controls (198.4550,110.5590) and
        (199.5730,116.9200) .. (203.6980,121.1140) .. controls (207.8220,125.2370) and
        (210.3380,125.2370) .. (212.0160,123.4200) .. controls (215.2310,119.9950) and
        (212.6450,116.7790) .. (210.6880,114.5430) .. controls (208.7310,112.3060) and
        (206.2140,113.0740) .. (203.9070,110.8380) .. controls (201.6010,108.6020) and
        (199.9920,108.7410) .. (199.0840,109.7900) -- cycle;
    \path[fill=c666666] (199.1540,109.8560) .. controls (198.5310,110.6180) and
      (199.6390,116.9210) .. (203.7260,121.0770) .. controls (207.8120,125.1630) and
      (210.3060,125.1630) .. (211.9680,123.3620) .. controls (215.1540,119.9680) and
      (212.5910,116.7820) .. (210.6520,114.5660) .. controls (208.7130,112.3490) and
      (206.2190,113.1110) .. (203.9330,110.8950) .. controls (201.6480,108.6790) and
      (200.0540,108.8170) .. (199.1540,109.8560) -- cycle;
    \path[fill=black] (121.0980,47.3824) .. controls (120.3780,49.5424) and
      (129.8100,52.4944) .. (131.8980,54.3665) .. controls (134.7060,56.8864) and
      (134.9220,61.8545) .. (138.2340,59.8384) .. controls (140.3940,58.5424) and
      (138.7380,55.6624) .. (134.7780,51.2704) .. controls (128.9460,44.7905) and
      (121.7460,45.5825) .. (121.0980,47.3824) -- cycle;
      \path[fill=c050505] (121.5870,47.5792) .. controls (120.8710,49.6524) and
        (129.8620,52.4790) .. (131.9230,54.3199) .. controls (134.6680,56.7766) and
        (134.8840,61.5686) .. (138.0690,59.6328) .. controls (140.1640,58.3785) and
        (138.5190,55.5607) .. (134.7270,51.3522) .. controls (129.1520,45.1498) and
        (122.2350,45.8481) .. (121.5870,47.5792) -- cycle;
      \path[fill=c0a0a0a] (122.0750,47.7758) .. controls (121.3650,49.7624) and
        (129.9140,52.4636) .. (131.9480,54.2735) .. controls (134.6310,56.6667) and
        (134.8460,61.2828) .. (137.9040,59.4270) .. controls (139.9330,58.2145) and
        (138.2990,55.4590) .. (134.6770,51.4341) .. controls (129.3570,45.5092) and
        (122.7240,46.1136) .. (122.0750,47.7758) -- cycle;
      \path[fill=c0f0f0f] (122.5630,47.9726) .. controls (121.8580,49.8723) and
        (129.9660,52.4481) .. (131.9730,54.2270) .. controls (134.5930,56.5567) and
        (134.8090,60.9969) .. (137.7380,59.2213) .. controls (139.7020,58.0506) and
        (138.0800,55.3573) .. (134.6260,51.5158) .. controls (129.5630,45.8685) and
        (123.2120,46.3792) .. (122.5630,47.9726) -- cycle;
      \path[fill=c141414] (123.0520,48.1693) .. controls (122.3510,49.9822) and
        (130.0180,52.4326) .. (131.9980,54.1805) .. controls (134.5550,56.4468) and
        (134.7710,60.7111) .. (137.5730,59.0157) .. controls (139.4710,57.8867) and
        (137.8600,55.2556) .. (134.5750,51.5977) .. controls (129.7680,46.2279) and
        (123.7010,46.6449) .. (123.0520,48.1693) -- cycle;
      \path[fill=c191919] (123.5400,48.3659) .. controls (122.8440,50.0922) and
        (130.0700,52.4172) .. (132.0230,54.1339) .. controls (134.5170,56.3369) and
        (134.7330,60.4252) .. (137.4080,58.8099) .. controls (139.2410,57.7227) and
        (137.6400,55.1540) .. (134.5240,51.6794) .. controls (129.9730,46.5872) and
        (124.1890,46.9105) .. (123.5400,48.3659) -- cycle;
      \path[fill=c1e1e1e] (124.0290,48.5627) .. controls (123.3370,50.2021) and
        (130.1220,52.4018) .. (132.0490,54.0875) .. controls (134.4790,56.2271) and
        (134.6950,60.1393) .. (137.2420,58.6042) .. controls (139.0100,57.5587) and
        (137.4210,55.0522) .. (134.4730,51.7613) .. controls (130.1790,46.9466) and
        (124.6780,47.1761) .. (124.0290,48.5627) -- cycle;
      \path[fill=c232323] (124.5170,48.7593) .. controls (123.8310,50.3121) and
        (130.1740,52.3863) .. (132.0740,54.0410) .. controls (134.4410,56.1172) and
        (134.6570,59.8535) .. (137.0770,58.3986) .. controls (138.7790,57.3948) and
        (137.2010,54.9506) .. (134.4230,51.8430) .. controls (130.3840,47.3059) and
        (125.1670,47.4417) .. (124.5170,48.7593) -- cycle;
      \path[fill=c282828] (125.0050,48.9561) .. controls (124.3240,50.4221) and
        (130.2260,52.3708) .. (132.0990,53.9944) .. controls (134.4030,56.0072) and
        (134.6190,59.5677) .. (136.9120,58.1928) .. controls (138.5490,57.2309) and
        (136.9820,54.8488) .. (134.3720,51.9249) .. controls (130.5900,47.6653) and
        (125.6550,47.7072) .. (125.0050,48.9561) -- cycle;
      \path[fill=c2d2d2d] (125.4940,49.1527) .. controls (124.8170,50.5320) and
        (130.2780,52.3554) .. (132.1240,53.9480) .. controls (134.3660,55.8973) and
        (134.5810,59.2818) .. (136.7470,57.9872) .. controls (138.3180,57.0669) and
        (136.7620,54.7472) .. (134.3210,52.0067) .. controls (130.7950,48.0246) and
        (126.1440,47.9728) .. (125.4940,49.1527) -- cycle;
      \path[fill=c333333] (125.9820,49.3495) .. controls (125.3100,50.6419) and
        (130.3300,52.3399) .. (132.1490,53.9015) .. controls (134.3280,55.7874) and
        (134.5430,58.9959) .. (136.5810,57.7815) .. controls (138.0870,56.9030) and
        (136.5430,54.6454) .. (134.2700,52.0885) .. controls (131.0010,48.3839) and
        (126.6330,48.2384) .. (125.9820,49.3495) -- cycle;
      \path[fill=c383838] (126.4710,49.5461) .. controls (125.8030,50.7519) and
        (130.3820,52.3245) .. (132.1740,53.8549) .. controls (134.2900,55.6776) and
        (134.5050,58.7101) .. (136.4160,57.5757) .. controls (137.8570,56.7390) and
        (136.3230,54.5437) .. (134.2190,52.1703) .. controls (131.2060,48.7433) and
        (127.1210,48.5041) .. (126.4710,49.5461) -- cycle;
      \path[fill=c3d3d3d] (126.9590,49.7429) .. controls (126.2970,50.8618) and
        (130.4340,52.3091) .. (132.1990,53.8085) .. controls (134.2520,55.5677) and
        (134.4670,58.4243) .. (136.2510,57.3701) .. controls (137.6260,56.5750) and
        (136.1040,54.4420) .. (134.1690,52.2520) .. controls (131.4120,49.1027) and
        (127.6100,48.7697) .. (126.9590,49.7429) -- cycle;
      \path[fill=c424242] (127.4470,49.9396) .. controls (126.7900,50.9718) and
        (130.4860,52.2936) .. (132.2240,53.7620) .. controls (134.2140,55.4577) and
        (134.4300,58.1384) .. (136.0850,57.1643) .. controls (137.3950,56.4111) and
        (135.8840,54.3403) .. (134.1180,52.3339) .. controls (131.6170,49.4620) and
        (128.0990,49.0352) .. (127.4470,49.9396) -- cycle;
      \path[fill=c474747] (127.9360,50.1363) .. controls (127.2830,51.0818) and
        (130.5380,52.2782) .. (132.2490,53.7155) .. controls (134.1760,55.3478) and
        (134.3920,57.8526) .. (135.9200,56.9586) .. controls (137.1640,56.2472) and
        (135.6650,54.2386) .. (134.0670,52.4156) .. controls (131.8230,49.8214) and
        (128.5870,49.3008) .. (127.9360,50.1363) -- cycle;
      \path[fill=c4c4c4c] (128.4240,50.3330) .. controls (127.7760,51.1917) and
        (130.5900,52.2627) .. (132.2740,53.6690) .. controls (134.1380,55.2379) and
        (134.3540,57.5667) .. (135.7550,56.7530) .. controls (136.9340,56.0832) and
        (135.4450,54.1369) .. (134.0160,52.4975) .. controls (132.0280,50.1807) and
        (129.0760,49.5664) .. (128.4240,50.3330) -- cycle;
      \path[fill=c515151] (128.9130,50.5297) .. controls (128.2690,51.3017) and
        (130.6420,52.2473) .. (132.2990,53.6225) .. controls (134.1010,55.1281) and
        (134.3160,57.2809) .. (135.5890,56.5472) .. controls (136.7030,55.9193) and
        (135.2250,54.0352) .. (133.9650,52.5793) .. controls (132.2330,50.5400) and
        (129.5650,49.8320) .. (128.9130,50.5297) -- cycle;
      \path[fill=c565656] (129.4010,50.7264) .. controls (128.7630,51.4116) and
        (130.6940,52.2318) .. (132.3240,53.5760) .. controls (134.0630,55.0182) and
        (134.2780,56.9950) .. (135.4240,56.3416) .. controls (136.4720,55.7553) and
        (135.0060,53.9336) .. (133.9150,52.6611) .. controls (132.4390,50.8994) and
        (130.0530,50.0977) .. (129.4010,50.7264) -- cycle;
      \path[fill=c5b5b5b] (129.8890,50.9231) .. controls (129.2560,51.5215) and
        (130.7460,52.2163) .. (132.3490,53.5294) .. controls (134.0250,54.9082) and
        (134.2400,56.7091) .. (135.2590,56.1359) .. controls (136.2420,55.5914) and
        (134.7860,53.8318) .. (133.8640,52.7429) .. controls (132.6440,51.2588) and
        (130.5420,50.3633) .. (129.8890,50.9231) -- cycle;
      \path[fill=c606060] (130.3780,51.1198) .. controls (129.7490,51.6315) and
        (130.7980,52.2009) .. (132.3740,53.4830) .. controls (133.9870,54.7983) and
        (134.2020,56.4233) .. (135.0940,55.9301) .. controls (136.0110,55.4274) and
        (134.5670,53.7302) .. (133.8130,52.8247) .. controls (132.8500,51.6181) and
        (131.0310,50.6288) .. (130.3780,51.1198) -- cycle;
    \path[fill=c666666] (130.8660,51.3165) .. controls (130.2420,51.7415) and
      (130.8500,52.1855) .. (132.3990,53.4365) .. controls (133.9490,54.6884) and
      (134.1640,56.1375) .. (134.9280,55.7245) .. controls (135.7800,55.2635) and
      (134.3470,53.6284) .. (133.7620,52.9065) .. controls (133.0550,51.9774) and
      (131.5190,50.8944) .. (130.8660,51.3165) -- cycle;
    \path[fill=black] (171.0660,46.5905) .. controls (170.9940,47.6704) and
      (174.0180,48.3185) .. (175.9620,49.0385) .. controls (177.9060,49.6865) and
      (181.7220,52.2784) .. (183.5220,54.2945) .. controls (185.3220,56.2385) and
      (189.2100,61.9984) .. (190.0740,60.4865) .. controls (190.9380,59.1184) and
      (188.0580,55.1584) .. (187.2660,53.5024) .. controls (186.4740,51.8465) and
      (184.0260,48.2465) .. (180.1380,47.0225) .. controls (176.7540,46.0145) and
      (171.1380,45.7264) .. (171.0660,46.5905) -- cycle;
      \path[fill=c060606] (171.3750,46.6671) .. controls (171.3040,47.7174) and
        (174.2440,48.3518) .. (176.1330,49.0547) .. controls (178.0230,49.6876) and
        (181.7310,52.2153) .. (183.4790,54.1768) .. controls (185.2250,56.0698) and
        (189.0130,61.6698) .. (189.8540,60.2036) .. controls (190.6930,58.8758) and
        (187.9180,55.0300) .. (187.1150,53.4081) .. controls (186.3150,51.7864) and
        (183.9400,48.3177) .. (180.1630,47.1075) .. controls (176.8820,46.1116) and
        (171.4460,45.8269) .. (171.3750,46.6671) -- cycle;
      \path[fill=c0c0c0c] (171.6840,46.7439) .. controls (171.6130,47.7643) and
        (174.4690,48.3852) .. (176.3040,49.0710) .. controls (178.1400,49.6888) and
        (181.7390,52.1521) .. (183.4350,54.0592) .. controls (185.1270,55.9012) and
        (188.8160,61.3412) .. (189.6330,59.9208) .. controls (190.4470,58.6332) and
        (187.7770,54.9016) .. (186.9630,53.3138) .. controls (186.1550,51.7263) and
        (183.8540,48.3889) .. (180.1870,47.1924) .. controls (177.0100,46.2088) and
        (171.7540,45.9274) .. (171.6840,46.7439) -- cycle;
      \path[fill=c131313] (171.9930,46.8206) .. controls (171.9220,47.8113) and
        (174.6950,48.4185) .. (176.4760,49.0872) .. controls (178.2570,49.6899) and
        (181.7470,52.0890) .. (183.3910,53.9415) .. controls (185.0300,55.7325) and
        (188.6190,61.0125) .. (189.4130,59.6381) .. controls (190.2010,58.3906) and
        (187.6370,54.7731) .. (186.8110,53.2194) .. controls (185.9950,51.6662) and
        (183.7670,48.4602) .. (180.2120,47.2775) .. controls (177.1370,46.3059) and
        (172.0630,46.0280) .. (171.9930,46.8206) -- cycle;
      \path[fill=c191919] (172.3020,46.8972) .. controls (172.2320,47.8582) and
        (174.9200,48.4518) .. (176.6470,49.1035) .. controls (178.3740,49.6911) and
        (181.7560,52.0258) .. (183.3470,53.8239) .. controls (184.9330,55.5639) and
        (188.4220,60.6838) .. (189.1920,59.3553) .. controls (189.9560,58.1480) and
        (187.4960,54.6447) .. (186.6600,53.1250) .. controls (185.8360,51.6060) and
        (183.6810,48.5315) .. (180.2360,47.3625) .. controls (177.2650,46.4031) and
        (172.3710,46.1284) .. (172.3020,46.8972) -- cycle;
      \path[fill=c1f1f1f] (172.6110,46.9740) .. controls (172.5410,47.9052) and
        (175.1460,48.4852) .. (176.8180,49.1197) .. controls (178.4900,49.6922) and
        (181.7640,51.9627) .. (183.3040,53.7062) .. controls (184.8350,55.3952) and
        (188.2250,60.3552) .. (188.9720,59.0724) .. controls (189.7100,57.9054) and
        (187.3560,54.5162) .. (186.5080,53.0307) .. controls (185.6760,51.5460) and
        (183.5940,48.6027) .. (180.2610,47.4474) .. controls (177.3920,46.5002) and
        (172.6790,46.2289) .. (172.6110,46.9740) -- cycle;
      \path[fill=c262626] (172.9200,47.0507) .. controls (172.8510,47.9521) and
        (175.3710,48.5186) .. (176.9890,49.1360) .. controls (178.6070,49.6934) and
        (181.7730,51.8995) .. (183.2600,53.5886) .. controls (184.7380,55.2266) and
        (188.0280,60.0266) .. (188.7520,58.7896) .. controls (189.4650,57.6628) and
        (187.2150,54.3878) .. (186.3560,52.9363) .. controls (185.5170,51.4859) and
        (183.5080,48.6740) .. (180.2860,47.5325) .. controls (177.5200,46.5974) and
        (172.9870,46.3294) .. (172.9200,47.0507) -- cycle;
      \path[fill=c2c2c2c] (173.2290,47.1273) .. controls (173.1600,47.9991) and
        (175.5970,48.5519) .. (177.1600,49.1522) .. controls (178.7240,49.6945) and
        (181.7810,51.8364) .. (183.2160,53.4709) .. controls (184.6400,55.0579) and
        (187.8310,59.6979) .. (188.5310,58.5068) .. controls (189.2190,57.4203) and
        (187.0750,54.2593) .. (186.2050,52.8420) .. controls (185.3570,51.4258) and
        (183.4220,48.7452) .. (180.3100,47.6175) .. controls (177.6480,46.6945) and
        (173.2950,46.4299) .. (173.2290,47.1273) -- cycle;
      \path[fill=c333333] (173.5370,47.2040) .. controls (173.4690,48.0461) and
        (175.8220,48.5852) .. (177.3310,49.1685) .. controls (178.8410,49.6956) and
        (181.7890,51.7733) .. (183.1730,53.3533) .. controls (184.5430,54.8893) and
        (187.6340,59.3692) .. (188.3110,58.2241) .. controls (188.9730,57.1776) and
        (186.9340,54.1309) .. (186.0530,52.7477) .. controls (185.1970,51.3657) and
        (183.3350,48.8165) .. (180.3350,47.7025) .. controls (177.7750,46.7917) and
        (173.6030,46.5305) .. (173.5370,47.2040) -- cycle;
      \path[fill=c393939] (173.8460,47.2808) .. controls (173.7790,48.0930) and
        (176.0480,48.6186) .. (177.5020,49.1847) .. controls (178.9580,49.6968) and
        (181.7980,51.7101) .. (183.1290,53.2356) .. controls (184.4460,54.7206) and
        (187.4370,59.0406) .. (188.0900,57.9413) .. controls (188.7280,56.9351) and
        (186.7940,54.0024) .. (185.9010,52.6533) .. controls (185.0380,51.3055) and
        (183.2490,48.8877) .. (180.3590,47.7874) .. controls (177.9030,46.8888) and
        (173.9110,46.6310) .. (173.8460,47.2808) -- cycle;
      \path[fill=c3f3f3f] (174.1550,47.3575) .. controls (174.0880,48.1400) and
        (176.2730,48.6519) .. (177.6730,49.2010) .. controls (179.0750,49.6980) and
        (181.8060,51.6469) .. (183.0850,53.1180) .. controls (184.3480,54.5520) and
        (187.2400,58.7119) .. (187.8700,57.6584) .. controls (188.4820,56.6924) and
        (186.6530,53.8739) .. (185.7500,52.5589) .. controls (184.8780,51.2455) and
        (183.1630,48.9590) .. (180.3840,47.8725) .. controls (178.0310,46.9860) and
        (174.2190,46.7314) .. (174.1550,47.3575) -- cycle;
      \path[fill=c464646] (174.4640,47.4341) .. controls (174.3980,48.1869) and
        (176.4990,48.6853) .. (177.8440,49.2172) .. controls (179.1920,49.6991) and
        (181.8150,51.5838) .. (183.0420,53.0003) .. controls (184.2510,54.3833) and
        (187.0430,58.3833) .. (187.6490,57.3756) .. controls (188.2370,56.4498) and
        (186.5130,53.7455) .. (185.5980,52.4646) .. controls (184.7190,51.1854) and
        (183.0760,49.0302) .. (180.4080,47.9575) .. controls (178.1580,47.0831) and
        (174.5270,46.8319) .. (174.4640,47.4341) -- cycle;
      \path[fill=c4c4c4c] (174.7730,47.5109) .. controls (174.7070,48.2339) and
        (176.7240,48.7187) .. (178.0150,49.2335) .. controls (179.3080,49.7003) and
        (181.8230,51.5206) .. (182.9980,52.8827) .. controls (184.1530,54.2147) and
        (186.8460,58.0547) .. (187.4290,57.0928) .. controls (187.9910,56.2072) and
        (186.3720,53.6170) .. (185.4460,52.3702) .. controls (184.5590,51.1252) and
        (182.9900,49.1014) .. (180.4330,48.0424) .. controls (178.2860,47.1803) and
        (174.8350,46.9324) .. (174.7730,47.5109) -- cycle;
      \path[fill=c525252] (175.0820,47.5876) .. controls (175.0160,48.2808) and
        (176.9500,48.7520) .. (178.1870,49.2497) .. controls (179.4250,49.7014) and
        (181.8310,51.4575) .. (182.9540,52.7650) .. controls (184.0560,54.0460) and
        (186.6490,57.7260) .. (187.2080,56.8101) .. controls (187.7450,55.9646) and
        (186.2320,53.4886) .. (185.2950,52.2759) .. controls (184.3990,51.0652) and
        (182.9040,49.1727) .. (180.4570,48.1274) .. controls (178.4140,47.2774) and
        (175.1440,47.0330) .. (175.0820,47.5876) -- cycle;
      \path[fill=c595959] (175.3910,47.6642) .. controls (175.3260,48.3278) and
        (177.1750,48.7853) .. (178.3580,49.2660) .. controls (179.5420,49.7025) and
        (181.8400,51.3943) .. (182.9100,52.6474) .. controls (183.9590,53.8773) and
        (186.4520,57.3973) .. (186.9880,56.5273) .. controls (187.5000,55.7220) and
        (186.0910,53.3601) .. (185.1430,52.1815) .. controls (184.2400,51.0050) and
        (182.8170,49.2440) .. (180.4820,48.2125) .. controls (178.5410,47.3746) and
        (175.4520,47.1335) .. (175.3910,47.6642) -- cycle;
      \path[fill=c5f5f5f] (175.7000,47.7409) .. controls (175.6350,48.3747) and
        (177.4010,48.8187) .. (178.5290,49.2822) .. controls (179.6590,49.7037) and
        (181.8480,51.3312) .. (182.8670,52.5297) .. controls (183.8610,53.7087) and
        (186.2550,57.0687) .. (186.7670,56.2444) .. controls (187.2540,55.4795) and
        (185.9510,53.2317) .. (184.9910,52.0872) .. controls (184.0800,50.9449) and
        (182.7310,49.3152) .. (180.5060,48.2975) .. controls (178.6690,47.4717) and
        (175.7600,47.2339) .. (175.7000,47.7409) -- cycle;
      \path[fill=c666666] (176.0090,47.8177) .. controls (175.9450,48.4217) and
        (177.6260,48.8521) .. (178.7000,49.2985) .. controls (179.7760,49.7048) and
        (181.8570,51.2680) .. (182.8230,52.4121) .. controls (183.7640,53.5400) and
        (186.0580,56.7401) .. (186.5470,55.9616) .. controls (187.0090,55.2368) and
        (185.8100,53.1032) .. (184.8400,51.9929) .. controls (183.9210,50.8849) and
        (182.6450,49.3864) .. (180.5310,48.3824) .. controls (178.7970,47.5688) and
        (176.0680,47.3344) .. (176.0090,47.8177) -- cycle;
      \path[fill=c6c6c6c] (176.3180,47.8943) .. controls (176.2540,48.4686) and
        (177.8520,48.8854) .. (178.8710,49.3147) .. controls (179.8930,49.7060) and
        (181.8650,51.2049) .. (182.7790,52.2944) .. controls (183.6660,53.3714) and
        (185.8610,56.4114) .. (186.3270,55.6788) .. controls (186.7630,54.9943) and
        (185.6700,52.9748) .. (184.6880,51.8985) .. controls (183.7610,50.8247) and
        (182.5580,49.4577) .. (180.5560,48.4674) .. controls (178.9240,47.6660) and
        (176.3760,47.4349) .. (176.3180,47.8943) -- cycle;
      \path[fill=c727272] (176.6260,47.9710) .. controls (176.5630,48.5156) and
        (178.0770,48.9187) .. (179.0420,49.3310) .. controls (180.0100,49.7072) and
        (181.8730,51.1418) .. (182.7360,52.1768) .. controls (183.5690,53.2028) and
        (185.6640,56.0827) .. (186.1060,55.3961) .. controls (186.5170,54.7516) and
        (185.5290,52.8463) .. (184.5370,51.8041) .. controls (183.6010,50.7646) and
        (182.4720,49.5290) .. (180.5800,48.5524) .. controls (179.0520,47.7632) and
        (176.6840,47.5355) .. (176.6260,47.9710) -- cycle;
      \path[fill=c797979] (176.9350,48.0478) .. controls (176.8730,48.5625) and
        (178.3030,48.9521) .. (179.2130,49.3472) .. controls (180.1260,49.7083) and
        (181.8820,51.0786) .. (182.6920,52.0591) .. controls (183.4720,53.0341) and
        (185.4670,55.7541) .. (185.8860,55.1133) .. controls (186.2720,54.5091) and
        (185.3890,52.7179) .. (184.3850,51.7098) .. controls (183.4420,50.7045) and
        (182.3860,49.6002) .. (180.6050,48.6375) .. controls (179.1800,47.8603) and
        (176.9920,47.6360) .. (176.9350,48.0478) -- cycle;
    \path[fill=c7f7f7f] (177.2440,48.1245) .. controls (177.1820,48.6095) and
      (178.5280,48.9854) .. (179.3840,49.3635) .. controls (180.2430,49.7094) and
      (181.8900,51.0154) .. (182.6480,51.9415) .. controls (183.3740,52.8654) and
      (185.2700,55.4254) .. (185.6650,54.8304) .. controls (186.0260,54.2664) and
      (185.2480,52.5894) .. (184.2330,51.6154) .. controls (183.2820,50.6444) and
      (182.2990,49.6714) .. (180.6290,48.7224) .. controls (179.3070,47.9575) and
      (177.3000,47.7365) .. (177.2440,48.1245) -- cycle;
    \path[fill=c995900] (281.2260,263.5260) .. controls (285.6900,270.0060) and
      (281.0820,278.4300) .. (284.3220,283.6140) .. controls (289.6500,292.1100) and
      (300.3780,300.6780) .. (304.4820,303.0540) .. controls (307.4340,304.8540) and
      (311.6100,306.5820) .. (311.4660,311.8380) .. controls (311.2500,317.8140) and
      (308.2980,319.3980) .. (306.6420,321.0540) .. controls (303.3300,324.3660) and
      (286.6260,332.6460) .. (275.5380,340.2060) .. controls (261.5700,349.7820) and
      (256.7460,354.0300) .. (252.2100,358.5660) .. controls (245.1540,365.6220) and
      (238.4580,367.9980) .. (227.7300,367.9980) .. controls (217.0020,367.9980) and
      (212.1780,365.7660) .. (208.8660,362.8140) .. controls (205.5540,359.9340) and
      (201.8820,352.5900) .. (202.2420,341.6460) .. controls (202.5300,330.7740) and
      (205.9860,320.8380) .. (207.6420,303.9180) .. controls (208.3620,296.6460) and
      (208.2900,287.6460) .. (208.2900,279.6540) .. controls (208.2900,269.5740) and
      (208.4340,261.0060) .. (210.5940,259.7100) .. controls (214.4820,257.2620) and
      (215.3460,257.1180) .. (219.9540,257.1180) .. controls (224.5620,257.1180) and
      (226.6500,257.4060) .. (228.1620,258.9180) .. controls (229.6020,260.3580) and
      (229.0260,263.6700) .. (228.5220,268.3500) .. controls (228.0900,273.0300) and
      (230.3220,274.5420) .. (232.5540,276.4860) .. controls (234.7860,278.3580) and
      (236.8020,280.7340) .. (243.8580,281.3100) .. controls (250.9140,281.8140) and
      (253.5060,280.6620) .. (256.8180,278.5740) .. controls (260.1300,276.4860) and
      (264.8100,272.7420) .. (266.5380,270.5820) .. controls (268.1940,268.4940) and
      (272.2980,261.2940) .. (273.0900,261.2940) .. controls (273.8100,261.2940) and
      (278.5620,259.6380) .. (281.2260,263.5260) -- cycle;
      \path[fill=c9e5e00] (281.2260,263.5260) .. controls (285.6900,270.0060) and
        (281.0820,278.4300) .. (284.3220,283.6140) .. controls (289.6500,292.1100) and
        (300.3780,300.6780) .. (304.4820,303.0540) .. controls (307.4340,304.8540) and
        (311.6100,306.5820) .. (311.4660,311.8380) .. controls (311.2500,317.8140) and
        (308.2980,319.3980) .. (306.6420,321.0540) .. controls (303.3300,324.3660) and
        (286.5220,332.6100) .. (275.3830,340.0620) .. controls (261.4980,349.4330) and
        (256.4800,353.6020) .. (251.8610,357.9650) .. controls (244.9490,364.7040) and
        (238.4470,367.0480) .. (227.9460,366.9830) .. controls (217.5350,366.9360) and
        (212.8300,364.8020) .. (209.5320,361.8640) .. controls (206.2350,358.9980) and
        (202.6020,351.8990) .. (202.9190,341.2860) .. controls (203.1240,330.4360) and
        (206.6200,320.6040) .. (208.1750,304.0080) .. controls (208.8410,296.6280) and
        (208.5820,287.7040) .. (208.5100,279.6330) .. controls (208.4340,269.5640) and
        (208.4380,261.0060) .. (210.5940,259.7100) .. controls (214.4820,257.2660) and
        (215.3460,257.1180) .. (219.9540,257.1180) .. controls (224.5620,257.1180) and
        (226.6540,257.4100) .. (228.1620,258.9180) .. controls (229.6530,260.4090) and
        (228.9290,263.8040) .. (228.3670,268.3250) .. controls (227.7370,272.9840) and
        (229.8250,274.8120) .. (232.1620,276.9330) .. controls (234.5020,278.9920) and
        (236.6510,281.4040) .. (243.6960,281.9550) .. controls (250.9290,282.4370) and
        (253.7110,281.1230) .. (257.0590,278.8480) .. controls (260.3610,276.6410) and
        (264.8320,272.9440) .. (266.5710,270.6800) .. controls (268.2120,268.5480) and
        (272.2980,261.2940) .. (273.0900,261.2940) .. controls (273.8100,261.2940) and
        (278.5620,259.6380) .. (281.2260,263.5260) -- cycle;
      \path[fill=ca36400] (281.2260,263.5260) .. controls (285.6900,270.0060) and
        (281.0820,278.4300) .. (284.3220,283.6140) .. controls (289.6500,292.1100) and
        (300.3780,300.6780) .. (304.4820,303.0540) .. controls (307.4340,304.8540) and
        (311.6100,306.5820) .. (311.4660,311.8380) .. controls (311.2500,317.8140) and
        (308.2980,319.3980) .. (306.6420,321.0540) .. controls (303.3300,324.3660) and
        (286.4170,332.5740) .. (275.2290,339.9180) .. controls (261.4260,349.0840) and
        (256.2130,353.1740) .. (251.5120,357.3640) .. controls (244.7440,363.7860) and
        (238.4370,366.0980) .. (228.1620,365.9680) .. controls (218.0680,365.8740) and
        (213.4810,363.8370) .. (210.1980,360.9140) .. controls (206.9150,358.0620) and
        (203.3220,351.2080) .. (203.5960,340.9260) .. controls (203.7180,330.0980) and
        (207.2530,320.3700) .. (208.7080,304.0980) .. controls (209.3200,296.6100) and
        (208.8730,287.7620) .. (208.7290,279.6110) .. controls (208.5780,269.5530) and
        (208.4410,261.0060) .. (210.5940,259.7100) .. controls (214.4820,257.2700) and
        (215.3460,257.1180) .. (219.9540,257.1180) .. controls (224.5620,257.1180) and
        (226.6570,257.4140) .. (228.1620,258.9180) .. controls (229.7030,260.4590) and
        (228.8320,263.9370) .. (228.2130,268.3000) .. controls (227.3850,272.9370) and
        (229.3290,275.0820) .. (231.7690,277.3790) .. controls (234.2170,279.6260) and
        (236.5000,282.0740) .. (243.5340,282.5990) .. controls (250.9430,283.0600) and
        (253.9170,281.5840) .. (257.3010,279.1220) .. controls (260.5910,276.7960) and
        (264.8530,273.1460) .. (266.6030,270.7770) .. controls (268.2300,268.6020) and
        (272.2980,261.2940) .. (273.0900,261.2940) .. controls (273.8100,261.2940) and
        (278.5620,259.6380) .. (281.2260,263.5260) -- cycle;
      \path[fill=ca86a00] (281.2260,263.5260) .. controls (285.6900,270.0060) and
        (281.0820,278.4300) .. (284.3220,283.6140) .. controls (289.6500,292.1100) and
        (300.3780,300.6780) .. (304.4820,303.0540) .. controls (307.4340,304.8540) and
        (311.6100,306.5820) .. (311.4660,311.8380) .. controls (311.2500,317.8140) and
        (308.2980,319.3980) .. (306.6420,321.0540) .. controls (303.3300,324.3660) and
        (286.3130,332.5380) .. (275.0740,339.7740) .. controls (261.3540,348.7350) and
        (255.9470,352.7450) .. (251.1630,356.7630) .. controls (244.5390,362.8680) and
        (238.4260,365.1470) .. (228.3780,364.9530) .. controls (218.6010,364.8120) and
        (214.1330,362.8720) .. (210.8640,359.9630) .. controls (207.5950,357.1260) and
        (204.0420,350.5170) .. (204.2730,340.5660) .. controls (204.3120,329.7590) and
        (207.8870,320.1360) .. (209.2410,304.1880) .. controls (209.7990,296.5920) and
        (209.1650,287.8190) .. (208.9490,279.5900) .. controls (208.7220,269.5420) and
        (208.4450,261.0060) .. (210.5940,259.7100) .. controls (214.4820,257.2730) and
        (215.3460,257.1180) .. (219.9540,257.1180) .. controls (224.5620,257.1180) and
        (226.6610,257.4170) .. (228.1620,258.9180) .. controls (229.7530,260.5100) and
        (228.7350,264.0700) .. (228.0580,268.2750) .. controls (227.0320,272.8900) and
        (228.8320,275.3520) .. (231.3770,277.8260) .. controls (233.9330,280.2590) and
        (236.3490,282.7430) .. (243.3720,283.2440) .. controls (250.9570,283.6830) and
        (254.1220,282.0450) .. (257.5420,279.3950) .. controls (260.8210,276.9510) and
        (264.8750,273.3470) .. (266.6350,270.8740) .. controls (268.2480,268.6560) and
        (272.2980,261.2940) .. (273.0900,261.2940) .. controls (273.8100,261.2940) and
        (278.5620,259.6380) .. (281.2260,263.5260) -- cycle;
      \path[fill=cad7000] (281.2260,263.5260) .. controls (285.6900,270.0060) and
        (281.0820,278.4300) .. (284.3220,283.6140) .. controls (289.6500,292.1100) and
        (300.3780,300.6780) .. (304.4820,303.0540) .. controls (307.4340,304.8540) and
        (311.6100,306.5820) .. (311.4660,311.8380) .. controls (311.2500,317.8140) and
        (308.2980,319.3980) .. (306.6420,321.0540) .. controls (303.3300,324.3660) and
        (286.2090,332.5020) .. (274.9190,339.6300) .. controls (261.2820,348.3860) and
        (255.6810,352.3170) .. (250.8130,356.1620) .. controls (244.3330,361.9500) and
        (238.4150,364.1970) .. (228.5940,363.9380) .. controls (219.1330,363.7500) and
        (214.7850,361.9070) .. (211.5300,359.0130) .. controls (208.2760,356.1900) and
        (204.7620,349.8260) .. (204.9490,340.2060) .. controls (204.9060,329.4210) and
        (208.5210,319.9020) .. (209.7730,304.2780) .. controls (210.2770,296.5740) and
        (209.4570,287.8770) .. (209.1690,279.5680) .. controls (208.8660,269.5310) and
        (208.4490,261.0060) .. (210.5940,259.7100) .. controls (214.4820,257.2770) and
        (215.3460,257.1180) .. (219.9540,257.1180) .. controls (224.5620,257.1180) and
        (226.6650,257.4210) .. (228.1620,258.9180) .. controls (229.8040,260.5600) and
        (228.6370,264.2030) .. (227.9030,268.2500) .. controls (226.6790,272.8430) and
        (228.3350,275.6220) .. (230.9850,278.2720) .. controls (233.6490,280.8930) and
        (236.1970,283.4130) .. (243.2100,283.8880) .. controls (250.9720,284.3060) and
        (254.3270,282.5060) .. (257.7830,279.6690) .. controls (261.0520,277.1060) and
        (264.8970,273.5490) .. (266.6680,270.9710) .. controls (268.2660,268.7100) and
        (272.2980,261.2940) .. (273.0900,261.2940) .. controls (273.8100,261.2940) and
        (278.5620,259.6380) .. (281.2260,263.5260) -- cycle;
      \path[fill=cb27500] (281.2260,263.5260) .. controls (285.6900,270.0060) and
        (281.0820,278.4300) .. (284.3220,283.6140) .. controls (289.6500,292.1100) and
        (300.3780,300.6780) .. (304.4820,303.0540) .. controls (307.4340,304.8540) and
        (311.6100,306.5820) .. (311.4660,311.8380) .. controls (311.2500,317.8140) and
        (308.2980,319.3980) .. (306.6420,321.0540) .. controls (303.3300,324.3660) and
        (286.1040,332.4660) .. (274.7640,339.4860) .. controls (261.2100,348.0360) and
        (255.4140,351.8880) .. (250.4640,355.5600) .. controls (244.1280,361.0320) and
        (238.4040,363.2460) .. (228.8100,362.9220) .. controls (219.6660,362.6880) and
        (215.4360,360.9420) .. (212.1960,358.0620) .. controls (208.9560,355.2540) and
        (205.4820,349.1340) .. (205.6260,339.8460) .. controls (205.5000,329.0820) and
        (209.1540,319.6680) .. (210.3060,304.3680) .. controls (210.7560,296.5560) and
        (209.7480,287.9340) .. (209.3880,279.5460) .. controls (209.0100,269.5200) and
        (208.4520,261.0060) .. (210.5940,259.7100) .. controls (214.4820,257.2800) and
        (215.3460,257.1180) .. (219.9540,257.1180) .. controls (224.5620,257.1180) and
        (226.6680,257.4240) .. (228.1620,258.9180) .. controls (229.8540,260.6100) and
        (228.5400,264.3360) .. (227.7480,268.2240) .. controls (226.3260,272.7960) and
        (227.8380,275.8920) .. (230.5920,278.7180) .. controls (233.3640,281.5260) and
        (236.0460,284.0820) .. (243.0480,284.5320) .. controls (250.9860,284.9280) and
        (254.5320,282.9660) .. (258.0240,279.9420) .. controls (261.2820,277.2600) and
        (264.9180,273.7500) .. (266.7000,271.0680) .. controls (268.2840,268.7640) and
        (272.2980,261.2940) .. (273.0900,261.2940) .. controls (273.8100,261.2940) and
        (278.5620,259.6380) .. (281.2260,263.5260) -- cycle;
      \path[fill=cb77b00] (281.2260,263.5260) .. controls (285.6900,270.0060) and
        (281.0820,278.4300) .. (284.3220,283.6140) .. controls (289.6500,292.1100) and
        (300.3780,300.6780) .. (304.4820,303.0540) .. controls (307.4340,304.8540) and
        (311.6100,306.5820) .. (311.4660,311.8380) .. controls (311.2500,317.8140) and
        (308.2980,319.3980) .. (306.6420,321.0540) .. controls (303.3300,324.3660) and
        (286.0000,332.4300) .. (274.6090,339.3420) .. controls (261.1380,347.6870) and
        (255.1480,351.4600) .. (250.1150,354.9590) .. controls (243.9230,360.1140) and
        (238.3930,362.2960) .. (229.0260,361.9070) .. controls (220.1990,361.6260) and
        (216.0880,359.9780) .. (212.8620,357.1120) .. controls (209.6370,354.3180) and
        (206.2020,348.4430) .. (206.3030,339.4860) .. controls (206.0940,328.7440) and
        (209.7880,319.4340) .. (210.8390,304.4580) .. controls (211.2350,296.5380) and
        (210.0400,287.9920) .. (209.6080,279.5250) .. controls (209.1540,269.5100) and
        (208.4560,261.0060) .. (210.5940,259.7100) .. controls (214.4820,257.2840) and
        (215.3460,257.1180) .. (219.9540,257.1180) .. controls (224.5620,257.1180) and
        (226.6720,257.4280) .. (228.1620,258.9180) .. controls (229.9050,260.6610) and
        (228.4430,264.4700) .. (227.5930,268.1990) .. controls (225.9730,272.7500) and
        (227.3410,276.1620) .. (230.2000,279.1650) .. controls (233.0800,282.1600) and
        (235.8950,284.7520) .. (242.8860,285.1770) .. controls (251.0010,285.5510) and
        (254.7370,283.4270) .. (258.2650,280.2160) .. controls (261.5130,277.4150) and
        (264.9400,273.9520) .. (266.7330,271.1660) .. controls (268.3020,268.8180) and
        (272.2980,261.2940) .. (273.0900,261.2940) .. controls (273.8100,261.2940) and
        (278.5620,259.6380) .. (281.2260,263.5260) -- cycle;
      \path[fill=cbc8100] (281.2260,263.5260) .. controls (285.6900,270.0060) and
        (281.0820,278.4300) .. (284.3220,283.6140) .. controls (289.6500,292.1100) and
        (300.3780,300.6780) .. (304.4820,303.0540) .. controls (307.4340,304.8540) and
        (311.6100,306.5820) .. (311.4660,311.8380) .. controls (311.2500,317.8140) and
        (308.2980,319.3980) .. (306.6420,321.0540) .. controls (303.3300,324.3660) and
        (285.8950,332.3940) .. (274.4550,339.1980) .. controls (261.0660,347.3380) and
        (254.8810,351.0320) .. (249.7660,354.3580) .. controls (243.7180,359.1960) and
        (238.3830,361.3460) .. (229.2420,360.8920) .. controls (220.7320,360.5640) and
        (216.7390,359.0130) .. (213.5280,356.1620) .. controls (210.3170,353.3820) and
        (206.9220,347.7520) .. (206.9800,339.1260) .. controls (206.6880,328.4060) and
        (210.4210,319.2000) .. (211.3720,304.5480) .. controls (211.7140,296.5200) and
        (210.3310,288.0500) .. (209.8270,279.5030) .. controls (209.2980,269.4990) and
        (208.4590,261.0060) .. (210.5940,259.7100) .. controls (214.4820,257.2880) and
        (215.3460,257.1180) .. (219.9540,257.1180) .. controls (224.5620,257.1180) and
        (226.6750,257.4320) .. (228.1620,258.9180) .. controls (229.9550,260.7110) and
        (228.3460,264.6030) .. (227.4390,268.1740) .. controls (225.6210,272.7030) and
        (226.8450,276.4320) .. (229.8070,279.6110) .. controls (232.7950,282.7940) and
        (235.7440,285.4220) .. (242.7240,285.8210) .. controls (251.0150,286.1740) and
        (254.9430,283.8880) .. (258.5070,280.4900) .. controls (261.7430,277.5700) and
        (264.9610,274.1540) .. (266.7650,271.2630) .. controls (268.3200,268.8720) and
        (272.2980,261.2940) .. (273.0900,261.2940) .. controls (273.8100,261.2940) and
        (278.5620,259.6380) .. (281.2260,263.5260) -- cycle;
      \path[fill=cc18700] (281.2260,263.5260) .. controls (285.6900,270.0060) and
        (281.0820,278.4300) .. (284.3220,283.6140) .. controls (289.6500,292.1100) and
        (300.3780,300.6780) .. (304.4820,303.0540) .. controls (307.4340,304.8540) and
        (311.6100,306.5820) .. (311.4660,311.8380) .. controls (311.2500,317.8140) and
        (308.2980,319.3980) .. (306.6420,321.0540) .. controls (303.3300,324.3660) and
        (285.7910,332.3580) .. (274.3000,339.0540) .. controls (260.9940,346.9890) and
        (254.6150,350.6030) .. (249.4170,353.7570) .. controls (243.5130,358.2780) and
        (238.3720,360.3950) .. (229.4580,359.8770) .. controls (221.2650,359.5020) and
        (217.3910,358.0480) .. (214.1940,355.2110) .. controls (210.9970,352.4460) and
        (207.6420,347.0610) .. (207.6570,338.7660) .. controls (207.2820,328.0670) and
        (211.0550,318.9660) .. (211.9050,304.6380) .. controls (212.1930,296.5020) and
        (210.6230,288.1070) .. (210.0470,279.4820) .. controls (209.4420,269.4880) and
        (208.4630,261.0060) .. (210.5940,259.7100) .. controls (214.4820,257.2910) and
        (215.3460,257.1180) .. (219.9540,257.1180) .. controls (224.5620,257.1180) and
        (226.6790,257.4350) .. (228.1620,258.9180) .. controls (230.0050,260.7620) and
        (228.2490,264.7360) .. (227.2840,268.1490) .. controls (225.2680,272.6560) and
        (226.3480,276.7020) .. (229.4150,280.0580) .. controls (232.5110,283.4270) and
        (235.5930,286.0910) .. (242.5620,286.4660) .. controls (251.0290,286.7970) and
        (255.1480,284.3490) .. (258.7480,280.7630) .. controls (261.9730,277.7250) and
        (264.9830,274.3550) .. (266.7970,271.3600) .. controls (268.3380,268.9260) and
        (272.2980,261.2940) .. (273.0900,261.2940) .. controls (273.8100,261.2940) and
        (278.5620,259.6380) .. (281.2260,263.5260) -- cycle;
      \path[fill=cc68c00] (281.2260,263.5260) .. controls (285.6900,270.0060) and
        (281.0820,278.4300) .. (284.3220,283.6140) .. controls (289.6500,292.1100) and
        (300.3780,300.6780) .. (304.4820,303.0540) .. controls (307.4340,304.8540) and
        (311.6100,306.5820) .. (311.4660,311.8380) .. controls (311.2500,317.8140) and
        (308.2980,319.3980) .. (306.6420,321.0540) .. controls (303.3300,324.3660) and
        (285.6870,332.3220) .. (274.1450,338.9100) .. controls (260.9220,346.6400) and
        (254.3490,350.1750) .. (249.0670,353.1560) .. controls (243.3070,357.3600) and
        (238.3610,359.4450) .. (229.6740,358.8620) .. controls (221.7970,358.4400) and
        (218.0430,357.0830) .. (214.8600,354.2610) .. controls (211.6780,351.5100) and
        (208.3620,346.3700) .. (208.3330,338.4060) .. controls (207.8760,327.7290) and
        (211.6890,318.7320) .. (212.4370,304.7280) .. controls (212.6710,296.4840) and
        (210.9150,288.1650) .. (210.2670,279.4600) .. controls (209.5860,269.4770) and
        (208.4670,261.0060) .. (210.5940,259.7100) .. controls (214.4820,257.2950) and
        (215.3460,257.1180) .. (219.9540,257.1180) .. controls (224.5620,257.1180) and
        (226.6830,257.4390) .. (228.1620,258.9180) .. controls (230.0560,260.8120) and
        (228.1510,264.8690) .. (227.1290,268.1240) .. controls (224.9150,272.6090) and
        (225.8510,276.9720) .. (229.0230,280.5040) .. controls (232.2270,284.0610) and
        (235.4410,286.7610) .. (242.4000,287.1100) .. controls (251.0440,287.4200) and
        (255.3530,284.8100) .. (258.9890,281.0370) .. controls (262.2040,277.8800) and
        (265.0050,274.5570) .. (266.8300,271.4570) .. controls (268.3560,268.9800) and
        (272.2980,261.2940) .. (273.0900,261.2940) .. controls (273.8100,261.2940) and
        (278.5620,259.6380) .. (281.2260,263.5260) -- cycle;
      \path[fill=ccc9200] (281.2260,263.5260) .. controls (285.6900,270.0060) and
        (281.0820,278.4300) .. (284.3220,283.6140) .. controls (289.6500,292.1100) and
        (300.3780,300.6780) .. (304.4820,303.0540) .. controls (307.4340,304.8540) and
        (311.6100,306.5820) .. (311.4660,311.8380) .. controls (311.2500,317.8140) and
        (308.2980,319.3980) .. (306.6420,321.0540) .. controls (303.3300,324.3660) and
        (285.5820,332.2860) .. (273.9900,338.7660) .. controls (260.8500,346.2900) and
        (254.0820,349.7460) .. (248.7180,352.5540) .. controls (243.1020,356.4420) and
        (238.3500,358.4940) .. (229.8900,357.8460) .. controls (222.3300,357.3780) and
        (218.6940,356.1180) .. (215.5260,353.3100) .. controls (212.3580,350.5740) and
        (209.0820,345.6780) .. (209.0100,338.0460) .. controls (208.4700,327.3900) and
        (212.3220,318.4980) .. (212.9700,304.8180) .. controls (213.1500,296.4660) and
        (211.2060,288.2220) .. (210.4860,279.4380) .. controls (209.7300,269.4660) and
        (208.4700,261.0060) .. (210.5940,259.7100) .. controls (214.4820,257.2980) and
        (215.3460,257.1180) .. (219.9540,257.1180) .. controls (224.5620,257.1180) and
        (226.6860,257.4420) .. (228.1620,258.9180) .. controls (230.1060,260.8620) and
        (228.0540,265.0020) .. (226.9740,268.0980) .. controls (224.5620,272.5620) and
        (225.3540,277.2420) .. (228.6300,280.9500) .. controls (231.9420,284.6940) and
        (235.2900,287.4300) .. (242.2380,287.7540) .. controls (251.0580,288.0420) and
        (255.5580,285.2700) .. (259.2300,281.3100) .. controls (262.4340,278.0340) and
        (265.0260,274.7580) .. (266.8620,271.5540) .. controls (268.3740,269.0340) and
        (272.2980,261.2940) .. (273.0900,261.2940) .. controls (273.8100,261.2940) and
        (278.5620,259.6380) .. (281.2260,263.5260) -- cycle;
      \path[fill=cd19800] (281.2260,263.5260) .. controls (285.6900,270.0060) and
        (281.0820,278.4300) .. (284.3220,283.6140) .. controls (289.6500,292.1100) and
        (300.3780,300.6780) .. (304.4820,303.0540) .. controls (307.4340,304.8540) and
        (311.6100,306.5820) .. (311.4660,311.8380) .. controls (311.2500,317.8140) and
        (308.2980,319.3980) .. (306.6420,321.0540) .. controls (303.3300,324.3660) and
        (285.4780,332.2500) .. (273.8350,338.6220) .. controls (260.7780,345.9410) and
        (253.8160,349.3180) .. (248.3690,351.9530) .. controls (242.8970,355.5240) and
        (238.3390,357.5440) .. (230.1060,356.8310) .. controls (222.8630,356.3160) and
        (219.3460,355.1540) .. (216.1920,352.3600) .. controls (213.0390,349.6380) and
        (209.8020,344.9870) .. (209.6870,337.6860) .. controls (209.0640,327.0520) and
        (212.9560,318.2640) .. (213.5030,304.9080) .. controls (213.6290,296.4480) and
        (211.4980,288.2800) .. (210.7060,279.4170) .. controls (209.8740,269.4560) and
        (208.4740,261.0060) .. (210.5940,259.7100) .. controls (214.4820,257.3020) and
        (215.3460,257.1180) .. (219.9540,257.1180) .. controls (224.5620,257.1180) and
        (226.6900,257.4460) .. (228.1620,258.9180) .. controls (230.1570,260.9130) and
        (227.9570,265.1360) .. (226.8190,268.0730) .. controls (224.2090,272.5160) and
        (224.8570,277.5120) .. (228.2380,281.3970) .. controls (231.6580,285.3280) and
        (235.1390,288.1000) .. (242.0760,288.3990) .. controls (251.0730,288.6650) and
        (255.7630,285.7310) .. (259.4710,281.5840) .. controls (262.6650,278.1890) and
        (265.0480,274.9600) .. (266.8950,271.6520) .. controls (268.3920,269.0880) and
        (272.2980,261.2940) .. (273.0900,261.2940) .. controls (273.8100,261.2940) and
        (278.5620,259.6380) .. (281.2260,263.5260) -- cycle;
      \path[fill=cd69e00] (281.2260,263.5260) .. controls (285.6900,270.0060) and
        (281.0820,278.4300) .. (284.3220,283.6140) .. controls (289.6500,292.1100) and
        (300.3780,300.6780) .. (304.4820,303.0540) .. controls (307.4340,304.8540) and
        (311.6100,306.5820) .. (311.4660,311.8380) .. controls (311.2500,317.8140) and
        (308.2980,319.3980) .. (306.6420,321.0540) .. controls (303.3300,324.3660) and
        (285.3730,332.2140) .. (273.6810,338.4780) .. controls (260.7060,345.5920) and
        (253.5490,348.8900) .. (248.0200,351.3520) .. controls (242.6920,354.6060) and
        (238.3290,356.5940) .. (230.3220,355.8160) .. controls (223.3960,355.2540) and
        (219.9970,354.1890) .. (216.8580,351.4100) .. controls (213.7190,348.7020) and
        (210.5220,344.2960) .. (210.3640,337.3260) .. controls (209.6580,326.7140) and
        (213.5890,318.0300) .. (214.0360,304.9980) .. controls (214.1080,296.4300) and
        (211.7890,288.3380) .. (210.9250,279.3950) .. controls (210.0180,269.4450) and
        (208.4770,261.0060) .. (210.5940,259.7100) .. controls (214.4820,257.3060) and
        (215.3460,257.1180) .. (219.9540,257.1180) .. controls (224.5620,257.1180) and
        (226.6930,257.4500) .. (228.1620,258.9180) .. controls (230.2070,260.9630) and
        (227.8600,265.2690) .. (226.6650,268.0480) .. controls (223.8570,272.4690) and
        (224.3610,277.7820) .. (227.8450,281.8430) .. controls (231.3730,285.9620) and
        (234.9880,288.7700) .. (241.9140,289.0430) .. controls (251.0870,289.2880) and
        (255.9690,286.1920) .. (259.7130,281.8580) .. controls (262.8950,278.3440) and
        (265.0690,275.1620) .. (266.9270,271.7490) .. controls (268.4100,269.1420) and
        (272.2980,261.2940) .. (273.0900,261.2940) .. controls (273.8100,261.2940) and
        (278.5620,259.6380) .. (281.2260,263.5260) -- cycle;
      \path[fill=cdba300] (281.2260,263.5260) .. controls (285.6900,270.0060) and
        (281.0820,278.4300) .. (284.3220,283.6140) .. controls (289.6500,292.1100) and
        (300.3780,300.6780) .. (304.4820,303.0540) .. controls (307.4340,304.8540) and
        (311.6100,306.5820) .. (311.4660,311.8380) .. controls (311.2500,317.8140) and
        (308.2980,319.3980) .. (306.6420,321.0540) .. controls (303.3300,324.3660) and
        (285.2690,332.1780) .. (273.5260,338.3340) .. controls (260.6340,345.2430) and
        (253.2830,348.4610) .. (247.6710,350.7510) .. controls (242.4870,353.6880) and
        (238.3180,355.6430) .. (230.5380,354.8010) .. controls (223.9290,354.1920) and
        (220.6490,353.2240) .. (217.5240,350.4590) .. controls (214.3990,347.7660) and
        (211.2420,343.6050) .. (211.0410,336.9660) .. controls (210.2520,326.3750) and
        (214.2230,317.7960) .. (214.5690,305.0880) .. controls (214.5870,296.4120) and
        (212.0810,288.3950) .. (211.1450,279.3740) .. controls (210.1620,269.4340) and
        (208.4810,261.0060) .. (210.5940,259.7100) .. controls (214.4820,257.3090) and
        (215.3460,257.1180) .. (219.9540,257.1180) .. controls (224.5620,257.1180) and
        (226.6970,257.4530) .. (228.1620,258.9180) .. controls (230.2570,261.0140) and
        (227.7630,265.4020) .. (226.5100,268.0230) .. controls (223.5040,272.4220) and
        (223.8640,278.0520) .. (227.4530,282.2900) .. controls (231.0890,286.5950) and
        (234.8370,289.4390) .. (241.7520,289.6880) .. controls (251.1010,289.9110) and
        (256.1740,286.6530) .. (259.9540,282.1310) .. controls (263.1250,278.4990) and
        (265.0910,275.3630) .. (266.9590,271.8460) .. controls (268.4280,269.1960) and
        (272.2980,261.2940) .. (273.0900,261.2940) .. controls (273.8100,261.2940) and
        (278.5620,259.6380) .. (281.2260,263.5260) -- cycle;
      \path[fill=ce0a900] (281.2260,263.5260) .. controls (285.6900,270.0060) and
        (281.0820,278.4300) .. (284.3220,283.6140) .. controls (289.6500,292.1100) and
        (300.3780,300.6780) .. (304.4820,303.0540) .. controls (307.4340,304.8540) and
        (311.6100,306.5820) .. (311.4660,311.8380) .. controls (311.2500,317.8140) and
        (308.2980,319.3980) .. (306.6420,321.0540) .. controls (303.3300,324.3660) and
        (285.1650,332.1420) .. (273.3710,338.1900) .. controls (260.5620,344.8940) and
        (253.0170,348.0330) .. (247.3210,350.1500) .. controls (242.2810,352.7700) and
        (238.3070,354.6930) .. (230.7540,353.7860) .. controls (224.4610,353.1300) and
        (221.3010,352.2590) .. (218.1900,349.5090) .. controls (215.0800,346.8300) and
        (211.9620,342.9140) .. (211.7170,336.6060) .. controls (210.8460,326.0370) and
        (214.8570,317.5620) .. (215.1010,305.1780) .. controls (215.0650,296.3940) and
        (212.3730,288.4530) .. (211.3650,279.3520) .. controls (210.3060,269.4230) and
        (208.4850,261.0060) .. (210.5940,259.7100) .. controls (214.4820,257.3130) and
        (215.3460,257.1180) .. (219.9540,257.1180) .. controls (224.5620,257.1180) and
        (226.7010,257.4570) .. (228.1620,258.9180) .. controls (230.3080,261.0640) and
        (227.6650,265.5350) .. (226.3550,267.9980) .. controls (223.1510,272.3750) and
        (223.3670,278.3220) .. (227.0610,282.7360) .. controls (230.8050,287.2290) and
        (234.6850,290.1090) .. (241.5900,290.3320) .. controls (251.1160,290.5340) and
        (256.3790,287.1140) .. (260.1950,282.4050) .. controls (263.3560,278.6540) and
        (265.1130,275.5650) .. (266.9920,271.9430) .. controls (268.4460,269.2500) and
        (272.2980,261.2940) .. (273.0900,261.2940) .. controls (273.8100,261.2940) and
        (278.5620,259.6380) .. (281.2260,263.5260) -- cycle;
      \path[fill=ce5af00] (281.2260,263.5260) .. controls (285.6900,270.0060) and
        (281.0820,278.4300) .. (284.3220,283.6140) .. controls (289.6500,292.1100) and
        (300.3780,300.6780) .. (304.4820,303.0540) .. controls (307.4340,304.8540) and
        (311.6100,306.5820) .. (311.4660,311.8380) .. controls (311.2500,317.8140) and
        (308.2980,319.3980) .. (306.6420,321.0540) .. controls (303.3300,324.3660) and
        (285.0600,332.1060) .. (273.2160,338.0460) .. controls (260.4900,344.5440) and
        (252.7500,347.6040) .. (246.9720,349.5480) .. controls (242.0760,351.8520) and
        (238.2960,353.7420) .. (230.9700,352.7700) .. controls (224.9940,352.0680) and
        (221.9520,351.2940) .. (218.8560,348.5580) .. controls (215.7600,345.8940) and
        (212.6820,342.2220) .. (212.3940,336.2460) .. controls (211.4400,325.6980) and
        (215.4900,317.3280) .. (215.6340,305.2680) .. controls (215.5440,296.3760) and
        (212.6640,288.5100) .. (211.5840,279.3300) .. controls (210.4500,269.4120) and
        (208.4880,261.0060) .. (210.5940,259.7100) .. controls (214.4820,257.3160) and
        (215.3460,257.1180) .. (219.9540,257.1180) .. controls (224.5620,257.1180) and
        (226.7040,257.4600) .. (228.1620,258.9180) .. controls (230.3580,261.1140) and
        (227.5680,265.6680) .. (226.2000,267.9720) .. controls (222.7980,272.3280) and
        (222.8700,278.5920) .. (226.6680,283.1820) .. controls (230.5200,287.8620) and
        (234.5340,290.7780) .. (241.4280,290.9760) .. controls (251.1300,291.1560) and
        (256.5840,287.5740) .. (260.4360,282.6780) .. controls (263.5860,278.8080) and
        (265.1340,275.7660) .. (267.0240,272.0400) .. controls (268.4640,269.3040) and
        (272.2980,261.2940) .. (273.0900,261.2940) .. controls (273.8100,261.2940) and
        (278.5620,259.6380) .. (281.2260,263.5260) -- cycle;
      \path[fill=ceab500] (281.2260,263.5260) .. controls (285.6900,270.0060) and
        (281.0820,278.4300) .. (284.3220,283.6140) .. controls (289.6500,292.1100) and
        (300.3780,300.6780) .. (304.4820,303.0540) .. controls (307.4340,304.8540) and
        (311.6100,306.5820) .. (311.4660,311.8380) .. controls (311.2500,317.8140) and
        (308.2980,319.3980) .. (306.6420,321.0540) .. controls (303.3300,324.3660) and
        (284.9560,332.0700) .. (273.0610,337.9020) .. controls (260.4180,344.1950) and
        (252.4840,347.1760) .. (246.6230,348.9470) .. controls (241.8710,350.9340) and
        (238.2850,352.7920) .. (231.1860,351.7550) .. controls (225.5270,351.0060) and
        (222.6040,350.3300) .. (219.5220,347.6080) .. controls (216.4410,344.9580) and
        (213.4020,341.5310) .. (213.0710,335.8860) .. controls (212.0340,325.3600) and
        (216.1240,317.0940) .. (216.1670,305.3580) .. controls (216.0230,296.3580) and
        (212.9560,288.5680) .. (211.8040,279.3090) .. controls (210.5940,269.4020) and
        (208.4920,261.0060) .. (210.5940,259.7100) .. controls (214.4820,257.3200) and
        (215.3460,257.1180) .. (219.9540,257.1180) .. controls (224.5620,257.1180) and
        (226.7080,257.4640) .. (228.1620,258.9180) .. controls (230.4090,261.1650) and
        (227.4710,265.8020) .. (226.0450,267.9470) .. controls (222.4450,272.2820) and
        (222.3730,278.8620) .. (226.2760,283.6290) .. controls (230.2360,288.4960) and
        (234.3830,291.4480) .. (241.2660,291.6210) .. controls (251.1450,291.7790) and
        (256.7890,288.0350) .. (260.6770,282.9520) .. controls (263.8170,278.9630) and
        (265.1560,275.9680) .. (267.0570,272.1380) .. controls (268.4820,269.3580) and
        (272.2980,261.2940) .. (273.0900,261.2940) .. controls (273.8100,261.2940) and
        (278.5620,259.6380) .. (281.2260,263.5260) -- cycle;
      \path[fill=cefba00] (281.2260,263.5260) .. controls (285.6900,270.0060) and
        (281.0820,278.4300) .. (284.3220,283.6140) .. controls (289.6500,292.1100) and
        (300.3780,300.6780) .. (304.4820,303.0540) .. controls (307.4340,304.8540) and
        (311.6100,306.5820) .. (311.4660,311.8380) .. controls (311.2500,317.8140) and
        (308.2980,319.3980) .. (306.6420,321.0540) .. controls (303.3300,324.3660) and
        (284.8510,332.0340) .. (272.9070,337.7580) .. controls (260.3460,343.8460) and
        (252.2170,346.7480) .. (246.2740,348.3460) .. controls (241.6660,350.0160) and
        (238.2750,351.8420) .. (231.4020,350.7400) .. controls (226.0600,349.9440) and
        (223.2550,349.3650) .. (220.1880,346.6580) .. controls (217.1210,344.0220) and
        (214.1220,340.8400) .. (213.7480,335.5260) .. controls (212.6280,325.0220) and
        (216.7570,316.8600) .. (216.7000,305.4480) .. controls (216.5020,296.3400) and
        (213.2470,288.6260) .. (212.0230,279.2870) .. controls (210.7380,269.3910) and
        (208.4950,261.0060) .. (210.5940,259.7100) .. controls (214.4820,257.3240) and
        (215.3460,257.1180) .. (219.9540,257.1180) .. controls (224.5620,257.1180) and
        (226.7110,257.4680) .. (228.1620,258.9180) .. controls (230.4590,261.2150) and
        (227.3740,265.9350) .. (225.8910,267.9220) .. controls (222.0930,272.2350) and
        (221.8770,279.1320) .. (225.8830,284.0750) .. controls (229.9510,289.1300) and
        (234.2320,292.1180) .. (241.1040,292.2650) .. controls (251.1590,292.4020) and
        (256.9950,288.4960) .. (260.9190,283.2260) .. controls (264.0470,279.1180) and
        (265.1770,276.1700) .. (267.0890,272.2350) .. controls (268.5000,269.4120) and
        (272.2980,261.2940) .. (273.0900,261.2940) .. controls (273.8100,261.2940) and
        (278.5620,259.6380) .. (281.2260,263.5260) -- cycle;
      \path[fill=cf4c000] (281.2260,263.5260) .. controls (285.6900,270.0060) and
        (281.0820,278.4300) .. (284.3220,283.6140) .. controls (289.6500,292.1100) and
        (300.3780,300.6780) .. (304.4820,303.0540) .. controls (307.4340,304.8540) and
        (311.6100,306.5820) .. (311.4660,311.8380) .. controls (311.2500,317.8140) and
        (308.2980,319.3980) .. (306.6420,321.0540) .. controls (303.3300,324.3660) and
        (284.7470,331.9980) .. (272.7520,337.6140) .. controls (260.2740,343.4970) and
        (251.9510,346.3190) .. (245.9250,347.7450) .. controls (241.4610,349.0980) and
        (238.2640,350.8910) .. (231.6180,349.7250) .. controls (226.5930,348.8820) and
        (223.9070,348.4000) .. (220.8540,345.7070) .. controls (217.8010,343.0860) and
        (214.8420,340.1490) .. (214.4250,335.1660) .. controls (213.2220,324.6830) and
        (217.3910,316.6260) .. (217.2330,305.5380) .. controls (216.9810,296.3220) and
        (213.5390,288.6830) .. (212.2430,279.2660) .. controls (210.8820,269.3800) and
        (208.4990,261.0060) .. (210.5940,259.7100) .. controls (214.4820,257.3270) and
        (215.3460,257.1180) .. (219.9540,257.1180) .. controls (224.5620,257.1180) and
        (226.7150,257.4710) .. (228.1620,258.9180) .. controls (230.5090,261.2660) and
        (227.2770,266.0680) .. (225.7360,267.8970) .. controls (221.7400,272.1880) and
        (221.3800,279.4020) .. (225.4910,284.5220) .. controls (229.6670,289.7630) and
        (234.0810,292.7870) .. (240.9420,292.9100) .. controls (251.1730,293.0250) and
        (257.2000,288.9570) .. (261.1600,283.4990) .. controls (264.2770,279.2730) and
        (265.1990,276.3710) .. (267.1210,272.3320) .. controls (268.5180,269.4660) and
        (272.2980,261.2940) .. (273.0900,261.2940) .. controls (273.8100,261.2940) and
        (278.5620,259.6380) .. (281.2260,263.5260) -- cycle;
      \path[fill=cf9c600] (281.2260,263.5260) .. controls (285.6900,270.0060) and
        (281.0820,278.4300) .. (284.3220,283.6140) .. controls (289.6500,292.1100) and
        (300.3780,300.6780) .. (304.4820,303.0540) .. controls (307.4340,304.8540) and
        (311.6100,306.5820) .. (311.4660,311.8380) .. controls (311.2500,317.8140) and
        (308.2980,319.3980) .. (306.6420,321.0540) .. controls (303.3300,324.3660) and
        (284.6430,331.9620) .. (272.5970,337.4700) .. controls (260.2020,343.1480) and
        (251.6850,345.8910) .. (245.5750,347.1440) .. controls (241.2550,348.1800) and
        (238.2530,349.9410) .. (231.8340,348.7100) .. controls (227.1250,347.8200) and
        (224.5590,347.4350) .. (221.5200,344.7570) .. controls (218.4820,342.1500) and
        (215.5620,339.4580) .. (215.1010,334.8060) .. controls (213.8160,324.3450) and
        (218.0250,316.3920) .. (217.7650,305.6280) .. controls (217.4590,296.3040) and
        (213.8310,288.7410) .. (212.4630,279.2440) .. controls (211.0260,269.3690) and
        (208.5030,261.0060) .. (210.5940,259.7100) .. controls (214.4820,257.3310) and
        (215.3460,257.1180) .. (219.9540,257.1180) .. controls (224.5620,257.1180) and
        (226.7190,257.4750) .. (228.1620,258.9180) .. controls (230.5600,261.3160) and
        (227.1790,266.2010) .. (225.5810,267.8720) .. controls (221.3870,272.1410) and
        (220.8830,279.6720) .. (225.0990,284.9680) .. controls (229.3830,290.3970) and
        (233.9290,293.4570) .. (240.7800,293.5540) .. controls (251.1880,293.6480) and
        (257.4050,289.4180) .. (261.4010,283.7730) .. controls (264.5080,279.4280) and
        (265.2210,276.5730) .. (267.1540,272.4290) .. controls (268.5360,269.5200) and
        (272.2980,261.2940) .. (273.0900,261.2940) .. controls (273.8100,261.2940) and
        (278.5620,259.6380) .. (281.2260,263.5260) -- cycle;
    \path[fill=cffcc00] (281.2260,263.5260) .. controls (285.6900,270.0060) and
      (281.0820,278.4300) .. (284.3220,283.6140) .. controls (289.6500,292.1100) and
      (300.3780,300.6780) .. (304.4820,303.0540) .. controls (307.4340,304.8540) and
      (311.6100,306.5820) .. (311.4660,311.8380) .. controls (311.2500,317.8140) and
      (308.2980,319.3980) .. (306.6420,321.0540) .. controls (303.3300,324.3660) and
      (284.5380,331.9260) .. (272.4420,337.3260) .. controls (260.1300,342.7980) and
      (251.4180,345.4620) .. (245.2260,346.5420) .. controls (241.0500,347.2620) and
      (238.2420,348.9900) .. (232.0500,347.6940) .. controls (227.6580,346.7580) and
      (225.2100,346.4700) .. (222.1860,343.8060) .. controls (219.1620,341.2140) and
      (216.2820,338.7660) .. (215.7780,334.4460) .. controls (214.4100,324.0060) and
      (218.6580,316.1580) .. (218.2980,305.7180) .. controls (217.9380,296.2860) and
      (214.1220,288.7980) .. (212.6820,279.2220) .. controls (211.1700,269.3580) and
      (208.5060,261.0060) .. (210.5940,259.7100) .. controls (214.4820,257.3340) and
      (215.3460,257.1180) .. (219.9540,257.1180) .. controls (224.5620,257.1180) and
      (226.7220,257.4780) .. (228.1620,258.9180) .. controls (230.6100,261.3660) and
      (227.0820,266.3340) .. (225.4260,267.8460) .. controls (221.0340,272.0940) and
      (220.3860,279.9420) .. (224.7060,285.4140) .. controls (229.0980,291.0300) and
      (233.7780,294.1260) .. (240.6180,294.1980) .. controls (251.2020,294.2700) and
      (257.6100,289.8780) .. (261.6420,284.0460) .. controls (264.7380,279.5820) and
      (265.2420,276.7740) .. (267.1860,272.5260) .. controls (268.5540,269.5740) and
      (272.2980,261.2940) .. (273.0900,261.2940) .. controls (273.8100,261.2940) and
      (278.5620,259.6380) .. (281.2260,263.5260) -- cycle;
    \path[fill=cffcc00] (213.8590,276.7590) .. controls (213.1500,276.5020) and
      (209.9280,261.5510) .. (211.6680,260.5200) .. controls (215.0190,258.5230) and
      (215.9210,258.2000) .. (220.0450,258.2000) .. controls (224.1690,258.2000) and
      (226.1020,258.5230) .. (227.3910,259.8110) .. controls (229.5170,261.9380) and
      (226.2310,266.5130) .. (224.9420,267.8020) .. controls (221.2050,271.4750) and
      (214.5680,277.0170) .. (213.8590,276.7590) -- cycle;
      \path[fill=cffcc02] (214.0880,276.2820) .. controls (213.3180,275.9570) and
        (209.9860,261.5640) .. (211.7200,260.5400) .. controls (215.0640,258.5490) and
        (215.9990,258.3290) .. (220.0420,258.3290) .. controls (224.1660,258.3290) and
        (226.0180,258.5980) .. (227.3240,259.9310) .. controls (229.4300,262.0670) and
        (226.2050,266.5040) .. (224.9260,267.7830) .. controls (221.1990,271.4460) and
        (214.8450,276.5950) .. (214.0880,276.2820) -- cycle;
      \path[fill=cffcc05] (214.3170,275.8060) .. controls (213.4850,275.4130) and
        (210.0440,261.5770) .. (211.7710,260.5590) .. controls (215.1090,258.5750) and
        (216.0760,258.4580) .. (220.0390,258.4580) .. controls (224.1630,258.4580) and
        (225.9350,258.6720) .. (227.2560,260.0500) .. controls (229.3430,262.1960) and
        (226.1800,266.4940) .. (224.9100,267.7640) .. controls (221.1920,271.4170) and
        (215.1220,276.1730) .. (214.3170,275.8060) -- cycle;
      \path[fill=cffcc07] (214.5450,275.3290) .. controls (213.6530,274.8680) and
        (210.1020,261.5900) .. (211.8230,260.5780) .. controls (215.1550,258.6010) and
        (216.1530,258.5870) .. (220.0360,258.5870) .. controls (224.1600,258.5870) and
        (225.8510,258.7460) .. (227.1880,260.1690) .. controls (229.2560,262.3250) and
        (226.1540,266.4840) .. (224.8940,267.7440) .. controls (221.1860,271.3880) and
        (215.3990,275.7510) .. (214.5450,275.3290) -- cycle;
      \path[fill=cffcd0a] (214.7740,274.8520) .. controls (213.8200,274.3240) and
        (210.1600,261.6030) .. (211.8740,260.5980) .. controls (215.2000,258.6260) and
        (216.2300,258.7160) .. (220.0320,258.7160) .. controls (224.1560,258.7160) and
        (225.7670,258.8200) .. (227.1210,260.2880) .. controls (229.1690,262.4540) and
        (226.1280,266.4750) .. (224.8780,267.7250) .. controls (221.1790,271.3590) and
        (215.6760,275.3290) .. (214.7740,274.8520) -- cycle;
      \path[fill=cffcd0c] (215.0030,274.3750) .. controls (213.9880,273.7790) and
        (210.2180,261.6160) .. (211.9260,260.6170) .. controls (215.2450,258.6520) and
        (216.3080,258.8450) .. (220.0290,258.8450) .. controls (224.1530,258.8450) and
        (225.6830,258.8940) .. (227.0530,260.4080) .. controls (229.0820,262.5830) and
        (226.1020,266.4650) .. (224.8620,267.7060) .. controls (221.1730,271.3300) and
        (215.9530,274.9070) .. (215.0030,274.3750) -- cycle;
      \path[fill=cffcd0f] (215.2320,273.8980) .. controls (214.1560,273.2350) and
        (210.2760,261.6290) .. (211.9780,260.6370) .. controls (215.2900,258.6780) and
        (216.3850,258.9740) .. (220.0260,258.9740) .. controls (224.1500,258.9740) and
        (225.6000,258.9680) .. (226.9850,260.5270) .. controls (228.9960,262.7120) and
        (226.0770,266.4560) .. (224.8460,267.6860) .. controls (221.1670,271.3010) and
        (216.2310,274.4850) .. (215.2320,273.8980) -- cycle;
      \path[fill=cffcd11] (215.4600,273.4210) .. controls (214.3230,272.6900) and
        (210.3340,261.6420) .. (212.0290,260.6560) .. controls (215.3350,258.7040) and
        (216.4620,259.1030) .. (220.0230,259.1030) .. controls (224.1470,259.1030) and
        (225.5160,259.0420) .. (226.9180,260.6460) .. controls (228.9090,262.8400) and
        (226.0510,266.4460) .. (224.8300,267.6670) .. controls (221.1600,271.2720) and
        (216.5080,274.0630) .. (215.4600,273.4210) -- cycle;
      \path[fill=cffce14] (215.6890,272.9450) .. controls (214.4910,272.1460) and
        (210.3920,261.6550) .. (212.0810,260.6750) .. controls (215.3800,258.7290) and
        (216.5400,259.2320) .. (220.0200,259.2320) .. controls (224.1440,259.2320) and
        (225.4320,259.1160) .. (226.8500,260.7650) .. controls (228.8220,262.9690) and
        (226.0250,266.4360) .. (224.8130,267.6480) .. controls (221.1540,271.2430) and
        (216.7850,273.6410) .. (215.6890,272.9450) -- cycle;
      \path[fill=cffce16] (215.9180,272.4680) .. controls (214.6580,271.6010) and
        (210.4500,261.6680) .. (212.1320,260.6950) .. controls (215.4250,258.7550) and
        (216.6170,259.3610) .. (220.0160,259.3610) .. controls (224.1400,259.3610) and
        (225.3480,259.1900) .. (226.7820,260.8850) .. controls (228.7350,263.0980) and
        (225.9990,266.4270) .. (224.7970,267.6280) .. controls (221.1470,271.2140) and
        (217.0620,273.2190) .. (215.9180,272.4680) -- cycle;
      \path[fill=cffce19] (216.1470,271.9910) .. controls (214.8260,271.0570) and
        (210.5080,261.6800) .. (212.1840,260.7140) .. controls (215.4700,258.7810) and
        (216.6940,259.4890) .. (220.0130,259.4890) .. controls (224.1370,259.4890) and
        (225.2650,259.2640) .. (226.7150,261.0040) .. controls (228.6480,263.2270) and
        (225.9740,266.4170) .. (224.7810,267.6090) .. controls (221.1410,271.1850) and
        (217.3390,272.7960) .. (216.1470,271.9910) -- cycle;
      \path[fill=cffce1c] (216.3750,271.5140) .. controls (214.9930,270.5120) and
        (210.5660,261.6930) .. (212.2350,260.7330) .. controls (215.5150,258.8070) and
        (216.7720,259.6180) .. (220.0100,259.6180) .. controls (224.1340,259.6180) and
        (225.1810,259.3390) .. (226.6470,261.1230) .. controls (228.5610,263.3560) and
        (225.9480,266.4070) .. (224.7650,267.5900) .. controls (221.1340,271.1560) and
        (217.6160,272.3740) .. (216.3750,271.5140) -- cycle;
      \path[fill=cffcf1e] (216.6040,271.0370) .. controls (215.1610,269.9680) and
        (210.6240,261.7060) .. (212.2870,260.7530) .. controls (215.5600,258.8320) and
        (216.8490,259.7470) .. (220.0070,259.7470) .. controls (224.1310,259.7470) and
        (225.0970,259.4130) .. (226.5790,261.2420) .. controls (228.4740,263.4850) and
        (225.9220,266.3980) .. (224.7490,267.5700) .. controls (221.1280,271.1270) and
        (217.8930,271.9520) .. (216.6040,271.0370) -- cycle;
      \path[fill=cffcf21] (216.8330,270.5600) .. controls (215.3280,269.4230) and
        (210.6820,261.7190) .. (212.3380,260.7720) .. controls (215.6060,258.8580) and
        (216.9260,259.8760) .. (220.0040,259.8760) .. controls (224.1280,259.8760) and
        (225.0130,259.4870) .. (226.5120,261.3620) .. controls (228.3870,263.6130) and
        (225.8960,266.3880) .. (224.7330,267.5510) .. controls (221.1210,271.0980) and
        (218.1700,271.5300) .. (216.8330,270.5600) -- cycle;
      \path[fill=cffcf23] (217.0620,270.0830) .. controls (215.4960,268.8790) and
        (210.7400,261.7320) .. (212.3900,260.7910) .. controls (215.6510,258.8840) and
        (217.0030,260.0050) .. (220.0000,260.0050) .. controls (224.1240,260.0050) and
        (224.9300,259.5610) .. (226.4440,261.4810) .. controls (228.3000,263.7420) and
        (225.8710,266.3780) .. (224.7170,267.5320) .. controls (221.1150,271.0690) and
        (218.4470,271.1080) .. (217.0620,270.0830) -- cycle;
      \path[fill=cffcf26] (217.2900,269.6070) .. controls (215.6630,268.3340) and
        (210.7980,261.7450) .. (212.4410,260.8110) .. controls (215.6960,258.9100) and
        (217.0810,260.1340) .. (219.9970,260.1340) .. controls (224.1210,260.1340) and
        (224.8460,259.6350) .. (226.3760,261.6000) .. controls (228.2130,263.8710) and
        (225.8450,266.3690) .. (224.7010,267.5120) .. controls (221.1080,271.0400) and
        (218.7240,270.6860) .. (217.2900,269.6070) -- cycle;
      \path[fill=cffd028] (217.5190,269.1300) .. controls (215.8310,267.7890) and
        (210.8560,261.7580) .. (212.4930,260.8300) .. controls (215.7410,258.9350) and
        (217.1580,260.2630) .. (219.9940,260.2630) .. controls (224.1180,260.2630) and
        (224.7620,259.7090) .. (226.3090,261.7190) .. controls (228.1260,264.0000) and
        (225.8190,266.3590) .. (224.6850,267.4930) .. controls (221.1020,271.0110) and
        (219.0010,270.2640) .. (217.5190,269.1300) -- cycle;
      \path[fill=cffd02b] (217.7480,268.6530) .. controls (215.9990,267.2450) and
        (210.9140,261.7710) .. (212.5450,260.8490) .. controls (215.7860,258.9610) and
        (217.2350,260.3920) .. (219.9910,260.3920) .. controls (224.1150,260.3920) and
        (224.6780,259.7830) .. (226.2410,261.8390) .. controls (228.0390,264.1290) and
        (225.7930,266.3490) .. (224.6690,267.4740) .. controls (221.0960,270.9830) and
        (219.2780,269.8420) .. (217.7480,268.6530) -- cycle;
      \path[fill=cffd02d] (217.9770,268.1760) .. controls (216.1660,266.7000) and
        (210.9720,261.7840) .. (212.5960,260.8690) .. controls (215.8310,258.9870) and
        (217.3130,260.5210) .. (219.9880,260.5210) .. controls (224.1120,260.5210) and
        (224.5950,259.8570) .. (226.1740,261.9580) .. controls (227.9520,264.2580) and
        (225.7680,266.3400) .. (224.6520,267.4540) .. controls (221.0890,270.9540) and
        (219.5550,269.4200) .. (217.9770,268.1760) -- cycle;
      \path[fill=cffd030] (218.2050,267.6990) .. controls (216.3340,266.1560) and
        (211.0300,261.7970) .. (212.6480,260.8880) .. controls (215.8760,259.0130) and
        (217.3900,260.6500) .. (219.9840,260.6500) .. controls (224.1080,260.6500) and
        (224.5110,259.9310) .. (226.1060,262.0770) .. controls (227.8650,264.3870) and
        (225.7420,266.3300) .. (224.6360,267.4350) .. controls (221.0830,270.9250) and
        (219.8320,268.9980) .. (218.2050,267.6990) -- cycle;
    \path[fill=cffd133] (218.4340,267.2220) .. controls (216.5010,265.6110) and
      (211.0880,261.8090) .. (212.6990,260.9070) .. controls (215.9210,259.0380) and
      (217.4670,260.7780) .. (219.9810,260.7780) .. controls (224.1050,260.7780) and
      (224.4270,260.0050) .. (226.0380,262.1960) .. controls (227.7780,264.5150) and
      (225.7160,266.3200) .. (224.6200,267.4150) .. controls (221.0760,270.8960) and
      (220.1090,268.5760) .. (218.4340,267.2220) -- cycle;
    \path[fill=cffcc00] (280.3650,264.3710) .. controls (284.1070,269.8310) and
      (280.3040,277.1310) .. (283.0030,281.4870) .. controls (276.1940,275.2300) and
      (273.0650,272.6530) .. (263.6790,281.8550) .. controls (266.2560,278.1130) and
      (266.8080,275.5980) .. (268.4030,272.0400) .. controls (269.5070,269.5860) and
      (272.7580,262.4700) .. (273.4330,262.4700) .. controls (274.0460,262.4700) and
      (278.1570,261.1200) .. (280.3650,264.3710) -- cycle;
      \path[fill=cffcc02] (280.3260,264.4090) .. controls (283.9810,269.7440) and
        (280.3610,277.0100) .. (282.9040,281.1320) .. controls (276.2360,275.0350) and
        (273.1750,272.4660) .. (263.9960,281.4550) .. controls (266.4170,277.9270) and
        (267.0220,275.3600) .. (268.6020,271.8620) .. controls (269.7240,269.4010) and
        (272.8450,262.5610) .. (273.5530,262.5520) .. controls (274.1800,262.5450) and
        (278.1690,261.2340) .. (280.3260,264.4090) -- cycle;
      \path[fill=cffcc05] (280.2870,264.4460) .. controls (283.8550,269.6560) and
        (280.4190,276.8880) .. (282.8050,280.7780) .. controls (276.2780,274.8410) and
        (273.2850,272.2790) .. (264.3130,281.0560) .. controls (266.5790,277.7410) and
        (267.2350,275.1220) .. (268.8000,271.6840) .. controls (269.9400,269.2160) and
        (272.9320,262.6520) .. (273.6730,262.6330) .. controls (274.3140,262.6190) and
        (278.1810,261.3480) .. (280.2870,264.4460) -- cycle;
      \path[fill=cffcc07] (280.2490,264.4840) .. controls (283.7280,269.5690) and
        (280.4760,276.7660) .. (282.7050,280.4230) .. controls (276.3190,274.6460) and
        (273.3950,272.0920) .. (264.6290,280.6560) .. controls (266.7400,277.5560) and
        (267.4480,274.8840) .. (268.9990,271.5060) .. controls (270.1560,269.0310) and
        (273.0200,262.7430) .. (273.7930,262.7140) .. controls (274.4490,262.6940) and
        (278.1920,261.4610) .. (280.2490,264.4840) -- cycle;
      \path[fill=cffcd0a] (280.2100,264.5210) .. controls (283.6020,269.4810) and
        (280.5330,276.6440) .. (282.6060,280.0690) .. controls (276.3610,274.4520) and
        (273.5050,271.9050) .. (264.9460,280.2570) .. controls (266.9010,277.3700) and
        (267.6620,274.6460) .. (269.1970,271.3280) .. controls (270.3730,268.8460) and
        (273.1070,262.8340) .. (273.9130,262.7950) .. controls (274.5830,262.7680) and
        (278.2040,261.5750) .. (280.2100,264.5210) -- cycle;
      \path[fill=cffcd0c] (280.1710,264.5590) .. controls (283.4760,269.3940) and
        (280.5900,276.5230) .. (282.5060,279.7140) .. controls (276.4020,274.2570) and
        (273.6150,271.7180) .. (265.2620,279.8570) .. controls (267.0620,277.1840) and
        (267.8750,274.4080) .. (269.3950,271.1500) .. controls (270.5890,268.6610) and
        (273.1940,262.9250) .. (274.0320,262.8760) .. controls (274.7170,262.8420) and
        (278.2150,261.6880) .. (280.1710,264.5590) -- cycle;
      \path[fill=cffcd0f] (280.1320,264.5960) .. controls (283.3500,269.3060) and
        (280.6480,276.4010) .. (282.4070,279.3600) .. controls (276.4440,274.0620) and
        (273.7250,271.5310) .. (265.5790,279.4580) .. controls (267.2230,276.9980) and
        (268.0890,274.1700) .. (269.5940,270.9720) .. controls (270.8060,268.4760) and
        (273.2810,263.0160) .. (274.1520,262.9570) .. controls (274.8510,262.9170) and
        (278.2270,261.8020) .. (280.1320,264.5960) -- cycle;
      \path[fill=cffcd11] (280.0930,264.6340) .. controls (283.2230,269.2190) and
        (280.7050,276.2790) .. (282.3080,279.0060) .. controls (276.4860,273.8680) and
        (273.8350,271.3440) .. (265.8960,279.0580) .. controls (267.3850,276.8130) and
        (268.3020,273.9330) .. (269.7920,270.7940) .. controls (271.0220,268.2910) and
        (273.3680,263.1070) .. (274.2720,263.0380) .. controls (274.9850,262.9910) and
        (278.2390,261.9160) .. (280.0930,264.6340) -- cycle;
      \path[fill=cffce14] (280.0540,264.6710) .. controls (283.0970,269.1310) and
        (280.7620,276.1580) .. (282.2080,278.6510) .. controls (276.5270,273.6730) and
        (273.9450,271.1570) .. (266.2120,278.6590) .. controls (267.5460,276.6270) and
        (268.5150,273.6950) .. (269.9910,270.6160) .. controls (271.2380,268.1060) and
        (273.4550,263.1980) .. (274.3920,263.1190) .. controls (275.1190,263.0660) and
        (278.2500,262.0290) .. (280.0540,264.6710) -- cycle;
      \path[fill=cffce16] (280.0150,264.7090) .. controls (282.9710,269.0440) and
        (280.8190,276.0360) .. (282.1090,278.2970) .. controls (276.5690,273.4790) and
        (274.0550,270.9700) .. (266.5290,278.2600) .. controls (267.7070,276.4410) and
        (268.7290,273.4570) .. (270.1890,270.4380) .. controls (271.4550,267.9210) and
        (273.5420,263.2890) .. (274.5120,263.2000) .. controls (275.2530,263.1400) and
        (278.2620,262.1430) .. (280.0150,264.7090) -- cycle;
      \path[fill=cffce19] (279.9760,264.7460) .. controls (282.8450,268.9560) and
        (280.8770,275.9140) .. (282.0100,277.9420) .. controls (276.6110,273.2840) and
        (274.1650,270.7830) .. (266.8460,277.8600) .. controls (267.8680,276.2550) and
        (268.9420,273.2190) .. (270.3880,270.2600) .. controls (271.6710,267.7360) and
        (273.6290,263.3800) .. (274.6320,263.2810) .. controls (275.3870,263.2140) and
        (278.2740,262.2560) .. (279.9760,264.7460) -- cycle;
      \path[fill=cffce1c] (279.9370,264.7840) .. controls (282.7190,268.8680) and
        (280.9340,275.7920) .. (281.9100,277.5880) .. controls (276.6520,273.0900) and
        (274.2750,270.5960) .. (267.1620,277.4610) .. controls (268.0290,276.0700) and
        (269.1560,272.9810) .. (270.5860,270.0820) .. controls (271.8880,267.5510) and
        (273.7160,263.4710) .. (274.7520,263.3620) .. controls (275.5210,263.2890) and
        (278.2850,262.3700) .. (279.9370,264.7840) -- cycle;
      \path[fill=cffcf1e] (279.8980,264.8210) .. controls (282.5920,268.7810) and
        (280.9910,275.6710) .. (281.8110,277.2330) .. controls (276.6940,272.8950) and
        (274.3850,270.4080) .. (267.4790,277.0610) .. controls (268.1910,275.8840) and
        (269.3690,272.7430) .. (270.7850,269.9040) .. controls (272.1040,267.3660) and
        (273.8030,263.5620) .. (274.8710,263.4430) .. controls (275.6550,263.3630) and
        (278.2970,262.4840) .. (279.8980,264.8210) -- cycle;
      \path[fill=cffcf21] (279.8600,264.8590) .. controls (282.4660,268.6930) and
        (281.0480,275.5490) .. (281.7120,276.8790) .. controls (276.7360,272.7010) and
        (274.4950,270.2210) .. (267.7960,276.6620) .. controls (268.3520,275.6980) and
        (269.5820,272.5050) .. (270.9830,269.7260) .. controls (272.3200,267.1810) and
        (273.8910,263.6530) .. (274.9910,263.5240) .. controls (275.7900,263.4380) and
        (278.3090,262.5970) .. (279.8600,264.8590) -- cycle;
      \path[fill=cffcf23] (279.8210,264.8960) .. controls (282.3400,268.6060) and
        (281.1060,275.4270) .. (281.6120,276.5240) .. controls (276.7770,272.5060) and
        (274.6050,270.0340) .. (268.1120,276.2620) .. controls (268.5130,275.5120) and
        (269.7960,272.2670) .. (271.1820,269.5480) .. controls (272.5370,266.9960) and
        (273.9780,263.7440) .. (275.1110,263.6050) .. controls (275.9240,263.5120) and
        (278.3200,262.7110) .. (279.8210,264.8960) -- cycle;
      \path[fill=cffcf26] (279.7820,264.9340) .. controls (282.2140,268.5180) and
        (281.1630,275.3050) .. (281.5130,276.1700) .. controls (276.8190,272.3120) and
        (274.7150,269.8470) .. (268.4290,275.8630) .. controls (268.6740,275.3270) and
        (270.0090,272.0300) .. (271.3800,269.3700) .. controls (272.7530,266.8110) and
        (274.0650,263.8350) .. (275.2310,263.6860) .. controls (276.0580,263.5860) and
        (278.3320,262.8240) .. (279.7820,264.9340) -- cycle;
      \path[fill=cffd028] (279.7430,264.9710) .. controls (282.0870,268.4310) and
        (281.2200,275.1840) .. (281.4140,275.8160) .. controls (276.8610,272.1170) and
        (274.8250,269.6600) .. (268.7460,275.4630) .. controls (268.8350,275.1410) and
        (270.2230,271.7920) .. (271.5780,269.1920) .. controls (272.9700,266.6260) and
        (274.1520,263.9260) .. (275.3510,263.7670) .. controls (276.1920,263.6610) and
        (278.3440,262.9380) .. (279.7430,264.9710) -- cycle;
      \path[fill=cffd02b] (279.7040,265.0090) .. controls (281.9610,268.3430) and
        (281.2770,275.0620) .. (281.3140,275.4610) .. controls (276.9020,271.9220) and
        (274.9350,269.4730) .. (269.0620,275.0640) .. controls (268.9970,274.9550) and
        (270.4360,271.5540) .. (271.7770,269.0140) .. controls (273.1860,266.4410) and
        (274.2390,264.0170) .. (275.4710,263.8480) .. controls (276.3260,263.7350) and
        (278.3550,263.0520) .. (279.7040,265.0090) -- cycle;
      \path[fill=cffd02d] (279.6650,265.0460) .. controls (281.8350,268.2560) and
        (281.3350,274.9400) .. (281.2150,275.1070) .. controls (276.9440,271.7280) and
        (275.0450,269.2860) .. (269.3790,274.6650) .. controls (269.1580,274.7690) and
        (270.6490,271.3160) .. (271.9750,268.8360) .. controls (273.4020,266.2560) and
        (274.3260,264.1080) .. (275.5910,263.9290) .. controls (276.4600,263.8100) and
        (278.3670,263.1650) .. (279.6650,265.0460) -- cycle;
      \path[fill=cffd030] (279.6260,265.0840) .. controls (281.7080,268.1680) and
        (281.3920,274.8190) .. (281.1160,274.7520) .. controls (276.9860,271.5330) and
        (275.1550,269.0990) .. (269.6960,274.2650) .. controls (269.3190,274.5830) and
        (270.8630,271.0780) .. (272.1740,268.6580) .. controls (273.6190,266.0710) and
        (274.4130,264.1990) .. (275.7100,264.0100) .. controls (276.5940,263.8840) and
        (278.3790,263.2790) .. (279.6260,265.0840) -- cycle;
    \path[fill=cffd133] (279.5870,265.1210) .. controls (281.5820,268.0800) and
      (281.4490,274.6970) .. (281.0160,274.3980) .. controls (277.0270,271.3390) and
      (275.2650,268.9120) .. (270.0120,273.8660) .. controls (269.4800,274.3980) and
      (271.0760,270.8400) .. (272.3720,268.4800) .. controls (273.8350,265.8860) and
      (274.5000,264.2900) .. (275.8300,264.0910) .. controls (276.7280,263.9580) and
      (278.3900,263.3920) .. (279.5870,265.1210) -- cycle;
    \path[fill=cffcc00] (283.4580,284.4780) .. controls (289.4340,292.3260) and
      (300.5220,300.7500) .. (304.4820,303.0540) .. controls (307.3620,304.7100) and
      (311.5380,306.6540) .. (311.4660,311.8380) .. controls (311.3220,317.7420) and
      (308.2980,319.3980) .. (306.6420,321.0540) .. controls (303.3300,324.3660) and
      (284.4660,331.9980) .. (272.4420,337.3260) .. controls (260.2020,342.7260) and
      (251.3460,345.5340) .. (245.2260,346.5420) .. controls (241.1220,347.1900) and
      (238.1700,348.9180) .. (232.0500,347.6940) .. controls (227.7300,346.8300) and
      (225.1380,346.3980) .. (222.1860,343.8060) .. controls (219.2340,341.2860) and
      (216.2100,338.6220) .. (215.7780,334.4460) .. controls (214.5540,324.0780) and
      (219.4500,317.5260) .. (224.2020,308.5260) .. controls (228.0900,301.3260) and
      (235.9380,299.8860) .. (240.8340,300.5340) .. controls (258.4020,302.9100) and
      (257.2500,291.8940) .. (262.0740,287.0700) .. controls (266.5380,282.6060) and
      (279.2820,279.0060) .. (283.4580,284.4780) -- cycle;
      \path[fill=cffcc02] (283.4060,284.5510) .. controls (289.3680,292.3810) and
        (300.4300,300.7850) .. (304.3810,303.0840) .. controls (307.2540,304.7360) and
        (311.4210,306.6750) .. (311.3490,311.8470) .. controls (311.2050,317.7370) and
        (308.1880,319.3890) .. (306.5360,321.0420) .. controls (303.2320,324.3460) and
        (284.1960,332.0350) .. (272.2960,337.1310) .. controls (260.0370,342.3060) and
        (251.6480,345.0040) .. (245.4940,346.0890) .. controls (241.4170,346.7730) and
        (238.3380,348.4830) .. (232.2360,347.2660) .. controls (227.9330,346.4100) and
        (225.4700,346.0340) .. (222.5290,343.4550) .. controls (219.5910,340.9450) and
        (216.7730,338.4930) .. (216.3450,334.3370) .. controls (215.1340,323.9970) and
        (219.7100,317.6800) .. (224.3890,308.6870) .. controls (228.2780,301.3120) and
        (236.1800,300.2530) .. (241.0540,300.9090) .. controls (258.5670,303.3240) and
        (257.2600,291.9500) .. (262.0730,287.1370) .. controls (266.5260,282.6840) and
        (279.2400,279.0920) .. (283.4060,284.5510) -- cycle;
      \path[fill=cffcc05] (283.3550,284.6250) .. controls (289.3030,292.4360) and
        (300.3380,300.8200) .. (304.2800,303.1130) .. controls (307.1460,304.7610) and
        (311.3030,306.6960) .. (311.2310,311.8560) .. controls (311.0880,317.7320) and
        (308.0780,319.3800) .. (306.4300,321.0290) .. controls (303.1330,324.3250) and
        (283.9260,332.0720) .. (272.1500,336.9360) .. controls (259.8720,341.8850) and
        (251.9490,344.4740) .. (245.7620,345.6350) .. controls (241.7110,346.3560) and
        (238.5060,348.0480) .. (232.4210,346.8370) .. controls (228.1350,345.9910) and
        (225.8030,345.6700) .. (222.8710,343.1040) .. controls (219.9470,340.6030) and
        (217.3350,338.3630) .. (216.9120,334.2270) .. controls (215.7140,323.9150) and
        (219.9700,317.8330) .. (224.5760,308.8480) .. controls (228.4660,301.2970) and
        (236.4210,300.6190) .. (241.2740,301.2840) .. controls (258.7320,303.7380) and
        (257.2700,292.0060) .. (262.0710,287.2040) .. controls (266.5140,282.7610) and
        (279.1980,279.1780) .. (283.3550,284.6250) -- cycle;
      \path[fill=cffcc07] (283.3030,284.6980) .. controls (289.2370,292.4900) and
        (300.2470,300.8550) .. (304.1790,303.1430) .. controls (307.0380,304.7870) and
        (311.1850,306.7170) .. (311.1130,311.8650) .. controls (310.9700,317.7270) and
        (307.9680,319.3720) .. (306.3230,321.0160) .. controls (303.0350,324.3050) and
        (283.6550,332.1090) .. (272.0040,336.7410) .. controls (259.7060,341.4650) and
        (252.2510,343.9440) .. (246.0300,345.1820) .. controls (242.0060,345.9380) and
        (238.6730,347.6130) .. (232.6070,346.4080) .. controls (228.3380,345.5710) and
        (226.1350,345.3060) .. (223.2140,342.7530) .. controls (220.3030,340.2610) and
        (217.8970,338.2340) .. (217.4790,334.1180) .. controls (216.2940,323.8330) and
        (220.2290,317.9860) .. (224.7630,309.0080) .. controls (228.6540,301.2830) and
        (236.6630,300.9850) .. (241.4940,301.6590) .. controls (258.8970,304.1530) and
        (257.2790,292.0610) .. (262.0690,287.2710) .. controls (266.5020,282.8390) and
        (279.1560,279.2640) .. (283.3030,284.6980) -- cycle;
      \path[fill=cffcd0a] (283.2510,284.7710) .. controls (289.1710,292.5450) and
        (300.1550,300.8900) .. (304.0770,303.1720) .. controls (306.9300,304.8130) and
        (311.0670,306.7380) .. (310.9960,311.8740) .. controls (310.8530,317.7220) and
        (307.8580,319.3630) .. (306.2170,321.0030) .. controls (302.9360,324.2840) and
        (283.3850,332.1460) .. (271.8580,336.5460) .. controls (259.5410,341.0440) and
        (252.5520,343.4140) .. (246.2980,344.7280) .. controls (242.3010,345.5210) and
        (238.8410,347.1780) .. (232.7920,345.9790) .. controls (228.5400,345.1510) and
        (226.4670,344.9430) .. (223.5560,342.4020) .. controls (220.6590,339.9200) and
        (218.4600,338.1040) .. (218.0460,334.0090) .. controls (216.8740,323.7520) and
        (220.4890,318.1400) .. (224.9500,309.1690) .. controls (228.8420,301.2680) and
        (236.9050,301.3510) .. (241.7130,302.0340) .. controls (259.0620,304.5670) and
        (257.2890,292.1170) .. (262.0680,287.3380) .. controls (266.4900,282.9160) and
        (279.1140,279.3500) .. (283.2510,284.7710) -- cycle;
      \path[fill=cffcd0c] (283.1990,284.8440) .. controls (289.1050,292.6000) and
        (300.0630,300.9250) .. (303.9760,303.2020) .. controls (306.8230,304.8380) and
        (310.9490,306.7590) .. (310.8780,311.8830) .. controls (310.7360,317.7170) and
        (307.7470,319.3540) .. (306.1110,320.9900) .. controls (302.8380,324.2630) and
        (283.1150,332.1830) .. (271.7120,336.3510) .. controls (259.3750,340.6240) and
        (252.8540,342.8840) .. (246.5650,344.2750) .. controls (242.5950,345.1040) and
        (239.0090,346.7430) .. (232.9780,345.5510) .. controls (228.7430,344.7310) and
        (226.7990,344.5790) .. (223.8990,342.0510) .. controls (221.0160,339.5780) and
        (219.0220,337.9750) .. (218.6120,333.8990) .. controls (217.4540,323.6700) and
        (220.7490,318.2930) .. (225.1360,309.3300) .. controls (229.0300,301.2540) and
        (237.1460,301.7180) .. (241.9330,302.4100) .. controls (259.2260,304.9810) and
        (257.2990,292.1730) .. (262.0660,287.4050) .. controls (266.4780,282.9940) and
        (279.0720,279.4360) .. (283.1990,284.8440) -- cycle;
      \path[fill=cffcd0f] (283.1470,284.9170) .. controls (289.0390,292.6540) and
        (299.9710,300.9600) .. (303.8750,303.2310) .. controls (306.7150,304.8640) and
        (310.8320,306.7800) .. (310.7610,311.8910) .. controls (310.6190,317.7120) and
        (307.6370,319.3450) .. (306.0050,320.9770) .. controls (302.7390,324.2430) and
        (282.8440,332.2200) .. (271.5660,336.1560) .. controls (259.2100,340.2030) and
        (253.1550,342.3540) .. (246.8330,343.8210) .. controls (242.8900,344.6870) and
        (239.1770,346.3080) .. (233.1640,345.1220) .. controls (228.9460,344.3110) and
        (227.1310,344.2150) .. (224.2410,341.7000) .. controls (221.3720,339.2360) and
        (219.5850,337.8450) .. (219.1790,333.7900) .. controls (218.0340,323.5880) and
        (221.0090,318.4460) .. (225.3230,309.4910) .. controls (229.2180,301.2390) and
        (237.3880,302.0840) .. (242.1530,302.7850) .. controls (259.3910,305.3950) and
        (257.3080,292.2280) .. (262.0650,287.4720) .. controls (266.4660,283.0710) and
        (279.0300,279.5220) .. (283.1470,284.9170) -- cycle;
      \path[fill=cffcd11] (283.0960,284.9900) .. controls (288.9730,292.7090) and
        (299.8790,300.9940) .. (303.7740,303.2600) .. controls (306.6070,304.8890) and
        (310.7140,306.8010) .. (310.6430,311.9000) .. controls (310.5010,317.7070) and
        (307.5270,319.3360) .. (305.8980,320.9650) .. controls (302.6410,324.2220) and
        (282.5740,332.2570) .. (271.4200,335.9600) .. controls (259.0450,339.7830) and
        (253.4570,341.8240) .. (247.1010,343.3680) .. controls (243.1840,344.2690) and
        (239.3440,345.8730) .. (233.3490,344.6930) .. controls (229.1480,343.8910) and
        (227.4630,343.8510) .. (224.5840,341.3490) .. controls (221.7280,338.8950) and
        (220.1470,337.7160) .. (219.7460,333.6800) .. controls (218.6140,323.5070) and
        (221.2680,318.5990) .. (225.5100,309.6510) .. controls (229.4060,301.2250) and
        (237.6290,302.4500) .. (242.3730,303.1600) .. controls (259.5560,305.8090) and
        (257.3180,292.2840) .. (262.0630,287.5390) .. controls (266.4540,283.1490) and
        (278.9880,279.6080) .. (283.0960,284.9900) -- cycle;
      \path[fill=cffce14] (283.0440,285.0630) .. controls (288.9070,292.7640) and
        (299.7870,301.0290) .. (303.6730,303.2900) .. controls (306.4990,304.9150) and
        (310.5960,306.8220) .. (310.5250,311.9090) .. controls (310.3840,317.7020) and
        (307.4170,319.3270) .. (305.7920,320.9520) .. controls (302.5420,324.2020) and
        (282.3030,332.2940) .. (271.2740,335.7650) .. controls (258.8790,339.3620) and
        (253.7580,341.2940) .. (247.3690,342.9140) .. controls (243.4790,343.8520) and
        (239.5120,345.4380) .. (233.5350,344.2640) .. controls (229.3510,343.4710) and
        (227.7950,343.4870) .. (224.9260,340.9980) .. controls (222.0850,338.5530) and
        (220.7090,337.5860) .. (220.3130,333.5710) .. controls (219.1940,323.4250) and
        (221.5280,318.7530) .. (225.6970,309.8120) .. controls (229.5940,301.2100) and
        (237.8710,302.8160) .. (242.5930,303.5350) .. controls (259.7210,306.2230) and
        (257.3280,292.3400) .. (262.0610,287.6060) .. controls (266.4410,283.2260) and
        (278.9460,279.6940) .. (283.0440,285.0630) -- cycle;
      \path[fill=cffce16] (282.9920,285.1360) .. controls (288.8420,292.8180) and
        (299.6950,301.0640) .. (303.5710,303.3190) .. controls (306.3910,304.9400) and
        (310.4780,306.8430) .. (310.4080,311.9180) .. controls (310.2670,317.6970) and
        (307.3070,319.3180) .. (305.6860,320.9390) .. controls (302.4440,324.1810) and
        (282.0330,332.3310) .. (271.1280,335.5700) .. controls (258.7140,338.9420) and
        (254.0600,340.7640) .. (247.6370,342.4610) .. controls (243.7740,343.4350) and
        (239.6800,345.0030) .. (233.7200,343.8360) .. controls (229.5530,343.0520) and
        (228.1280,343.1230) .. (225.2690,340.6470) .. controls (222.4410,338.2110) and
        (221.2720,337.4560) .. (220.8800,333.4610) .. controls (219.7740,323.3430) and
        (221.7880,318.9060) .. (225.8840,309.9730) .. controls (229.7820,301.1960) and
        (238.1130,303.1830) .. (242.8120,303.9100) .. controls (259.8860,306.6370) and
        (257.3380,292.3950) .. (262.0600,287.6730) .. controls (266.4290,283.3040) and
        (278.9040,279.7800) .. (282.9920,285.1360) -- cycle;
      \path[fill=cffce19] (282.9400,285.2090) .. controls (288.7760,292.8730) and
        (299.6030,301.0990) .. (303.4700,303.3490) .. controls (306.2830,304.9660) and
        (310.3610,306.8640) .. (310.2900,311.9270) .. controls (310.1500,317.6920) and
        (307.1970,319.3090) .. (305.5800,320.9260) .. controls (302.3450,324.1600) and
        (281.7630,332.3680) .. (270.9820,335.3750) .. controls (258.5490,338.5210) and
        (254.3610,340.2340) .. (247.9050,342.0070) .. controls (244.0680,343.0170) and
        (239.8480,344.5680) .. (233.9060,343.4070) .. controls (229.7560,342.6320) and
        (228.4600,342.7590) .. (225.6110,340.2960) .. controls (222.7970,337.8700) and
        (221.8340,337.3270) .. (221.4470,333.3520) .. controls (220.3540,323.2620) and
        (222.0480,319.0590) .. (226.0710,310.1330) .. controls (229.9700,301.1810) and
        (238.3540,303.5490) .. (243.0320,304.2850) .. controls (260.0510,307.0510) and
        (257.3470,292.4510) .. (262.0580,287.7400) .. controls (266.4170,283.3810) and
        (278.8620,279.8660) .. (282.9400,285.2090) -- cycle;
      \path[fill=cffce1c] (282.8880,285.2820) .. controls (288.7100,292.9270) and
        (299.5110,301.1340) .. (303.3690,303.3780) .. controls (306.1750,304.9920) and
        (310.2430,306.8850) .. (310.1730,311.9350) .. controls (310.0320,317.6870) and
        (307.0870,319.3000) .. (305.4730,320.9130) .. controls (302.2470,324.1400) and
        (281.4920,332.4050) .. (270.8360,335.1800) .. controls (258.3830,338.1010) and
        (254.6630,339.7040) .. (248.1730,341.5540) .. controls (244.3630,342.6000) and
        (240.0150,344.1330) .. (234.0910,342.9780) .. controls (229.9580,342.2120) and
        (228.7920,342.3950) .. (225.9540,339.9450) .. controls (223.1540,337.5280) and
        (222.3970,337.1970) .. (222.0140,333.2420) .. controls (220.9340,323.1800) and
        (222.3070,319.2130) .. (226.2580,310.2940) .. controls (230.1580,301.1670) and
        (238.5960,303.9150) .. (243.2520,304.6600) .. controls (260.2160,307.4650) and
        (257.3570,292.5070) .. (262.0570,287.8070) .. controls (266.4050,283.4590) and
        (278.8200,279.9520) .. (282.8880,285.2820) -- cycle;
      \path[fill=cffcf1e] (282.8370,285.3550) .. controls (288.6440,292.9820) and
        (299.4190,301.1690) .. (303.2680,303.4080) .. controls (306.0670,305.0170) and
        (310.1250,306.9060) .. (310.0550,311.9440) .. controls (309.9150,317.6820) and
        (306.9760,319.2910) .. (305.3670,320.9010) .. controls (302.1480,324.1190) and
        (281.2220,332.4420) .. (270.6900,334.9850) .. controls (258.2180,337.6800) and
        (254.9640,339.1740) .. (248.4400,341.1000) .. controls (244.6570,342.1830) and
        (240.1830,343.6980) .. (234.2770,342.5490) .. controls (230.1610,341.7920) and
        (229.1240,342.0310) .. (226.2960,339.5940) .. controls (223.5100,337.1860) and
        (222.9590,337.0680) .. (222.5800,333.1330) .. controls (221.5140,323.0980) and
        (222.5670,319.3660) .. (226.4440,310.4550) .. controls (230.3460,301.1520) and
        (238.8370,304.2820) .. (243.4720,305.0350) .. controls (260.3800,307.8790) and
        (257.3670,292.5620) .. (262.0550,287.8740) .. controls (266.3930,283.5360) and
        (278.7780,280.0380) .. (282.8370,285.3550) -- cycle;
      \path[fill=cffcf21] (282.7850,285.4280) .. controls (288.5780,293.0370) and
        (299.3280,301.2040) .. (303.1670,303.4370) .. controls (305.9590,305.0430) and
        (310.0070,306.9270) .. (309.9370,311.9530) .. controls (309.7980,317.6770) and
        (306.8660,319.2820) .. (305.2610,320.8880) .. controls (302.0500,324.0990) and
        (280.9520,332.4790) .. (270.5440,334.7900) .. controls (258.0530,337.2600) and
        (255.2660,338.6440) .. (248.7080,340.6470) .. controls (244.9520,341.7660) and
        (240.3510,343.2630) .. (234.4620,342.1210) .. controls (230.3630,341.3720) and
        (229.4560,341.6670) .. (226.6390,339.2430) .. controls (223.8660,336.8450) and
        (223.5210,336.9380) .. (223.1470,333.0240) .. controls (222.0940,323.0170) and
        (222.8270,319.5190) .. (226.6310,310.6160) .. controls (230.5340,301.1380) and
        (239.0790,304.6480) .. (243.6920,305.4100) .. controls (260.5450,308.2930) and
        (257.3760,292.6180) .. (262.0530,287.9410) .. controls (266.3810,283.6130) and
        (278.7360,280.1230) .. (282.7850,285.4280) -- cycle;
      \path[fill=cffcf23] (282.7330,285.5010) .. controls (288.5120,293.0910) and
        (299.2360,301.2380) .. (303.0650,303.4670) .. controls (305.8510,305.0680) and
        (309.8900,306.9480) .. (309.8200,311.9620) .. controls (309.6810,317.6720) and
        (306.7560,319.2730) .. (305.1550,320.8750) .. controls (301.9510,324.0780) and
        (280.6810,332.5160) .. (270.3980,334.5950) .. controls (257.8870,336.8390) and
        (255.5670,338.1140) .. (248.9760,340.1930) .. controls (245.2470,341.3480) and
        (240.5190,342.8270) .. (234.6480,341.6920) .. controls (230.5660,340.9520) and
        (229.7880,341.3030) .. (226.9810,338.8920) .. controls (224.2220,336.5030) and
        (224.0840,336.8090) .. (223.7140,332.9140) .. controls (222.6740,322.9350) and
        (223.0870,319.6720) .. (226.8180,310.7760) .. controls (230.7220,301.1230) and
        (239.3210,305.0140) .. (243.9110,305.7850) .. controls (260.7100,308.7070) and
        (257.3860,292.6740) .. (262.0520,288.0080) .. controls (266.3690,283.6910) and
        (278.6940,280.2090) .. (282.7330,285.5010) -- cycle;
      \path[fill=cffcf26] (282.6810,285.5740) .. controls (288.4470,293.1460) and
        (299.1440,301.2730) .. (302.9640,303.4960) .. controls (305.7430,305.0940) and
        (309.7720,306.9690) .. (309.7020,311.9710) .. controls (309.5630,317.6670) and
        (306.6460,319.2640) .. (305.0480,320.8620) .. controls (301.8530,324.0570) and
        (280.4110,332.5530) .. (270.2510,334.3990) .. controls (257.7220,336.4190) and
        (255.8690,337.5840) .. (249.2440,339.7400) .. controls (245.5410,340.9310) and
        (240.6860,342.3920) .. (234.8330,341.2630) .. controls (230.7680,340.5320) and
        (230.1200,340.9390) .. (227.3240,338.5410) .. controls (224.5790,336.1610) and
        (224.6460,336.6790) .. (224.2810,332.8050) .. controls (223.2540,322.8530) and
        (223.3460,319.8260) .. (227.0050,310.9370) .. controls (230.9100,301.1090) and
        (239.5620,305.3800) .. (244.1310,306.1600) .. controls (260.8750,309.1210) and
        (257.3960,292.7290) .. (262.0500,288.0750) .. controls (266.3570,283.7680) and
        (278.6520,280.2950) .. (282.6810,285.5740) -- cycle;
      \path[fill=cffd028] (282.6290,285.6480) .. controls (288.3810,293.2010) and
        (299.0520,301.3080) .. (302.8630,303.5260) .. controls (305.6350,305.1190) and
        (309.6540,306.9900) .. (309.5850,311.9800) .. controls (309.4460,317.6620) and
        (306.5360,319.2550) .. (304.9420,320.8490) .. controls (301.7540,324.0370) and
        (280.1410,332.5900) .. (270.1050,334.2040) .. controls (257.5570,335.9980) and
        (256.1700,337.0540) .. (249.5120,339.2860) .. controls (245.8360,340.5140) and
        (240.8540,341.9570) .. (235.0190,340.8340) .. controls (230.9710,340.1120) and
        (230.4530,340.5750) .. (227.6660,338.1900) .. controls (224.9350,335.8200) and
        (225.2090,336.5500) .. (224.8480,332.6950) .. controls (223.8340,322.7720) and
        (223.6060,319.9790) .. (227.1920,311.0980) .. controls (231.0980,301.0940) and
        (239.8040,305.7470) .. (244.3510,306.5350) .. controls (261.0400,309.5350) and
        (257.4050,292.7850) .. (262.0490,288.1420) .. controls (266.3450,283.8460) and
        (278.6100,280.3810) .. (282.6290,285.6480) -- cycle;
      \path[fill=cffd02b] (282.5780,285.7210) .. controls (288.3150,293.2550) and
        (298.9600,301.3430) .. (302.7620,303.5550) .. controls (305.5270,305.1450) and
        (309.5360,307.0110) .. (309.4670,311.9880) .. controls (309.3290,317.6570) and
        (306.4260,319.2470) .. (304.8360,320.8360) .. controls (301.6560,324.0160) and
        (279.8700,332.6270) .. (269.9590,334.0090) .. controls (257.3910,335.5780) and
        (256.4720,336.5230) .. (249.7800,338.8330) .. controls (246.1300,340.0960) and
        (241.0220,341.5220) .. (235.2050,340.4060) .. controls (231.1740,339.6930) and
        (230.7850,340.2110) .. (228.0090,337.8390) .. controls (225.2910,335.4780) and
        (225.7710,336.4200) .. (225.4150,332.5860) .. controls (224.4140,322.6900) and
        (223.8660,320.1320) .. (227.3790,311.2580) .. controls (231.2860,301.0800) and
        (240.0450,306.1130) .. (244.5710,306.9100) .. controls (261.2050,309.9490) and
        (257.4150,292.8400) .. (262.0470,288.2090) .. controls (266.3330,283.9230) and
        (278.5680,280.4670) .. (282.5780,285.7210) -- cycle;
      \path[fill=cffd02d] (282.5260,285.7940) .. controls (288.2490,293.3100) and
        (298.8680,301.3780) .. (302.6610,303.5840) .. controls (305.4190,305.1710) and
        (309.4190,307.0320) .. (309.3490,311.9970) .. controls (309.2120,317.6520) and
        (306.3160,319.2380) .. (304.7300,320.8240) .. controls (301.5570,323.9960) and
        (279.6000,332.6640) .. (269.8130,333.8140) .. controls (257.2260,335.1570) and
        (256.7730,335.9930) .. (250.0480,338.3790) .. controls (246.4250,339.6790) and
        (241.1900,341.0870) .. (235.3900,339.9770) .. controls (231.3760,339.2730) and
        (231.1170,339.8470) .. (228.3510,337.4880) .. controls (225.6480,335.1360) and
        (226.3330,336.2910) .. (225.9820,332.4760) .. controls (224.9940,322.6080) and
        (224.1260,320.2860) .. (227.5660,311.4190) .. controls (231.4740,301.0650) and
        (240.2870,306.4790) .. (244.7910,307.2850) .. controls (261.3700,310.3630) and
        (257.4250,292.8960) .. (262.0450,288.2760) .. controls (266.3200,284.0010) and
        (278.5260,280.5530) .. (282.5260,285.7940) -- cycle;
      \path[fill=cffd030] (282.4740,285.8670) .. controls (288.1830,293.3650) and
        (298.7760,301.4130) .. (302.5590,303.6140) .. controls (305.3110,305.1960) and
        (309.3010,307.0530) .. (309.2320,312.0060) .. controls (309.0940,317.6470) and
        (306.2050,319.2290) .. (304.6240,320.8110) .. controls (301.4590,323.9750) and
        (279.3300,332.7010) .. (269.6670,333.6190) .. controls (257.0610,334.7360) and
        (257.0750,335.4630) .. (250.3150,337.9260) .. controls (246.7200,339.2620) and
        (241.3570,340.6520) .. (235.5760,339.5480) .. controls (231.5790,338.8530) and
        (231.4490,339.4830) .. (228.6940,337.1370) .. controls (226.0040,334.7950) and
        (226.8960,336.1610) .. (226.5480,332.3670) .. controls (225.5740,322.5270) and
        (224.3850,320.4390) .. (227.7520,311.5800) .. controls (231.6620,301.0510) and
        (240.5290,306.8450) .. (245.0100,307.6600) .. controls (261.5340,310.7770) and
        (257.4350,292.9520) .. (262.0440,288.3430) .. controls (266.3080,284.0780) and
        (278.4840,280.6390) .. (282.4740,285.8670) -- cycle;
    \path[fill=cffd133] (282.4220,285.9400) .. controls (288.1170,293.4190) and
      (298.6840,301.4480) .. (302.4580,303.6430) .. controls (305.2030,305.2220) and
      (309.1830,307.0740) .. (309.1140,312.0150) .. controls (308.9770,317.6420) and
      (306.0950,319.2200) .. (304.5170,320.7980) .. controls (301.3600,323.9540) and
      (279.0590,332.7380) .. (269.5210,333.4240) .. controls (256.8950,334.3160) and
      (257.3760,334.9330) .. (250.5830,337.4720) .. controls (247.0140,338.8450) and
      (241.5250,340.2170) .. (235.7610,339.1190) .. controls (231.7810,338.4330) and
      (231.7810,339.1190) .. (229.0360,336.7860) .. controls (226.3600,334.4530) and
      (227.4580,336.0310) .. (227.1150,332.2570) .. controls (226.1540,322.4450) and
      (224.6450,320.5920) .. (227.9390,311.7410) .. controls (231.8500,301.0360) and
      (240.7700,307.2120) .. (245.2300,308.0350) .. controls (261.6990,311.1910) and
      (257.4440,293.0070) .. (262.0420,288.4100) .. controls (266.2960,284.1560) and
      (278.4420,280.7250) .. (282.4220,285.9400) -- cycle;
    \path[fill=c995900] (30.0901,274.6140) .. controls (23.6101,279.2940) and
      (7.1942,278.8620) .. (3.0182,284.3340) .. controls (-1.0858,289.8060) and
      (3.1622,297.7260) .. (3.0902,312.4140) .. controls (3.0902,318.6780) and
      (2.0102,323.4300) .. (1.2902,327.2460) .. controls (0.2822,332.0700) and
      (-0.3658,335.4540) .. (1.6502,338.9100) .. controls (5.3222,345.0300) and
      (11.2261,346.5420) .. (44.9941,353.5980) .. controls (63.0661,357.3420) and
      (80.1301,367.0620) .. (91.5781,367.9980) .. controls (103.0260,368.8620) and
      (105.4740,365.0460) .. (112.4580,358.8540) .. controls (119.3700,352.6620) and
      (121.6020,354.6060) .. (121.3860,340.9980) .. controls (121.1700,327.4620) and
      (112.4580,323.4300) .. (102.5940,307.8060) .. controls (92.7301,292.1820) and
      (91.1461,289.3020) .. (84.5941,278.9340) .. controls (78.0421,268.7100) and
      (65.0821,250.1340) .. (57.6661,249.9180) .. controls (51.7621,249.7740) and
      (48.4501,252.9420) .. (44.7781,256.6860) .. controls (41.1061,260.4300) and
      (36.5701,269.9340) .. (30.0901,274.6140) -- cycle;
      \path[fill=c9e5e00] (30.1945,274.9020) .. controls (23.7973,279.6040) and
        (7.7198,278.6900) .. (3.1334,284.4140) .. controls (-1.0210,289.8280) and
        (3.3602,297.6900) .. (3.2306,312.3680) .. controls (3.2054,318.5990) and
        (2.0786,323.2500) .. (1.2938,327.2460) .. controls (0.2570,332.1170) and
        (-0.3838,335.4470) .. (1.6430,338.8640) .. controls (5.4302,345.0270) and
        (11.3377,345.9880) .. (45.0985,353.0330) .. controls (63.1597,356.7700) and
        (79.6405,366.3390) .. (91.4413,367.1450) .. controls (102.6270,367.9370) and
        (105.0060,364.3160) .. (111.9040,358.1850) .. controls (118.8010,351.9890) and
        (120.9290,353.9080) .. (120.7630,340.7790) .. controls (120.6880,327.2570) and
        (112.4580,323.4300) .. (102.5940,307.8060) .. controls (92.7301,292.1820) and
        (91.1461,289.3020) .. (84.5941,278.9340) .. controls (78.0421,268.7100) and
        (65.0821,250.1340) .. (57.6661,249.9180) .. controls (51.7621,249.7740) and
        (48.4501,252.9420) .. (44.7781,256.6860) .. controls (41.1061,260.4300) and
        (36.6241,270.1650) .. (30.1945,274.9020) -- cycle;
      \path[fill=ca36400] (30.2989,275.1900) .. controls (23.9845,279.9140) and
        (8.2454,278.5170) .. (3.2486,284.4930) .. controls (-0.9562,289.8500) and
        (3.5582,297.6540) .. (3.3710,312.3210) .. controls (3.3206,318.5200) and
        (2.1470,323.0700) .. (1.2974,327.2460) .. controls (0.2318,332.1640) and
        (-0.4018,335.4400) .. (1.6358,338.8170) .. controls (5.5382,345.0230) and
        (11.4493,345.4340) .. (45.2029,352.4680) .. controls (63.2533,356.1980) and
        (79.1509,365.6150) .. (91.3045,366.2920) .. controls (102.2270,367.0120) and
        (104.5380,363.5850) .. (111.3490,357.5150) .. controls (118.2330,351.3160) and
        (120.2560,353.2100) .. (120.1410,340.5590) .. controls (120.2050,327.0520) and
        (112.4580,323.4300) .. (102.5940,307.8060) .. controls (92.7301,292.1820) and
        (91.1461,289.3020) .. (84.5941,278.9340) .. controls (78.0421,268.7100) and
        (65.0821,250.1340) .. (57.6661,249.9180) .. controls (51.7621,249.7740) and
        (48.4501,252.9420) .. (44.7781,256.6860) .. controls (41.1061,260.4300) and
        (36.6781,270.3950) .. (30.2989,275.1900) -- cycle;
      \path[fill=ca86a00] (30.4033,275.4780) .. controls (24.1717,280.2230) and
        (8.7710,278.3440) .. (3.3638,284.5720) .. controls (-0.8914,289.8710) and
        (3.7562,297.6180) .. (3.5114,312.2740) .. controls (3.4358,318.4410) and
        (2.2154,322.8900) .. (1.3010,327.2460) .. controls (0.2066,332.2110) and
        (-0.4198,335.4330) .. (1.6286,338.7700) .. controls (5.6462,345.0200) and
        (11.5609,344.8790) .. (45.3073,351.9030) .. controls (63.3469,355.6250) and
        (78.6613,364.8920) .. (91.1677,365.4390) .. controls (101.8270,366.0870) and
        (104.0700,362.8540) .. (110.7950,356.8460) .. controls (117.6640,350.6430) and
        (119.5830,352.5110) .. (119.5180,340.3400) .. controls (119.7230,326.8470) and
        (112.4580,323.4300) .. (102.5940,307.8060) .. controls (92.7301,292.1820) and
        (91.1461,289.3020) .. (84.5941,278.9340) .. controls (78.0421,268.7100) and
        (65.0821,250.1340) .. (57.6661,249.9180) .. controls (51.7621,249.7740) and
        (48.4501,252.9420) .. (44.7781,256.6860) .. controls (41.1061,260.4300) and
        (36.7321,270.6260) .. (30.4033,275.4780) -- cycle;
      \path[fill=cad7000] (30.5077,275.7660) .. controls (24.3589,280.5330) and
        (9.2966,278.1710) .. (3.4790,284.6510) .. controls (-0.8266,289.8930) and
        (3.9542,297.5820) .. (3.6518,312.2270) .. controls (3.5510,318.3620) and
        (2.2838,322.7100) .. (1.3046,327.2460) .. controls (0.1814,332.2580) and
        (-0.4378,335.4260) .. (1.6214,338.7230) .. controls (5.7542,345.0160) and
        (11.6725,344.3250) .. (45.4117,351.3380) .. controls (63.4405,355.0530) and
        (78.1717,364.1680) .. (91.0309,364.5860) .. controls (101.4280,365.1620) and
        (103.6020,362.1230) .. (110.2410,356.1760) .. controls (117.0950,349.9700) and
        (118.9090,351.8130) .. (118.8950,340.1200) .. controls (119.2410,326.6420) and
        (112.4580,323.4300) .. (102.5940,307.8060) .. controls (92.7301,292.1820) and
        (91.1461,289.3020) .. (84.5941,278.9340) .. controls (78.0421,268.7100) and
        (65.0821,250.1340) .. (57.6661,249.9180) .. controls (51.7621,249.7740) and
        (48.4501,252.9420) .. (44.7781,256.6860) .. controls (41.1061,260.4300) and
        (36.7861,270.8560) .. (30.5077,275.7660) -- cycle;
      \path[fill=cb27500] (30.6121,276.0540) .. controls (24.5461,280.8420) and
        (9.8222,277.9980) .. (3.5942,284.7300) .. controls (-0.7618,289.9140) and
        (4.1522,297.5460) .. (3.7922,312.1800) .. controls (3.6662,318.2820) and
        (2.3522,322.5300) .. (1.3082,327.2460) .. controls (0.1562,332.3040) and
        (-0.4558,335.4180) .. (1.6142,338.6760) .. controls (5.8622,345.0120) and
        (11.7841,343.7700) .. (45.5161,350.7720) .. controls (63.5341,354.4800) and
        (77.6821,363.4440) .. (90.8941,363.7320) .. controls (101.0280,364.2360) and
        (103.1340,361.3920) .. (109.6860,355.5060) .. controls (116.5260,349.2960) and
        (118.2360,351.1140) .. (118.2720,339.9000) .. controls (118.7580,326.4360) and
        (112.4580,323.4300) .. (102.5940,307.8060) .. controls (92.7301,292.1820) and
        (91.1461,289.3020) .. (84.5941,278.9340) .. controls (78.0421,268.7100) and
        (65.0821,250.1340) .. (57.6661,249.9180) .. controls (51.7621,249.7740) and
        (48.4501,252.9420) .. (44.7781,256.6860) .. controls (41.1061,260.4300) and
        (36.8401,271.0860) .. (30.6121,276.0540) -- cycle;
      \path[fill=cb77b00] (30.7165,276.3420) .. controls (24.7333,281.1520) and
        (10.3478,277.8260) .. (3.7094,284.8100) .. controls (-0.6970,289.9360) and
        (4.3502,297.5100) .. (3.9326,312.1340) .. controls (3.7814,318.2030) and
        (2.4206,322.3500) .. (1.3118,327.2460) .. controls (0.1310,332.3510) and
        (-0.4738,335.4110) .. (1.6070,338.6300) .. controls (5.9702,345.0090) and
        (11.8957,343.2160) .. (45.6205,350.2070) .. controls (63.6277,353.9080) and
        (77.1925,362.7210) .. (90.7573,362.8790) .. controls (100.6290,363.3110) and
        (102.6660,360.6620) .. (109.1320,354.8370) .. controls (115.9570,348.6230) and
        (117.5630,350.4160) .. (117.6490,339.6810) .. controls (118.2760,326.2310) and
        (112.4580,323.4300) .. (102.5940,307.8060) .. controls (92.7301,292.1820) and
        (91.1461,289.3020) .. (84.5941,278.9340) .. controls (78.0421,268.7100) and
        (65.0821,250.1340) .. (57.6661,249.9180) .. controls (51.7621,249.7740) and
        (48.4501,252.9420) .. (44.7781,256.6860) .. controls (41.1061,260.4300) and
        (36.8941,271.3170) .. (30.7165,276.3420) -- cycle;
      \path[fill=cbc8100] (30.8209,276.6300) .. controls (24.9205,281.4620) and
        (10.8734,277.6530) .. (3.8246,284.8890) .. controls (-0.6322,289.9580) and
        (4.5482,297.4740) .. (4.0730,312.0870) .. controls (3.8966,318.1240) and
        (2.4890,322.1700) .. (1.3154,327.2460) .. controls (0.1058,332.3980) and
        (-0.4918,335.4040) .. (1.5998,338.5830) .. controls (6.0782,345.0050) and
        (12.0073,342.6620) .. (45.7249,349.6420) .. controls (63.7213,353.3360) and
        (76.7029,361.9970) .. (90.6205,362.0260) .. controls (100.2290,362.3860) and
        (102.1980,359.9310) .. (108.5770,354.1670) .. controls (115.3890,347.9500) and
        (116.8900,349.7180) .. (117.0270,339.4610) .. controls (117.7930,326.0260) and
        (112.4580,323.4300) .. (102.5940,307.8060) .. controls (92.7301,292.1820) and
        (91.1461,289.3020) .. (84.5941,278.9340) .. controls (78.0421,268.7100) and
        (65.0821,250.1340) .. (57.6661,249.9180) .. controls (51.7621,249.7740) and
        (48.4501,252.9420) .. (44.7781,256.6860) .. controls (41.1061,260.4300) and
        (36.9481,271.5470) .. (30.8209,276.6300) -- cycle;
      \path[fill=cc18700] (30.9253,276.9180) .. controls (25.1077,281.7710) and
        (11.3990,277.4800) .. (3.9398,284.9680) .. controls (-0.5674,289.9790) and
        (4.7462,297.4380) .. (4.2134,312.0400) .. controls (4.0118,318.0450) and
        (2.5574,321.9900) .. (1.3190,327.2460) .. controls (0.0806,332.4450) and
        (-0.5098,335.3970) .. (1.5926,338.5360) .. controls (6.1862,345.0020) and
        (12.1189,342.1070) .. (45.8293,349.0770) .. controls (63.8149,352.7630) and
        (76.2133,361.2740) .. (90.4837,361.1730) .. controls (99.8294,361.4610) and
        (101.7300,359.2000) .. (108.0230,353.4980) .. controls (114.8200,347.2770) and
        (116.2170,349.0190) .. (116.4040,339.2420) .. controls (117.3110,325.8210) and
        (112.4580,323.4300) .. (102.5940,307.8060) .. controls (92.7301,292.1820) and
        (91.1461,289.3020) .. (84.5941,278.9340) .. controls (78.0421,268.7100) and
        (65.0821,250.1340) .. (57.6661,249.9180) .. controls (51.7621,249.7740) and
        (48.4501,252.9420) .. (44.7781,256.6860) .. controls (41.1061,260.4300) and
        (37.0021,271.7780) .. (30.9253,276.9180) -- cycle;
      \path[fill=cc68c00] (31.0297,277.2060) .. controls (25.2949,282.0810) and
        (11.9245,277.3070) .. (4.0550,285.0470) .. controls (-0.5026,290.0010) and
        (4.9442,297.4020) .. (4.3538,311.9930) .. controls (4.1270,317.9660) and
        (2.6258,321.8100) .. (1.3226,327.2460) .. controls (0.0554,332.4920) and
        (-0.5278,335.3900) .. (1.5854,338.4890) .. controls (6.2942,344.9980) and
        (12.2305,341.5530) .. (45.9337,348.5120) .. controls (63.9085,352.1910) and
        (75.7237,360.5500) .. (90.3469,360.3200) .. controls (99.4298,360.5360) and
        (101.2620,358.4690) .. (107.4690,352.8280) .. controls (114.2510,346.6040) and
        (115.5430,348.3210) .. (115.7810,339.0220) .. controls (116.8290,325.6160) and
        (112.4580,323.4300) .. (102.5940,307.8060) .. controls (92.7301,292.1820) and
        (91.1461,289.3020) .. (84.5941,278.9340) .. controls (78.0421,268.7100) and
        (65.0821,250.1340) .. (57.6661,249.9180) .. controls (51.7621,249.7740) and
        (48.4501,252.9420) .. (44.7781,256.6860) .. controls (41.1061,260.4300) and
        (37.0561,272.0080) .. (31.0297,277.2060) -- cycle;
      \path[fill=ccc9200] (31.1341,277.4940) .. controls (25.4821,282.3900) and
        (12.4501,277.1340) .. (4.1702,285.1260) .. controls (-0.4378,290.0220) and
        (5.1422,297.3660) .. (4.4942,311.9460) .. controls (4.2422,317.8860) and
        (2.6942,321.6300) .. (1.3262,327.2460) .. controls (0.0302,332.5380) and
        (-0.5458,335.3820) .. (1.5782,338.4420) .. controls (6.4022,344.9940) and
        (12.3421,340.9980) .. (46.0381,347.9460) .. controls (64.0021,351.6180) and
        (75.2341,359.8260) .. (90.2101,359.4660) .. controls (99.0302,359.6100) and
        (100.7940,357.7380) .. (106.9140,352.1580) .. controls (113.6820,345.9300) and
        (114.8700,347.6220) .. (115.1580,338.8020) .. controls (116.3460,325.4100) and
        (112.4580,323.4300) .. (102.5940,307.8060) .. controls (92.7301,292.1820) and
        (91.1461,289.3020) .. (84.5941,278.9340) .. controls (78.0421,268.7100) and
        (65.0821,250.1340) .. (57.6661,249.9180) .. controls (51.7621,249.7740) and
        (48.4501,252.9420) .. (44.7781,256.6860) .. controls (41.1061,260.4300) and
        (37.1101,272.2380) .. (31.1341,277.4940) -- cycle;
      \path[fill=cd19800] (31.2385,277.7820) .. controls (25.6693,282.7000) and
        (12.9757,276.9620) .. (4.2854,285.2060) .. controls (-0.3730,290.0440) and
        (5.3402,297.3300) .. (4.6346,311.9000) .. controls (4.3574,317.8070) and
        (2.7626,321.4500) .. (1.3298,327.2460) .. controls (0.0050,332.5850) and
        (-0.5638,335.3750) .. (1.5710,338.3960) .. controls (6.5102,344.9910) and
        (12.4537,340.4440) .. (46.1425,347.3810) .. controls (64.0957,351.0460) and
        (74.7445,359.1030) .. (90.0733,358.6130) .. controls (98.6306,358.6850) and
        (100.3260,357.0080) .. (106.3600,351.4890) .. controls (113.1130,345.2570) and
        (114.1970,346.9240) .. (114.5350,338.5830) .. controls (115.8640,325.2050) and
        (112.4580,323.4300) .. (102.5940,307.8060) .. controls (92.7301,292.1820) and
        (91.1461,289.3020) .. (84.5941,278.9340) .. controls (78.0421,268.7100) and
        (65.0821,250.1340) .. (57.6661,249.9180) .. controls (51.7621,249.7740) and
        (48.4501,252.9420) .. (44.7781,256.6860) .. controls (41.1061,260.4300) and
        (37.1641,272.4690) .. (31.2385,277.7820) -- cycle;
      \path[fill=cd69e00] (31.3429,278.0700) .. controls (25.8565,283.0100) and
        (13.5013,276.7890) .. (4.4006,285.2850) .. controls (-0.3082,290.0660) and
        (5.5382,297.2940) .. (4.7750,311.8530) .. controls (4.4726,317.7280) and
        (2.8310,321.2700) .. (1.3334,327.2460) .. controls (-0.0202,332.6320) and
        (-0.5818,335.3680) .. (1.5638,338.3490) .. controls (6.6182,344.9870) and
        (12.5653,339.8900) .. (46.2469,346.8160) .. controls (64.1893,350.4740) and
        (74.2549,358.3790) .. (89.9365,357.7600) .. controls (98.2310,357.7600) and
        (99.8582,356.2770) .. (105.8050,350.8190) .. controls (112.5450,344.5840) and
        (113.5240,346.2260) .. (113.9130,338.3630) .. controls (115.3810,325.0000) and
        (112.4580,323.4300) .. (102.5940,307.8060) .. controls (92.7301,292.1820) and
        (91.1461,289.3020) .. (84.5941,278.9340) .. controls (78.0421,268.7100) and
        (65.0821,250.1340) .. (57.6661,249.9180) .. controls (51.7621,249.7740) and
        (48.4501,252.9420) .. (44.7781,256.6860) .. controls (41.1061,260.4300) and
        (37.2181,272.6990) .. (31.3429,278.0700) -- cycle;
      \path[fill=cdba300] (31.4473,278.3580) .. controls (26.0437,283.3190) and
        (14.0269,276.6160) .. (4.5158,285.3640) .. controls (-0.2434,290.0870) and
        (5.7362,297.2580) .. (4.9154,311.8060) .. controls (4.5878,317.6490) and
        (2.8994,321.0900) .. (1.3370,327.2460) .. controls (-0.0454,332.6790) and
        (-0.5998,335.3610) .. (1.5566,338.3020) .. controls (6.7262,344.9840) and
        (12.6769,339.3350) .. (46.3513,346.2510) .. controls (64.2829,349.9010) and
        (73.7653,357.6560) .. (89.7997,356.9070) .. controls (97.8314,356.8350) and
        (99.3902,355.5460) .. (105.2510,350.1500) .. controls (111.9760,343.9110) and
        (112.8510,345.5270) .. (113.2900,338.1440) .. controls (114.8990,324.7950) and
        (112.4580,323.4300) .. (102.5940,307.8060) .. controls (92.7301,292.1820) and
        (91.1461,289.3020) .. (84.5941,278.9340) .. controls (78.0421,268.7100) and
        (65.0821,250.1340) .. (57.6661,249.9180) .. controls (51.7621,249.7740) and
        (48.4501,252.9420) .. (44.7781,256.6860) .. controls (41.1061,260.4300) and
        (37.2721,272.9300) .. (31.4473,278.3580) -- cycle;
      \path[fill=ce0a900] (31.5517,278.6460) .. controls (26.2309,283.6290) and
        (14.5525,276.4430) .. (4.6310,285.4430) .. controls (-0.1786,290.1090) and
        (5.9342,297.2220) .. (5.0558,311.7590) .. controls (4.7030,317.5700) and
        (2.9678,320.9100) .. (1.3406,327.2460) .. controls (-0.0706,332.7260) and
        (-0.6178,335.3540) .. (1.5494,338.2550) .. controls (6.8342,344.9800) and
        (12.7885,338.7810) .. (46.4557,345.6860) .. controls (64.3765,349.3290) and
        (73.2757,356.9320) .. (89.6629,356.0540) .. controls (97.4318,355.9100) and
        (98.9222,354.8150) .. (104.6970,349.4800) .. controls (111.4070,343.2380) and
        (112.1770,344.8290) .. (112.6670,337.9240) .. controls (114.4170,324.5900) and
        (112.4580,323.4300) .. (102.5940,307.8060) .. controls (92.7301,292.1820) and
        (91.1461,289.3020) .. (84.5941,278.9340) .. controls (78.0421,268.7100) and
        (65.0821,250.1340) .. (57.6661,249.9180) .. controls (51.7621,249.7740) and
        (48.4501,252.9420) .. (44.7781,256.6860) .. controls (41.1061,260.4300) and
        (37.3261,273.1600) .. (31.5517,278.6460) -- cycle;
      \path[fill=ce5af00] (31.6561,278.9340) .. controls (26.4181,283.9380) and
        (15.0781,276.2700) .. (4.7462,285.5220) .. controls (-0.1138,290.1300) and
        (6.1322,297.1860) .. (5.1962,311.7120) .. controls (4.8182,317.4900) and
        (3.0362,320.7300) .. (1.3442,327.2460) .. controls (-0.0958,332.7720) and
        (-0.6358,335.3460) .. (1.5422,338.2080) .. controls (6.9422,344.9760) and
        (12.9001,338.2260) .. (46.5601,345.1200) .. controls (64.4701,348.7560) and
        (72.7861,356.2080) .. (89.5261,355.2000) .. controls (97.0322,354.9840) and
        (98.4542,354.0840) .. (104.1420,348.8100) .. controls (110.8380,342.5640) and
        (111.5040,344.1300) .. (112.0440,337.7040) .. controls (113.9340,324.3840) and
        (112.4580,323.4300) .. (102.5940,307.8060) .. controls (92.7301,292.1820) and
        (91.1461,289.3020) .. (84.5941,278.9340) .. controls (78.0421,268.7100) and
        (65.0821,250.1340) .. (57.6661,249.9180) .. controls (51.7621,249.7740) and
        (48.4501,252.9420) .. (44.7781,256.6860) .. controls (41.1061,260.4300) and
        (37.3801,273.3900) .. (31.6561,278.9340) -- cycle;
      \path[fill=ceab500] (31.7605,279.2220) .. controls (26.6053,284.2480) and
        (15.6037,276.0980) .. (4.8614,285.6020) .. controls (-0.0490,290.1520) and
        (6.3302,297.1500) .. (5.3366,311.6660) .. controls (4.9334,317.4110) and
        (3.1046,320.5500) .. (1.3478,327.2460) .. controls (-0.1210,332.8190) and
        (-0.6538,335.3390) .. (1.5350,338.1620) .. controls (7.0502,344.9730) and
        (13.0117,337.6720) .. (46.6645,344.5550) .. controls (64.5637,348.1840) and
        (72.2965,355.4850) .. (89.3893,354.3470) .. controls (96.6326,354.0590) and
        (97.9862,353.3540) .. (103.5880,348.1410) .. controls (110.2690,341.8910) and
        (110.8310,343.4320) .. (111.4210,337.4850) .. controls (113.4520,324.1790) and
        (112.4580,323.4300) .. (102.5940,307.8060) .. controls (92.7301,292.1820) and
        (91.1461,289.3020) .. (84.5941,278.9340) .. controls (78.0421,268.7100) and
        (65.0821,250.1340) .. (57.6661,249.9180) .. controls (51.7621,249.7740) and
        (48.4501,252.9420) .. (44.7781,256.6860) .. controls (41.1061,260.4300) and
        (37.4341,273.6210) .. (31.7605,279.2220) -- cycle;
      \path[fill=cefba00] (31.8649,279.5100) .. controls (26.7925,284.5580) and
        (16.1293,275.9250) .. (4.9766,285.6810) .. controls (0.0158,290.1740) and
        (6.5282,297.1140) .. (5.4770,311.6190) .. controls (5.0486,317.3320) and
        (3.1730,320.3700) .. (1.3514,327.2460) .. controls (-0.1462,332.8660) and
        (-0.6718,335.3320) .. (1.5278,338.1150) .. controls (7.1582,344.9690) and
        (13.1233,337.1180) .. (46.7689,343.9900) .. controls (64.6573,347.6120) and
        (71.8069,354.7610) .. (89.2525,353.4940) .. controls (96.2329,353.1340) and
        (97.5182,352.6230) .. (103.0330,347.4710) .. controls (109.7010,341.2180) and
        (110.1580,342.7340) .. (110.7990,337.2650) .. controls (112.9690,323.9740) and
        (112.4580,323.4300) .. (102.5940,307.8060) .. controls (92.7301,292.1820) and
        (91.1461,289.3020) .. (84.5941,278.9340) .. controls (78.0421,268.7100) and
        (65.0821,250.1340) .. (57.6661,249.9180) .. controls (51.7621,249.7740) and
        (48.4501,252.9420) .. (44.7781,256.6860) .. controls (41.1061,260.4300) and
        (37.4881,273.8510) .. (31.8649,279.5100) -- cycle;
      \path[fill=cf4c000] (31.9693,279.7980) .. controls (26.9797,284.8670) and
        (16.6549,275.7520) .. (5.0918,285.7600) .. controls (0.0806,290.1950) and
        (6.7262,297.0780) .. (5.6174,311.5720) .. controls (5.1638,317.2530) and
        (3.2414,320.1900) .. (1.3550,327.2460) .. controls (-0.1714,332.9130) and
        (-0.6898,335.3250) .. (1.5206,338.0680) .. controls (7.2662,344.9660) and
        (13.2349,336.5630) .. (46.8733,343.4250) .. controls (64.7509,347.0390) and
        (71.3173,354.0380) .. (89.1157,352.6410) .. controls (95.8333,352.2090) and
        (97.0501,351.8920) .. (102.4790,346.8020) .. controls (109.1320,340.5450) and
        (109.4850,342.0350) .. (110.1760,337.0460) .. controls (112.4870,323.7690) and
        (112.4580,323.4300) .. (102.5940,307.8060) .. controls (92.7301,292.1820) and
        (91.1461,289.3020) .. (84.5941,278.9340) .. controls (78.0421,268.7100) and
        (65.0821,250.1340) .. (57.6661,249.9180) .. controls (51.7621,249.7740) and
        (48.4501,252.9420) .. (44.7781,256.6860) .. controls (41.1061,260.4300) and
        (37.5421,274.0820) .. (31.9693,279.7980) -- cycle;
      \path[fill=cf9c600] (32.0737,280.0860) .. controls (27.1669,285.1770) and
        (17.1805,275.5790) .. (5.2070,285.8390) .. controls (0.1454,290.2170) and
        (6.9242,297.0420) .. (5.7578,311.5250) .. controls (5.2790,317.1740) and
        (3.3098,320.0100) .. (1.3586,327.2460) .. controls (-0.1966,332.9600) and
        (-0.7078,335.3180) .. (1.5134,338.0210) .. controls (7.3742,344.9620) and
        (13.3465,336.0090) .. (46.9777,342.8600) .. controls (64.8445,346.4670) and
        (70.8277,353.3140) .. (88.9789,351.7880) .. controls (95.4337,351.2840) and
        (96.5821,351.1610) .. (101.9250,346.1320) .. controls (108.5630,339.8720) and
        (108.8110,341.3370) .. (109.5530,336.8260) .. controls (112.0050,323.5640) and
        (112.4580,323.4300) .. (102.5940,307.8060) .. controls (92.7301,292.1820) and
        (91.1461,289.3020) .. (84.5941,278.9340) .. controls (78.0421,268.7100) and
        (65.0821,250.1340) .. (57.6661,249.9180) .. controls (51.7621,249.7740) and
        (48.4501,252.9420) .. (44.7781,256.6860) .. controls (41.1061,260.4300) and
        (37.5961,274.3120) .. (32.0737,280.0860) -- cycle;
    \path[fill=cffcc00] (32.1781,280.3740) .. controls (27.3541,285.4860) and
      (17.7061,275.4060) .. (5.3222,285.9180) .. controls (0.2102,290.2380) and
      (7.1222,297.0060) .. (5.8982,311.4780) .. controls (5.3942,317.0940) and
      (3.3782,319.8300) .. (1.3622,327.2460) .. controls (-0.2218,333.0060) and
      (-0.7258,335.3100) .. (1.5062,337.9740) .. controls (7.4822,344.9580) and
      (13.4581,335.4540) .. (47.0821,342.2940) .. controls (64.9381,345.8940) and
      (70.3381,352.5900) .. (88.8421,350.9340) .. controls (95.0341,350.3580) and
      (96.1141,350.4300) .. (101.3700,345.4620) .. controls (107.9940,339.1980) and
      (108.1380,340.6380) .. (108.9300,336.6060) .. controls (111.5220,323.3580) and
      (112.4580,323.4300) .. (102.5940,307.8060) .. controls (92.7301,292.1820) and
      (91.1461,289.3020) .. (84.5941,278.9340) .. controls (78.0421,268.7100) and
      (65.0821,250.1340) .. (57.6661,249.9180) .. controls (51.7621,249.7740) and
      (48.4501,252.9420) .. (44.7781,256.6860) .. controls (41.1061,260.4300) and
      (37.6501,274.5420) .. (32.1781,280.3740) -- cycle;
    \path[fill=cffcc00] (35.2968,286.0000) .. controls (31.0185,290.5380) and
      (17.0170,284.6390) .. (10.4052,289.6300) .. controls (5.6734,293.1310) and
      (12.8036,297.5390) .. (11.8313,310.5030) .. controls (11.4424,315.4290) and
      (8.0069,316.3370) .. (9.4330,323.1430) .. controls (10.5350,328.2640) and
      (11.7018,328.1340) .. (13.6463,330.4680) .. controls (18.9616,336.6910) and
      (17.7302,332.1530) .. (48.0017,338.2460) .. controls (64.0126,341.4880) and
      (68.9392,347.5160) .. (85.5982,346.0250) .. controls (91.1728,345.5070) and
      (92.2100,345.5710) .. (96.8772,341.0990) .. controls (102.8410,335.5240) and
      (99.7941,336.6910) .. (100.4420,333.1260) .. controls (102.7110,321.2630) and
      (100.8960,319.1890) .. (92.0155,305.1230) .. controls (83.1351,291.0560) and
      (83.9130,290.2790) .. (78.0790,281.0090) .. controls (72.2450,271.8690) and
      (64.2071,255.2100) .. (57.5306,255.0800) .. controls (52.2800,254.9500) and
      (49.2333,257.8020) .. (45.9276,261.1730) .. controls (42.6215,264.5440) and
      (40.1582,280.8150) .. (35.2968,286.0000) -- cycle;
      \path[fill=cffcc02] (35.5911,286.0910) .. controls (31.1299,290.7930) and
        (16.9863,284.8950) .. (10.5310,289.7970) .. controls (5.8302,293.3100) and
        (13.1092,297.8460) .. (12.1398,310.4760) .. controls (11.7390,315.3730) and
        (8.4067,316.3530) .. (9.6815,323.1100) .. controls (10.6686,328.2030) and
        (12.0397,328.0480) .. (14.0154,330.3110) .. controls (19.7016,336.4420) and
        (18.6887,332.1370) .. (48.1347,338.0540) .. controls (64.0530,341.2650) and
        (68.9502,347.2690) .. (85.5071,345.7870) .. controls (91.0485,345.2740) and
        (91.9853,345.3240) .. (96.6618,340.9380) .. controls (102.5270,335.5310) and
        (99.5214,336.6510) .. (100.0740,332.8810) .. controls (102.1410,321.0930) and
        (100.5170,318.8990) .. (91.8862,305.2310) .. controls (83.0558,291.2480) and
        (83.7435,290.5320) .. (77.9442,281.3180) .. controls (72.1450,272.2330) and
        (64.2435,255.5130) .. (57.6068,255.3840) .. controls (52.3875,255.2550) and
        (49.3589,258.0900) .. (46.0729,261.4410) .. controls (42.7865,264.7920) and
        (40.4293,280.9450) .. (35.5911,286.0910) -- cycle;
      \path[fill=cffcc05] (35.8854,286.1820) .. controls (31.2413,291.0480) and
        (16.9556,285.1510) .. (10.6567,289.9630) .. controls (5.9869,293.4900) and
        (13.4148,298.1540) .. (12.4484,310.4500) .. controls (12.0356,315.3170) and
        (8.8065,316.3680) .. (9.9301,323.0760) .. controls (10.8021,328.1420) and
        (12.3775,327.9620) .. (14.3845,330.1540) .. controls (20.4415,336.1940) and
        (19.6473,332.1210) .. (48.2676,337.8620) .. controls (64.0934,341.0420) and
        (68.9613,347.0220) .. (85.4160,345.5480) .. controls (90.9242,345.0420) and
        (91.7606,345.0770) .. (96.4464,340.7780) .. controls (102.2140,335.5380) and
        (99.2486,336.6110) .. (99.7063,332.6370) .. controls (101.5710,320.9220) and
        (100.1380,318.6100) .. (91.7568,305.3390) .. controls (82.9766,291.4400) and
        (83.5739,290.7860) .. (77.8094,281.6270) .. controls (72.0450,272.5960) and
        (64.2799,255.8160) .. (57.6829,255.6880) .. controls (52.4949,255.5590) and
        (49.4845,258.3770) .. (46.2182,261.7080) .. controls (42.9515,265.0390) and
        (40.7003,281.0750) .. (35.8854,286.1820) -- cycle;
      \path[fill=cffcc07] (36.1797,286.2730) .. controls (31.3527,291.3030) and
        (16.9249,285.4070) .. (10.7825,290.1290) .. controls (6.1437,293.6700) and
        (13.7204,298.4610) .. (12.7569,310.4230) .. controls (12.3321,315.2610) and
        (9.2063,316.3840) .. (10.1787,323.0430) .. controls (10.9356,328.0810) and
        (12.7153,327.8760) .. (14.7536,329.9970) .. controls (21.1814,335.9460) and
        (20.6058,332.1060) .. (48.4006,337.6710) .. controls (64.1338,340.8190) and
        (68.9723,346.7740) .. (85.3249,345.3100) .. controls (90.7998,344.8090) and
        (91.5359,344.8300) .. (96.2310,340.6170) .. controls (101.9000,335.5450) and
        (98.9758,336.5710) .. (99.3383,332.3930) .. controls (101.0010,320.7510) and
        (99.7585,318.3200) .. (91.6275,305.4470) .. controls (82.8973,291.6320) and
        (83.4043,291.0400) .. (77.6746,281.9360) .. controls (71.9449,272.9600) and
        (64.3162,256.1190) .. (57.7591,255.9910) .. controls (52.6023,255.8640) and
        (49.6101,258.6650) .. (46.3635,261.9750) .. controls (43.1165,265.2860) and
        (40.9713,281.2050) .. (36.1797,286.2730) -- cycle;
      \path[fill=cffcd0a] (36.4740,286.3640) .. controls (31.4641,291.5580) and
        (16.8942,285.6630) .. (10.9082,290.2950) .. controls (6.3005,293.8490) and
        (14.0259,298.7690) .. (13.0654,310.3960) .. controls (12.6287,315.2050) and
        (9.6061,316.4000) .. (10.4273,323.0090) .. controls (11.0692,328.0200) and
        (13.0532,327.7900) .. (15.1226,329.8400) .. controls (21.9213,335.6970) and
        (21.5643,332.0900) .. (48.5335,337.4790) .. controls (64.1742,340.5970) and
        (68.9833,346.5270) .. (85.2338,345.0720) .. controls (90.6755,344.5770) and
        (91.3112,344.5830) .. (96.0156,340.4570) .. controls (101.5860,335.5520) and
        (98.7031,336.5310) .. (98.9704,332.1490) .. controls (100.4320,320.5800) and
        (99.3792,318.0300) .. (91.4982,305.5550) .. controls (82.8180,291.8240) and
        (83.2348,291.2940) .. (77.5398,282.2450) .. controls (71.8449,273.3230) and
        (64.3526,256.4220) .. (57.8353,256.2950) .. controls (52.7098,256.1680) and
        (49.7358,258.9520) .. (46.5088,262.2430) .. controls (43.2815,265.5330) and
        (41.2424,281.3360) .. (36.4740,286.3640) -- cycle;
      \path[fill=cffcd0c] (36.7682,286.4540) .. controls (31.5755,291.8130) and
        (16.8635,285.9190) .. (11.0340,290.4620) .. controls (6.4573,294.0290) and
        (14.3315,299.0760) .. (13.3739,310.3700) .. controls (12.9252,315.1490) and
        (10.0059,316.4150) .. (10.6759,322.9760) .. controls (11.2027,327.9580) and
        (13.3910,327.7040) .. (15.4917,329.6830) .. controls (22.6613,335.4490) and
        (22.5228,332.0740) .. (48.6665,337.2870) .. controls (64.2145,340.3740) and
        (68.9943,346.2800) .. (85.1427,344.8330) .. controls (90.5512,344.3450) and
        (91.0864,344.3360) .. (95.8001,340.2960) .. controls (101.2720,335.5590) and
        (98.4303,336.4910) .. (98.6024,331.9040) .. controls (99.8616,320.4100) and
        (99.0000,317.7410) .. (91.3688,305.6640) .. controls (82.7387,292.0160) and
        (83.0652,291.5470) .. (77.4050,282.5540) .. controls (71.7449,273.6860) and
        (64.3890,256.7250) .. (57.9114,256.5980) .. controls (52.8172,256.4720) and
        (49.8614,259.2400) .. (46.6541,262.5100) .. controls (43.4465,265.7810) and
        (41.5134,281.4660) .. (36.7682,286.4540) -- cycle;
      \path[fill=cffcd0f] (37.0625,286.5450) .. controls (31.6868,292.0680) and
        (16.8328,286.1750) .. (11.1597,290.6280) .. controls (6.6141,294.2090) and
        (14.6371,299.3840) .. (13.6824,310.3430) .. controls (13.2218,315.0930) and
        (10.4057,316.4310) .. (10.9245,322.9420) .. controls (11.3362,327.8970) and
        (13.7288,327.6180) .. (15.8608,329.5260) .. controls (23.4012,335.2010) and
        (23.4813,332.0580) .. (48.7994,337.0950) .. controls (64.2549,340.1510) and
        (69.0053,346.0330) .. (85.0516,344.5950) .. controls (90.4269,344.1120) and
        (90.8617,344.0890) .. (95.5847,340.1360) .. controls (100.9590,335.5660) and
        (98.1575,336.4510) .. (98.2344,331.6600) .. controls (99.2917,320.2390) and
        (98.6207,317.4510) .. (91.2395,305.7720) .. controls (82.6594,292.2080) and
        (82.8956,291.8010) .. (77.2702,282.8630) .. controls (71.6448,274.0500) and
        (64.4253,257.0280) .. (57.9876,256.9020) .. controls (52.9246,256.7770) and
        (49.9870,259.5270) .. (46.7994,262.7780) .. controls (43.6115,266.0280) and
        (41.7844,281.5960) .. (37.0625,286.5450) -- cycle;
      \path[fill=cffcd11] (37.3568,286.6360) .. controls (31.7982,292.3230) and
        (16.8021,286.4310) .. (11.2854,290.7940) .. controls (6.7709,294.3880) and
        (14.9427,299.6910) .. (13.9909,310.3160) .. controls (13.5184,315.0370) and
        (10.8056,316.4470) .. (11.1730,322.9090) .. controls (11.4697,327.8360) and
        (14.0666,327.5320) .. (16.2299,329.3690) .. controls (24.1411,334.9520) and
        (24.4398,332.0430) .. (48.9324,336.9030) .. controls (64.2953,339.9290) and
        (69.0163,345.7860) .. (84.9605,344.3570) .. controls (90.3025,343.8800) and
        (90.6370,343.8420) .. (95.3693,339.9760) .. controls (100.6450,335.5730) and
        (97.8848,336.4110) .. (97.8664,331.4160) .. controls (98.7218,320.0680) and
        (98.2415,317.1610) .. (91.1101,305.8800) .. controls (82.5801,292.4000) and
        (82.7260,292.0550) .. (77.1354,283.1720) .. controls (71.5448,274.4130) and
        (64.4617,257.3300) .. (58.0637,257.2060) .. controls (53.0321,257.0810) and
        (50.1126,259.8150) .. (46.9447,263.0450) .. controls (43.7765,266.2750) and
        (42.0555,281.7270) .. (37.3568,286.6360) -- cycle;
      \path[fill=cffce14] (37.6511,286.7270) .. controls (31.9096,292.5780) and
        (16.7714,286.6870) .. (11.4112,290.9600) .. controls (6.9276,294.5680) and
        (15.2482,299.9990) .. (14.2994,310.2900) .. controls (13.8149,314.9810) and
        (11.2054,316.4620) .. (11.4216,322.8750) .. controls (11.6033,327.7750) and
        (14.4045,327.4460) .. (16.5989,329.2120) .. controls (24.8810,334.7040) and
        (25.3983,332.0270) .. (49.0653,336.7110) .. controls (64.3357,339.7060) and
        (69.0274,345.5380) .. (84.8694,344.1180) .. controls (90.1782,343.6470) and
        (90.4123,343.5950) .. (95.1539,339.8150) .. controls (100.3310,335.5800) and
        (97.6120,336.3710) .. (97.4985,331.1720) .. controls (98.1519,319.8980) and
        (97.8622,316.8720) .. (90.9808,305.9880) .. controls (82.5009,292.5930) and
        (82.5565,292.3090) .. (77.0006,283.4810) .. controls (71.4448,274.7770) and
        (64.4981,257.6330) .. (58.1399,257.5090) .. controls (53.1395,257.3860) and
        (50.2382,260.1020) .. (47.0900,263.3120) .. controls (43.9415,266.5220) and
        (42.3265,281.8570) .. (37.6511,286.7270) -- cycle;
      \path[fill=cffce16] (37.9454,286.8180) .. controls (32.0210,292.8330) and
        (16.7406,286.9430) .. (11.5369,291.1270) .. controls (7.0844,294.7480) and
        (15.5538,300.3060) .. (14.6079,310.2630) .. controls (14.1115,314.9250) and
        (11.6052,316.4780) .. (11.6702,322.8420) .. controls (11.7368,327.7140) and
        (14.7423,327.3600) .. (16.9680,329.0550) .. controls (25.6210,334.4560) and
        (26.3568,332.0110) .. (49.1982,336.5190) .. controls (64.3761,339.4830) and
        (69.0384,345.2910) .. (84.7783,343.8800) .. controls (90.0539,343.4150) and
        (90.1876,343.3480) .. (94.9385,339.6550) .. controls (100.0170,335.5870) and
        (97.3392,336.3310) .. (97.1305,330.9270) .. controls (97.5820,319.7270) and
        (97.4830,316.5820) .. (90.8514,306.0960) .. controls (82.4216,292.7850) and
        (82.3869,292.5620) .. (76.8658,283.7900) .. controls (71.3447,275.1400) and
        (64.5344,257.9360) .. (58.2160,257.8130) .. controls (53.2469,257.6900) and
        (50.3638,260.3890) .. (47.2352,263.5800) .. controls (44.1064,266.7690) and
        (42.5975,281.9870) .. (37.9454,286.8180) -- cycle;
      \path[fill=cffce19] (38.2396,286.9080) .. controls (32.1324,293.0880) and
        (16.7099,287.1990) .. (11.6627,291.2930) .. controls (7.2412,294.9270) and
        (15.8594,300.6140) .. (14.9164,310.2370) .. controls (14.4080,314.8690) and
        (12.0050,316.4940) .. (11.9188,322.8090) .. controls (11.8703,327.6530) and
        (15.0801,327.2740) .. (17.3371,328.8970) .. controls (26.3609,334.2070) and
        (27.3153,331.9950) .. (49.3312,336.3270) .. controls (64.4164,339.2600) and
        (69.0494,345.0440) .. (84.6872,343.6420) .. controls (89.9295,343.1830) and
        (89.9628,343.1010) .. (94.7230,339.4940) .. controls (99.7035,335.5940) and
        (97.0665,336.2910) .. (96.7625,330.6830) .. controls (97.0120,319.5560) and
        (97.1037,316.2920) .. (90.7221,306.2050) .. controls (82.3423,292.9770) and
        (82.2173,292.8160) .. (76.7310,284.0990) .. controls (71.2447,275.5040) and
        (64.5708,258.2390) .. (58.2922,258.1160) .. controls (53.3544,257.9940) and
        (50.4894,260.6770) .. (47.3805,263.8470) .. controls (44.2714,267.0170) and
        (42.8686,282.1170) .. (38.2396,286.9080) -- cycle;
      \path[fill=cffce1c] (38.5339,286.9990) .. controls (32.2438,293.3430) and
        (16.6792,287.4550) .. (11.7884,291.4590) .. controls (7.3980,295.1070) and
        (16.1650,300.9210) .. (15.2249,310.2100) .. controls (14.7046,314.8130) and
        (12.4048,316.5090) .. (12.1674,322.7750) .. controls (12.0039,327.5920) and
        (15.4180,327.1880) .. (17.7062,328.7400) .. controls (27.1008,333.9590) and
        (28.2738,331.9800) .. (49.4641,336.1350) .. controls (64.4568,339.0380) and
        (69.0604,344.7970) .. (84.5961,343.4030) .. controls (89.8052,342.9500) and
        (89.7381,342.8540) .. (94.5076,339.3340) .. controls (99.3897,335.6010) and
        (96.7937,336.2510) .. (96.3946,330.4390) .. controls (96.4421,319.3850) and
        (96.7245,316.0030) .. (90.5927,306.3130) .. controls (82.2630,293.1690) and
        (82.0478,293.0700) .. (76.5962,284.4080) .. controls (71.1446,275.8670) and
        (64.6071,258.5420) .. (58.3683,258.4200) .. controls (53.4618,258.2990) and
        (50.6150,260.9640) .. (47.5258,264.1140) .. controls (44.4364,267.2640) and
        (43.1396,282.2480) .. (38.5339,286.9990) -- cycle;
      \path[fill=cffcf1e] (38.8282,287.0900) .. controls (32.3552,293.5990) and
        (16.6485,287.7110) .. (11.9142,291.6250) .. controls (7.5548,295.2860) and
        (16.4705,301.2290) .. (15.5335,310.1830) .. controls (15.0012,314.7570) and
        (12.8046,316.5250) .. (12.4160,322.7420) .. controls (12.1374,327.5310) and
        (15.7558,327.1020) .. (18.0752,328.5830) .. controls (27.8407,333.7100) and
        (29.2324,331.9640) .. (49.5971,335.9430) .. controls (64.4972,338.8150) and
        (69.0714,344.5500) .. (84.5050,343.1650) .. controls (89.6809,342.7180) and
        (89.5134,342.6070) .. (94.2922,339.1730) .. controls (99.0759,335.6080) and
        (96.5210,336.2110) .. (96.0266,330.1950) .. controls (95.8722,319.2150) and
        (96.3452,315.7130) .. (90.4634,306.4210) .. controls (82.1837,293.3610) and
        (81.8782,293.3230) .. (76.4614,284.7170) .. controls (71.0446,276.2310) and
        (64.6435,258.8450) .. (58.4445,258.7240) .. controls (53.5693,258.6030) and
        (50.7406,261.2520) .. (47.6711,264.3820) .. controls (44.6014,267.5110) and
        (43.4107,282.3780) .. (38.8282,287.0900) -- cycle;
      \path[fill=cffcf21] (39.1225,287.1810) .. controls (32.4665,293.8540) and
        (16.6178,287.9660) .. (12.0399,291.7920) .. controls (7.7116,295.4660) and
        (16.7761,301.5370) .. (15.8420,310.1570) .. controls (15.2977,314.7010) and
        (13.2044,316.5410) .. (12.6645,322.7080) .. controls (12.2709,327.4690) and
        (16.0936,327.0160) .. (18.4443,328.4260) .. controls (28.5807,333.4620) and
        (30.1909,331.9480) .. (49.7300,335.7510) .. controls (64.5376,338.5920) and
        (69.0824,344.3020) .. (84.4139,342.9270) .. controls (89.5565,342.4860) and
        (89.2887,342.3600) .. (94.0768,339.0130) .. controls (98.7622,335.6150) and
        (96.2482,336.1710) .. (95.6586,329.9500) .. controls (95.3023,319.0440) and
        (95.9660,315.4230) .. (90.3340,306.5290) .. controls (82.1044,293.5530) and
        (81.7086,293.5770) .. (76.3266,285.0260) .. controls (70.9446,276.5940) and
        (64.6799,259.1480) .. (58.5206,259.0270) .. controls (53.6767,258.9080) and
        (50.8662,261.5390) .. (47.8164,264.6490) .. controls (44.7664,267.7580) and
        (43.6817,282.5080) .. (39.1225,287.1810) -- cycle;
      \path[fill=cffcf23] (39.4168,287.2720) .. controls (32.5779,294.1090) and
        (16.5871,288.2220) .. (12.1657,291.9580) .. controls (7.8684,295.6460) and
        (17.0817,301.8440) .. (16.1505,310.1300) .. controls (15.5943,314.6450) and
        (13.6042,316.5560) .. (12.9131,322.6750) .. controls (12.4045,327.4080) and
        (16.4315,326.9300) .. (18.8134,328.2690) .. controls (29.3206,333.2140) and
        (31.1494,331.9320) .. (49.8630,335.5590) .. controls (64.5780,338.3700) and
        (69.0934,344.0550) .. (84.3228,342.6880) .. controls (89.4322,342.2530) and
        (89.0640,342.1130) .. (93.8614,338.8520) .. controls (98.4484,335.6210) and
        (95.9754,336.1310) .. (95.2907,329.7060) .. controls (94.7324,318.8730) and
        (95.5867,315.1340) .. (90.2047,306.6370) .. controls (82.0251,293.7450) and
        (81.5391,293.8310) .. (76.1918,285.3350) .. controls (70.8445,276.9570) and
        (64.7162,259.4500) .. (58.5968,259.3310) .. controls (53.7841,259.2120) and
        (50.9918,261.8270) .. (47.9617,264.9160) .. controls (44.9314,268.0060) and
        (43.9527,282.6390) .. (39.4168,287.2720) -- cycle;
      \path[fill=cffcf26] (39.7110,287.3630) .. controls (32.6893,294.3640) and
        (16.5564,288.4780) .. (12.2914,292.1240) .. controls (8.0251,295.8250) and
        (17.3873,302.1520) .. (16.4590,310.1030) .. controls (15.8908,314.5890) and
        (14.0041,316.5720) .. (13.1617,322.6410) .. controls (12.5380,327.3470) and
        (16.7693,326.8440) .. (19.1825,328.1120) .. controls (30.0605,332.9650) and
        (32.1079,331.9160) .. (49.9959,335.3670) .. controls (64.6183,338.1470) and
        (69.1045,343.8080) .. (84.2317,342.4500) .. controls (89.3079,342.0210) and
        (88.8392,341.8660) .. (93.6459,338.6920) .. controls (98.1346,335.6280) and
        (95.7027,336.0910) .. (94.9227,329.4620) .. controls (94.1624,318.7030) and
        (95.2075,314.8440) .. (90.0754,306.7460) .. controls (81.9459,293.9370) and
        (81.3695,294.0850) .. (76.0570,285.6440) .. controls (70.7445,277.3210) and
        (64.7526,259.7530) .. (58.6730,259.6340) .. controls (53.8916,259.5160) and
        (51.1174,262.1140) .. (48.1070,265.1840) .. controls (45.0964,268.2530) and
        (44.2238,282.7690) .. (39.7110,287.3630) -- cycle;
      \path[fill=cffd028] (40.0053,287.4530) .. controls (32.8007,294.6190) and
        (16.5257,288.7340) .. (12.4172,292.2900) .. controls (8.1819,296.0050) and
        (17.6928,302.4590) .. (16.7675,310.0770) .. controls (16.1874,314.5330) and
        (14.4039,316.5880) .. (13.4103,322.6080) .. controls (12.6715,327.2860) and
        (17.1071,326.7580) .. (19.5515,327.9550) .. controls (30.8004,332.7170) and
        (33.0664,331.9010) .. (50.1289,335.1750) .. controls (64.6587,337.9240) and
        (69.1155,343.5610) .. (84.1406,342.2120) .. controls (89.1836,341.7880) and
        (88.6145,341.6180) .. (93.4305,338.5320) .. controls (97.8208,335.6350) and
        (95.4299,336.0510) .. (94.5547,329.2180) .. controls (93.5925,318.5320) and
        (94.8282,314.5540) .. (89.9460,306.8540) .. controls (81.8666,294.1290) and
        (81.1999,294.3380) .. (75.9222,285.9530) .. controls (70.6445,277.6840) and
        (64.7890,260.0560) .. (58.7491,259.9380) .. controls (53.9990,259.8210) and
        (51.2430,262.4020) .. (48.2523,265.4510) .. controls (45.2614,268.5000) and
        (44.4948,282.8990) .. (40.0053,287.4530) -- cycle;
      \path[fill=cffd02b] (40.2996,287.5440) .. controls (32.9121,294.8740) and
        (16.4950,288.9900) .. (12.5429,292.4570) .. controls (8.3387,296.1850) and
        (17.9984,302.7670) .. (17.0760,310.0500) .. controls (16.4840,314.4770) and
        (14.8037,316.6030) .. (13.6589,322.5740) .. controls (12.8050,327.2250) and
        (17.4449,326.6720) .. (19.9206,327.7980) .. controls (31.5404,332.4690) and
        (34.0249,331.8850) .. (50.2618,334.9830) .. controls (64.6991,337.7020) and
        (69.1265,343.3130) .. (84.0495,341.9740) .. controls (89.0592,341.5560) and
        (88.3898,341.3710) .. (93.2151,338.3710) .. controls (97.5071,335.6420) and
        (95.1571,336.0110) .. (94.1867,328.9730) .. controls (93.0226,318.3610) and
        (94.4490,314.2650) .. (89.8167,306.9620) .. controls (81.7873,294.3210) and
        (81.0303,294.5920) .. (75.7873,286.2620) .. controls (70.5444,278.0480) and
        (64.8253,260.3590) .. (58.8253,260.2420) .. controls (54.1064,260.1250) and
        (51.3686,262.6890) .. (48.3976,265.7180) .. controls (45.4264,268.7470) and
        (44.7658,283.0300) .. (40.2996,287.5440) -- cycle;
      \path[fill=cffd02d] (40.5939,287.6350) .. controls (33.0235,295.1290) and
        (16.4643,289.2460) .. (12.6686,292.6230) .. controls (8.4955,296.3640) and
        (18.3040,303.0740) .. (17.3845,310.0230) .. controls (16.7805,314.4210) and
        (15.2035,316.6190) .. (13.9075,322.5410) .. controls (12.9386,327.1640) and
        (17.7828,326.5860) .. (20.2897,327.6410) .. controls (32.2803,332.2200) and
        (34.9834,331.8690) .. (50.3947,334.7910) .. controls (64.7395,337.4790) and
        (69.1375,343.0660) .. (83.9584,341.7350) .. controls (88.9349,341.3240) and
        (88.1651,341.1240) .. (92.9997,338.2110) .. controls (97.1933,335.6490) and
        (94.8844,335.9710) .. (93.8188,328.7290) .. controls (92.4527,318.1910) and
        (94.0697,313.9750) .. (89.6873,307.0700) .. controls (81.7080,294.5130) and
        (80.8608,294.8460) .. (75.6525,286.5710) .. controls (70.4444,278.4110) and
        (64.8617,260.6620) .. (58.9014,260.5450) .. controls (54.2139,260.4300) and
        (51.4942,262.9770) .. (48.5429,265.9860) .. controls (45.5914,268.9950) and
        (45.0369,283.1600) .. (40.5939,287.6350) -- cycle;
      \path[fill=cffd030] (40.8882,287.7260) .. controls (33.1348,295.3840) and
        (16.4335,289.5020) .. (12.7944,292.7890) .. controls (8.6523,296.5440) and
        (18.6096,303.3820) .. (17.6930,309.9970) .. controls (17.0771,314.3650) and
        (15.6033,316.6340) .. (14.1560,322.5070) .. controls (13.0721,327.1030) and
        (18.1206,326.5000) .. (20.6588,327.4840) .. controls (33.0202,331.9720) and
        (35.9419,331.8530) .. (50.5277,334.5990) .. controls (64.7799,337.2560) and
        (69.1485,342.8190) .. (83.8672,341.4970) .. controls (88.8106,341.0910) and
        (87.9404,340.8770) .. (92.7843,338.0500) .. controls (96.8795,335.6560) and
        (94.6116,335.9310) .. (93.4508,328.4850) .. controls (91.8828,318.0200) and
        (93.6905,313.6850) .. (89.5580,307.1780) .. controls (81.6287,294.7050) and
        (80.6912,295.1000) .. (75.5177,286.8800) .. controls (70.3444,278.7750) and
        (64.8981,260.9650) .. (58.9776,260.8490) .. controls (54.3213,260.7340) and
        (51.6198,263.2640) .. (48.6881,266.2530) .. controls (45.7563,269.2420) and
        (45.3079,283.2900) .. (40.8882,287.7260) -- cycle;
    \path[fill=cffd133] (41.1824,287.8170) .. controls (33.2462,295.6390) and
      (16.4028,289.7580) .. (12.9201,292.9550) .. controls (8.8091,296.7240) and
      (18.9151,303.6890) .. (18.0015,309.9700) .. controls (17.3736,314.3090) and
      (16.0031,316.6500) .. (14.4046,322.4740) .. controls (13.2056,327.0420) and
      (18.4584,326.4130) .. (21.0278,327.3270) .. controls (33.7601,331.7240) and
      (36.9004,331.8380) .. (50.6606,334.4070) .. controls (64.8202,337.0330) and
      (69.1595,342.5720) .. (83.7761,341.2590) .. controls (88.6862,340.8590) and
      (87.7156,340.6300) .. (92.5688,337.8900) .. controls (96.5657,335.6630) and
      (94.3388,335.8910) .. (93.0828,328.2410) .. controls (91.3128,317.8490) and
      (93.3112,313.3960) .. (89.4286,307.2860) .. controls (81.5494,294.8970) and
      (80.5216,295.3530) .. (75.3829,287.1890) .. controls (70.2443,279.1380) and
      (64.9344,261.2670) .. (59.0537,261.1520) .. controls (54.4287,261.0380) and
      (51.7454,263.5510) .. (48.8334,266.5200) .. controls (45.9213,269.4890) and
      (45.5789,283.4200) .. (41.1824,287.8170) -- cycle;
    \path[fill=black] (66.3781,219.6780) .. controls (65.4421,221.7660) and
      (64.6501,239.6940) .. (69.3301,246.6780) .. controls (74.0101,253.5900) and
      (72.6421,257.5500) .. (67.4581,252.2940) .. controls (62.0581,247.1820) and
      (58.5301,239.4780) .. (58.4581,234.1500) .. controls (58.4581,231.0540) and
      (60.8341,218.3100) .. (61.7701,216.9420) .. controls (62.7781,215.5020) and
      (67.0981,218.0940) .. (66.3781,219.6780) -- cycle;
      \path[fill=c030303] (66.2881,220.0280) .. controls (65.3845,222.3680) and
        (64.6321,239.7270) .. (69.2977,246.6960) .. controls (73.9633,253.5980) and
        (72.4837,257.1870) .. (67.4581,252.0530) .. controls (62.2273,247.0560) and
        (58.8505,239.5830) .. (58.7173,234.2660) .. controls (58.6741,231.1520) and
        (60.9529,218.8470) .. (61.8493,217.4750) .. controls (62.8105,216.0820) and
        (66.9757,218.4650) .. (66.2881,220.0280) -- cycle;
      \path[fill=c070707] (66.1981,220.3770) .. controls (65.3269,222.9690) and
        (64.6141,239.7590) .. (69.2653,246.7140) .. controls (73.9165,253.6050) and
        (72.3253,256.8230) .. (67.4581,251.8120) .. controls (62.3965,246.9300) and
        (59.1709,239.6870) .. (58.9765,234.3810) .. controls (58.8901,231.2490) and
        (61.0717,219.3830) .. (61.9285,218.0080) .. controls (62.8429,216.6620) and
        (66.8533,218.8360) .. (66.1981,220.3770) -- cycle;
      \path[fill=c0b0b0b] (66.1081,220.7260) .. controls (65.2693,223.5700) and
        (64.5961,239.7920) .. (69.2329,246.7320) .. controls (73.8697,253.6120) and
        (72.1669,256.4600) .. (67.4581,251.5710) .. controls (62.5657,246.8040) and
        (59.4913,239.7920) .. (59.2357,234.4960) .. controls (59.1061,231.3460) and
        (61.1905,219.9200) .. (62.0077,218.5410) .. controls (62.8753,217.2410) and
        (66.7309,219.2070) .. (66.1081,220.7260) -- cycle;
      \path[fill=c0f0f0f] (66.0181,221.0750) .. controls (65.2117,224.1710) and
        (64.5781,239.8240) .. (69.2005,246.7500) .. controls (73.8229,253.6190) and
        (72.0085,256.0960) .. (67.4581,251.3300) .. controls (62.7349,246.6780) and
        (59.8117,239.8960) .. (59.4949,234.6110) .. controls (59.3221,231.4430) and
        (61.3093,220.4560) .. (62.0869,219.0740) .. controls (62.9077,217.8210) and
        (66.6085,219.5780) .. (66.0181,221.0750) -- cycle;
      \path[fill=c131313] (65.9281,221.4240) .. controls (65.1541,224.7720) and
        (64.5601,239.8560) .. (69.1681,246.7680) .. controls (73.7761,253.6260) and
        (71.8501,255.7320) .. (67.4581,251.0880) .. controls (62.9041,246.5520) and
        (60.1321,240.0000) .. (59.7541,234.7260) .. controls (59.5381,231.5400) and
        (61.4281,220.9920) .. (62.1661,219.6060) .. controls (62.9401,218.4000) and
        (66.4861,219.9480) .. (65.9281,221.4240) -- cycle;
      \path[fill=c161616] (65.8381,221.7740) .. controls (65.0965,225.3740) and
        (64.5421,239.8890) .. (69.1357,246.7860) .. controls (73.7293,253.6340) and
        (71.6917,255.3690) .. (67.4581,250.8470) .. controls (63.0733,246.4260) and
        (60.4525,240.1050) .. (60.0133,234.8420) .. controls (59.7541,231.6380) and
        (61.5469,221.5290) .. (62.2453,220.1390) .. controls (62.9725,218.9800) and
        (66.3637,220.3190) .. (65.8381,221.7740) -- cycle;
      \path[fill=c1a1a1a] (65.7481,222.1230) .. controls (65.0389,225.9750) and
        (64.5241,239.9210) .. (69.1033,246.8040) .. controls (73.6825,253.6410) and
        (71.5333,255.0050) .. (67.4581,250.6060) .. controls (63.2425,246.3000) and
        (60.7729,240.2090) .. (60.2725,234.9570) .. controls (59.9701,231.7350) and
        (61.6657,222.0650) .. (62.3245,220.6720) .. controls (63.0049,219.5600) and
        (66.2413,220.6900) .. (65.7481,222.1230) -- cycle;
      \path[fill=c1e1e1e] (65.6581,222.4720) .. controls (64.9813,226.5760) and
        (64.5061,239.9540) .. (69.0709,246.8220) .. controls (73.6357,253.6480) and
        (71.3749,254.6420) .. (67.4581,250.3650) .. controls (63.4117,246.1740) and
        (61.0933,240.3140) .. (60.5317,235.0720) .. controls (60.1861,231.8320) and
        (61.7845,222.6020) .. (62.4037,221.2050) .. controls (63.0373,220.1390) and
        (66.1189,221.0610) .. (65.6581,222.4720) -- cycle;
      \path[fill=c222222] (65.5681,222.8210) .. controls (64.9237,227.1770) and
        (64.4881,239.9860) .. (69.0385,246.8400) .. controls (73.5889,253.6550) and
        (71.2165,254.2780) .. (67.4581,250.1240) .. controls (63.5809,246.0480) and
        (61.4137,240.4180) .. (60.7909,235.1870) .. controls (60.4021,231.9290) and
        (61.9033,223.1380) .. (62.4829,221.7380) .. controls (63.0697,220.7190) and
        (65.9965,221.4320) .. (65.5681,222.8210) -- cycle;
      \path[fill=c262626] (65.4781,223.1700) .. controls (64.8661,227.7780) and
        (64.4701,240.0180) .. (69.0061,246.8580) .. controls (73.5421,253.6620) and
        (71.0581,253.9140) .. (67.4581,249.8820) .. controls (63.7501,245.9220) and
        (61.7341,240.5220) .. (61.0501,235.3020) .. controls (60.6181,232.0260) and
        (62.0221,223.6740) .. (62.5621,222.2700) .. controls (63.1021,221.2980) and
        (65.8741,221.8020) .. (65.4781,223.1700) -- cycle;
      \path[fill=c2a2a2a] (65.3881,223.5200) .. controls (64.8085,228.3800) and
        (64.4521,240.0510) .. (68.9737,246.8760) .. controls (73.4953,253.6700) and
        (70.8997,253.5510) .. (67.4581,249.6410) .. controls (63.9193,245.7960) and
        (62.0545,240.6270) .. (61.3093,235.4180) .. controls (60.8341,232.1240) and
        (62.1409,224.2110) .. (62.6413,222.8030) .. controls (63.1345,221.8780) and
        (65.7517,222.1730) .. (65.3881,223.5200) -- cycle;
      \path[fill=c2d2d2d] (65.2981,223.8690) .. controls (64.7509,228.9810) and
        (64.4341,240.0830) .. (68.9413,246.8940) .. controls (73.4485,253.6770) and
        (70.7413,253.1870) .. (67.4581,249.4000) .. controls (64.0885,245.6700) and
        (62.3749,240.7310) .. (61.5685,235.5330) .. controls (61.0501,232.2210) and
        (62.2597,224.7470) .. (62.7205,223.3360) .. controls (63.1669,222.4580) and
        (65.6293,222.5440) .. (65.2981,223.8690) -- cycle;
      \path[fill=c313131] (65.2081,224.2180) .. controls (64.6933,229.5820) and
        (64.4161,240.1160) .. (68.9089,246.9120) .. controls (73.4017,253.6840) and
        (70.5829,252.8240) .. (67.4581,249.1590) .. controls (64.2577,245.5440) and
        (62.6953,240.8360) .. (61.8277,235.6480) .. controls (61.2661,232.3180) and
        (62.3785,225.2840) .. (62.7997,223.8690) .. controls (63.1993,223.0370) and
        (65.5069,222.9150) .. (65.2081,224.2180) -- cycle;
      \path[fill=c353535] (65.1181,224.5670) .. controls (64.6357,230.1830) and
        (64.3981,240.1480) .. (68.8765,246.9300) .. controls (73.3549,253.6910) and
        (70.4245,252.4600) .. (67.4581,248.9180) .. controls (64.4269,245.4180) and
        (63.0157,240.9400) .. (62.0869,235.7630) .. controls (61.4821,232.4150) and
        (62.4973,225.8200) .. (62.8789,224.4020) .. controls (63.2317,223.6170) and
        (65.3845,223.2860) .. (65.1181,224.5670) -- cycle;
      \path[fill=c393939] (65.0281,224.9160) .. controls (64.5781,230.7840) and
        (64.3801,240.1800) .. (68.8441,246.9480) .. controls (73.3081,253.6980) and
        (70.2661,252.0960) .. (67.4581,248.6760) .. controls (64.5961,245.2920) and
        (63.3361,241.0440) .. (62.3461,235.8780) .. controls (61.6981,232.5120) and
        (62.6161,226.3560) .. (62.9581,224.9340) .. controls (63.2641,224.1960) and
        (65.2621,223.6560) .. (65.0281,224.9160) -- cycle;
      \path[fill=c3d3d3d] (64.9381,225.2660) .. controls (64.5205,231.3860) and
        (64.3621,240.2130) .. (68.8117,246.9660) .. controls (73.2613,253.7060) and
        (70.1077,251.7330) .. (67.4581,248.4350) .. controls (64.7653,245.1660) and
        (63.6565,241.1490) .. (62.6053,235.9940) .. controls (61.9141,232.6100) and
        (62.7349,226.8930) .. (63.0373,225.4670) .. controls (63.2965,224.7760) and
        (65.1397,224.0270) .. (64.9381,225.2660) -- cycle;
      \path[fill=c414141] (64.8481,225.6150) .. controls (64.4629,231.9870) and
        (64.3441,240.2450) .. (68.7793,246.9840) .. controls (73.2145,253.7130) and
        (69.9493,251.3690) .. (67.4581,248.1940) .. controls (64.9345,245.0400) and
        (63.9769,241.2530) .. (62.8645,236.1090) .. controls (62.1301,232.7070) and
        (62.8537,227.4290) .. (63.1165,226.0000) .. controls (63.3289,225.3560) and
        (65.0173,224.3980) .. (64.8481,225.6150) -- cycle;
      \path[fill=c444444] (64.7581,225.9640) .. controls (64.4053,232.5880) and
        (64.3261,240.2780) .. (68.7469,247.0020) .. controls (73.1677,253.7200) and
        (69.7909,251.0060) .. (67.4581,247.9530) .. controls (65.1037,244.9140) and
        (64.2973,241.3580) .. (63.1237,236.2240) .. controls (62.3461,232.8040) and
        (62.9725,227.9660) .. (63.1957,226.5330) .. controls (63.3613,225.9350) and
        (64.8949,224.7690) .. (64.7581,225.9640) -- cycle;
      \path[fill=c484848] (64.6681,226.3130) .. controls (64.3477,233.1890) and
        (64.3081,240.3100) .. (68.7145,247.0200) .. controls (73.1209,253.7270) and
        (69.6325,250.6420) .. (67.4581,247.7120) .. controls (65.2729,244.7880) and
        (64.6177,241.4620) .. (63.3829,236.3390) .. controls (62.5621,232.9010) and
        (63.0913,228.5020) .. (63.2749,227.0660) .. controls (63.3937,226.5150) and
        (64.7725,225.1400) .. (64.6681,226.3130) -- cycle;
    \path[fill=c4c4c4c] (64.5781,226.6620) .. controls (64.2901,233.7900) and
      (64.2901,240.3420) .. (68.6821,247.0380) .. controls (73.0741,253.7340) and
      (69.4741,250.2780) .. (67.4581,247.4700) .. controls (65.4421,244.6620) and
      (64.9381,241.5660) .. (63.6421,236.4540) .. controls (62.7781,232.9980) and
      (63.2101,229.0380) .. (63.3541,227.5980) .. controls (63.4261,227.0940) and
      (64.6501,225.5100) .. (64.5781,226.6620) -- cycle;
    \path[fill=black] (66.3781,219.6780) .. controls (70.6981,210.2460) and
      (73.0741,200.0940) .. (79.2661,190.2300) .. controls (85.3861,180.4380) and
      (82.9381,176.6220) .. (78.4021,181.5900) .. controls (73.8661,186.5580) and
      (68.8981,197.0700) .. (68.8981,197.0700) .. controls (68.8981,197.0700) and
      (63.0661,206.2860) .. (61.7701,216.9420) .. controls (61.5541,218.7420) and
      (65.6581,221.1900) .. (66.3781,219.6780) -- cycle;
      \path[fill=c020202] (66.5633,219.0680) .. controls (70.8429,209.5680) and
        (73.1788,199.8210) .. (79.1490,190.2660) .. controls (85.0508,180.7780) and
        (82.7585,176.9870) .. (78.3857,181.7940) .. controls (73.9827,186.6410) and
        (69.0845,197.0300) .. (69.0179,197.1750) .. controls (69.0179,197.1750) and
        (63.3806,206.1670) .. (62.1403,216.4140) .. controls (61.9321,218.1840) and
        (65.8611,220.5590) .. (66.5633,219.0680) -- cycle;
      \path[fill=c050505] (66.7485,218.4570) .. controls (70.9877,208.8900) and
        (73.2835,199.5480) .. (79.0319,190.3010) .. controls (84.7155,181.1180) and
        (82.5789,177.3510) .. (78.3693,181.9980) .. controls (74.0992,186.7230) and
        (69.2708,196.9890) .. (69.1376,197.2800) .. controls (69.1376,197.2800) and
        (63.6950,206.0470) .. (62.5104,215.8850) .. controls (62.3100,217.6270) and
        (66.0641,219.9270) .. (66.7485,218.4570) -- cycle;
      \path[fill=c070707] (66.9336,217.8460) .. controls (71.1324,208.2120) and
        (73.3882,199.2740) .. (78.9148,190.3360) .. controls (84.3802,181.4580) and
        (82.3993,177.7150) .. (78.3530,182.2010) .. controls (74.2158,186.8060) and
        (69.4571,196.9490) .. (69.2573,197.3850) .. controls (69.2573,197.3850) and
        (64.0094,205.9270) .. (62.8806,215.3570) .. controls (62.6879,217.0690) and
        (66.2672,219.2950) .. (66.9336,217.8460) -- cycle;
      \path[fill=c0a0a0a] (67.1188,217.2350) .. controls (71.2772,207.5340) and
        (73.4929,199.0010) .. (78.7976,190.3710) .. controls (84.0448,181.7980) and
        (82.2197,178.0790) .. (78.3366,182.4050) .. controls (74.3323,186.8880) and
        (69.6434,196.9080) .. (69.3770,197.4900) .. controls (69.3770,197.4900) and
        (64.3239,205.8070) .. (63.2507,214.8280) .. controls (63.0659,216.5110) and
        (66.4702,218.6630) .. (67.1188,217.2350) -- cycle;
      \path[fill=c0c0c0c] (67.3039,216.6240) .. controls (71.4220,206.8560) and
        (73.5976,198.7270) .. (78.6805,190.4060) .. controls (83.7095,182.1380) and
        (82.0401,178.4440) .. (78.3202,182.6090) .. controls (74.4489,186.9700) and
        (69.8298,196.8680) .. (69.4968,197.5940) .. controls (69.4968,197.5940) and
        (64.6383,205.6870) .. (63.6209,214.3000) .. controls (63.4438,215.9530) and
        (66.6732,218.0310) .. (67.3039,216.6240) -- cycle;
      \path[fill=c0f0f0f] (67.4891,216.0140) .. controls (71.5667,206.1780) and
        (73.7022,198.4540) .. (78.5634,190.4410) .. controls (83.3742,182.4780) and
        (81.8604,178.8080) .. (78.3038,182.8120) .. controls (74.5654,187.0530) and
        (70.0161,196.8270) .. (69.6165,197.6990) .. controls (69.6165,197.6990) and
        (64.9527,205.5670) .. (63.9910,213.7710) .. controls (63.8217,215.3950) and
        (66.8762,217.3990) .. (67.4891,216.0140) -- cycle;
      \path[fill=c111111] (67.6743,215.4030) .. controls (71.7115,205.5000) and
        (73.8069,198.1800) .. (78.4463,190.4760) .. controls (83.0389,182.8180) and
        (81.6808,179.1720) .. (78.2874,183.0160) .. controls (74.6820,187.1350) and
        (70.2024,196.7870) .. (69.7362,197.8040) .. controls (69.7362,197.8040) and
        (65.2672,205.4480) .. (64.3612,213.2430) .. controls (64.1996,214.8370) and
        (67.0792,216.7670) .. (67.6743,215.4030) -- cycle;
      \path[fill=c141414] (67.8594,214.7920) .. controls (71.8563,204.8220) and
        (73.9116,197.9070) .. (78.3291,190.5110) .. controls (82.7035,183.1580) and
        (81.5012,179.5360) .. (78.2710,183.2200) .. controls (74.7985,187.2180) and
        (70.3887,196.7460) .. (69.8559,197.9090) .. controls (69.8559,197.9090) and
        (65.5816,205.3280) .. (64.7313,212.7140) .. controls (64.5776,214.2790) and
        (67.2822,216.1350) .. (67.8594,214.7920) -- cycle;
      \path[fill=c161616] (68.0446,214.1810) .. controls (72.0010,204.1440) and
        (74.0163,197.6330) .. (78.2120,190.5460) .. controls (82.3682,183.4980) and
        (81.3216,179.9010) .. (78.2546,183.4230) .. controls (74.9151,187.3000) and
        (70.5751,196.7060) .. (69.9757,198.0140) .. controls (69.9757,198.0140) and
        (65.8960,205.2080) .. (65.1015,212.1860) .. controls (64.9555,213.7210) and
        (67.4852,215.5030) .. (68.0446,214.1810) -- cycle;
      \path[fill=c191919] (68.2297,213.5700) .. controls (72.1458,203.4650) and
        (74.1210,197.3600) .. (78.0949,190.5810) .. controls (82.0329,183.8380) and
        (81.1420,180.2650) .. (78.2382,183.6270) .. controls (75.0316,187.3820) and
        (70.7614,196.6650) .. (70.0954,198.1180) .. controls (70.0954,198.1180) and
        (66.2105,205.0880) .. (65.4716,211.6570) .. controls (65.3334,213.1630) and
        (67.6882,214.8710) .. (68.2297,213.5700) -- cycle;
      \path[fill=c1c1c1c] (68.4149,212.9600) .. controls (72.2906,202.7870) and
        (74.2257,197.0870) .. (77.9778,190.6160) .. controls (81.6976,184.1780) and
        (80.9624,180.6290) .. (78.2218,183.8310) .. controls (75.1482,187.4650) and
        (70.9477,196.6240) .. (70.2151,198.2230) .. controls (70.2151,198.2230) and
        (66.5249,204.9680) .. (65.8418,211.1290) .. controls (65.7114,212.6050) and
        (67.8912,214.2400) .. (68.4149,212.9600) -- cycle;
      \path[fill=c1e1e1e] (68.6001,212.3490) .. controls (72.4353,202.1090) and
        (74.3304,196.8130) .. (77.8606,190.6510) .. controls (81.3622,184.5180) and
        (80.7828,180.9930) .. (78.2054,184.0340) .. controls (75.2647,187.5470) and
        (71.1340,196.5840) .. (70.3348,198.3280) .. controls (70.3348,198.3280) and
        (66.8394,204.8480) .. (66.2119,210.6000) .. controls (66.0893,212.0470) and
        (68.0942,213.6080) .. (68.6001,212.3490) -- cycle;
      \path[fill=c212121] (68.7852,211.7380) .. controls (72.5801,201.4310) and
        (74.4350,196.5400) .. (77.7435,190.6860) .. controls (81.0269,184.8580) and
        (80.6031,181.3580) .. (78.1890,184.2380) .. controls (75.3813,187.6300) and
        (71.3204,196.5430) .. (70.4546,198.4330) .. controls (70.4546,198.4330) and
        (67.1538,204.7280) .. (66.5821,210.0720) .. controls (66.4672,211.4890) and
        (68.2972,212.9760) .. (68.7852,211.7380) -- cycle;
      \path[fill=c232323] (68.9704,211.1270) .. controls (72.7248,200.7530) and
        (74.5397,196.2660) .. (77.6264,190.7210) .. controls (80.6916,185.1980) and
        (80.4235,181.7220) .. (78.1726,184.4420) .. controls (75.4978,187.7120) and
        (71.5067,196.5030) .. (70.5743,198.5380) .. controls (70.5743,198.5380) and
        (67.4682,204.6090) .. (66.9522,209.5430) .. controls (66.8452,210.9310) and
        (68.5002,212.3440) .. (68.9704,211.1270) -- cycle;
      \path[fill=c262626] (69.1555,210.5160) .. controls (72.8696,200.0750) and
        (74.6444,195.9930) .. (77.5093,190.7560) .. controls (80.3563,185.5380) and
        (80.2439,182.0860) .. (78.1562,184.6450) .. controls (75.6144,187.7940) and
        (71.6930,196.4620) .. (70.6940,198.6420) .. controls (70.6940,198.6420) and
        (67.7827,204.4890) .. (67.3224,209.0150) .. controls (67.2231,210.3730) and
        (68.7032,211.7120) .. (69.1555,210.5160) -- cycle;
      \path[fill=c282828] (69.3407,209.9060) .. controls (73.0144,199.3970) and
        (74.7491,195.7190) .. (77.3921,190.7910) .. controls (80.0209,185.8780) and
        (80.0643,182.4500) .. (78.1398,184.8490) .. controls (75.7309,187.8770) and
        (71.8793,196.4220) .. (70.8137,198.7470) .. controls (70.8137,198.7470) and
        (68.0971,204.3690) .. (67.6925,208.4860) .. controls (67.6010,209.8150) and
        (68.9062,211.0800) .. (69.3407,209.9060) -- cycle;
      \path[fill=c2b2b2b] (69.5259,209.2950) .. controls (73.1591,198.7190) and
        (74.8538,195.4460) .. (77.2750,190.8260) .. controls (79.6856,186.2180) and
        (79.8847,182.8150) .. (78.1234,185.0530) .. controls (75.8475,187.9590) and
        (72.0657,196.3810) .. (70.9335,198.8520) .. controls (70.9335,198.8520) and
        (68.4115,204.2490) .. (68.0627,207.9580) .. controls (67.9789,209.2570) and
        (69.1092,210.4480) .. (69.5259,209.2950) -- cycle;
      \path[fill=c2d2d2d] (69.7110,208.6840) .. controls (73.3039,198.0410) and
        (74.9585,195.1720) .. (77.1579,190.8610) .. controls (79.3503,186.5580) and
        (79.7051,183.1790) .. (78.1070,185.2560) .. controls (75.9640,188.0420) and
        (72.2520,196.3410) .. (71.0532,198.9570) .. controls (71.0532,198.9570) and
        (68.7260,204.1290) .. (68.4328,207.4290) .. controls (68.3569,208.6990) and
        (69.3122,209.8160) .. (69.7110,208.6840) -- cycle;
      \path[fill=c303030] (69.8962,208.0730) .. controls (73.4487,197.3630) and
        (75.0631,194.8990) .. (77.0408,190.8960) .. controls (79.0150,186.8980) and
        (79.5254,183.5430) .. (78.0906,185.4600) .. controls (76.0806,188.1240) and
        (72.4383,196.3000) .. (71.1729,199.0620) .. controls (71.1729,199.0620) and
        (69.0404,204.0090) .. (68.8030,206.9010) .. controls (68.7348,208.1410) and
        (69.5152,209.1840) .. (69.8962,208.0730) -- cycle;
    \path[fill=c333333] (70.0813,207.4620) .. controls (73.5934,196.6840) and
      (75.1678,194.6250) .. (76.9236,190.9310) .. controls (78.6796,187.2370) and
      (79.3458,183.9070) .. (78.0742,185.6630) .. controls (76.1971,188.2060) and
      (72.6246,196.2590) .. (71.2926,199.1660) .. controls (71.2926,199.1660) and
      (69.3548,203.8890) .. (69.1731,206.3720) .. controls (69.1127,207.5830) and
      (69.7182,208.5520) .. (70.0813,207.4620) -- cycle;
    \path[fill=black] (250.6260,226.5900) .. controls (252.7140,220.6860) and
      (252.4980,205.6380) .. (247.1700,195.9180) .. controls (245.2980,192.3900) and
      (243.4980,188.2860) .. (242.4180,188.0700) .. controls (241.2660,187.8540) and
      (239.1780,190.1580) .. (239.3940,190.5180) .. controls (239.6820,191.0940) and
      (249.4020,204.7740) .. (247.3860,223.3500) .. controls (247.2420,224.8620) and
      (250.1220,227.9580) .. (250.6260,226.5900) -- cycle;
      \path[fill=c030303] (250.5320,225.8280) .. controls (252.5580,220.1090) and
        (252.2850,205.4480) .. (247.1890,196.1650) .. controls (245.3730,192.7580) and
        (243.6530,188.7910) .. (242.5940,188.5750) .. controls (241.4720,188.3610) and
        (239.4690,190.5970) .. (239.6690,190.9420) .. controls (239.9270,191.5040) and
        (249.2740,204.7200) .. (247.3980,222.6940) .. controls (247.2680,224.1820) and
        (250.0440,227.1510) .. (250.5320,225.8280) -- cycle;
      \path[fill=c070707] (250.4370,225.0650) .. controls (252.4030,219.5320) and
        (252.0730,205.2570) .. (247.2090,196.4110) .. controls (245.4480,193.1250) and
        (243.8080,189.2960) .. (242.7690,189.0790) .. controls (241.6780,188.8680) and
        (239.7600,191.0350) .. (239.9440,191.3660) .. controls (240.1720,191.9140) and
        (249.1460,204.6650) .. (247.4100,222.0380) .. controls (247.2930,223.5020) and
        (249.9660,226.3430) .. (250.4370,225.0650) -- cycle;
      \path[fill=c0b0b0b] (250.3430,224.3020) .. controls (252.2470,218.9550) and
        (251.8600,205.0670) .. (247.2280,196.6570) .. controls (245.5220,193.4920) and
        (243.9630,189.8010) .. (242.9450,189.5840) .. controls (241.8840,189.3740) and
        (240.0500,191.4730) .. (240.2190,191.7900) .. controls (240.4160,192.3240) and
        (249.0180,204.6100) .. (247.4230,221.3820) .. controls (247.3190,222.8220) and
        (249.8880,225.5350) .. (250.3430,224.3020) -- cycle;
      \path[fill=c0f0f0f] (250.2480,223.5390) .. controls (252.0910,218.3770) and
        (251.6470,204.8760) .. (247.2470,196.9030) .. controls (245.5970,193.8590) and
        (244.1170,190.3050) .. (243.1200,190.0880) .. controls (242.0900,189.8810) and
        (240.3410,191.9110) .. (240.4940,192.2140) .. controls (240.6610,192.7340) and
        (248.8900,204.5550) .. (247.4350,220.7250) .. controls (247.3440,222.1420) and
        (249.8100,224.7270) .. (250.2480,223.5390) -- cycle;
      \path[fill=c131313] (250.1540,222.7760) .. controls (251.9350,217.8000) and
        (251.4340,204.6860) .. (247.2660,197.1500) .. controls (245.6710,194.2270) and
        (244.2720,190.8100) .. (243.2960,190.5930) .. controls (242.2960,190.3880) and
        (240.6320,192.3490) .. (240.7690,192.6380) .. controls (240.9050,193.1440) and
        (248.7620,204.5000) .. (247.4470,220.0690) .. controls (247.3690,221.4620) and
        (249.7320,223.9190) .. (250.1540,222.7760) -- cycle;
      \path[fill=c161616] (250.0590,222.0130) .. controls (251.7790,217.2230) and
        (251.2220,204.4950) .. (247.2850,197.3960) .. controls (245.7460,194.5940) and
        (244.4270,191.3150) .. (243.4720,191.0970) .. controls (242.5030,190.8940) and
        (240.9230,192.7880) .. (241.0450,193.0620) .. controls (241.1500,193.5540) and
        (248.6340,204.4450) .. (247.4590,219.4130) .. controls (247.3950,220.7820) and
        (249.6550,223.1110) .. (250.0590,222.0130) -- cycle;
      \path[fill=c1a1a1a] (249.9650,221.2510) .. controls (251.6240,216.6460) and
        (251.0090,204.3050) .. (247.3050,197.6420) .. controls (245.8210,194.9610) and
        (244.5820,191.8200) .. (243.6470,191.6020) .. controls (242.7090,191.4010) and
        (241.2130,193.2260) .. (241.3200,193.4860) .. controls (241.3950,193.9640) and
        (248.5060,204.3900) .. (247.4710,218.7570) .. controls (247.4200,220.1010) and
        (249.5770,222.3030) .. (249.9650,221.2510) -- cycle;
      \path[fill=c1e1e1e] (249.8700,220.4880) .. controls (251.4680,216.0680) and
        (250.7960,204.1140) .. (247.3240,197.8880) .. controls (245.8950,195.3280) and
        (244.7370,192.3240) .. (243.8230,192.1060) .. controls (242.9150,191.9080) and
        (241.5040,193.6640) .. (241.5950,193.9100) .. controls (241.6390,194.3740) and
        (248.3770,204.3350) .. (247.4830,218.1000) .. controls (247.4460,219.4210) and
        (249.4990,221.4950) .. (249.8700,220.4880) -- cycle;
      \path[fill=c222222] (249.7760,219.7250) .. controls (251.3120,215.4910) and
        (250.5830,203.9240) .. (247.3430,198.1350) .. controls (245.9700,195.6960) and
        (244.8910,192.8290) .. (243.9980,192.6110) .. controls (243.1210,192.4140) and
        (241.7950,194.1020) .. (241.8700,194.3340) .. controls (241.8840,194.7840) and
        (248.2490,204.2800) .. (247.4950,217.4440) .. controls (247.4710,218.7410) and
        (249.4210,220.6870) .. (249.7760,219.7250) -- cycle;
      \path[fill=c262626] (249.6810,218.9620) .. controls (251.1560,214.9140) and
        (250.3710,203.7330) .. (247.3620,198.3810) .. controls (246.0450,196.0630) and
        (245.0460,193.3340) .. (244.1740,193.1150) .. controls (243.3270,192.9210) and
        (242.0860,194.5400) .. (242.1450,194.7570) .. controls (242.1290,195.1930) and
        (248.1210,204.2250) .. (247.5070,216.7880) .. controls (247.4970,218.0610) and
        (249.3430,219.8790) .. (249.6810,218.9620) -- cycle;
      \path[fill=c2a2a2a] (249.5870,218.1990) .. controls (251.0000,214.3370) and
        (250.1580,203.5430) .. (247.3810,198.6270) .. controls (246.1190,196.4300) and
        (245.2010,193.8390) .. (244.3490,193.6190) .. controls (243.5330,193.4280) and
        (242.3760,194.9790) .. (242.4200,195.1810) .. controls (242.3730,195.6030) and
        (247.9930,204.1700) .. (247.5190,216.1320) .. controls (247.5220,217.3810) and
        (249.2650,219.0720) .. (249.5870,218.1990) -- cycle;
      \path[fill=c2d2d2d] (249.4920,217.4360) .. controls (250.8450,213.7590) and
        (249.9450,203.3520) .. (247.4010,198.8730) .. controls (246.1940,196.7970) and
        (245.3560,194.3430) .. (244.5250,194.1240) .. controls (243.7390,193.9340) and
        (242.6670,195.4170) .. (242.6950,195.6050) .. controls (242.6180,196.0130) and
        (247.8650,204.1150) .. (247.5310,215.4750) .. controls (247.5480,216.7010) and
        (249.1870,218.2640) .. (249.4920,217.4360) -- cycle;
      \path[fill=c313131] (249.3980,216.6730) .. controls (250.6890,213.1820) and
        (249.7320,203.1620) .. (247.4200,199.1200) .. controls (246.2690,197.1650) and
        (245.5110,194.8480) .. (244.7000,194.6280) .. controls (243.9450,194.4410) and
        (242.9580,195.8550) .. (242.9700,196.0290) .. controls (242.8630,196.4230) and
        (247.7370,204.0600) .. (247.5440,214.8190) .. controls (247.5730,216.0200) and
        (249.1090,217.4560) .. (249.3980,216.6730) -- cycle;
      \path[fill=c353535] (249.3030,215.9110) .. controls (250.5330,212.6050) and
        (249.5200,202.9710) .. (247.4390,199.3660) .. controls (246.3430,197.5320) and
        (245.6650,195.3530) .. (244.8760,195.1330) .. controls (244.1510,194.9480) and
        (243.2490,196.2930) .. (243.2450,196.4530) .. controls (243.1070,196.8330) and
        (247.6090,204.0050) .. (247.5560,214.1630) .. controls (247.5990,215.3400) and
        (249.0310,216.6480) .. (249.3030,215.9110) -- cycle;
      \path[fill=c393939] (249.2090,215.1480) .. controls (250.3770,212.0280) and
        (249.3070,202.7810) .. (247.4580,199.6120) .. controls (246.4180,197.8990) and
        (245.8200,195.8580) .. (245.0510,195.6370) .. controls (244.3570,195.4540) and
        (243.5390,196.7310) .. (243.5200,196.8770) .. controls (243.3520,197.2430) and
        (247.4810,203.9500) .. (247.5680,213.5070) .. controls (247.6240,214.6600) and
        (248.9530,215.8400) .. (249.2090,215.1480) -- cycle;
      \path[fill=c3d3d3d] (249.1140,214.3850) .. controls (250.2210,211.4500) and
        (249.0940,202.5900) .. (247.4770,199.8580) .. controls (246.4930,198.2660) and
        (245.9750,196.3620) .. (245.2270,196.1420) .. controls (244.5630,195.9610) and
        (243.8300,197.1700) .. (243.7950,197.3010) .. controls (243.5970,197.6530) and
        (247.3530,203.8950) .. (247.5800,212.8500) .. controls (247.6490,213.9800) and
        (248.8750,215.0320) .. (249.1140,214.3850) -- cycle;
      \path[fill=c414141] (249.0200,213.6220) .. controls (250.0660,210.8730) and
        (248.8810,202.4000) .. (247.4970,200.1050) .. controls (246.5670,198.6340) and
        (246.1300,196.8670) .. (245.4030,196.6460) .. controls (244.7690,196.4670) and
        (244.1210,197.6080) .. (244.0700,197.7250) .. controls (243.8410,198.0630) and
        (247.2250,203.8400) .. (247.5920,212.1940) .. controls (247.6750,213.3000) and
        (248.7970,214.2240) .. (249.0200,213.6220) -- cycle;
      \path[fill=c444444] (248.9250,212.8590) .. controls (249.9100,210.2960) and
        (248.6690,202.2090) .. (247.5160,200.3510) .. controls (246.6420,199.0010) and
        (246.2850,197.3720) .. (245.5780,197.1510) .. controls (244.9750,196.9740) and
        (244.4120,198.0460) .. (244.3450,198.1490) .. controls (244.0860,198.4730) and
        (247.0960,203.7850) .. (247.6040,211.5380) .. controls (247.7000,212.6200) and
        (248.7190,213.4160) .. (248.9250,212.8590) -- cycle;
      \path[fill=c484848] (248.8310,212.0960) .. controls (249.7540,209.7190) and
        (248.4560,202.0190) .. (247.5350,200.5970) .. controls (246.7170,199.3680) and
        (246.4390,197.8770) .. (245.7540,197.6550) .. controls (245.1810,197.4810) and
        (244.7020,198.4840) .. (244.6200,198.5730) .. controls (244.3310,198.8830) and
        (246.9680,203.7300) .. (247.6160,210.8820) .. controls (247.7260,211.9400) and
        (248.6410,212.6080) .. (248.8310,212.0960) -- cycle;
    \path[fill=c4c4c4c] (248.7360,211.3330) .. controls (249.5980,209.1410) and
      (248.2430,201.8280) .. (247.5540,200.8430) .. controls (246.7910,199.7350) and
      (246.5940,198.3810) .. (245.9290,198.1590) .. controls (245.3870,197.9870) and
      (244.9930,198.9220) .. (244.8950,198.9960) .. controls (244.5750,199.2920) and
      (246.8400,203.6750) .. (247.6280,210.2250) .. controls (247.7510,211.2590) and
      (248.5630,211.8000) .. (248.7360,211.3330) -- cycle;
    \path[fill=black] (242.4180,188.0700) .. controls (232.1940,174.1020) and
      (218.9460,169.5660) .. (220.3140,173.9580) .. controls (220.3140,173.9580) and
      (230.4660,179.7180) .. (239.3940,190.5180) .. controls (241.1220,192.6060) and
      (244.0020,190.2300) .. (242.4180,188.0700) -- cycle;
      \path[fill=c030303] (241.9680,187.6840) .. controls (232.0440,174.1890) and
        (219.0740,169.7720) .. (220.4290,173.9840) .. controls (220.4460,173.9950) and
        (230.3310,179.5980) .. (239.0380,190.0550) .. controls (240.7160,192.0710) and
        (243.5050,189.7700) .. (241.9680,187.6840) -- cycle;
      \path[fill=c070707] (241.5180,187.2970) .. controls (231.8940,174.2750) and
        (219.2010,169.9780) .. (220.5430,174.0100) .. controls (220.5790,174.0330) and
        (230.1950,179.4770) .. (238.6820,189.5920) .. controls (240.3100,191.5350) and
        (243.0070,189.3090) .. (241.5180,187.2970) -- cycle;
      \path[fill=c0b0b0b] (241.0680,186.9100) .. controls (231.7440,174.3610) and
        (219.3290,170.1840) .. (220.6580,174.0360) .. controls (220.7110,174.0700) and
        (230.0590,179.3560) .. (238.3260,189.1290) .. controls (239.9040,190.9990) and
        (242.5090,188.8490) .. (241.0680,186.9100) -- cycle;
      \path[fill=c0f0f0f] (240.6180,186.5240) .. controls (231.5940,174.4470) and
        (219.4560,170.3900) .. (220.7720,174.0620) .. controls (220.8430,174.1070) and
        (229.9230,179.2360) .. (237.9700,188.6660) .. controls (239.4980,190.4640) and
        (242.0120,188.3880) .. (240.6180,186.5240) -- cycle;
      \path[fill=c131313] (240.1670,186.1370) .. controls (231.4440,174.5340) and
        (219.5840,170.5960) .. (220.8870,174.0880) .. controls (220.9750,174.1440) and
        (229.7880,179.1150) .. (237.6140,188.2020) .. controls (239.0920,189.9280) and
        (241.5140,187.9270) .. (240.1670,186.1370) -- cycle;
      \path[fill=c161616] (239.7170,185.7500) .. controls (231.2940,174.6200) and
        (219.7120,170.8020) .. (221.0010,174.1140) .. controls (221.1070,174.1810) and
        (229.6520,178.9940) .. (237.2580,187.7390) .. controls (238.6860,189.3930) and
        (241.0170,187.4670) .. (239.7170,185.7500) -- cycle;
      \path[fill=c1a1a1a] (239.2670,185.3640) .. controls (231.1440,174.7060) and
        (219.8390,171.0080) .. (221.1160,174.1400) .. controls (221.2400,174.2180) and
        (229.5160,178.8740) .. (236.9020,187.2760) .. controls (238.2800,188.8570) and
        (240.5190,187.0060) .. (239.2670,185.3640) -- cycle;
      \path[fill=c1e1e1e] (238.8170,184.9770) .. controls (230.9930,174.7920) and
        (219.9670,171.2140) .. (221.2300,174.1660) .. controls (221.3720,174.2550) and
        (229.3810,178.7530) .. (236.5450,186.8130) .. controls (237.8740,188.3210) and
        (240.0210,186.5460) .. (238.8170,184.9770) -- cycle;
      \path[fill=c222222] (238.3670,184.5900) .. controls (230.8430,174.8790) and
        (220.0940,171.4200) .. (221.3450,174.1920) .. controls (221.5040,174.2920) and
        (229.2450,178.6320) .. (236.1890,186.3500) .. controls (237.4680,187.7860) and
        (239.5240,186.0850) .. (238.3670,184.5900) -- cycle;
      \path[fill=c262626] (237.9170,184.2030) .. controls (230.6930,174.9650) and
        (220.2220,171.6250) .. (221.4590,174.2180) .. controls (221.6360,174.3290) and
        (229.1090,178.5110) .. (235.8330,185.8860) .. controls (237.0620,187.2500) and
        (239.0260,185.6240) .. (237.9170,184.2030) -- cycle;
      \path[fill=c2a2a2a] (237.4670,183.8170) .. controls (230.5430,175.0510) and
        (220.3490,171.8310) .. (221.5740,174.2440) .. controls (221.7680,174.3660) and
        (228.9740,178.3910) .. (235.4770,185.4230) .. controls (236.6560,186.7140) and
        (238.5290,185.1640) .. (237.4670,183.8170) -- cycle;
      \path[fill=c2d2d2d] (237.0160,183.4300) .. controls (230.3930,175.1370) and
        (220.4770,172.0370) .. (221.6880,174.2700) .. controls (221.9010,174.4030) and
        (228.8380,178.2700) .. (235.1210,184.9600) .. controls (236.2500,186.1790) and
        (238.0310,184.7030) .. (237.0160,183.4300) -- cycle;
      \path[fill=c313131] (236.5660,183.0430) .. controls (230.2430,175.2240) and
        (220.6040,172.2430) .. (221.8030,174.2960) .. controls (222.0330,174.4400) and
        (228.7020,178.1490) .. (234.7650,184.4970) .. controls (235.8440,185.6430) and
        (237.5330,184.2430) .. (236.5660,183.0430) -- cycle;
      \path[fill=c353535] (236.1160,182.6570) .. controls (230.0930,175.3100) and
        (220.7320,172.4490) .. (221.9170,174.3220) .. controls (222.1650,174.4770) and
        (228.5660,178.0290) .. (234.4090,184.0340) .. controls (235.4380,185.1070) and
        (237.0360,183.7820) .. (236.1160,182.6570) -- cycle;
      \path[fill=c393939] (235.6660,182.2700) .. controls (229.9430,175.3960) and
        (220.8590,172.6550) .. (222.0320,174.3480) .. controls (222.2970,174.5140) and
        (228.4310,177.9080) .. (234.0530,183.5700) .. controls (235.0310,184.5720) and
        (236.5380,183.3210) .. (235.6660,182.2700) -- cycle;
      \path[fill=c3d3d3d] (235.2160,181.8830) .. controls (229.7930,175.4820) and
        (220.9870,172.8610) .. (222.1460,174.3740) .. controls (222.4290,174.5510) and
        (228.2950,177.7870) .. (233.6970,183.1070) .. controls (234.6250,184.0360) and
        (236.0410,182.8610) .. (235.2160,181.8830) -- cycle;
      \path[fill=c414141] (234.7660,181.4970) .. controls (229.6430,175.5690) and
        (221.1150,173.0670) .. (222.2610,174.4000) .. controls (222.5620,174.5880) and
        (228.1590,177.6670) .. (233.3410,182.6440) .. controls (234.2190,183.5000) and
        (235.5430,182.4000) .. (234.7660,181.4970) -- cycle;
      \path[fill=c444444] (234.3160,181.1100) .. controls (229.4920,175.6550) and
        (221.2420,173.2730) .. (222.3750,174.4260) .. controls (222.6940,174.6250) and
        (228.0240,177.5460) .. (232.9840,182.1810) .. controls (233.8130,182.9650) and
        (235.0450,181.9400) .. (234.3160,181.1100) -- cycle;
      \path[fill=c484848] (233.8650,180.7230) .. controls (229.3420,175.7410) and
        (221.3700,173.4790) .. (222.4900,174.4510) .. controls (222.8260,174.6620) and
        (227.8880,177.4250) .. (232.6280,181.7180) .. controls (233.4070,182.4290) and
        (234.5480,181.4790) .. (233.8650,180.7230) -- cycle;
    \path[fill=c4c4c4c] (233.4150,180.3360) .. controls (229.1920,175.8270) and
      (221.4970,173.6840) .. (222.6040,174.4770) .. controls (222.9580,174.6990) and
      (227.7520,177.3040) .. (232.2720,181.2540) .. controls (233.0010,181.8930) and
      (234.0500,181.0180) .. (233.4150,180.3360) -- cycle;
    \path[fill=black] (250.6260,226.5900) .. controls (250.7700,232.8540) and
      (244.8660,248.8380) .. (242.6340,248.2620) .. controls (240.1140,247.6860) and
      (243.1380,243.0060) .. (245.4420,235.0860) .. controls (246.3780,231.7740) and
      (247.0980,223.6380) .. (247.3860,223.3500) .. controls (248.3220,222.4140) and
      (250.6260,225.0780) .. (250.6260,226.5900) -- cycle;
      \path[fill=c050505] (250.4420,227.1120) .. controls (250.5450,233.1950) and
        (244.8050,248.7060) .. (242.6460,248.1300) .. controls (240.2240,247.5630) and
        (243.1930,242.9840) .. (245.4600,235.2020) .. controls (246.3590,232.0250) and
        (247.0550,224.2550) .. (247.3470,223.9530) .. controls (248.2600,223.0260) and
        (250.4930,225.4800) .. (250.4420,227.1120) -- cycle;
      \path[fill=c0a0a0a] (250.2580,227.6340) .. controls (250.3200,233.5360) and
        (244.7440,248.5730) .. (242.6570,247.9980) .. controls (240.3350,247.4400) and
        (243.2490,242.9620) .. (245.4780,235.3170) .. controls (246.3390,232.2760) and
        (247.0120,224.8710) .. (247.3090,224.5560) .. controls (248.1970,223.6380) and
        (250.3610,225.8820) .. (250.2580,227.6340) -- cycle;
      \path[fill=c0f0f0f] (250.0740,228.1550) .. controls (250.0950,233.8770) and
        (244.6830,248.4400) .. (242.6690,247.8660) .. controls (240.4450,247.3170) and
        (243.3040,242.9400) .. (245.4970,235.4330) .. controls (246.3200,232.5260) and
        (246.9700,225.4870) .. (247.2700,225.1590) .. controls (248.1350,224.2490) and
        (250.2280,226.2840) .. (250.0740,228.1550) -- cycle;
      \path[fill=c141414] (249.8900,228.6770) .. controls (249.8700,234.2170) and
        (244.6220,248.3070) .. (242.6800,247.7340) .. controls (240.5550,247.1930) and
        (243.3590,242.9180) .. (245.5150,235.5480) .. controls (246.3010,232.7770) and
        (246.9270,226.1030) .. (247.2310,225.7620) .. controls (248.0720,224.8610) and
        (250.0950,226.6860) .. (249.8900,228.6770) -- cycle;
      \path[fill=c191919] (249.7060,229.1980) .. controls (249.6440,234.5580) and
        (244.5610,248.1750) .. (242.6920,247.6020) .. controls (240.6650,247.0700) and
        (243.4140,242.8960) .. (245.5330,235.6640) .. controls (246.2810,233.0270) and
        (246.8840,226.7190) .. (247.1920,226.3650) .. controls (248.0100,225.4720) and
        (249.9620,227.0880) .. (249.7060,229.1980) -- cycle;
      \path[fill=c1e1e1e] (249.5220,229.7200) .. controls (249.4190,234.8990) and
        (244.5000,248.0420) .. (242.7040,247.4700) .. controls (240.7750,246.9470) and
        (243.4690,242.8740) .. (245.5510,235.7790) .. controls (246.2620,233.2780) and
        (246.8410,227.3350) .. (247.1530,226.9680) .. controls (247.9480,226.0840) and
        (249.8290,227.4900) .. (249.5220,229.7200) -- cycle;
      \path[fill=c232323] (249.3380,230.2420) .. controls (249.1940,235.2400) and
        (244.4390,247.9090) .. (242.7150,247.3380) .. controls (240.8850,246.8230) and
        (243.5250,242.8520) .. (245.5690,235.8950) .. controls (246.2420,233.5280) and
        (246.7980,227.9510) .. (247.1140,227.5700) .. controls (247.8850,226.6950) and
        (249.6970,227.8910) .. (249.3380,230.2420) -- cycle;
      \path[fill=c282828] (249.1540,230.7630) .. controls (248.9690,235.5800) and
        (244.3770,247.7760) .. (242.7270,247.2060) .. controls (240.9950,246.7000) and
        (243.5800,242.8300) .. (245.5870,236.0100) .. controls (246.2230,233.7790) and
        (246.7550,228.5670) .. (247.0750,228.1730) .. controls (247.8230,227.3070) and
        (249.5640,228.2930) .. (249.1540,230.7630) -- cycle;
      \path[fill=c2d2d2d] (248.9700,231.2850) .. controls (248.7440,235.9210) and
        (244.3160,247.6440) .. (242.7380,247.0740) .. controls (241.1060,246.5770) and
        (243.6350,242.8080) .. (245.6050,236.1260) .. controls (246.2040,234.0290) and
        (246.7120,229.1830) .. (247.0370,228.7760) .. controls (247.7600,227.9180) and
        (249.4310,228.6950) .. (248.9700,231.2850) -- cycle;
      \path[fill=c333333] (248.7860,231.8060) .. controls (248.5190,236.2620) and
        (244.2550,247.5110) .. (242.7500,246.9420) .. controls (241.2160,246.4530) and
        (243.6900,242.7860) .. (245.6230,236.2410) .. controls (246.1840,234.2800) and
        (246.6690,229.7990) .. (246.9980,229.3790) .. controls (247.6980,228.5300) and
        (249.2980,229.0970) .. (248.7860,231.8060) -- cycle;
      \path[fill=c383838] (248.6020,232.3280) .. controls (248.2940,236.6030) and
        (244.1940,247.3780) .. (242.7610,246.8100) .. controls (241.3260,246.3300) and
        (243.7450,242.7640) .. (245.6410,236.3570) .. controls (246.1650,234.5310) and
        (246.6260,230.4150) .. (246.9590,229.9820) .. controls (247.6350,229.1410) and
        (249.1650,229.4990) .. (248.6020,232.3280) -- cycle;
      \path[fill=c3d3d3d] (248.4180,232.8500) .. controls (248.0680,236.9430) and
        (244.1330,247.2450) .. (242.7730,246.6780) .. controls (241.4360,246.2070) and
        (243.8010,242.7420) .. (245.6590,236.4720) .. controls (246.1450,234.7810) and
        (246.5830,231.0310) .. (246.9200,230.5850) .. controls (247.5730,229.7530) and
        (249.0330,229.9010) .. (248.4180,232.8500) -- cycle;
      \path[fill=c424242] (248.2340,233.3710) .. controls (247.8430,237.2840) and
        (244.0720,247.1130) .. (242.7840,246.5460) .. controls (241.5460,246.0840) and
        (243.8560,242.7200) .. (245.6780,236.5880) .. controls (246.1260,235.0320) and
        (246.5410,231.6470) .. (246.8810,231.1880) .. controls (247.5100,230.3650) and
        (248.9000,230.3020) .. (248.2340,233.3710) -- cycle;
      \path[fill=c474747] (248.0500,233.8930) .. controls (247.6180,237.6250) and
        (244.0110,246.9800) .. (242.7960,246.4140) .. controls (241.6560,245.9600) and
        (243.9110,242.6980) .. (245.6960,236.7030) .. controls (246.1070,235.2820) and
        (246.4980,232.2630) .. (246.8420,231.7900) .. controls (247.4480,230.9760) and
        (248.7670,230.7040) .. (248.0500,233.8930) -- cycle;
      \path[fill=c4c4c4c] (247.8660,234.4140) .. controls (247.3930,237.9660) and
        (243.9500,246.8470) .. (242.8070,246.2820) .. controls (241.7660,245.8370) and
        (243.9660,242.6760) .. (245.7140,236.8190) .. controls (246.0870,235.5330) and
        (246.4550,232.8790) .. (246.8030,232.3930) .. controls (247.3850,231.5880) and
        (248.6340,231.1060) .. (247.8660,234.4140) -- cycle;
      \path[fill=c515151] (247.6820,234.9360) .. controls (247.1680,238.3060) and
        (243.8890,246.7140) .. (242.8190,246.1500) .. controls (241.8770,245.7140) and
        (244.0210,242.6540) .. (245.7320,236.9340) .. controls (246.0680,235.7830) and
        (246.4120,233.4950) .. (246.7650,232.9960) .. controls (247.3230,232.1990) and
        (248.5010,231.5080) .. (247.6820,234.9360) -- cycle;
      \path[fill=c565656] (247.4980,235.4580) .. controls (246.9430,238.6470) and
        (243.8280,246.5820) .. (242.8310,246.0180) .. controls (241.9870,245.5900) and
        (244.0770,242.6320) .. (245.7500,237.0500) .. controls (246.0480,236.0340) and
        (246.3690,234.1110) .. (246.7260,233.5990) .. controls (247.2610,232.8110) and
        (248.3690,231.9100) .. (247.4980,235.4580) -- cycle;
      \path[fill=c5b5b5b] (247.3140,235.9790) .. controls (246.7180,238.9880) and
        (243.7660,246.4490) .. (242.8420,245.8860) .. controls (242.0970,245.4670) and
        (244.1320,242.6100) .. (245.7680,237.1650) .. controls (246.0290,236.2840) and
        (246.3260,234.7270) .. (246.6870,234.2020) .. controls (247.1980,233.4220) and
        (248.2360,232.3120) .. (247.3140,235.9790) -- cycle;
      \path[fill=c606060] (247.1300,236.5010) .. controls (246.4920,239.3290) and
        (243.7050,246.3160) .. (242.8540,245.7540) .. controls (242.2070,245.3440) and
        (244.1870,242.5870) .. (245.7860,237.2810) .. controls (246.0100,236.5350) and
        (246.2830,235.3430) .. (246.6480,234.8050) .. controls (247.1360,234.0340) and
        (248.1030,232.7140) .. (247.1300,236.5010) -- cycle;
    \path[fill=c666666] (246.9460,237.0220) .. controls (246.2670,239.6690) and
      (243.6440,246.1830) .. (242.8650,245.6210) .. controls (242.3170,245.2200) and
      (244.2420,242.5650) .. (245.8040,237.3960) .. controls (245.9900,236.7850) and
      (246.2400,235.9590) .. (246.6090,235.4070) .. controls (247.0730,234.6450) and
      (247.9700,233.1150) .. (246.9460,237.0220) -- cycle;
    \path[fill=black] (228.6660,188.8620) .. controls (227.1290,190.4840) and
      (225.7630,198.8530) .. (229.6050,201.7560) .. controls (233.4480,204.5740) and
      (240.1940,199.3650) .. (240.1940,196.3760) .. controls (240.1080,189.6300) and
      (230.2030,187.1540) .. (228.6660,188.8620) -- cycle;
      \path[fill=c010101] (228.8030,189.0030) .. controls (227.2440,190.6140) and
        (225.9270,198.8250) .. (229.6600,201.6690) .. controls (233.3920,204.4330) and
        (240.0090,199.3960) .. (240.0440,196.3900) .. controls (239.9970,189.8150) and
        (230.3630,187.3080) .. (228.8030,189.0030) -- cycle;
      \path[fill=c030303] (228.9400,189.1430) .. controls (227.3580,190.7440) and
        (226.0910,198.7970) .. (229.7150,201.5820) .. controls (233.3350,204.2910) and
        (239.8240,199.4270) .. (239.8930,196.4040) .. controls (239.8860,190.0000) and
        (230.5220,187.4610) .. (228.9400,189.1430) -- cycle;
      \path[fill=c050505] (229.0770,189.2840) .. controls (227.4730,190.8730) and
        (226.2540,198.7680) .. (229.7690,201.4950) .. controls (233.2790,204.1490) and
        (239.6380,199.4580) .. (239.7430,196.4180) .. controls (239.7740,190.1850) and
        (230.6820,187.6150) .. (229.0770,189.2840) -- cycle;
      \path[fill=c070707] (229.2130,189.4240) .. controls (227.5870,191.0030) and
        (226.4180,198.7400) .. (229.8240,201.4080) .. controls (233.2220,204.0070) and
        (239.4530,199.4890) .. (239.5920,196.4320) .. controls (239.6630,190.3700) and
        (230.8410,187.7680) .. (229.2130,189.4240) -- cycle;
      \path[fill=c090909] (229.3500,189.5640) .. controls (227.7010,191.1320) and
        (226.5820,198.7110) .. (229.8790,201.3210) .. controls (233.1660,203.8650) and
        (239.2680,199.5200) .. (239.4420,196.4460) .. controls (239.5510,190.5550) and
        (231.0010,187.9220) .. (229.3500,189.5640) -- cycle;
      \path[fill=c0b0b0b] (229.4870,189.7050) .. controls (227.8160,191.2620) and
        (226.7460,198.6830) .. (229.9330,201.2340) .. controls (233.1090,203.7230) and
        (239.0830,199.5510) .. (239.2920,196.4600) .. controls (239.4400,190.7400) and
        (231.1600,188.0750) .. (229.4870,189.7050) -- cycle;
      \path[fill=c0d0d0d] (229.6240,189.8450) .. controls (227.9300,191.3910) and
        (226.9090,198.6540) .. (229.9880,201.1470) .. controls (233.0530,203.5810) and
        (238.8970,199.5810) .. (239.1410,196.4740) .. controls (239.3290,190.9250) and
        (231.3200,188.2290) .. (229.6240,189.8450) -- cycle;
      \path[fill=c0f0f0f] (229.7610,189.9860) .. controls (228.0450,191.5210) and
        (227.0730,198.6260) .. (230.0430,201.0600) .. controls (232.9960,203.4400) and
        (238.7120,199.6120) .. (238.9910,196.4880) .. controls (239.2170,191.1100) and
        (231.4790,188.3820) .. (229.7610,189.9860) -- cycle;
      \path[fill=c111111] (229.8970,190.1260) .. controls (228.1590,191.6500) and
        (227.2370,198.5970) .. (230.0980,200.9730) .. controls (232.9400,203.2980) and
        (238.5270,199.6430) .. (238.8400,196.5020) .. controls (239.1060,191.2950) and
        (231.6390,188.5360) .. (229.8970,190.1260) -- cycle;
      \path[fill=c131313] (230.0340,190.2660) .. controls (228.2740,191.7800) and
        (227.4010,198.5690) .. (230.1520,200.8850) .. controls (232.8830,203.1560) and
        (238.3420,199.6740) .. (238.6900,196.5160) .. controls (238.9950,191.4800) and
        (231.7980,188.6890) .. (230.0340,190.2660) -- cycle;
      \path[fill=c151515] (230.1710,190.4070) .. controls (228.3880,191.9100) and
        (227.5640,198.5400) .. (230.2070,200.7980) .. controls (232.8270,203.0140) and
        (238.1560,199.7050) .. (238.5390,196.5300) .. controls (238.8830,191.6650) and
        (231.9580,188.8430) .. (230.1710,190.4070) -- cycle;
      \path[fill=c161616] (230.3080,190.5470) .. controls (228.5030,192.0390) and
        (227.7280,198.5120) .. (230.2620,200.7110) .. controls (232.7700,202.8720) and
        (237.9710,199.7360) .. (238.3890,196.5440) .. controls (238.7720,191.8500) and
        (232.1170,188.9960) .. (230.3080,190.5470) -- cycle;
      \path[fill=c181818] (230.4450,190.6880) .. controls (228.6170,192.1690) and
        (227.8920,198.4840) .. (230.3160,200.6240) .. controls (232.7140,202.7300) and
        (237.7860,199.7660) .. (238.2380,196.5580) .. controls (238.6610,192.0350) and
        (232.2770,189.1500) .. (230.4450,190.6880) -- cycle;
      \path[fill=c1a1a1a] (230.5810,190.8280) .. controls (228.7320,192.2980) and
        (228.0560,198.4550) .. (230.3710,200.5370) .. controls (232.6570,202.5890) and
        (237.6010,199.7970) .. (238.0880,196.5720) .. controls (238.5490,192.2200) and
        (232.4360,189.3030) .. (230.5810,190.8280) -- cycle;
      \path[fill=c1c1c1c] (230.7180,190.9680) .. controls (228.8460,192.4280) and
        (228.2190,198.4270) .. (230.4260,200.4500) .. controls (232.6010,202.4470) and
        (237.4150,199.8280) .. (237.9370,196.5860) .. controls (238.4380,192.4050) and
        (232.5960,189.4570) .. (230.7180,190.9680) -- cycle;
      \path[fill=c1e1e1e] (230.8550,191.1090) .. controls (228.9600,192.5570) and
        (228.3830,198.3980) .. (230.4800,200.3630) .. controls (232.5440,202.3050) and
        (237.2300,199.8590) .. (237.7870,196.6000) .. controls (238.3270,192.5900) and
        (232.7550,189.6100) .. (230.8550,191.1090) -- cycle;
      \path[fill=c202020] (230.9920,191.2490) .. controls (229.0750,192.6870) and
        (228.5470,198.3700) .. (230.5350,200.2760) .. controls (232.4880,202.1630) and
        (237.0450,199.8900) .. (237.6370,196.6140) .. controls (238.2150,192.7750) and
        (232.9150,189.7640) .. (230.9920,191.2490) -- cycle;
      \path[fill=c222222] (231.1290,191.3900) .. controls (229.1890,192.8160) and
        (228.7110,198.3410) .. (230.5900,200.1890) .. controls (232.4310,202.0210) and
        (236.8600,199.9210) .. (237.4860,196.6280) .. controls (238.1040,192.9600) and
        (233.0740,189.9170) .. (231.1290,191.3900) -- cycle;
      \path[fill=c242424] (231.2650,191.5300) .. controls (229.3040,192.9460) and
        (228.8740,198.3130) .. (230.6450,200.1020) .. controls (232.3750,201.8790) and
        (236.6740,199.9520) .. (237.3360,196.6420) .. controls (237.9930,193.1450) and
        (233.2340,190.0710) .. (231.2650,191.5300) -- cycle;
    \path[fill=c262626] (231.4020,191.6700) .. controls (229.4180,193.0750) and
      (229.0380,198.2840) .. (230.6990,200.0140) .. controls (232.3180,201.7370) and
      (236.4890,199.9820) .. (237.1850,196.6560) .. controls (237.8810,193.3300) and
      (233.3930,190.2240) .. (231.4020,191.6700) -- cycle;
    \path[fill=black] (228.3060,257.6940) .. controls (229.8180,259.2060) and
      (231.1140,255.4620) .. (232.6260,254.2380) .. controls (234.1380,253.0140) and
      (236.5860,250.3500) .. (241.4820,250.3500) .. controls (246.3780,250.3500) and
      (246.0180,250.2060) .. (245.8020,248.3340) .. controls (245.6580,246.5340) and
      (244.2180,246.6780) .. (239.8980,247.3260) .. controls (235.5780,247.9020) and
      (232.6980,250.1340) .. (230.9700,252.1500) .. controls (229.3140,254.0940) and
      (227.4420,256.8300) .. (228.3060,257.6940) -- cycle;
      \path[fill=c050505] (228.6390,257.3270) .. controls (230.0980,258.7770) and
        (231.3420,255.2070) .. (232.8440,254.0200) .. controls (234.3450,252.8340) and
        (236.7740,250.2880) .. (241.4830,250.2880) .. controls (246.1930,250.2870) and
        (245.8980,250.1600) .. (245.6920,248.3480) .. controls (245.5540,246.6050) and
        (244.1490,246.7550) .. (239.9430,247.3690) .. controls (235.7370,247.9120) and
        (232.9140,250.0710) .. (231.2310,252.0170) .. controls (229.6160,253.8950) and
        (227.7940,256.4920) .. (228.6390,257.3270) -- cycle;
      \path[fill=c0a0a0a] (228.9720,256.9590) .. controls (230.3790,258.3470) and
        (231.5710,254.9510) .. (233.0610,253.8020) .. controls (234.5520,252.6530) and
        (236.9620,250.2260) .. (241.4850,250.2250) .. controls (246.0070,250.2240) and
        (245.7780,250.1140) .. (245.5810,248.3620) .. controls (245.4500,246.6750) and
        (244.0800,246.8320) .. (239.9870,247.4110) .. controls (235.8950,247.9210) and
        (233.1300,250.0070) .. (231.4920,251.8830) .. controls (229.9180,253.6950) and
        (228.1450,256.1540) .. (228.9720,256.9590) -- cycle;
      \path[fill=c0f0f0f] (229.3050,256.5910) .. controls (230.6590,257.9180) and
        (231.7990,254.6960) .. (233.2790,253.5840) .. controls (234.7590,252.4720) and
        (237.1500,250.1630) .. (241.4860,250.1620) .. controls (245.8220,250.1600) and
        (245.6570,250.0680) .. (245.4710,248.3760) .. controls (245.3450,246.7450) and
        (244.0110,246.9090) .. (240.0320,247.4530) .. controls (236.0530,247.9310) and
        (233.3460,249.9440) .. (231.7520,251.7500) .. controls (230.2210,253.4950) and
        (228.4960,255.8150) .. (229.3050,256.5910) -- cycle;
      \path[fill=c141414] (229.6380,256.2230) .. controls (230.9390,257.4880) and
        (232.0270,254.4400) .. (233.4960,253.3660) .. controls (234.9660,252.2910) and
        (237.3380,250.1010) .. (241.4870,250.0990) .. controls (245.6370,250.0970) and
        (245.5370,250.0220) .. (245.3600,248.3900) .. controls (245.2410,246.8150) and
        (243.9410,246.9860) .. (240.0760,247.4950) .. controls (236.2110,247.9400) and
        (233.5620,249.8800) .. (232.0130,251.6160) .. controls (230.5230,253.2950) and
        (228.8480,255.4770) .. (229.6380,256.2230) -- cycle;
      \path[fill=c191919] (229.9700,255.8560) .. controls (231.2190,257.0580) and
        (232.2550,254.1850) .. (233.7140,253.1480) .. controls (235.1720,252.1110) and
        (237.5260,250.0390) .. (241.4880,250.0360) .. controls (245.4510,250.0330) and
        (245.4170,249.9760) .. (245.2490,248.4040) .. controls (245.1360,246.8850) and
        (243.8720,247.0630) .. (240.1210,247.5370) .. controls (236.3700,247.9500) and
        (233.7770,249.8160) .. (232.2740,251.4820) .. controls (230.8250,253.0950) and
        (229.1990,255.1380) .. (229.9700,255.8560) -- cycle;
      \path[fill=c1e1e1e] (230.3030,255.4880) .. controls (231.4990,256.6290) and
        (232.4840,253.9290) .. (233.9320,252.9300) .. controls (235.3790,251.9300) and
        (237.7140,249.9760) .. (241.4900,249.9730) .. controls (245.2660,249.9700) and
        (245.2960,249.9300) .. (245.1390,248.4180) .. controls (245.0320,246.9560) and
        (243.8030,247.1400) .. (240.1650,247.5790) .. controls (236.5280,247.9590) and
        (233.9930,249.7530) .. (232.5350,251.3490) .. controls (231.1270,252.8950) and
        (229.5510,254.8000) .. (230.3030,255.4880) -- cycle;
      \path[fill=c232323] (230.6360,255.1200) .. controls (231.7790,256.1990) and
        (232.7120,253.6740) .. (234.1490,252.7110) .. controls (235.5860,251.7490) and
        (237.9020,249.9140) .. (241.4910,249.9100) .. controls (245.0800,249.9070) and
        (245.1760,249.8840) .. (245.0280,248.4310) .. controls (244.9280,247.0260) and
        (243.7340,247.2170) .. (240.2100,247.6210) .. controls (236.6860,247.9690) and
        (234.2090,249.6890) .. (232.7950,251.2150) .. controls (231.4290,252.6950) and
        (229.9020,254.4610) .. (230.6360,255.1200) -- cycle;
      \path[fill=c282828] (230.9690,254.7520) .. controls (232.0590,255.7690) and
        (232.9400,253.4180) .. (234.3670,252.4930) .. controls (235.7930,251.5680) and
        (238.0900,249.8520) .. (241.4920,249.8480) .. controls (244.8950,249.8430) and
        (245.0560,249.8380) .. (244.9180,248.4450) .. controls (244.8230,247.0960) and
        (243.6650,247.2940) .. (240.2540,247.6630) .. controls (236.8450,247.9780) and
        (234.4250,249.6250) .. (233.0560,251.0820) .. controls (231.7310,252.4950) and
        (230.2530,254.1230) .. (230.9690,254.7520) -- cycle;
      \path[fill=c2d2d2d] (231.3020,254.3850) .. controls (232.3400,255.3400) and
        (233.1680,253.1630) .. (234.5840,252.2750) .. controls (236.0000,251.3880) and
        (238.2780,249.7890) .. (241.4930,249.7850) .. controls (244.7100,249.7800) and
        (244.9360,249.7920) .. (244.8070,248.4590) .. controls (244.7190,247.1660) and
        (243.5950,247.3710) .. (240.2990,247.7050) .. controls (237.0030,247.9880) and
        (234.6410,249.5620) .. (233.3170,250.9480) .. controls (232.0330,252.2950) and
        (230.6050,253.7840) .. (231.3020,254.3850) -- cycle;
      \path[fill=c333333] (231.6350,254.0170) .. controls (232.6200,254.9100) and
        (233.3970,252.9070) .. (234.8020,252.0570) .. controls (236.2070,251.2070) and
        (238.4660,249.7270) .. (241.4950,249.7220) .. controls (244.5240,249.7160) and
        (244.8150,249.7450) .. (244.6970,248.4730) .. controls (244.6150,247.2360) and
        (243.5260,247.4480) .. (240.3430,247.7470) .. controls (237.1610,247.9970) and
        (234.8570,249.4980) .. (233.5780,250.8140) .. controls (232.3350,252.0950) and
        (230.9560,253.4460) .. (231.6350,254.0170) -- cycle;
      \path[fill=c383838] (231.9680,253.6490) .. controls (232.9000,254.4800) and
        (233.6250,252.6510) .. (235.0190,251.8390) .. controls (236.4140,251.0260) and
        (238.6540,249.6650) .. (241.4960,249.6590) .. controls (244.3390,249.6530) and
        (244.6950,249.6990) .. (244.5860,248.4870) .. controls (244.5100,247.3070) and
        (243.4570,247.5250) .. (240.3880,247.7900) .. controls (237.3200,248.0070) and
        (235.0730,249.4340) .. (233.8380,250.6810) .. controls (232.6370,251.8950) and
        (231.3080,253.1080) .. (231.9680,253.6490) -- cycle;
      \path[fill=c3d3d3d] (232.3000,253.2810) .. controls (233.1800,254.0510) and
        (233.8530,252.3960) .. (235.2370,251.6210) .. controls (236.6200,250.8450) and
        (238.8420,249.6020) .. (241.4970,249.5960) .. controls (244.1530,249.5900) and
        (244.5750,249.6530) .. (244.4760,248.5010) .. controls (244.4060,247.3770) and
        (243.3880,247.6020) .. (240.4320,247.8320) .. controls (237.4780,248.0160) and
        (235.2880,249.3710) .. (234.0990,250.5470) .. controls (232.9390,251.6950) and
        (231.6590,252.7690) .. (232.3000,253.2810) -- cycle;
      \path[fill=c424242] (232.6330,252.9140) .. controls (233.4600,253.6210) and
        (234.0810,252.1400) .. (235.4540,251.4020) .. controls (236.8270,250.6650) and
        (239.0300,249.5400) .. (241.4980,249.5330) .. controls (243.9680,249.5260) and
        (244.4540,249.6070) .. (244.3650,248.5140) .. controls (244.3020,247.4470) and
        (243.3190,247.6790) .. (240.4770,247.8740) .. controls (237.6360,248.0260) and
        (235.5040,249.3070) .. (234.3600,250.4140) .. controls (233.2420,251.4950) and
        (232.0100,252.4310) .. (232.6330,252.9140) -- cycle;
      \path[fill=c474747] (232.9660,252.5460) .. controls (233.7400,253.1910) and
        (234.3100,251.8850) .. (235.6720,251.1840) .. controls (237.0340,250.4840) and
        (239.2180,249.4780) .. (241.5000,249.4710) .. controls (243.7830,249.4630) and
        (244.3340,249.5610) .. (244.2550,248.5280) .. controls (244.1970,247.5170) and
        (243.2490,247.7560) .. (240.5210,247.9160) .. controls (237.7940,248.0350) and
        (235.7200,249.2430) .. (234.6210,250.2800) .. controls (233.5440,251.2950) and
        (232.3620,252.0920) .. (232.9660,252.5460) -- cycle;
      \path[fill=c4c4c4c] (233.2990,252.1780) .. controls (234.0200,252.7620) and
        (234.5380,251.6290) .. (235.8890,250.9660) .. controls (237.2410,250.3030) and
        (239.4050,249.4150) .. (241.5010,249.4080) .. controls (243.5970,249.3990) and
        (244.2140,249.5150) .. (244.1440,248.5420) .. controls (244.0930,247.5870) and
        (243.1800,247.8330) .. (240.5660,247.9580) .. controls (237.9530,248.0450) and
        (235.9360,249.1800) .. (234.8810,250.1460) .. controls (233.8460,251.0950) and
        (232.7130,251.7540) .. (233.2990,252.1780) -- cycle;
      \path[fill=c515151] (233.6320,251.8100) .. controls (234.3010,252.3320) and
        (234.7660,251.3740) .. (236.1070,250.7480) .. controls (237.4480,250.1220) and
        (239.5930,249.3530) .. (241.5020,249.3450) .. controls (243.4120,249.3360) and
        (244.0930,249.4690) .. (244.0330,248.5560) .. controls (243.9890,247.6580) and
        (243.1110,247.9100) .. (240.6100,248.0000) .. controls (238.1110,248.0540) and
        (236.1520,249.1160) .. (235.1420,250.0130) .. controls (234.1480,250.8950) and
        (233.0650,251.4150) .. (233.6320,251.8100) -- cycle;
      \path[fill=c565656] (233.9650,251.4430) .. controls (234.5810,251.9020) and
        (234.9940,251.1180) .. (236.3250,250.5300) .. controls (237.6550,249.9420) and
        (239.7810,249.2900) .. (241.5030,249.2820) .. controls (243.2260,249.2730) and
        (243.9730,249.4230) .. (243.9230,248.5700) .. controls (243.8840,247.7280) and
        (243.0420,247.9870) .. (240.6550,248.0420) .. controls (238.2690,248.0640) and
        (236.3680,249.0520) .. (235.4030,249.8790) .. controls (234.4500,250.6950) and
        (233.4160,251.0770) .. (233.9650,251.4430) -- cycle;
      \path[fill=c5b5b5b] (234.2980,251.0750) .. controls (234.8610,251.4730) and
        (235.2230,250.8630) .. (236.5420,250.3120) .. controls (237.8620,249.7610) and
        (239.9690,249.2280) .. (241.5050,249.2190) .. controls (243.0410,249.2090) and
        (243.8530,249.3770) .. (243.8120,248.5840) .. controls (243.7800,247.7980) and
        (242.9730,248.0640) .. (240.6990,248.0840) .. controls (238.4280,248.0730) and
        (236.5840,248.9890) .. (235.6640,249.7460) .. controls (234.7520,250.4950) and
        (233.7670,250.7380) .. (234.2980,251.0750) -- cycle;
      \path[fill=c606060] (234.6300,250.7070) .. controls (235.1410,251.0430) and
        (235.4510,250.6070) .. (236.7600,250.0940) .. controls (238.0680,249.5800) and
        (240.1570,249.1660) .. (241.5060,249.1560) .. controls (242.8560,249.1460) and
        (243.7330,249.3310) .. (243.7020,248.5980) .. controls (243.6760,247.8680) and
        (242.9030,248.1410) .. (240.7440,248.1260) .. controls (238.5860,248.0830) and
        (236.7990,248.9250) .. (235.9240,249.6120) .. controls (235.0540,250.2950) and
        (234.1190,250.4000) .. (234.6300,250.7070) -- cycle;
    \path[fill=c666666] (234.9630,250.3390) .. controls (235.4210,250.6130) and
      (235.6790,250.3510) .. (236.9770,249.8750) .. controls (238.2750,249.3990) and
      (240.3450,249.1030) .. (241.5070,249.0930) .. controls (242.6700,249.0820) and
      (243.6120,249.2840) .. (243.5910,248.6110) .. controls (243.5710,247.9380) and
      (242.8340,248.2180) .. (240.7880,248.1680) .. controls (238.7440,248.0920) and
      (237.0150,248.8610) .. (236.1850,249.4780) .. controls (235.3560,250.0950) and
      (234.4700,250.0610) .. (234.9630,250.3390) -- cycle;
    \path[fill=black] (247.8180,248.2620) .. controls (247.8180,250.5660) and
      (252.4980,251.3580) .. (256.9620,252.0060) .. controls (261.3540,252.6540) and
      (264.8820,253.5180) .. (265.0980,258.1260) .. controls (265.3140,262.6620) and
      (264.1620,265.9020) .. (266.1780,265.5420) .. controls (270.4980,264.7500) and
      (272.0820,260.0700) .. (272.0100,257.9100) .. controls (272.0100,255.7500) and
      (268.8420,251.8620) .. (263.1540,249.7020) .. controls (258.9780,248.1180) and
      (255.9540,247.3980) .. (252.2820,247.2540) .. controls (247.3860,247.0380) and
      (247.8180,248.2620) .. (247.8180,248.2620) -- cycle;
      \path[fill=c030303] (247.9440,248.3010) .. controls (247.9560,250.5400) and
        (252.5870,251.2710) .. (256.9930,251.9400) .. controls (261.3450,252.6140) and
        (264.7430,253.4500) .. (265.1040,257.9420) .. controls (265.4480,262.3570) and
        (264.2500,265.4560) .. (266.2390,265.1310) .. controls (270.3840,264.3910) and
        (271.9270,259.9300) .. (271.8470,257.7990) .. controls (271.8320,255.6470) and
        (268.7290,251.8880) .. (263.1130,249.7600) .. controls (258.9840,248.1960) and
        (256.0030,247.4620) .. (252.3730,247.3080) .. controls (247.6190,247.0910) and
        (247.9350,248.2600) .. (247.9440,248.3010) -- cycle;
      \path[fill=c070707] (248.0700,248.3390) .. controls (248.0940,250.5130) and
        (252.6760,251.1840) .. (257.0230,251.8730) .. controls (261.3350,252.5740) and
        (264.6040,253.3810) .. (265.1100,257.7580) .. controls (265.5810,262.0520) and
        (264.3390,265.0090) .. (266.3000,264.7200) .. controls (270.2700,264.0310) and
        (271.7730,259.7900) .. (271.6840,257.6870) .. controls (271.6550,255.5430) and
        (268.6150,251.9140) .. (263.0730,249.8170) .. controls (258.9910,248.2740) and
        (256.0510,247.5260) .. (252.4640,247.3610) .. controls (247.8520,247.1430) and
        (248.0530,248.2570) .. (248.0700,248.3390) -- cycle;
      \path[fill=c0b0b0b] (248.1960,248.3770) .. controls (248.2310,250.4860) and
        (252.7650,251.0970) .. (257.0540,251.8070) .. controls (261.3250,252.5340) and
        (264.4650,253.3120) .. (265.1170,257.5730) .. controls (265.7150,261.7470) and
        (264.4270,264.5620) .. (266.3610,264.3090) .. controls (270.1570,263.6720) and
        (271.6180,259.6490) .. (271.5210,257.5750) .. controls (271.4770,255.4390) and
        (268.5020,251.9400) .. (263.0320,249.8750) .. controls (258.9970,248.3520) and
        (256.1000,247.5900) .. (252.5550,247.4150) .. controls (248.0850,247.1960) and
        (248.1700,248.2540) .. (248.1960,248.3770) -- cycle;
      \path[fill=c0f0f0f] (248.3220,248.4160) .. controls (248.3690,250.4590) and
        (252.8540,251.0100) .. (257.0840,251.7400) .. controls (261.3160,252.4940) and
        (264.3260,253.2440) .. (265.1230,257.3890) .. controls (265.8480,261.4410) and
        (264.5150,264.1160) .. (266.4220,263.8980) .. controls (270.0430,263.3120) and
        (271.4630,259.5090) .. (271.3580,257.4630) .. controls (271.2990,255.3350) and
        (268.3880,251.9660) .. (262.9910,249.9320) .. controls (259.0030,248.4300) and
        (256.1490,247.6540) .. (252.6460,247.4680) .. controls (248.3180,247.2480) and
        (248.2870,248.2510) .. (248.3220,248.4160) -- cycle;
      \path[fill=c131313] (248.4480,248.4540) .. controls (248.5070,250.4320) and
        (252.9430,250.9230) .. (257.1150,251.6730) .. controls (261.3060,252.4540) and
        (264.1880,253.1750) .. (265.1290,257.2050) .. controls (265.9820,261.1360) and
        (264.6030,263.6690) .. (266.4830,263.4870) .. controls (269.9290,262.9520) and
        (271.3080,259.3690) .. (271.1950,257.3510) .. controls (271.1210,255.2320) and
        (268.2750,251.9920) .. (262.9500,249.9900) .. controls (259.0090,248.5080) and
        (256.1970,247.7180) .. (252.7370,247.5210) .. controls (248.5510,247.3000) and
        (248.4040,248.2480) .. (248.4480,248.4540) -- cycle;
      \path[fill=c161616] (248.5740,248.4920) .. controls (248.6450,250.4050) and
        (253.0330,250.8360) .. (257.1460,251.6070) .. controls (261.2960,252.4140) and
        (264.0490,253.1070) .. (265.1350,257.0200) .. controls (266.1160,260.8310) and
        (264.6910,263.2220) .. (266.5440,263.0760) .. controls (269.8150,262.5930) and
        (271.1530,259.2280) .. (271.0320,257.2400) .. controls (270.9440,255.1280) and
        (268.1620,252.0180) .. (262.9090,250.0470) .. controls (259.0160,248.5860) and
        (256.2460,247.7820) .. (252.8290,247.5750) .. controls (248.7840,247.3530) and
        (248.5210,248.2450) .. (248.5740,248.4920) -- cycle;
      \path[fill=c1a1a1a] (248.7000,248.5310) .. controls (248.7820,250.3780) and
        (253.1220,250.7490) .. (257.1760,251.5400) .. controls (261.2870,252.3730) and
        (263.9100,253.0380) .. (265.1410,256.8360) .. controls (266.2490,260.5260) and
        (264.7800,262.7760) .. (266.6050,262.6650) .. controls (269.7010,262.2330) and
        (270.9980,259.0880) .. (270.8690,257.1280) .. controls (270.7660,255.0240) and
        (268.0480,252.0440) .. (262.8690,250.1050) .. controls (259.0220,248.6630) and
        (256.2940,247.8450) .. (252.9200,247.6280) .. controls (249.0170,247.4050) and
        (248.6380,248.2420) .. (248.7000,248.5310) -- cycle;
      \path[fill=c1e1e1e] (248.8260,248.5690) .. controls (248.9200,250.3510) and
        (253.2110,250.6620) .. (257.2070,251.4740) .. controls (261.2770,252.3330) and
        (263.7710,252.9690) .. (265.1470,256.6520) .. controls (266.3830,260.2200) and
        (264.8680,262.3290) .. (266.6660,262.2540) .. controls (269.5870,261.8730) and
        (270.8430,258.9480) .. (270.7060,257.0160) .. controls (270.5880,254.9200) and
        (267.9350,252.0700) .. (262.8280,250.1620) .. controls (259.0280,248.7410) and
        (256.3430,247.9090) .. (253.0110,247.6820) .. controls (249.2490,247.4580) and
        (248.7550,248.2390) .. (248.8260,248.5690) -- cycle;
      \path[fill=c222222] (248.9520,248.6070) .. controls (249.0580,250.3240) and
        (253.3000,250.5750) .. (257.2370,251.4070) .. controls (261.2670,252.2930) and
        (263.6320,252.9010) .. (265.1530,256.4670) .. controls (266.5160,259.9150) and
        (264.9560,261.8820) .. (266.7270,261.8430) .. controls (269.4730,261.5140) and
        (270.6890,258.8070) .. (270.5430,256.9040) .. controls (270.4100,254.8170) and
        (267.8210,252.0960) .. (262.7870,250.2200) .. controls (259.0340,248.8190) and
        (256.3920,247.9730) .. (253.1020,247.7350) .. controls (249.4820,247.5100) and
        (248.8730,248.2360) .. (248.9520,248.6070) -- cycle;
      \path[fill=c262626] (249.0780,248.6450) .. controls (249.1960,250.2970) and
        (253.3890,250.4870) .. (257.2680,251.3400) .. controls (261.2580,252.2530) and
        (263.4930,252.8320) .. (265.1590,256.2830) .. controls (266.6500,259.6100) and
        (265.0440,261.4350) .. (266.7880,261.4320) .. controls (269.3590,261.1540) and
        (270.5340,258.6670) .. (270.3800,256.7920) .. controls (270.2330,254.7130) and
        (267.7080,252.1210) .. (262.7460,250.2770) .. controls (259.0410,248.8970) and
        (256.4400,248.0370) .. (253.1930,247.7880) .. controls (249.7150,247.5620) and
        (248.9900,248.2330) .. (249.0780,248.6450) -- cycle;
      \path[fill=c2a2a2a] (249.2040,248.6840) .. controls (249.3330,250.2710) and
        (253.4780,250.4000) .. (257.2980,251.2740) .. controls (261.2480,252.2130) and
        (263.3540,252.7630) .. (265.1650,256.0990) .. controls (266.7830,259.3050) and
        (265.1320,260.9890) .. (266.8490,261.0210) .. controls (269.2450,260.7940) and
        (270.3790,258.5270) .. (270.2170,256.6810) .. controls (270.0550,254.6090) and
        (267.5940,252.1470) .. (262.7050,250.3340) .. controls (259.0470,248.9750) and
        (256.4890,248.1010) .. (253.2840,247.8420) .. controls (249.9480,247.6150) and
        (249.1070,248.2310) .. (249.2040,248.6840) -- cycle;
      \path[fill=c2d2d2d] (249.3300,248.7220) .. controls (249.4710,250.2440) and
        (253.5670,250.3130) .. (257.3290,251.2070) .. controls (261.2380,252.1730) and
        (263.2150,252.6950) .. (265.1710,255.9140) .. controls (266.9170,258.9990) and
        (265.2210,260.5420) .. (266.9100,260.6100) .. controls (269.1310,260.4350) and
        (270.2240,258.3860) .. (270.0540,256.5690) .. controls (269.8770,254.5050) and
        (267.4810,252.1730) .. (262.6650,250.3920) .. controls (259.0530,249.0530) and
        (256.5370,248.1650) .. (253.3750,247.8950) .. controls (250.1810,247.6670) and
        (249.2240,248.2280) .. (249.3300,248.7220) -- cycle;
      \path[fill=c313131] (249.4560,248.7600) .. controls (249.6090,250.2170) and
        (253.6560,250.2260) .. (257.3590,251.1410) .. controls (261.2290,252.1330) and
        (263.0760,252.6260) .. (265.1780,255.7300) .. controls (267.0500,258.6940) and
        (265.3090,260.0950) .. (266.9710,260.1990) .. controls (269.0180,260.0750) and
        (270.0690,258.2460) .. (269.8910,256.4570) .. controls (269.6990,254.4020) and
        (267.3670,252.1990) .. (262.6240,250.4490) .. controls (259.0590,249.1300) and
        (256.5860,248.2290) .. (253.4660,247.9490) .. controls (250.4140,247.7200) and
        (249.3410,248.2250) .. (249.4560,248.7600) -- cycle;
      \path[fill=c353535] (249.5820,248.7990) .. controls (249.7470,250.1900) and
        (253.7450,250.1390) .. (257.3900,251.0740) .. controls (261.2190,252.0920) and
        (262.9370,252.5570) .. (265.1840,255.5460) .. controls (267.1840,258.3890) and
        (265.3970,259.6490) .. (267.0320,259.7880) .. controls (268.9040,259.7150) and
        (269.9140,258.1060) .. (269.7280,256.3450) .. controls (269.5220,254.2980) and
        (267.2540,252.2250) .. (262.5830,250.5070) .. controls (259.0660,249.2080) and
        (256.6350,248.2920) .. (253.5570,248.0020) .. controls (250.6470,247.7720) and
        (249.4580,248.2220) .. (249.5820,248.7990) -- cycle;
      \path[fill=c393939] (249.7080,248.8370) .. controls (249.8840,250.1630) and
        (253.8340,250.0520) .. (257.4210,251.0070) .. controls (261.2090,252.0520) and
        (262.7980,252.4890) .. (265.1900,255.3610) .. controls (267.3170,258.0840) and
        (265.4850,259.2020) .. (267.0920,259.3770) .. controls (268.7900,259.3560) and
        (269.7590,257.9650) .. (269.5650,256.2330) .. controls (269.3440,254.1940) and
        (267.1410,252.2510) .. (262.5420,250.5640) .. controls (259.0720,249.2860) and
        (256.6830,248.3560) .. (253.6480,248.0550) .. controls (250.8800,247.8240) and
        (249.5750,248.2190) .. (249.7080,248.8370) -- cycle;
      \path[fill=c3d3d3d] (249.8340,248.8750) .. controls (250.0220,250.1360) and
        (253.9230,249.9650) .. (257.4510,250.9410) .. controls (261.2000,252.0120) and
        (262.6590,252.4200) .. (265.1960,255.1770) .. controls (267.4510,257.7780) and
        (265.5730,258.7550) .. (267.1530,258.9660) .. controls (268.6760,258.9960) and
        (269.6050,257.8250) .. (269.4020,256.1220) .. controls (269.1660,254.0900) and
        (267.0270,252.2770) .. (262.5010,250.6220) .. controls (259.0780,249.3640) and
        (256.7320,248.4200) .. (253.7390,248.1090) .. controls (251.1130,247.8770) and
        (249.6930,248.2160) .. (249.8340,248.8750) -- cycle;
      \path[fill=c414141] (249.9600,248.9140) .. controls (250.1600,250.1090) and
        (254.0120,249.8780) .. (257.4820,250.8740) .. controls (261.1900,251.9720) and
        (262.5200,252.3510) .. (265.2020,254.9920) .. controls (267.5850,257.4730) and
        (265.6620,258.3090) .. (267.2140,258.5550) .. controls (268.5620,258.6360) and
        (269.4500,257.6850) .. (269.2390,256.0100) .. controls (268.9880,253.9870) and
        (266.9140,252.3030) .. (262.4610,250.6790) .. controls (259.0850,249.4420) and
        (256.7800,248.4840) .. (253.8300,248.1620) .. controls (251.3460,247.9290) and
        (249.8100,248.2130) .. (249.9600,248.9140) -- cycle;
      \path[fill=c444444] (250.0860,248.9520) .. controls (250.2980,250.0820) and
        (254.1010,249.7910) .. (257.5120,250.8080) .. controls (261.1810,251.9320) and
        (262.3810,252.2830) .. (265.2080,254.8080) .. controls (267.7180,257.1680) and
        (265.7500,257.8620) .. (267.2750,258.1440) .. controls (268.4480,258.2770) and
        (269.2950,257.5440) .. (269.0760,255.8980) .. controls (268.8110,253.8830) and
        (266.8000,252.3290) .. (262.4200,250.7370) .. controls (259.0910,249.5200) and
        (256.8290,248.5480) .. (253.9210,248.2160) .. controls (251.5780,247.9820) and
        (249.9270,248.2100) .. (250.0860,248.9520) -- cycle;
      \path[fill=c484848] (250.2120,248.9900) .. controls (250.4350,250.0550) and
        (254.1900,249.7040) .. (257.5430,250.7410) .. controls (261.1710,251.8920) and
        (262.2420,252.2140) .. (265.2140,254.6240) .. controls (267.8520,256.8630) and
        (265.8380,257.4150) .. (267.3360,257.7330) .. controls (268.3340,257.9170) and
        (269.1400,257.4040) .. (268.9130,255.7860) .. controls (268.6330,253.7790) and
        (266.6870,252.3550) .. (262.3790,250.7940) .. controls (259.0970,249.5980) and
        (256.8780,248.6120) .. (254.0120,248.2690) .. controls (251.8110,248.0340) and
        (250.0440,248.2070) .. (250.2120,248.9900) -- cycle;
    \path[fill=c4c4c4c] (250.3380,249.0280) .. controls (250.5730,250.0280) and
      (254.2790,249.6160) .. (257.5730,250.6740) .. controls (261.1610,251.8510) and
      (262.1030,252.1450) .. (265.2200,254.4390) .. controls (267.9850,256.5570) and
      (265.9260,256.9680) .. (267.3970,257.3220) .. controls (268.2200,257.5570) and
      (268.9850,257.2630) .. (268.7500,255.6740) .. controls (268.4550,253.6750) and
      (266.5730,252.3800) .. (262.3380,250.8510) .. controls (259.1030,249.6750) and
      (256.9260,248.6750) .. (254.1030,248.3220) .. controls (252.0440,248.0860) and
      (250.1610,248.2040) .. (250.3380,249.0280) -- cycle;
    \path[fill=c4c4c4c] (265.1610,253.8510) .. controls (266.8080,254.8510) and
      (266.5140,255.9100) .. (267.5730,256.6160) .. controls (268.1020,256.9680) and
      (268.9850,256.9680) .. (268.4550,255.6160) .. controls (267.7490,254.0280) and
      (267.1610,253.1450) .. (263.5140,251.6740) .. controls (261.1610,250.7330) and
      (261.6320,251.7330) .. (265.1610,253.8510) -- cycle;
      \path[fill=c505050] (265.2050,253.8640) .. controls (266.8120,254.8390) and
        (266.5250,255.8720) .. (267.5580,256.5610) .. controls (268.0740,256.9040) and
        (268.9350,256.9040) .. (268.4180,255.5850) .. controls (267.7290,254.0360) and
        (267.1560,253.1750) .. (263.5980,251.7400) .. controls (261.3030,250.8220) and
        (261.7620,251.7980) .. (265.2050,253.8640) -- cycle;
      \path[fill=c545454] (265.2490,253.8760) .. controls (266.8150,254.8270) and
        (266.5350,255.8340) .. (267.5420,256.5050) .. controls (268.0450,256.8400) and
        (268.8850,256.8400) .. (268.3810,255.5540) .. controls (267.7100,254.0440) and
        (267.1510,253.2040) .. (263.6820,251.8050) .. controls (261.4450,250.9110) and
        (261.8920,251.8620) .. (265.2490,253.8760) -- cycle;
      \path[fill=c575757] (265.2920,253.8880) .. controls (266.8180,254.8140) and
        (266.5460,255.7960) .. (267.5270,256.4500) .. controls (268.0170,256.7760) and
        (268.8350,256.7760) .. (268.3440,255.5230) .. controls (267.6900,254.0520) and
        (267.1450,253.2340) .. (263.7660,251.8710) .. controls (261.5860,250.9990) and
        (262.0230,251.9260) .. (265.2920,253.8880) -- cycle;
      \path[fill=c5b5b5b] (265.3360,253.9000) .. controls (266.8210,254.8020) and
        (266.5560,255.7570) .. (267.5110,256.3940) .. controls (267.9890,256.7120) and
        (268.7850,256.7120) .. (268.3070,255.4920) .. controls (267.6700,254.0600) and
        (267.1400,253.2630) .. (263.8500,251.9360) .. controls (261.7280,251.0880) and
        (262.1530,251.9900) .. (265.3360,253.9000) -- cycle;
      \path[fill=c5f5f5f] (265.3790,253.9120) .. controls (266.8250,254.7900) and
        (266.5670,255.7190) .. (267.4960,256.3380) .. controls (267.9600,256.6470) and
        (268.7350,256.6470) .. (268.2700,255.4610) .. controls (267.6500,254.0670) and
        (267.1340,253.2930) .. (263.9340,252.0020) .. controls (261.8690,251.1760) and
        (262.2830,252.0540) .. (265.3790,253.9120) -- cycle;
      \path[fill=c636363] (265.4230,253.9240) .. controls (266.8280,254.7770) and
        (266.5770,255.6810) .. (267.4810,256.2830) .. controls (267.9320,256.5830) and
        (268.6850,256.5830) .. (268.2330,255.4300) .. controls (267.6310,254.0750) and
        (267.1290,253.3220) .. (264.0180,252.0670) .. controls (262.0110,251.2650) and
        (262.4130,252.1180) .. (265.4230,253.9240) -- cycle;
      \path[fill=c676767] (265.4670,253.9370) .. controls (266.8310,254.7650) and
        (266.5880,255.6420) .. (267.4650,256.2270) .. controls (267.9030,256.5190) and
        (268.6350,256.5190) .. (268.1960,255.3990) .. controls (267.6110,254.0830) and
        (267.1240,253.3520) .. (264.1020,252.1330) .. controls (262.1530,251.3530) and
        (262.5430,252.1820) .. (265.4670,253.9370) -- cycle;
      \path[fill=c6b6b6b] (265.5100,253.9490) .. controls (266.8350,254.7530) and
        (266.5980,255.6040) .. (267.4500,256.1720) .. controls (267.8750,256.4550) and
        (268.5850,256.4550) .. (268.1590,255.3680) .. controls (267.5910,254.0910) and
        (267.1180,253.3810) .. (264.1860,252.1980) .. controls (262.2940,251.4420) and
        (262.6730,252.2460) .. (265.5100,253.9490) -- cycle;
      \path[fill=c6e6e6e] (265.5540,253.9610) .. controls (266.8380,254.7400) and
        (266.6090,255.5660) .. (267.4340,256.1160) .. controls (267.8470,256.3910) and
        (268.5350,256.3910) .. (268.1220,255.3370) .. controls (267.5720,254.0990) and
        (267.1130,253.4100) .. (264.2700,252.2640) .. controls (262.4360,251.5300) and
        (262.8030,252.3100) .. (265.5540,253.9610) -- cycle;
      \path[fill=c727272] (265.5980,253.9730) .. controls (266.8410,254.7280) and
        (266.6190,255.5270) .. (267.4190,256.0600) .. controls (267.8180,256.3260) and
        (268.4850,256.3260) .. (268.0850,255.3050) .. controls (267.5520,254.1060) and
        (267.1080,253.4400) .. (264.3540,252.3290) .. controls (262.5780,251.6190) and
        (262.9330,252.3740) .. (265.5980,253.9730) -- cycle;
      \path[fill=c767676] (265.6410,253.9850) .. controls (266.8450,254.7160) and
        (266.6300,255.4890) .. (267.4030,256.0050) .. controls (267.7900,256.2620) and
        (268.4350,256.2620) .. (268.0480,255.2740) .. controls (267.5320,254.1140) and
        (267.1020,253.4690) .. (264.4380,252.3950) .. controls (262.7190,251.7080) and
        (263.0630,252.4380) .. (265.6410,253.9850) -- cycle;
      \path[fill=c7a7a7a] (265.6850,253.9970) .. controls (266.8480,254.7030) and
        (266.6400,255.4510) .. (267.3880,255.9490) .. controls (267.7610,256.1980) and
        (268.3850,256.1980) .. (268.0110,255.2430) .. controls (267.5120,254.1220) and
        (267.0970,253.4990) .. (264.5220,252.4600) .. controls (262.8610,251.7960) and
        (263.1930,252.5020) .. (265.6850,253.9970) -- cycle;
      \path[fill=c7e7e7e] (265.7290,254.0090) .. controls (266.8510,254.6910) and
        (266.6510,255.4130) .. (267.3720,255.8940) .. controls (267.7330,256.1340) and
        (268.3350,256.1340) .. (267.9740,255.2120) .. controls (267.4920,254.1300) and
        (267.0920,253.5280) .. (264.6060,252.5260) .. controls (263.0030,251.8850) and
        (263.3240,252.5660) .. (265.7290,254.0090) -- cycle;
      \path[fill=c828282] (265.7720,254.0220) .. controls (266.8540,254.6790) and
        (266.6610,255.3740) .. (267.3570,255.8380) .. controls (267.7050,256.0700) and
        (268.2850,256.0700) .. (267.9370,255.1810) .. controls (267.4730,254.1380) and
        (267.0860,253.5580) .. (264.6900,252.5910) .. controls (263.1440,251.9730) and
        (263.4540,252.6300) .. (265.7720,254.0220) -- cycle;
      \path[fill=c858585] (265.8160,254.0340) .. controls (266.8580,254.6660) and
        (266.6720,255.3360) .. (267.3410,255.7820) .. controls (267.6760,256.0050) and
        (268.2340,256.0050) .. (267.8990,255.1500) .. controls (267.4530,254.1450) and
        (267.0810,253.5870) .. (264.7740,252.6570) .. controls (263.2860,252.0620) and
        (263.5840,252.6940) .. (265.8160,254.0340) -- cycle;
      \path[fill=c898989] (265.8600,254.0460) .. controls (266.8610,254.6540) and
        (266.6820,255.2980) .. (267.3260,255.7270) .. controls (267.6480,255.9410) and
        (268.1840,255.9410) .. (267.8620,255.1190) .. controls (267.4330,254.1530) and
        (267.0760,253.6170) .. (264.8580,252.7220) .. controls (263.4280,252.1500) and
        (263.7140,252.7580) .. (265.8600,254.0460) -- cycle;
      \path[fill=c8d8d8d] (265.9030,254.0580) .. controls (266.8640,254.6420) and
        (266.6930,255.2590) .. (267.3110,255.6710) .. controls (267.6190,255.8770) and
        (268.1340,255.8770) .. (267.8250,255.0880) .. controls (267.4130,254.1610) and
        (267.0700,253.6460) .. (264.9420,252.7880) .. controls (263.5690,252.2390) and
        (263.8440,252.8220) .. (265.9030,254.0580) -- cycle;
      \path[fill=c919191] (265.9470,254.0700) .. controls (266.8680,254.6290) and
        (266.7030,255.2210) .. (267.2950,255.6160) .. controls (267.5910,255.8130) and
        (268.0840,255.8130) .. (267.7880,255.0570) .. controls (267.3940,254.1690) and
        (267.0650,253.6760) .. (265.0260,252.8530) .. controls (263.7110,252.3270) and
        (263.9740,252.8860) .. (265.9470,254.0700) -- cycle;
      \path[fill=c959595] (265.9910,254.0820) .. controls (266.8710,254.6170) and
        (266.7140,255.1830) .. (267.2800,255.5600) .. controls (267.5630,255.7490) and
        (268.0340,255.7490) .. (267.7510,255.0260) .. controls (267.3740,254.1770) and
        (267.0600,253.7050) .. (265.1100,252.9190) .. controls (263.8530,252.4160) and
        (264.1040,252.9500) .. (265.9910,254.0820) -- cycle;
    \path[fill=c999999] (266.0340,254.0940) .. controls (266.8740,254.6040) and
      (266.7240,255.1440) .. (267.2640,255.5040) .. controls (267.5340,255.6840) and
      (267.9840,255.6840) .. (267.7140,254.9940) .. controls (267.3540,254.1840) and
      (267.0540,253.7340) .. (265.1940,252.9840) .. controls (263.9940,252.5040) and
      (264.2340,253.0140) .. (266.0340,254.0940) -- cycle;

\end{tikzpicture}

% vim:ft=tex

        \end{column}
    \end{columns}
\end{frame}

\section{Geschichte}

\frame{\tableofcontents[currentsection]}

\subsection{GNU is Not Unix}

\begin{frame}{UNIX}
    \begin{itemize}
        \item Bell Laboratories, später AT\&T (1969)
        \item Workstations, Server, Großrechner
        \item Unixartige Systeme
            \begin{itemize}
                \item HP-UX (Hewlett-Packard)
                \item AIX (IBM)
                \item IRIX (Silicon Graphics)
                \item Solaris (Sun/Oracle)
                \item BSD, Berkeley Software Distribution
                \item macOS (Apple)
            \end{itemize}
        \item Kernel, Dateisystem, Netzwerkstack
        \item \enquote{alles} ist eine Datei
        \item Shell
    \end{itemize}
\end{frame}

\begin{frame}{GNU und die GPL}
    \begin{itemize}
        \item Unix bis Ende der 70er frei verteilt
        \item Richard Stallman (1983)
        \item \href{https://www.fsf.org/}{Free Software Foundation} (1985)
        \item POSIX (Portable Operating System Interface, ab 1985)
        \item gcc, gdb, coreutils, Emacs, …
        \item General Public License (v1 1989, v2 1991, v3 2007)
        \item GNU/Hurd (1990)
        \item \href{https://fsfe.org/}{Free Software Foundation Europe} (2001)
    \end{itemize}
\end{frame}

\subsection{Linux}

\begin{frame}{Die ersten 15 Jahre}
    \begin{itemize}
        \item Linus Torvalds programmiert Terminalemulator zum Zugriff
            auf Unix-Server der Uni Helsinki mit Minix und GNU Compiler
            (1991)
        \item v0.99 unter GPL (1992)
        \item Verbreitung auf Disketten und mit Dokumentation in \LaTeX
        \item Nutzung zusammen mit GNU am häufigsten: GNU/Linux
        \item Entwicklung von Distributionen (Slackware und Debian 1993,
            RedHat und SuSE 1994)
        \item v1.0 mit Netzwerk (1994)
        \item Portierung auf andere Plattformen (1995)
        \item Multiprozessorunterstützung (1996)
        \item große Desktopumgebungen KDE (1998) und Gnome (1999)
        \item Unterstützung großer Firmen wie IBM, Compaq, Oracle
        \item immer mehr Anwendungen
        \item Linux-Foundation (2007)
    \end{itemize}
\end{frame}

\section{Linux}

\frame{\tableofcontents[currentsection]}

\subsection{Kernel}

\begin{frame}{Linux ist …}
    \begin{itemize}
        \item ein in C geschriebener Betriebssystemkern
            \begin{itemize}
                \item Treiber
                \item Scheduler
                \item Speicherverwaltung
                \item Netzwerkstack
            \end{itemize}
        \pause
        \item portabel
            \begin{itemize}
                \item verschiedene Architekturen (ARM, Intel, Mips,
                    Motorola, PowerPC, SPARC, …)
                \item unterschiedliche Geräte (Smartphone,
                    Industrierechner, NAS, Router, Server,
                    \strong{Desktop}, Supercomputer, Kassensysteme,
                    embedded Devices, Auto, …)
            \end{itemize}
        \pause
        \item die landläufige Bezeichnung für den Kernel und das \enquote{drumrum}
        \pause
        \item Freie Software
    \end{itemize}
\end{frame}

\subsection{Distributionen}

\begin{frame}{Distributionen}
    \begin{itemize}
        \item Zusammenstellung von Software
            \begin{itemize}
                \item Kernel
                \item Systemprogramme
                \item Bibliotheken
                \item Anwendungsprogramme (Browser, Office,
                    Entwicklungsumgebung, Spiele, Multimedia, Chat, …)
            \end{itemize}
        \pause
        \item Paketmanager (.deb, .rpm, portage, …)
        \pause
        \item Installer
        \item Dokumentation
        \item Community und/oder kommerzieller Support
    \end{itemize}
\end{frame}

\begin{frame}{Distributoren}
    \begin{itemize}
        \item Debianartige:
            \href{https://www.debian.org/}{Debian},
            \href{https://www.ubuntu.com/}{Ubuntu},
            \href{https://linuxmint.com/}{Mint}
        \pause
        \item RPM-basierte:
            \href{https://www.opensuse.org/}{openSUSE},
            \href{https://www.redhat.com/}{Red Hat},
            \href{https://getfedora.org/}{Fedora},
            \href{https://www.centos.org/}{CentOS},
            \href{https://www.mageia.org/de/}{Mageia}
        \pause
        \item andere:
            \href{http://www.slackware.com/}{Slackware},
            \href{https://www.gentoo.org/}{Gentoo},
            \href{https://www.sabayon.org/}{Sabayon},
            \href{https://www.archlinux.org/}{Arch},
            \href{https://www.alpinelinux.org/}{alpine}
        \pause
        \item embedded:
            \href{https://openwrt.org/}{OpenWRT},
            \href{http://www.fli4l.de/}{fli4l},
            \href{https://www.ptxdist.org/}{PTXdist},
            \href{https://www.android.com/}{Android},
            \href{https://www.yoctoproject.org/}{Yocto},
            \href{https://buildroot.org/}{Buildroot}
        \pause
        \item Live-CD:
            \href{https://grml.org/}{Grml},
            \href{http://www.knopper.net/knoppix/}{Knoppix},
            \href{https://partedmagic.com/}{Parted Magic},
            \href{http://siduction.org/}{Siduction}
        \pause
        \item professionelle Distributionen vs. Community (und gemischt)
        \item spezielle Distributionen für Musik, Gamer, Router, …
        \item siehe auch \url{http://distrowatch.com/}
    \end{itemize}
\end{frame}

\section{Freie Software}

\frame{\tableofcontents[currentsection]}

\begin{frame}{Free/Libre and OpenSource Software}
    \begin{block}{Freie Software nach Definition von GNU, FSF, FSFE}
        \begin{description}
            \item[Freiheit 0] \strong{Verwenden.} {\small Die Freiheit,
                das Programm auszuführen wie man möchte, für jeden
                Zweck.}
            \item[Freiheit 1] \strong{Verstehen.} {\small Die Freiheit,
                die Funktionsweise eines Programms zu untersuchen, und
                es an seine Bedürfnisse anzupassen.}
            \item[Freiheit 2] \strong{Verbreiten.} {\small Die Freiheit,
                Kopien weiterzugeben und damit seinen Mitmenschen zu
                helfen.}
            \item[Freiheit 3] \strong{Verbessern.} {\small Die Freiheit,
                ein Programm zu verbessern, und die Verbesserungen
                an die Öffentlichkeit weiterzugeben, sodass die
                gesamte Gesellschaft profitiert.}
        \end{description}
    \end{block}
\end{frame}

\subsection{Oberfläche}

\begin{frame}{Grafische Oberfläche}
    \begin{block}{Desktopumgebung}
        \begin{itemize}
            \item \href{https://www.kde.org/}{KDE} (Qt)
            \item \href{https://www.gnome.org/}{Gnome} (GTK)
            \item \href{https://xfce.org/}{Xfce}
            \item
                \href{https://lxde.org/}{LXDE}\,/\,\href{https://lxqt.org/}{LXQt}
            \item \href{https://github.com/linuxmint/Cinnamon}{Cinnamon}
                (Linux Mint)
            \item \href{http://mate-desktop.org/}{Mate}
        \end{itemize}
    \end{block}
    \pause
    \begin{block}{Window-Manager}
        \begin{itemize}
            \item \href{http://fluxbox.org/}{Fluxbox}
            \item \href{https://www.enlightenment.org/}{Enlightenment}
            \item \href{https://awesomewm.org/}{awesome}
        \end{itemize}
    \end{block}
\end{frame}

\begin{frame}{Shell}
    \begin{itemize}
        \item Kommandozeileninterpreter
        \item unerlässlisch für Systeme ohne grafische Oberfläche
            (Server, embedded)
        \item Unix-Philosophie
        \item Skripte
        \item Komfortfunktionen (completion, history, job control, …)
        \item viele praktische Tools für alle erdenklichen Aufgaben
            \begin{itemize}
                \item cp, df, ls, mkdir, mv, rm, cat, head, md5sum,
                    sort, grep, tail, wc, date, du, ping, …
            \end{itemize}
        \item spezielle Anwendungen für die Shell
            \begin{itemize}
                \item \href{http://www.mutt.org/}{mutt}
                \item \href{https://irssi.org/}{irssi}
                \item \href{https://www.vim.org/}{Vim}
                \item …
            \end{itemize}
    \end{itemize}
\end{frame}

\subsection{Anwendungen}

\begin{frame}{Office}
    \framesubtitle{OpenOffice/\href{https://de.libreoffice.org/}{LibreOffice}}
    \only<1>
    {
        \pgfuseimage{officecalc1}
    }
    \only<2>
    {
        \begin{block}{Alternative für …}
            \begin{description}[Tabellenkalkulation]
                \item[Textverarbeitung] \strong{Writer} statt Word
                \item[Tabellenkalkulation] \strong{Calc} statt Excel
                \item[Präsentation] \strong{Impress} statt
                    PowerPoint\footnote{Diese Folien sind mit
                    \LaTeX\ Beamer erstellt.}
                \item[Datenbank] \strong{Base} statt Access
            \end{description}
        \end{block}
    }
\end{frame}

\begin{frame}{Browser}
    \only<1>
    {
        \pgfuseimage{firefox2}
    }
    \only<2>
    {
        \begin{block}{Frei}
            \begin{itemize}
                \item \href{https://www.mozilla.org/firefox/}{Mozilla
                    Firefox}
                \item \href{https://chromium.org/}{Chromium}
                \item \href{http://www.midori-browser.org/}{Midori}
                \item …
            \end{itemize}
        \end{block}
        \begin{block}{Proprietär}
            \begin{itemize}
                \item Google Chrome
                \item Opera
            \end{itemize}
        \end{block}
        \begin{block}{Alternative für …}
            \begin{itemize}
                \item Microsoft Internet Explorer bzw. Edge
                \item Apple Safari
            \end{itemize}
        \end{block}
    }
\end{frame}

\begin{frame}{E-Mail}
    \only<1>
    {
        \pgfuseimage{thunderbird1}
    }
    \only<2>
    {
        \begin{block}{Alternativen}
            \begin{itemize}
                \item \href{https://www.thunderbird.net/}{Thunderbird}
                \item \href{http://www.mutt.org/}{mutt}
                \item \href{http://www.claws-mail.org/}{Claws Mail}
                \item \href{http://kontact.org/}{KMail}
                \item \href{http://sylpheed.sraoss.jp/en/}{Sylpheed}
                \item
                    \href{https://wiki.gnome.org/Apps/Evolution}{Evolution}
                \item …
            \end{itemize}
        \end{block}
        \begin{block}{Für …}
            \begin{itemize}
                \item Microsoft Outlook
            \end{itemize}
        \end{block}
    }
\end{frame}

\begin{frame}{Vektorgrafik}
    \framesubtitle{\href{https://inkscape.org/}{Inkscape}}
    Alternative zu Adobe Illustrator, Corel Draw, …
    \pgfuseimage{inkscape1}
\end{frame}

\begin{frame}{Musik und Video}
    \framesubtitle{\href{https://www.videolan.org/vlc/}{VLC Media Player}}
    \pgfuseimage{vlc1}
\end{frame}

\begin{frame}{Bildbearbeitung}
    \framesubtitle{\href{https://www.gimp.org/}{GIMP}}
    \pgfuseimage{gimp1}
\end{frame}

\begin{frame}{PDF}
    Mehrere Alternativen zu Adobe Reader, u.\,a.
    \href{https://okular.kde.org/}{Okular}
    \pgfuseimage{okular1}
\end{frame}

\begin{frame}{Diverse Anwendungen}
    \begin{description}[Lernprogramme]
        \item[Dateimanager]
            \href{https://userbase.kde.org/Dolphin}{Dolphin},
            \href{https://konqueror.org/}{Konqueror},
            \href{https://wiki.gnome.org/Apps/Files}{Nautilus},
            \href{https://midnight-commander.org/}{Midnight Commander},
            …
        \item[Bilder]
            \href{https://gitlab.gnome.org/GNOME/simple-scan}{SimpleScan},
            \href{https://www.digikam.org/}{digiKam},
            \href{https://userbase.kde.org/Gwenview}{Gwenview}, …
        \item[CAD] \href{https://librecad.org/}{LibreCAD},
            \href{https://www.freecadweb.org/}{FreeCAD},
            \href{http://www.openscad.org/}{OpenSCAD},
            \href{http://kicad-pcb.org/}{KiCad}, …
        \item[Internet] Clients für FTP, IRC, XMPP, Twitter, Bittorrent,
            Usenet, VNC, RSS, …
        \item[Multimedia] Tools für Recording, Videoschnitt, Notensatz,
            3D-Modellierung, CD Brennen, Fernsehen, …
        \item[Lernprogramme] Taschenrechner, Formeleditor,
            Vokabeltrainer, …
    \end{description}
\end{frame}

\subsection{Spiele}

\begin{frame}{Spielen unter Linux}
    \begin{itemize}
        \item \href{https://store.steampowered.com/linux}{Steam}
        \item Minesweeper, Solitair und Co.
        \item \href{https://www.scummvm.org/}{ScummVM}
        \item \href{https://www.winehq.org/}{Wine}
        \item …
    \end{itemize}
\end{frame}

\section{Hilfe}

\frame{\tableofcontents[currentsection]}

\begin{frame}{Hilfe}
    \begin{itemize}
        \item \texttt{alex@hal9000:\textasciitilde\$ man woman}
        \item \texttt{/usr/share/doc}
        \item Parameter \texttt{-h} / \texttt{-{}-help}
        \item Wikis
        \item Mailinglisten
        \item IRC
        \item Foren
        \item regionale Nutzergruppen und Stammtische
        \item your local Hackerspace …
    \end{itemize}
\end{frame}

\section{Kontakt}

\begin{frame}{Kontakt}
    \begin{center}
        \begin{description}[Jabber/XMPP MUC]
            \item[WWW] \url{http://www.netz39.de/}
            \item[Twitter] @netz39
            \item[E-Mail] kontakt@netz39.de
            \item[Mailingliste] list@netz39.de
            \item[Jabber/XMPP MUC] lounge@conference.jabber.n39.eu
        \end{description}
    \end{center}
\end{frame}

\appendix

\section{Lizenz}

\begin{frame}{Lizenz}
    \begin{block}{Bilder}
        \begin{block}{The Linux Mascot}
            Penguin Tux by \href{mailto:lewing@isc.tamu.edu}{Larry Ewing}
            and \href{http://isc.tamu.edu/~lewing/linux/}{The GIMP},
            vectorized by
            \href{http://www.home.unix-ag.org/simon/}{Simon Budig},
            converted to TikZ by
            \href{http://www.texample.net/weblog/2012/apr/28/tux-tex-tikz/}{Stefan Kottwitz}.
        \end{block}
    \end{block}
    \begin{block}{Folien}
        Die Folien sind freigegeben unter
        \href{http://creativecommons.org/licenses/by-sa/3.0/de/}{Creative
        Commons Namensnennung-Weitergabe unter gleichen Bedingungen 3.0
        Deutschland Lizenz}, (CC\ BY-SA 3.0). Download der Folien unter:
        \url{https://github.com/netz39/Talks}
    \end{block}
\end{frame}

\end{document}
