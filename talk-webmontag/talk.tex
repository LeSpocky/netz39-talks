%%%%%%%%%%%%%%%%%%%%%%%%%%%%%%%%%%%%%%%%%%%%%%%%%%%%%%%%%%%%%%%%%%%%%%%%
%%% documentclass and packages
%%%%%%%%%%%%%%%%%%%%%%%%%%%%%%%%%%%%%%%%%%%%%%%%%%%%%%%%%%%%%%%%%%%%%%%%
\RequirePackage{atbegshi}           % workaround for newer PGF versions
\documentclass[hyperref={pdfpagelabels=false}]{beamer}
% https://sourceforge.net/tracker/index.php?func=detail&aid=1848912&group_id=92412&atid=600660
\usepackage{lmodern}
\usepackage[T1]{fontenc}
\usepackage[utf8]{inputenc}
\usepackage{textcomp}
\usepackage[ngerman]{babel}
\usepackage[babel,english=american,german=guillemets]{csquotes}	% french
%\usepackage{microtype}

% colors for listings
\definecolor{lightergray}{gray}{.95}
\definecolor{darkblue}{rgb}{0,0,0.5}
\definecolor{darkgreen}{rgb}{0,0.5,0}
\definecolor{darkred}{rgb}{0.5,0,0}
\definecolor{darkerblue}{rgb}{0,0,0.4}
\definecolor{darkergreen}{rgb}{0,0.4,0}
\definecolor{darkerred}{rgb}{0.4,0,0}

\usepackage{listings}
\lstloadlanguages{HTML,XML}
\lstset{
    basicstyle=\ttfamily\small\mdseries,
    keywordstyle=\bfseries\color{darkblue},
    identifierstyle=,
    commentstyle=\color{darkgray},
    stringstyle=\itshape\color{darkred},
    frame=none,
    showstringspaces=false,
    tabsize=4,
    backgroundcolor=\color{lightergray},
}

%%%%%%%%%%%%%%%%%%%%%%%%%%%%%%%%%%%%%%%%%%%%%%%%%%%%%%%%%%%%%%%%%%%%%%%%
%%% preparations for beamer
%%%%%%%%%%%%%%%%%%%%%%%%%%%%%%%%%%%%%%%%%%%%%%%%%%%%%%%%%%%%%%%%%%%%%%%%
\useinnertheme{default}
\useoutertheme{infolines}
%\usecolortheme[rgb={0.28,0.37,0.52}]{structure}
\usecolortheme[rgb={0.18,0.23,0.33}]{structure}
%\usecolortheme{beaver}
\usefonttheme{structurebold}

%%%%%%%%%%%%%%%%%%%%%%%%%%%%%%%%%%%%%%%%%%%%%%%%%%%%%%%%%%%%%%%%%%%%%%%%
%%% images
%%%%%%%%%%%%%%%%%%%%%%%%%%%%%%%%%%%%%%%%%%%%%%%%%%%%%%%%%%%%%%%%%%%%%%%%
\pgfdeclareimage[width=0.65\textwidth]{tweetlili}{tweet_lili_2009}

%%%%%%%%%%%%%%%%%%%%%%%%%%%%%%%%%%%%%%%%%%%%%%%%%%%%%%%%%%%%%%%%%%%%%%%%
%%% title, author, date
%%%%%%%%%%%%%%%%%%%%%%%%%%%%%%%%%%%%%%%%%%%%%%%%%%%%%%%%%%%%%%%%%%%%%%%%
\title[Netz39]{Netz39}
\subtitle{Real Life Virtuality -- Ein Hackerspace für Magdeburg}
\author{Alexander Dahl}
\institute[netz39.de]{\url{http://www.netz39.de/}}
\date{2012-05-14}
\subject{subj}
\keywords{foo, bar}

%%%%%%%%%%%%%%%%%%%%%%%%%%%%%%%%%%%%%%%%%%%%%%%%%%%%%%%%%%%%%%%%%%%%%%%%
%%% document
%%%%%%%%%%%%%%%%%%%%%%%%%%%%%%%%%%%%%%%%%%%%%%%%%%%%%%%%%%%%%%%%%%%%%%%%
\begin{document}

\begin{frame}
	\titlepage
\end{frame}

\begin{frame}{Überblick}
    \tableofcontents
\end{frame}

\section{Hackerspaces}

\subsection{Definition}

\begin{frame}{Ein was?}
    \begin{block}{hackerspaces.org:}
        \begin{quote}
            Hackerspaces are community-operated physical places, where people
            can meet and work on their projects.
        \end{quote}
    \end{block}
    \pause
    \begin{block}{de.wikipedia.org:}
        \begin{quote}
            Ein Hackerspace (von Hacker und Space, engl. für Raum) oder
            Hackspace ist ein physischer, häufig offener Raum, in dem sich
            Hacker und Interessierte treffen und austauschen können.
            Mitglieder mit Interessen an Wissenschaft, Technologie und
            digitaler Kunst organisieren sich meist in Vereinen.
        \end{quote}
    \end{block}
\end{frame}

\begin{frame}{Ein \dots was?}
    (add pictures here)
\end{frame}

\subsection{Geschichte}

\begin{frame}{Ein bisschen Geschichte}
    \begin{block}{World}
        \begin{itemize}
            \item 80er Jahre: Chaos Computer Club
            \item 2007: Chaos Communication Camp und 24C3
                \cite{Ohlig2007}
            \item 2009: CRE 134 \cite{Pritlove2009}
        \end{itemize}
    \end{block}
    \pause
    \begin{block}{Hello}
        %\vspace{0.5em}
        \begin{center}
            \pgfuseimage{tweetlili}
        \end{center}
    \end{block}
\end{frame}

\section{Netz39}

\subsection{past, present, future}

\begin{frame}{Netz39}
    \begin{quote}
        According to David Wells in \textbf{The Penguin Dictionary of Curious and
        Interesting Numbers}, 39 is the smallest mathematically uninteresting
        number. The book claims that it is also the first number that is
        simultaneously both interesting and uninteresting, thereby avoiding the
        paradox. \cite{Wikipedia2010}
    \end{quote}
\end{frame}{Netz39}

\begin{frame}{gestern, heute, morgen}
    \begin{block}{past}
        \begin{itemize}
            \item FSA 2009
            \item Twitter, Wiki, Mailingliste
            \item Stammtische
        \end{itemize}
    \end{block}
    \pause
    \begin{block}{present}
        \begin{itemize}
            \item Verein gegründet
            \item aktive Suche nach Location
            \item Aufmerksamkeit erregen
        \end{itemize}
    \end{block}
    \pause
    \begin{block}{future}
        \begin{itemize}
            \item Gemeinnützigkeit
            \item Projekte
        \end{itemize}
    \end{block}
\end{frame}

\subsection{Verein}

\begin{frame}{Verein}
    \begin{block}{How to participate}
        \begin{itemize}
            \item neue Mitglieder willkommen
            \item Mitgliedsbeitrag pro Monat: 30 € oder 8 €
            \item Mitgliedsantrag online
            \pause
            \item Fördermitgliedschaft
            \item Spenden
        \end{itemize}
    \end{block}
    \pause
    \begin{block}{What you get}
        \begin{itemize}
            \item Raum
            \item Leute
            \item Know-How
        \end{itemize}
    \end{block}
\end{frame}

\section{Kontakt}

\begin{frame}{Kontakt}
    \begin{description}[Twitter/identi.ca]
        \item[WWW] \url{http://www.netz39.de/}
        \item[Twitter/identi.ca] @netz39
        \item[E-Mail] kontakt@netz39.de
        \item[Mailingliste] list@netz39.de
        \item[IRC] \#netz39 auf freenode
    \end{description}

    \vspace{1em}
    \small
    Die Folien sind freigegeben unter \emph{Creative Commons
    Namensnennung-Weitergabe unter gleichen Bedingungen 3.0 Deutschland
    Lizenz.} (BY-SA).
    \normalsize
\end{frame}

\section*{Referenzen}

\begin{frame}{Referenzen}
    \begin{thebibliography}{XX}
        %\beamertemplatebookbibitems
        %\beamertemplatearticlebibitems
        \beamertemplatetextbibitems
        \bibitem[Ohlig 2007]{Ohlig2007}
            Jens Ohlig, Lars Weiler.
            \newblock {\em Building a Hacker Space}.
            \footnotesize
            \newblock \url{http://events.ccc.de/congress/2007/Fahrplan/events/2133.en.html}
            \normalsize
        \bibitem[Pritlove 2009]{Pritlove2009}
            Tim Pritlove, Astera, Johannes Grenzfurthner.
            \newblock {\em CRE134 Hackerspaces}.
            \newblock \url{http://cre.fm/cre134}
        \bibitem[Wikipedia 2010]{Wikipedia2010}
            Wikipedia
            \newblock {\em 39 (number)}
            \scriptsize
            \newblock \url{https://en.wikipedia.org/w/index.php?title=39\_\%28number\%29\&oldid=344011024}
            \normalsize
    \end{thebibliography}
\end{frame}

\end{document}
