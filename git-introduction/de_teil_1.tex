%%%%%%%%%%%%%%%%%%%%%%%%%%%%%%%%%%%%%%%%%%%%%%%%%%%%%%%%%%%%%%%%%%%%%%%%
%%% documentclass and packages
%%%%%%%%%%%%%%%%%%%%%%%%%%%%%%%%%%%%%%%%%%%%%%%%%%%%%%%%%%%%%%%%%%%%%%%%
\RequirePackage{atbegshi}           % workaround for newer PGF versions
\documentclass{beamer}
% https://sourceforge.net/tracker/index.php?func=detail&aid=1848912&group_id=92412&atid=600660
\usepackage{lmodern}
\usepackage[T1]{fontenc}
\usepackage[utf8]{inputenc}
\usepackage{textcomp}
\usepackage[ngerman]{babel}
\usepackage[babel,english=american,german=guillemets]{csquotes}	% french
\usepackage{microtype}

\usepackage{tikz}
%\usepackage[european]{circuitikz}
%\usepackage[locale=DE]{siunitx}
%\usepackage{tikz-timing}

% colors for listings
\definecolor{lightergray}{gray}{.95}
\definecolor{darkblue}{rgb}{0,0,0.5}
\definecolor{darkgreen}{rgb}{0,0.5,0}
\definecolor{darkred}{rgb}{0.5,0,0}
\definecolor{darkerblue}{rgb}{0,0,0.4}
\definecolor{darkergreen}{rgb}{0,0.4,0}
\definecolor{darkerred}{rgb}{0.4,0,0}

%\usepackage{listings}
%\lstloadlanguages{C}
%\lstset{
%    basicstyle=\ttfamily\small\mdseries,
%    keywordstyle=\bfseries\color{darkblue},
%    identifierstyle=,
%    commentstyle=\color{darkgray},
%    stringstyle=\itshape\color{darkred},
%    frame=none,
%    showstringspaces=false,
%    tabsize=4,
%    backgroundcolor=\color{lightergray},
%}

%%%%%%%%%%%%%%%%%%%%%%%%%%%%%%%%%%%%%%%%%%%%%%%%%%%%%%%%%%%%%%%%%%%%%%%%
%%% preparations for beamer
%%%%%%%%%%%%%%%%%%%%%%%%%%%%%%%%%%%%%%%%%%%%%%%%%%%%%%%%%%%%%%%%%%%%%%%%
\useinnertheme{default}
\useoutertheme{infolines}
%\usecolortheme[rgb={0.28,0.37,0.52}]{structure}
\usecolortheme[rgb={0.18,0.23,0.33}]{structure}
%\usecolortheme{beaver}
\usefonttheme{structurebold}

%%% Ränder vergrößern für's Café Central
%\setbeamersize{text margin left=1.2cm}
%\setbeamersize{text margin right=1.2cm}

%%%%%%%%%%%%%%%%%%%%%%%%%%%%%%%%%%%%%%%%%%%%%%%%%%%%%%%%%%%%%%%%%%%%%%%%
%%% images
%%%%%%%%%%%%%%%%%%%%%%%%%%%%%%%%%%%%%%%%%%%%%%%%%%%%%%%%%%%%%%%%%%%%%%%%
%\pgfdeclareimage[height=0.7\paperheight]{durchlassstromgrbl}{durchlassstrom_gr-bl}
%\pgfdeclareimage[height=0.7\paperheight]{durchlassstromrot}{durchlassstrom_rot}

\titlegraphic{%
    \begin{tikzpicture}[
    y=0.8pt, 
    x=0.8pt, 
    yscale=-1, 
    inner sep=0pt, 
    outer sep=0pt, 
    scale=0.2
]
\begin{scope}[shift={(-448.05262,-67.70133)}]
  \path[color=black,fill=black,even odd rule,line width=16.000pt]
    (534.1205,161.8816) -- (534.1205,172.2877) -- (581.3732,172.2877) .. controls
    (571.8452,181.8308) and (562.3036,191.3603) .. (552.7463,200.8741) --
    (565.3389,200.8741) .. controls (585.4511,200.8741) and (601.7400,217.2036) ..
    (601.7400,237.3158) .. controls (601.7400,257.4280) and (585.4511,273.7170) ..
    (565.3389,273.7170) .. controls (545.2267,273.7170) and (528.8972,257.4280) ..
    (528.8972,237.3158) -- (518.4911,237.3158) .. controls (518.4911,263.1743) and
    (539.4804,284.1231) .. (565.3389,284.1231) .. controls (591.1974,284.1231) and
    (612.1462,263.1743) .. (612.1462,237.3158) .. controls (612.1462,215.3233) and
    (596.9939,196.8355) .. (576.5548,191.8042) -- (606.5180,161.8816) -- cycle;
  \begin{scope}[cm={{-1.0,0.0,0.0,-1.0,(1130.6373,475.18252)}}]
    \path[color=black,fill=black,even odd rule,line width=16.000pt]
      (565.3389,208.6889) .. controls (549.5365,208.6889) and (536.7120,221.5134) ..
      (536.7120,237.3158) .. controls (536.7120,249.4467) and (544.2722,259.8031) ..
      (554.9328,263.9587) -- (554.9328,252.2164) .. controls (550.2090,248.9346) and
      (547.1181,243.5056) .. (547.1181,237.3158) .. controls (547.1181,227.2597) and
      (555.2828,219.0950) .. (565.3389,219.0950) .. controls (575.3950,219.0950) and
      (583.5192,227.2597) .. (583.5192,237.3158) .. controls (583.5192,243.4955) and
      (580.4441,248.9328) .. (575.7450,252.2164) -- (575.7450,263.9587) .. controls
      (586.3945,259.8031) and (593.9253,249.4467) .. (593.9253,237.3158) .. controls
      (593.9253,221.5134) and (581.1413,208.6889) .. (565.3389,208.6889) -- cycle;
    \path[color=black,fill=black,even odd rule,line width=16.000pt]
      (565.3389,232.0925) .. controls (562.4657,232.0925) and (560.1156,234.4426) ..
      (560.1156,237.3158) -- (560.1156,268.4937) -- (570.5217,268.4937) --
      (570.5217,237.3158) .. controls (570.5217,234.4426) and (568.2121,232.0925) ..
      (565.3389,232.0925) -- cycle;
  \end{scope}
  \path[fill=black] (630.8529,149.4914) .. controls (625.1306,149.5458) and
    (620.4890,150.3447) .. (619.3535,151.0301) .. controls (617.0825,152.4009) and
    (616.5461,155.6877) .. (617.2480,156.4558) .. controls (617.9500,157.2237) and
    (628.6785,164.1143) .. (633.3633,166.3761) .. controls (645.9778,172.4665) and
    (632.0756,195.7720) .. (621.2161,191.6018) .. controls (613.5199,188.6464) and
    (604.7892,179.8446) .. (603.3192,179.7785) .. controls (601.8492,179.7137) and
    (599.9557,180.7048) .. (599.4726,183.2607) .. controls (598.9895,185.8166) and
    (602.4810,198.1576) .. (611.2554,207.7576) .. controls (621.7846,219.2778) and
    (638.1817,206.5491) .. (638.1817,217.0704) --
    (638.1817,265.9427)arc(179.961:0.039:13.018) -- (664.2172,203.5060) --
    (664.2172,188.9294) .. controls (664.2172,183.4623) and (670.9201,161.5406) ..
    (648.7498,152.3663) .. controls (643.3754,150.1424) and (636.5752,149.4370) ..
    (630.8529,149.4914) -- cycle;
\end{scope}
\path[color=black,fill=black,even odd rule,line width=8.333pt]
  (150.0000,23.1213) .. controls (79.9291,23.1213) and (23.1213,79.9291) ..
  (23.1213,150.0000) .. controls (23.1213,220.0708) and (79.9291,276.8787) ..
  (150.0000,276.8787) .. controls (220.0708,276.8787) and (276.8787,220.0708) ..
  (276.8787,150.0000) .. controls (276.8787,79.9291) and (220.0708,23.1213) ..
  (150.0000,23.1213) -- cycle(150.0000,40.5068) .. controls (210.4603,40.5068)
  and (259.4540,89.5397) .. (259.4540,150.0000) .. controls (259.4540,210.4603)
  and (210.4603,259.4540) .. (150.0000,259.4540) .. controls (89.5278,259.8428)
  and (41.3547,203.5210) .. (40.5068,150.0000) .. controls (40.5068,89.5397) and
  (89.5397,40.5068) .. (150.0001,40.5068) -- cycle;
\end{tikzpicture}

% vim: set ft=tex :

}

%%%%%%%%%%%%%%%%%%%%%%%%%%%%%%%%%%%%%%%%%%%%%%%%%%%%%%%%%%%%%%%%%%%%%%%%
%%% title, author, date
%%%%%%%%%%%%%%%%%%%%%%%%%%%%%%%%%%%%%%%%%%%%%%%%%%%%%%%%%%%%%%%%%%%%%%%%
\title{Versionsverwaltung mit Git}
\subtitle{Einführung}
\author{Alexander Dahl}
\institute{Netz39 e.\,V.}
\date{2014}
%\subject{subj}
\keywords{Git, Versionsverwaltung}

%%%%%%%%%%%%%%%%%%%%%%%%%%%%%%%%%%%%%%%%%%%%%%%%%%%%%%%%%%%%%%%%%%%%%%%%
%%% document
%%%%%%%%%%%%%%%%%%%%%%%%%%%%%%%%%%%%%%%%%%%%%%%%%%%%%%%%%%%%%%%%%%%%%%%%
\begin{document}

\begin{frame}
	\titlepage
\end{frame}

\logo{
    \begin{tikzpicture}[
    y=0.8pt, 
    x=0.8pt, 
    yscale=-1, 
    inner sep=0pt, 
    outer sep=0pt, 
    scale=0.1,
    opacity=0.5
]
\begin{scope}[shift={(-448.05262,-67.70133)}]
  \path[color=black,fill=black,even odd rule,line width=16.000pt]
    (534.1205,161.8816) -- (534.1205,172.2877) -- (581.3732,172.2877) .. controls
    (571.8452,181.8308) and (562.3036,191.3603) .. (552.7463,200.8741) --
    (565.3389,200.8741) .. controls (585.4511,200.8741) and (601.7400,217.2036) ..
    (601.7400,237.3158) .. controls (601.7400,257.4280) and (585.4511,273.7170) ..
    (565.3389,273.7170) .. controls (545.2267,273.7170) and (528.8972,257.4280) ..
    (528.8972,237.3158) -- (518.4911,237.3158) .. controls (518.4911,263.1743) and
    (539.4804,284.1231) .. (565.3389,284.1231) .. controls (591.1974,284.1231) and
    (612.1462,263.1743) .. (612.1462,237.3158) .. controls (612.1462,215.3233) and
    (596.9939,196.8355) .. (576.5548,191.8042) -- (606.5180,161.8816) -- cycle;
  \begin{scope}[cm={{-1.0,0.0,0.0,-1.0,(1130.6373,475.18252)}}]
    \path[color=black,fill=black,even odd rule,line width=16.000pt]
      (565.3389,208.6889) .. controls (549.5365,208.6889) and (536.7120,221.5134) ..
      (536.7120,237.3158) .. controls (536.7120,249.4467) and (544.2722,259.8031) ..
      (554.9328,263.9587) -- (554.9328,252.2164) .. controls (550.2090,248.9346) and
      (547.1181,243.5056) .. (547.1181,237.3158) .. controls (547.1181,227.2597) and
      (555.2828,219.0950) .. (565.3389,219.0950) .. controls (575.3950,219.0950) and
      (583.5192,227.2597) .. (583.5192,237.3158) .. controls (583.5192,243.4955) and
      (580.4441,248.9328) .. (575.7450,252.2164) -- (575.7450,263.9587) .. controls
      (586.3945,259.8031) and (593.9253,249.4467) .. (593.9253,237.3158) .. controls
      (593.9253,221.5134) and (581.1413,208.6889) .. (565.3389,208.6889) -- cycle;
    \path[color=black,fill=black,even odd rule,line width=16.000pt]
      (565.3389,232.0925) .. controls (562.4657,232.0925) and (560.1156,234.4426) ..
      (560.1156,237.3158) -- (560.1156,268.4937) -- (570.5217,268.4937) --
      (570.5217,237.3158) .. controls (570.5217,234.4426) and (568.2121,232.0925) ..
      (565.3389,232.0925) -- cycle;
  \end{scope}
  \path[fill=black] (630.8529,149.4914) .. controls (625.1306,149.5458) and
    (620.4890,150.3447) .. (619.3535,151.0301) .. controls (617.0825,152.4009) and
    (616.5461,155.6877) .. (617.2480,156.4558) .. controls (617.9500,157.2237) and
    (628.6785,164.1143) .. (633.3633,166.3761) .. controls (645.9778,172.4665) and
    (632.0756,195.7720) .. (621.2161,191.6018) .. controls (613.5199,188.6464) and
    (604.7892,179.8446) .. (603.3192,179.7785) .. controls (601.8492,179.7137) and
    (599.9557,180.7048) .. (599.4726,183.2607) .. controls (598.9895,185.8166) and
    (602.4810,198.1576) .. (611.2554,207.7576) .. controls (621.7846,219.2778) and
    (638.1817,206.5491) .. (638.1817,217.0704) --
    (638.1817,265.9427)arc(179.961:0.039:13.018) -- (664.2172,203.5060) --
    (664.2172,188.9294) .. controls (664.2172,183.4623) and (670.9201,161.5406) ..
    (648.7498,152.3663) .. controls (643.3754,150.1424) and (636.5752,149.4370) ..
    (630.8529,149.4914) -- cycle;
\end{scope}
\path[color=black,fill=black,even odd rule,line width=8.333pt]
  (150.0000,23.1213) .. controls (79.9291,23.1213) and (23.1213,79.9291) ..
  (23.1213,150.0000) .. controls (23.1213,220.0708) and (79.9291,276.8787) ..
  (150.0000,276.8787) .. controls (220.0708,276.8787) and (276.8787,220.0708) ..
  (276.8787,150.0000) .. controls (276.8787,79.9291) and (220.0708,23.1213) ..
  (150.0000,23.1213) -- cycle(150.0000,40.5068) .. controls (210.4603,40.5068)
  and (259.4540,89.5397) .. (259.4540,150.0000) .. controls (259.4540,210.4603)
  and (210.4603,259.4540) .. (150.0000,259.4540) .. controls (89.5278,259.8428)
  and (41.3547,203.5210) .. (40.5068,150.0000) .. controls (40.5068,89.5397) and
  (89.5397,40.5068) .. (150.0001,40.5068) -- cycle;
\end{tikzpicture}

% vim: set ft=tex :

}

\begin{frame}{Überblick}
    \tableofcontents
\end{frame}

\section{Versionsverwaltung}

\subsection{Warum}

%\begin{frame}{LED Characteristic LRTB G6TG}
%    \begin{columns}
%        \begin{column}{0.5\textwidth}
%            \begin{figure}
%                \pgfuseimage{durchlassstromrot}
%            \end{figure}
%        \end{column}
%        \begin{column}{0.5\textwidth}
%            \begin{figure}
%                \pgfuseimage{durchlassstromgrbl}
%            \end{figure}
%        \end{column}
%    \end{columns}
%\end{frame}

\begin{frame}{Nach eine wahren Begebenheit}
    Paul und Bill sind unsicher, wer wann den Quellcodebaum kopiert hat
    und von welchem gemeinsamen Stand man ausgegangen war. Sie fragen
    sich ob Steve noch was daran geändert hatte und wann. Alan hatte
    irgendeinen Stand mal ins Git getan, welches die anderen drei aber
    nicht nutzen. Paul schlägt vor im Dateisystem zu schauen, wann die
    Dateien zuletzt geändert wurden und von Hand den Code zu
    vergleichen. Alan schlägt vor, die verschieden existierenden Stände
    in Zweige ins Git zu tun, um mit dessen Hilfe die Änderungen
    nachvollziehen zu können. Paul und Bill verstehen nicht, was er
    meint. Es stellt sich raus, dass Quellcode auch in Zip-Dateien
    verschickt wurde. Der Satz \enquote{Hattest Du nicht damals…?} fällt
    häufiger.

    Eine Stunde später ist Paul immernoch nicht klar, welchen Codebaum
    er damals Bill gegeben hatte und warum. Er sagt er hätte mal in der
    Version 2.3 versucht Schritt für Schritt zwei Codebäume von Hand
    zusammenzuführen und war nach zwei, drei Dateien stehen geblieben.

    Die Frage, ob eine Änderung Pauls vom letzten Jahr noch von Bill
    übernommen wurde, bleibt offen.
\end{frame}

\begin{frame}{Versionsverwaltung?}
    \begin{itemize}
        \item Was?
        \item Wann?
        \item Wer?
        \item Wo?
        \item Warum?
    \end{itemize}
\end{frame}

\section{Contact}

\begin{frame}{Contact}
    \begin{center}
        \begin{description}[Twitter/identi.ca]
            \item[WWW] \url{http://www.netz39.de/}
            \item[Twitter/identi.ca] @netz39
            \item[E-Mail] kontakt@netz39.de
            \item[Mailingliste] list@netz39.de
            \item[IRC] \#netz39 on freenode
        \end{description}
    \end{center}
\end{frame}

\appendix

\section{License}

\begin{frame}{License}
    \begin{block}{Slides and Pictures}
        This work is licensed under a \emph{Creative Commons
        Attribution-ShareAlike 3.0 Unported License}, (CC-BY-SA 3.0).
        Source at: \url{https://github.com/netz39/Talks}
    \end{block}
\end{frame}

\end{document}
